% Copyright 2023 The terCAD team. All rights reserved.
% Use of this content is governed by a CC BY-NC-ND 4.0 license that can be found in the LICENSE file.

\subsection{Configuring Deployment}
\markboth{Configuring Deployment}{Configuring Deployment}


\subsubsection{Build Web Package}

Creating a web package is handled by `flutter build -v web --release` command. It'll be more strict to our code, and 
that might lead to additional failures as "non-constant instances of IconData".

\begin{lstlisting}
// ./lib/_classes/data/goal_app_data.dart
factory GoalAppData.fromJson(Map<String, dynamic> json) {
  return GoalAppData(
    icon: json['icon'] != null
      ? IconData(json['icon'], fontFamily: 'MaterialIcons') // Error
      : null,
// ... other code
\end{lstlisting}

\noindent To fix this issue, we need to ensure that all instances of IconData are used in a constant context, that can 
be solved by the next modification:
\begin{lstlisting}
// ./lib/_mixins/formatter_mixin.dart
mixin FormatterMixin {
  static final Map<String, IconData> _cache = {};

  static IconData getIconFromString(int icon) {
    if (_cache.containsKey(icon)) {
      return _cache[icon]!;
    } else {
      const String fontFamily = 'MaterialIcons';
      return IconData(icon, fontFamily: fontFamily);
    }
  }
  }(*@ \stopnumber @*)

// ./lib/_classes/data/goal_app_data.dart
factory GoalAppData.fromJson(Map<String, dynamic> json) {
  return GoalAppData(
    icon: json['icon'] != null
      ? FormatterMixin.getIconFromString(json['icon'])
      : null,
\end{lstlisting}

Now, when we'll call `getIconFromString` with a dynamic icon name, it will create the `IconData`-instance for that name 
if it doesn't already exist and cache it in the `\_cache`-map. If the same icon name is used again in the future, 
it will return the cached `IconData`-instance, allowing tree shaking to work correctly.

And, finally, we suppress the error itself because, when we're dealing with dynamic values that cannot be statically 
analyzed at compile time, tree shaking is not possible for those parts of the code:

\begin{lstlisting}[language=bash]
flutter build -v web --release --no-tree-shake-icons
\end{lstlisting}

Another problem is that our production code is obfuscated. That means that `AccountAppData` will be converted to 
something like `minified:iP`; and our restoring from a transactions is failing, but can be fixed via `runtimeType`
property of classes:

\begin{lstlisting}
// ./lib/_classes/data/transaction_log.dart
  static void init(AppData store, String type, Map<String, dynamic> data) {
    final goal = GoalAppData(title: '', initial: 0.0).runtimeType.toString();
    final account = AccountAppData(title: '', type: '').runtimeType.toString();
    final bill = BillAppData(title: '', account: '', category: '').runtimeType.toString();
    final budget = BudgetAppData(title: '').runtimeType.toString();
    final currency = CurrencyAppData(title: '').runtimeType.toString();
    final typeToClass = {
      goal: (data) => GoalAppData.fromJson(data),
      account: (data) => AccountAppData.fromJson(data),
      bill: (data) => BillAppData.fromJson(data),
      budget: (data) => BudgetAppData.fromJson(data),
      currency: (data) => CurrencyAppData.fromJson(data),
    };
// ... other code
\end{lstlisting}

Finally, we can no longer ignore the notification "Incorrect use of ParentDataWidget" while it has become crucial, and 
failure to address it may result in rendered widgets appearing as greyed-out pages.

Regarding deployment, while we are following a Serverless approach, which means there is no dedicated server to 
deploy into, we can improve usage of our GitHub Pages for deployment:

\begin{lstlisting}[language=yaml]
name: Web Deployment

on:
  push:
    branches:
      - main

jobs:
  build:
    runs-on: ubuntu-latest

    steps:
      - uses: actions/checkout@v3
      - uses: subosito/flutter-action@v2
        with:
          channel: 'stable'
          cache: true
      - run: flutter --version

      - name: Install Dependencies
        run: flutter pub get

      - name: Build Web Package
        run: flutter build -v web --release --no-tree-shake-icons --base-href="/app-finance/"

      - name: Add Coverage Report to the Package
        run: cp -R ./coverage ./build/web/coverage

      - name: Update GitHub Pages 
        uses: peaceiris/actions-gh-pages@v3
        with:
          github_token: ${{ secrets.GITHUB_TOKEN }}
          publish_dir: ./build/web
\end{lstlisting}

\noindent Valuable point here is to update `base-href` for our Web package. It has to be aligned with GitHub Pages 
notation:

\begin{lstlisting}[language=bash]
https://<account-name>.github.io/<repository-name>/
\end{lstlisting}
