% Copyright 2023 The terCAD team. All rights reserved.
% Use of this content is governed by a CC BY-NC-ND 4.0 license that can be found in the LICENSE file.

\subsection{Configuring Deployment}
\markboth{Configuring Deployment}{Configuring Deployment}


\subsubsection{Build Web Package}

Creating a web package is handled by `flutter build -v web --release` command. It'll be more strict to our code, and 
that might lead to additional failures as "non-constant instances of IconData".

\begin{lstlisting}
// ./lib/_classes/data/goal_app_data.dart
factory GoalAppData.fromJson(Map<String, dynamic> json) {
  return GoalAppData(
    icon: json['icon'] != null
      ? IconData(json['icon'], fontFamily: 'MaterialIcons') // Error
      : null,
// ... other code
\end{lstlisting}

\noindent To fix this issue, we need to ensure that all instances of IconData are used in a constant context, that can 
be solved by the next modification:
\begin{lstlisting}
// ./lib/_mixins/formatter_mixin.dart
mixin FormatterMixin {
  static final Map<String, IconData> _cache = {};

  static IconData getIconFromString(int icon) {
    if (_cache.containsKey(icon)) {
      return _cache[icon]!;
    } else {
      const String fontFamily = 'MaterialIcons';
      return IconData(icon, fontFamily: fontFamily);
    }
  }
  }(*@ \stopnumber @*)

// ./lib/_classes/data/goal_app_data.dart
factory GoalAppData.fromJson(Map<String, dynamic> json) {
  return GoalAppData(
    icon: json['icon'] != null
      ? FormatterMixin.getIconFromString(json['icon'])
      : null,
\end{lstlisting}

Now, when we'll call `getIconFromString` with a dynamic icon name, it will create the `IconData`-instance for that name 
if it doesn't already exist and cache it in the `\_cache`-map. If the same icon name is used again in the future, 
it will return the cached `IconData`-instance, allowing tree shaking to work correctly.

And, finally, we suppress the error itself because, when we're dealing with dynamic values that cannot be statically 
analyzed at compile time, tree shaking is not possible for those parts of the code:

\begin{lstlisting}[language=bash]
flutter build -v web --release --no-tree-shake-icons
\end{lstlisting}
