% Copyright 2023 The terCAD team. All rights reserved.
% Use of this content is governed by a CC BY-NC-ND 4.0 license that can be found in the LICENSE file.

\subsection{Adding Telemetry} \label{telemetry}
\markboth{Gating}{Adding Telemetry}

The application telemetry provide insights (a data-driven decision-making) into user interactions with different app 
features, illuminating areas requiring enhancements or the introduction of new features. Essentially, it serves as a 
compass guiding an app's ongoing development, ensuring that decisions are based on empirical data rather than 
assumptions. Analytics help pinpoint bottlenecks, crashes, and areas that hinder our app's seamless operation. 


\subsubsection{Activating Google Analytics}

To understand user behavior (user experience), their geographical locations (for localization prioritization), and the 
nature of issues they encounter (exception handling), external analytical tools are often employed. Google Analytics 
is a prominent example of such tools (without neglecting other options, as Amplitude, Flurry, Mixpanel, etc.).

The process is started from creating a Firebase account and a new project there (by clicking on the relative 
button and proceeding with obvious steps; just don't forget to enable Google Analytics while creating the project): 
\href{https://console.firebase.google.com}{https://console.firebase.google.com}.

Then we have to add packages \q{flutter pub add firebase\_core} and \q{flutter pub add firebase\_analytics}, activate 
FlutterFire CLI  via \q{dart pub global activate flutterfire\_cli} and \q{npm install -g firebase-tools} to make it 
(\q{flutterfire}-command) globally accessible for our environment. As a note for Windows users, be attentive to 
notifications like that (add \q{bin}-folder to PATH global variable, \cref{img:fs-windows-path}):

\begin{lstlisting}[language=terminal]
# Notification for Windows
Warning: Pub installs executables into C:\Users\...\AppData\Local\Pub\Cache\bin, which is not on your path.
# For Linux
Warning: Pub installs executables into $HOME/.pub-cache/bin, which is not on your path.
You can fix that by adding this to your shell's config file (.bashrc, .bash_profile, etc.):
  export PATH="$PATH":"$HOME/.pub-cache/bin"
\end{lstlisting}

\noindent By triggering sequence of \q{firebase login} and \q{flutterfire configure} from a command line 
we'll be able to configure the project.

\begin{lstlisting}[language=terminal]
$ firebase login

i  Firebase optionally collects CLI and Emulator Suite usage and error reporting information to help improve our 
products. Data is collected in accordance with Google`s privacy policy (https://policies.google.com/privacy) and 
is not used to identify you.

? Allow Firebase to collect CLI and Emulator Suite usage and error reporting information? Yes
i  To change your data collection preference at any time, run \q{firebase logout} and log in again.

Visit this URL on this device to log in:
https://accounts.google.com/o/oauth2/auth?...

Waiting for authentication...

+  Success! Logged in as ...

$ flutterfire configure

i Found 1 Firebase projects.
V Select a Firebase project to configure your Flutter application with  fingrom-9030a (Fingrom)
V Which platforms should your configuration support (use arrow keys & space to select)? - macos, web, ios, android
# ... other info

Firebase configuration file lib/firebase_options.dart generated successfully with the following Firebase apps:

Platform  Firebase App Id
web       ...
android   ...
ios       ...
macos     ...

Learn more about using this file and next steps from the documentation:
  https://firebase.google.com/docs/flutter/setup
\end{lstlisting}

\noindent Once completed, we should import the generated file and adjust \q{main}-method:

\begin{lstlisting}
// ./lib/main.dart
import 'package:app_finance/firebase_options.dart';
import 'package:firebase_core/firebase_core.dart';

void main() async {
  WidgetsFlutterBinding.ensureInitialized();
  await Firebase.initializeApp(
    options: DefaultFirebaseOptions.currentPlatform,
  );
  runApp(/* app initialization*/);
}
\end{lstlisting}

\noindent That might be the case for web, android, ios, and macos; but we have additionally linux and windows that are 
not supported by now, so, let's patch \q{firebase\_options.dart}-file to return null instead of throwing an exception:

\begin{lstlisting}
// ./lib/firebase_options.dart
class DefaultFirebaseOptions {
  static FirebaseOptions? get currentPlatform {
    if (kIsWeb) {
      return web;
    }
    switch (defaultTargetPlatform) {
      case TargetPlatform.android:
        return android;
      case TargetPlatform.iOS:
        return ios;
      case TargetPlatform.macOS:
        return macos;
      case TargetPlatform.windows:
      case TargetPlatform.linux:
      default:
        return null;
    }
  }
\end{lstlisting}

\noindent Alternatively, own cross-functional component can be implemented by following Firebase REST API 
(\href{https://firebase.google.com/docs/reference/rest/database}{https://firebase.google.com/docs/})... but it's a 
topic for another discussion.

After all of that modifications, it's needed to evaluate \q{flutter clean} to avoid any errors; and right after the
\q{flutter run}-command we would be able to view a log report on 
\href{https://analytics.google.com}{https://analytics.google.com}, \cref{img:pt-analytics}.

\img{prototyping/google-analytics-report}{Google Analytics results}{img:pt-analytics}

\noindent The negative impact here is that \q{firebase\_analytics} library force to use 19 version as a minimal SDK for 
Android, so, it's needed to be updated \q{android/app/build.gradle} file by adjusting \q{minSdkVersion}-property:

\begin{lstlisting}[language=yaml]
## ./android/app/build.gradle
android {
  defaultConfig {
\end{lstlisting}
{
\xpretocmd{\lstlisting}{\vspace{-12pt}}{}{}
\begin{lstlisting}[firstnumber=4, backgroundcolor=\color{backred}]
(*@\kdiff{-}@*)    minSdkVersion flutter.minSdkVersion
\end{lstlisting}
\begin{lstlisting}[firstnumber=4, backgroundcolor=\color{backgreen}]
(*@\kdiff{+}@*)    minSdkVersion 19
\end{lstlisting}
\begin{lstlisting}[firstnumber=5]
  }
}
\end{lstlisting}
}

\noindent It's not yet completed unless we follow additional instructions from  
\href{https://firebase.google.com/docs/guides/}{https://firebase.google.com/docs/guides/} to configure
Firebase Analytics per each thread (Android, iOS, Macos, Web); as for Android:

\begin{lstlisting}[language=yaml]
## ./android/build.gradle
buildscript {
  repositories {
\end{lstlisting}
{
\xpretocmd{\lstlisting}{\vspace{-12pt}}{}{}
\begin{lstlisting}[firstnumber=4, language=yaml, backgroundcolor=\color{backgreen}]
      google()
\end{lstlisting}
\begin{lstlisting}[firstnumber=5, language=yaml]
      # ... other settings
  }
  dependencies {
    # ... other settings
\end{lstlisting}
\begin{lstlisting}[firstnumber=9, language=yaml, backgroundcolor=\color{backgreen}]
    classpath 'com.google.gms:google-services:4.3.3'
\end{lstlisting}
\begin{lstlisting}[firstnumber=10, language=yaml]
  }(*@ \stopnumber @*)
}

## ./android/app/build.gradle
\end{lstlisting}
\begin{lstlisting}[firstnumber=2, language=yaml, backgroundcolor=\color{backgreen}]
apply plugin: 'com.google.gms.google-services'
\end{lstlisting}
\begin{lstlisting}[firstnumber=3, language=yaml]
# ... other settings
dependencies {
  # ... other settings
\end{lstlisting}
\begin{lstlisting}[firstnumber=6, language=yaml, backgroundcolor=\color{backgreen}]
  implementation 'com.google.firebase:firebase-analytics:17.4.1'
\end{lstlisting}
\begin{lstlisting}[firstnumber=7, language=yaml]
}
\end{lstlisting}
}


\subsubsection{Enabling Crashlytics Reports}

In addition to overseeing usage statistics, the system has the capability to monitor and log errors that may arise 
during a program execution. This entails capturing details about the nature and origin of errors, facilitating efficient 
debugging and troubleshooting. By enabling error tracking through the \q{firebase\_crashlytics}-package, developers gain 
valuable insights into potential issues, enabling them to identify and address problems effectively:

\begin{lstlisting}
void main() async {
  runZonedGuarded<Future<void>>(() async {
    await Firebase.initializeApp();
    if (kIsWeb) {
      FlutterError.onError = (details) {
        FlutterError.presentError(details);
        FirebaseAnalytics.instance.logEvent(
          name: 'flutter-error', 
          parameters: {'error': details.toString()},
        );
      };
    } else {
      FlutterError.onError =
        FirebaseCrashlytics.instance.recordFlutterFatalError;
    }
    // ... other stuff
  }, (error, stack) =>
    FirebaseCrashlytics.instance.recordError(error,stack,fatal:true)
  );
}
\end{lstlisting}

\noindent Crashlytics might return an error for iOS integration as "Crashlytics has detected a missing dSYM for version".
That would require additional adjustments for the \q{Run Script} build phase:

\begin{lstlisting}[language=terminal]
# Run Script: 
${PODS_ROOT}/FirebaseCrashlytics/run

# Input Files: 
$(SRCROOT)/$(BUILT_PRODUCTS_DIR)/$(INFOPLIST_PATH)
\end{lstlisting}

\noindent Crash-free metrics stand as the sole measure to gauge the stability of our application (\cref{img:g-crash}), 
and it is imperative that we proactively maintain it at a minimum of 100\% (and there's no mistake about the number 
here, the app should work with the precision of an atomic clock if we're talking about a user-centric design).

\img{prototyping/crashlytics-report}{Google Firebase: Crash-free users report}{img:g-crash}
