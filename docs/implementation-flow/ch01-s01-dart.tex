% Copyright 2023 The terCAD team. All rights reserved.
% Use of this content is governed by a CC BY-NC-ND 4.0 license that can be found in the LICENSE file.

\subsection{Dart}
\markboth{Bootcamping}{Dart} \label{dart}

Dart is known for its simplicity and readability as object-oriented and class-based language. It offers C-style syntax 
and object-oriented programming (OOP) concept (inheritance, polymorphism, and encapsulation).

\paragraph{Dart Virtual Machine (VM)} includes a just-in-time (JIT) compiler for fast development cycles and an 
ahead-of-time (AOT) compiler for optimized production performance by translating \q{Dart}-code into a native machine 
code. Its web compiler translates Dart into JavaScript or Web Assembly. With a support of multi-threaded 
execution, VM provides efficient utilization of modern multi-core processors via \q{Isolates} (\emph{a separate threads, 
that communicate via message passing}) for concurrent executions.

Dart Virtual Machine is designed as a platform-agnostic interpreter (\emph{desktop environments [Windows, macOS, Linux], 
mobile devices [Android, iOS], and web browsers}) with a modular and extensible architecture (capabilities to extend 
core-components as primitives and classes). It includes components such as Dart Core Libraries, Garbage Collector 
(\emph{with separating objects into young and old generations, automatically reclaims memory occupied by objects that 
are no longer reachable}), and Dart Development Compiler (DDC, \emph{designed to compile Dart to JavaScript / TypeScript 
by enabling real-time code changes and efficient debugging for web applications}).

\paragraph{Dart Software Development Kit (SDK)} provides a command-line interface (CLI), a package manager, 
and a collection of utility libraries that simplify common tasks. Package Manager (\q{pub}) controls libraries and 
their dependencies via \q{pubspec.yaml}-file by using public (\href{https://pub.dev}{https://pub.dev}) and private 
(\q{dart pub add private\_package --hosted https://...}) storages.

\paragraph{Entire End-to-End Stack}

Dart is well-equipped to cover the full stack development. \q{Flutter}, powered by Dart, enables developers to write 
code once and run it on multiple platforms, providing a unified user experience. \q{Aqueduct} provides a robust 
foundation for building scalable and high-performance backend services and APIs. Dart's versatility extends 
seamlessly to a database layer, offering compatibility with a wide range of database systems, both SQL and 
noSQL databases (as \q{Aqueduct ORM} for PostgreSQL, \q{Realm} for MongoDB), as well as databases built specifically 
using Dart as their foundation (like \q{DartinoDB}).


\newpage
\subsubsection{Checking Primitives} \label{dart-prim}

Primitives (built-in data types) behave as objects with own methods and properties, and can be even extended 
by \q{extension}:

\begin{lstlisting}
int counter = 123; // counter.toDouble() => 123.0
double pi = 3.141592; // pi.clamp(0, 3) => 3.0
bool isVisible = true; // true.toString() => 'true'
String text = 'Some content'; // text.codeUnits // List<int>
dynamic tmp = 1; tmp = 'test'; tmp = false; // ...
// 'StringBuffer' as a way to construct strings sequentially
final spread = StringBuffer();
spread.write('some text...'); // concatenate
spread.writeln('... another content'); // add indentation at the end
// But long text can be also broken and implicitly spliced back
String test = 'some long text...'
  '...continuation';
// Convert other types to String by interpolation (via '$' sign)
String sample = '$counter $isVisible';
// Extend functionality of 'double'-type (except 'null' - a keyword)
extension DoubleExt on double {
  // 0.12.pcnt => '12.0%'
  String get pcnt => '${this * 100}%';
}
\end{lstlisting}

\noindent Collections are represented by \q{List}, \q{Map}, \q{Set}, \q{Records}, and \q{Queue}. Where \q{List} is an 
ordered collection of objects; \q{Map}, as a collection of key/value pairs, is used to retrieve a value by its 
associated key with a maintained key uniqueness; as well as \q{Set} -- to control uniqueness of variables. \q{Queue} 
implements both stack and queue behavior (where, ListQueue -- keeps a cyclic buffer of elements, DoubleLinkedQueue -- 
to guarantee constant time on 'add', 'remove-at-ends' and 'peek'-operations). As a distinctive feature of collections, 
their iterativeness is declared by a subtype:

\begin{itemize}
  \item \q{HashMap} and \q{HashSet}, unordered, both provide an access to items (by key) in (potentially) amortized 
  constant time;
  \item \q{LinkedHashMap} and \q{LinkedHashSet} iterate in key insertion order;
  \item \q{SplayTreeMap} and \q{SplayTreeSet} are based on a self-balancing binary tree that allows most single-entry 
  operations in amortized logarithmic time;
\end{itemize}

\begin{lstlisting}
List<int> data = [1, 2, 3]; // data.last; // 3
Set<String> scope = {'a', 'b', 'a'}; // scope.length; // 2
HashMap<String, int> hash = {'test': 123}; // hash.values; // [123]
// Records - immutable, not iterable, with an object-based access
var map = (text: 'sample', at: 123); // map.text; // 'sample'
\end{lstlisting}


\subsubsection{Extracting Patterns}

Pattern matching in programming involves comparing data structures against predefined patterns, which enables 
conditional execution based on the match:

\begin{lstlisting}
const a = 'a';
const b = 'b';
final obj = [a, b];

if (obj.length == 2 && obj[0] == a && obj[1] == b) {
  print('$a, $b');
}
// Or, it can be transformed into:
switch (obj) {
  case [a, b]:
    print('$a, $b');
    break;
  // ... other conditions
}
\end{lstlisting}

\noindent Destructuring, on the other hand, offers a means to extract individual elements from complex data structures 
like arrays or objects and assign them to variables. This technique is especially valuable when dealing with complex 
objects, allowing you to fetch their properties efficiently:

\begin{lstlisting}
// Operation with basics
final users = ['User 1', 'User 2', 'User 3'];
final [adminName, userName, guestName] = users;

// To declare new variables from a nested structure
var (name, [code, type]) = ('User 1', [123, true]);

// Object extraction
class Transaction { 
  String description;
  String category;
  double amount;
}

const scope = [
  Transaction('Buy groceries', 'Groceries', 50.0),
  // ... others
];

for (final Transaction(:category, :amount) in scope) {
  print('$category: $amount');
}
\end{lstlisting}


\subsubsection{Operating with Dates}

Let's revise a few build-in capabilities of dates' type from an example:\\
- IF \q{end}-date is lower or equal to \q{current}-date THEN return 1\\
- IF \q{start}-date is equal to \q{current}-date THEN return 0\\
- ELSE return a relative value in the range of \q{0...1} by calculating the float value based on the ratio of the number 
of days between \q{start} and \q{current} to the total number of days between \q{start} and \q{end}.

\begin{lstlisting}
calculateValue(DateTime start, DateTime end, DateTime current) {
  if (end.isBefore(current) || end.isAtSameMomentAs(current))
    return 1;
  if (start.isAtSameMomentAs(current))
    return 0;  
  int total = end.difference(start).inDays;
  int diff = current.difference(start).inDays;
  return diff / total;
}
\end{lstlisting}

\noindent \q{DateFormat} is used to convert and parse dates into a specific format, such as 'yyyy-MM-dd' 
(\emph{4 characters for a year, dash, 2 symbols for a month, dash, last two to identify the date}), while considering 
localization preferences.

\begin{lstlisting}
DateTime dt = DateTime(2023, 11, 1);
var text = DateFormat.yMMMEd().format(dt);
print(text);  // Wed, Nov 1, 2023
// Parsing stringified date back without exceptions on failure
DateTime? value = DateTime.tryParse(text);
\end{lstlisting}

\noindent And, we may extend \q{DateTime} by an insensible usage of \q{DateFormat}-class:

\begin{lstlisting}
extension DateTimeExt on DateTime {
  // DateTime(/* ... */).toMonthDay(); 
  String toMonthDay(Locale? locale) =>
      DateFormat.MMMMd(locale ?? 'en_US').format(this);
}
\end{lstlisting}

\noindent \q{DateTime}-class doesn't inherently handle time zones, but it can be covered by external libraries like 
\q{timezone} or \q{intl}. To facilitate time processing, one can utilize the "Unix" timestamp, also known as an epoch 
timestamp, through the properties \q{microsecondsSinceEpoch} and \q{millisecondsSinceEpoch}. 
 
Additionally, for application testing purposes, consider the use of libraries like \q{fake\_datetime} to mock the 
current time, enabling more effective testing scenarios.

\newpage
\subsubsection{Understanding Operations}

Dart provides a wide range of built-in functions and libraries to cover various operations and functionalities:

\begin{lstlisting}
double num1 = 10.0; double num2 = 3.0; double? num3;
// Addition '+' and subtraction '-'
// Multiplication '*' and division '/'
// '~/' Integer Division (Floor Division)
result = num1 ~/ num2; // 3 (int)
// '%' Modulus (Remainder)
result = num1 % num2; // 1.0
// '??' null-check statement
num3 ?? num1; // return 'num1' if 'num3' is null
num3 ??= 2.0; // set 'num3' if it's null
// Tilde (Bitwise NOT)
int result = ~(-1); // 0
// Bitwise AND '&', OR '|', XOR '^'
int resultAND = 5 & 3; // 1
// Left '<<' and Right '>>' Shift
int resultLeft = 8 << 2; // 32
int resultRight = 8 >> 1; // 4
// 'sqrt' Square Root
result = sqrt(num3); // 1.4142135623730951
// 'pow' Exponentiation
result = pow(num3, 3); // 8.0
// 'sin' Sine, 'cos' Cosine, 'tan' Tangent
// 'log' Natural Logarithm
result = log(num3); // 0.6931471805599453
// 'log10' Base-10 Logarithm
result = log(num3) / ln10; // 0.3010299956639812
/// Overloading: null + 5 = 5
extension Ext<T extends num> on T? {
  T? operator +(T val) => this != null ? this + val : val;
}
/// Overloading '|' to merge maps
extension Merge<T, K> on Map<T, K> {
  Map<T, K> operator |(Map<T, K> other) => 
      {...this}..addEntries(other.entries);
}
\end{lstlisting}

\noindent In addition, each type has own operands:

\begin{lstlisting}
// Range limitation: '-10.0' to '0.0'; '100.0' to '3.0'
return (num as double).clamp(0, 3);
// Dates comparison
DateTime(now.year, now.month).isAfter(createdAt);
createdAt.isBefore(DateTime.now());
// Transformations
final m = {'Sample': 1, 'sample': 2};
CanonicalizedMap.from(m, (k) => k.toLowerCase()) // {'Sample': 1}
\end{lstlisting}


\newpage
\subsubsection{Overloading Operators}

"Magic methods" are often referred to as "operator overloading" or "special methods." They allow to declare a custom 
behavior for built-in operations:

\begin{itemize}
  \item \q{toString} returns a string representation of an object, can be used for a serialization and deserialization 
  process of an object;
  \item \q{call} allows an object to be treated as a function;
  \item \q{hashCode} returns a hash code for an object (to use it as a key for \q{Map} and \q{Set}, and to override \q{==});
  \item \q{operator}: overload basic operands (as \q{==} to compare, \q{+} to sum, etc.);
  \item \q{get} and \q{set} -- to override the behavior of getting and setting properties.
\end{itemize}

\begin{lstlisting}
class Person {
  // Private property - 'null' or 'DateTime'
  DateTime? _createdAt;
  // Required from a constructor since cannot be null
  String name;
  // Post-initialization
  late DateTime _createdAt = DateTime.now();
  // var person = Person('Tom');
  Person(this.name);
  // person() // 'Hello from Tom!'
  String call() => 'Hello from $name!';
  // person.createdAt = DateTime(2023, 01, 01);
  set createdAt(DateTime date) => _createdAt = date;
  // print(person.createdAt); // 2023-01-01 00:00:00
  DateTime get createdAt => _createdAt;
  // print(Person('Tom') == Person('Terry')); // false
  @override // Pre-requisite for any operator change
  int get hashCode => hashValues(name); // core-method to hash value
  @override // 'covariant' limits comparison to the same class
  bool operator ==(covariant Person other) => other.name == name;
  // person = Person.fromString('Tom');
  factory Person.fromString(String name) {
    return Person(name);
  }
  // print(person); // 'Tom'
  @override
  String toString() => name;
}
\end{lstlisting}


\subsubsection{Declaring Input Arguments}

A few options are available for the arguments declaration:

\begin{lstlisting}
// Ordered scope
/// Sample: add('test', null) 
/// Sample: add('test', 123) 
void add(String value, int? id);

// With optional arguments
/// Sample: add('test')
void add(String value, [int? id]);
void add(String value, [int id = 123]); // preset for 'id'

// Named attributes
/// Sample: add(value: 'test')
/// Sample: add(id: 1, value: 'test')
void add({String value, int? id});
void add({String value, int id = 123});

// Mix
/// Sample: add('test', id: 123)
void add(String value, {int? id});
\end{lstlisting}

\noindent In the context of classes, their properties can be protected (with \q{@protected}-annotation, accessible for 
extends), private (started from underscore; neither visible, nor accessible from extends), and public (others); 
with extra options as \q{final} (immutable), \q{const} (not changeable for run-time), \q{static} (to 
grant access without object initialization), and \q{late} (postponed initialization):

\begin{lstlisting}
class Person {
  // Only static fields can be declared as const
  static const gender = 'unknown';
  // Private property
  String? _priv;
  // Immutable after an object creation
  final String name;
  // Delayed assignment
  late final int age;
  // Post-initialization for a dynamic content
  late cast = PersonCast(age);
  // constructor
  Person(
    this.name, {
    int? age
    }
  // Post-processing (random age if not set)
  ) : this.age = age ?? Random().nextInt(120);
}
\end{lstlisting}


\subsubsection{Asserting Functions Definition}

By utilizing a type definition (\q{typedef}), we can simplify the representation of a complex structure as arguments or 
declare the expected structure of a function, which is then propagated as an argument:

\begin{lstlisting}
typedef SetViewFunction = String Function(Currency input);
typedef Nested = Map<String, List<Person>>;

class CurrencySelector {
  final Nested key; // 'final' to set once, immutable then
  SetViewFunction? setView; // accept nullable by '?'
  const CurrencySelector({
    this.key,
    this.setView,
  }) {
    setView ??= setDefaultView; // Set if 'null'
  }
  // Short notation ('=>') for "one-line" methods
  String setDefaultView(Currency input) => /* ... */;
}
\end{lstlisting}

\noindent Once the function is declared with a \q{void}-result, it can be extended through an extension of
\q{VoidCallback}.


\subsubsection{Casting Dynamic Structures}

Generics can be useful in specific cases, serving as a deferred type declaration:

\begin{lstlisting}
extension StringExt on String {
  // Letter in uppercase highlights a type variability
  /// Sample: '[1,2,3]'.toList<int>(); // [1, 2, 3] 
  List<T> toList<T>() { // Limitation: <T extends num>
    final data = length > 0 ? json.decode(this) : [];
    // Convert to declared Type
    return data.cast<T>();
  }
}
\end{lstlisting}

\noindent Alternatively, \q{dynamic} can be employed to accommodate various types:

\begin{lstlisting}
extension StringExt on String {
  /// Sample: '[1,"2",3.0]'.toList(); // [1, '2', 3.0] 
  List<dynamic> toList() =>
      length > 0 ? json.decode(this) : [];
}
\end{lstlisting}

\noindent Furthermore, we may perform type casting using \q{is} or \q{as}:

\begin{lstlisting}
(val as double).toInt();
if (val is Function) val();
\end{lstlisting}


\subsubsection{Initializating Classes}

Every object is an instance of a class, and all classes (except \q{null}) inherit from the \q{Object}-class. While each 
class typically has only one superclass (via \q{extends}), it might cover multiple interfaces by using \q{implements}.
Moreover, dart implements a form of multiple inheritance via \q{Mixin}s. Extension methods provide a means to 
enhance a class's functionality without altering the class or creating a subclass.

\begin{lstlisting}
class WebDavProtocol
    // inherits superclass methods and properties
    extends AbstractProtocol 
    // class extended by methods from mixin
    with FileImportMixin, FileExportMixin 
    // structure is controlled by interfaces
    implements InterfaceProtocol, InterfaceFile {
  // ... content
}
// Interface ('abstract' to define a pure interface)
abstract interface class InterfaceFile {
  String getContentSync(File data);
}
// Mixin
mixin FileImportMixin {
  Future<String?> importFile(List<String> ext) async { /* ... */ }
}
\end{lstlisting}
  
\noindent Additionally, class modifiers enable a control over how a class can be extended or specialized:

\begin{itemize}
  \item \q{abstract} -- prevents instantiation, allows methods definition without their implementation;
  \item \q{base} -- disallows implementation (\q{implements}), only inheritance (\q{extends});
  \item \q{interface} -- guarantees interface implementation, but not inheritance;
  \item \q{final} -- subtyping is forbidden from that class;
  \item \q{sealed} -- provides enumerable set of subtypes (to be used in \q{switch});
  \item \q{@immutable}-notation -- force all properties to be \q{final}.
\end{itemize}

\noindent With a few restrictions (cannot be combined): \q{abstract} with \q{sealed}; \q{interface}, \q{final} or 
\q{sealed} with \q{mixin}.

\noindent By using \q{extends}-operand (especially for \q{StatefullWidget}s pairs, \ref{flutter-state}), it might be 
raised an error that defined property is not accessible:

\begin{lstlisting}
// AccountEditPage <- AccountAddPage <- AbstractPage
// AccountEditPageState <- AccountAddPageState <- AbstractPageState

abstract class AbstractPage<T> extends StatefulWidget {
  int selectedMenu = 0;
  // Be accurate with "UniqueKey"-method usage for nested widgets
  // That will trigger "initState" each time of store updates 
  AbstractPage() : super(key: UniqueKey());
}

abstract class AbstractPageState<T extends AbstractPage> 
      extends State<T> {
  @override
  Widget build(BuildContext context) => Text('test');
}

class AccountAddPage extends AbstractPage {
  String? title;

  AccountAddPage({this.title}) : super();

  @override
  AccountAddPageState createState() => AccountAddPageState();
}

class AccountAddPageState<T extends AccountAddPage> extends AbstractPageState<AccountAddPage> {
  String? get title => widget.title; // OK
}

class AccountEditPage extends AccountAddPage {
  String uuid;

  AccountEditPage({required this.uuid}) : super();

  @override
  AccountEditPageState createState() => AccountEditPageState();
}

class AccountEditPageState extends 
      AccountAddPageState<AccountEditPage> {
  // ERROR: The getter 'uuid' isn't defined for 'AccountAddPage'
  String get uuid => widget.uuid; 
  // FIX: (widget as AccountEditPage).uuid
}
\end{lstlisting}


\subsubsection{Differentiating Constructors}

Classes may have a several constructors and factories to simplify or differentiate their flow:

\begin{lstlisting}
// Search-widget types
enum CurrencySelectorType {
  searchAnchor,
  searchAnchorMin,
  searchAnchorBar,
}
// Class declaration
class CurrencySelector<K extends CurrencySelectorItem> extends StatefulWidget {
  final CurrencySelectorType searchType;
  final String? value;
  // ... other properties

  // SearchAnchor
  const CurrencySelector({ /* ... */}) : 
      searchType = CurrencySelectorType.searchAnchor;
  // SearchAnchor(isFullScreen: false)
  const CurrencySelector.min({ /* ... */}) : 
      searchType = CurrencySelectorType.searchAnchorMin;
  // SearchAnchor.bar
  const CurrencySelector.bar({ /* ... */}) : 
      searchType = CurrencySelectorType.searchAnchorBar;
  // Differentiate State-manager per taken type
  @override
  CurrencySelectorState createState() => switch (searchType) {
        CurrencySelectorType.searchAnchorBar =>
          CurrencySelectorBarState<CurrencySelector, K>(),
        CurrencySelectorType.searchAnchorMin =>
          CurrencySelectorMinState<CurrencySelector, K>(),
        _ => CurrencySelectorState<CurrencySelector, K>(),
      };
  // Instantiate class from its string representation 
  factory CurrencySelector<K>.fromString(String data) {
    return CurrencySelector<K>(/* ... */);
  }
  // Create object by using structured data
  factory CurrencySelector<K>.fromJson(Map<String, dynamic> json) =>
      CurrencySelector<K>(/* ... */);
}
\end{lstlisting}


\newpage
\subsubsection{Enumerating Classifications}

Enumerated types, often referred to as enums, are a distinct category of classes designed for the representation of a 
predefined set of constant values. As an example, to validate function arguments or switch statement:

\begin{lstlisting}
enum Status { active, inactive, pending }
void processStatus(Status status) {
  switch (status) {
    case Status.active:   /* ... */ break;
    case Status.inactive: /* ... */ break;
    case Status.pending:  /* ... */ break;
  }
}
void main() => processStatus(Status.active);
\end{lstlisting}

\noindent Using enums in function parameters enforces valid enum values, ensuring that only acceptable options are used 
as inputs. Enums also facilitate seamless conversion between enum values and strings, and enable associating additional 
information with enum values (of any data type):

\begin{lstlisting}
enum AppAccountType { type1, type2, type3 }
// Convert to string, and back
String type = AppAccountType.type1.toString();
AppAccountType? accountType = AppAccountType.values
    .where((v) => v.toString() == type).firstOrNull;
// 'enum' with values
enum AppCardType { debit('Debit Card'), credit('Credit Card') }
print(AppCardType.debit.value); // Debit Card
\end{lstlisting}

\noindent Sealed classes may be used in the same way as \q{enum} by defining a union class to leverage a pattern matching.

\begin{lstlisting}
sealed class DefaultState { /* ... */ }
class LoadingState extends DefaultState { /* ... */ }
class LoadedState extends DefaultState { /* ... */ }
class ErrorState extends DefaultState { /* ... */ }
// Matching
String message = switch (state) {
    LoadingState() => 'Loading...',
    // Taking a property state from the class
    ErrorState(message:var msg) => 'Error taken: ${msg}',
    // To access the object in a case block by its type 
    LoadedState obj => obj.data,    
    _ => 'Unhandled state (all others)'
  };
\end{lstlisting}


\newpage
\subsubsection{Managing Components}

The \q{import}-statement is used to include code from our own project or external modules (packages) into the 
application. When we're importing code from the own project, it can be used relative paths to specify the location of 
the file or module:

\begin{lstlisting}
import 'relative_path_to_file.dart';
\end{lstlisting}

\noindent However, using relative paths is not recommended when we aim to maintain cross-platform compatibility. 
Instead, we should employ the package name as a prefix to the path, seamlessly, just as it's done for all external 
components:

\begin{lstlisting}
import 'package:app_finance/relative_path_to_file.dart'; // our
import 'package:flutter/material.dart'; // Flutter SDK
import 'package:flutter_svg/flutter_svg.dart'; // external
// Generated by the 'flutter build' command
import 'package:flutter_gen/gen_l10n/app_localization.dart';
\end{lstlisting}

\noindent To address naming collisions between different libraries, we may employ the technique of named imports:

\begin{lstlisting}
import 'package:flutter/material.dart';
import 'package:intl/intl.dart' as intl; // TextDirection collision

class GaugePainter extends CustomPainter {
  final intl.NumberFormat formatter;
  // ... other code
\end{lstlisting}

\noindent The useful optimization of the startup time of our application, when dealing with large or rarely used 
packages, is the usage of deferred imports. Deferred loading means that the code from the imported package will not be 
loaded immediately when the application starts. Instead, it will be loaded on-demand, typically when we explicitly 
request it in the code:

\begin{lstlisting}
import 'package:heavy_package/heavy_package.dart' deferred as heavy;
// To initialize library
final lib = await heavy.loadLibrary();
\end{lstlisting}

\noindent An essential element of the application size optimization involves the reduction of superfluous imports. 
We should refrain from importing complete libraries when we only require a fraction of their functionality:

\begin{lstlisting}
import 'dart:async' show Future, Stream;
\end{lstlisting}

\noindent In certain scenarios, it becomes necessary to address environment-specific requirements by employing distinct 
sets of libraries:

\begin{lstlisting}
import 'package:file_picker_web/file_picker_web.dart' 
  if (dart.library.io) 'package:file_picker/file_picker.dart';
\end{lstlisting}


\newpage
\subsubsection{Asserting in Tests}

Assertions in tests help to ensure the correctness and reliability of given code by matching the expected result.

\begin{lstlisting}
// Compare two dynamic values
expect(1, 0); // Expected: <0>; Actual: <1>
// Partial comparison
expect(result, contains('some text'));
expect(result, matches('part'));
expect(result, matches(RegExp('^\w+$')));
// Occurrence check
expect(result, isIn('some text'));
expect(result, everyElement(expected)); // List
expect(result, anyElement(expected)); // List
expect(result, containsValue(expected)); // List
expect(result, containsPair(position, value)); // Map
// Numbers check (or objects with operators: >, <, ==)
expect(result, isZero); // isPositive, ...
expect(result, inInclusiveRange(0, 100));
expect(result, greaterThan(0));
// Simple type check
expect(result, isNull); // isList, isTrue, ...
// Class type check
expect(result, isA<SomeClass>());
expect(result, isNot(isA<SomeClass>())); // Inverse
expect(result, same(someObject)); // Same instance check
// Stream assertions
expect(stream, emits(expected));
expect(stream, emitsInOrder(expectedList));
expect(stream, neverEmits(notExpected));
// Async assertions, when result is a Future
await expectLater(result, completion(expected));

// Exception check (call() => throw Exception())
expect(() => call(), throwsA(isException));

// Customize failed output
expect(result, expected, reason: 'Expected ___!');

// Skip comparison (deactivate test)
expect(result, 0, skip: true); // Or, set String as a comment
\end{lstlisting}

\noindent In addition, \q{allOf}-matcher allows combining multiple matchers, \q{anyOf} -- check if any of the conditions 
are met. While \q{expect}-function is used to compare the results of some operations, \q{mocktail}-package declares 
\q{verify}-function to ensure that some other functions were used (or not) during the test execution.
