% Copyright 2023 The terCAD team. All rights reserved.
% Use of this content is governed by a CC BY-NC-ND 4.0 license that can be found in the LICENSE file.

\subsection{Dart}
\markboth{Bootcamping}{Dart}

\subsubsection{Overloading operators}

In Dart, "magic methods" are often referred to as "operator overloading" or "special methods." They allow you to define 
custom behaviors for built-in operations:

\begin{itemize}
  \item \q{toString} returns a string representation of an object, can be used for a serialization and deserialization 
  process of an object;
  \item \q{call} allows an object to be treated as a function;
  \item \q{hashCode} returns a hash code for an object (to use object as a key for \q{Map} and \q{Set}, and override \q{==});
  \item \q{operator==} compares two objects for equality;
  \item \q{get} and \q{set} -- to override the behavior of getting and setting properties.
\end{itemize}

\begin{lstlisting}
class Person {
  // Required from constructor
  String name;
  // Post-initialization
  late DateTime _createdAt = DateTime.now();
  // var person = Person('Tom');
  Person(this.name);
  // person() // 'Hello from Tom!'
  String call() => 'Hello from $name!';
  // person.createdAt = DateTime(2023, 01, 01);
  set createdAt(DateTime date) => _createdAt = date;
  // print(person.createdAt); // 2023-01-01 00:00:00
  DateTime get createdAt => _createdAt;
  // print(Person('Tom') == Person('Terry')); // false
  @override
  int get hashCode => int.parse(name.codeUnits.join(''));
  @override
  bool operator ==(Object other) => other is Person && other.name == name;
  // person = Person.fromString('Tom');
  factory Person.fromString(String name) {
    return Person(name);
  }
  // print(person); // 'Tom'
  @override
  String toString() => name;
}
\end{lstlisting}
