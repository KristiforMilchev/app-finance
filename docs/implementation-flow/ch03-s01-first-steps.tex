% Copyright 2023 The terCAD team. All rights reserved.
% Use of this content is governed by a CC BY-NC-ND 4.0 license that can be found in the LICENSE file.

\subsection{[TBD] First Steps}
\markboth{Implementing Core Functionality}{First Steps}

We're starting from knowing nothing neither about Flutter, nor Dart.

The better way to start is to start from a template, so, let's create a sample project with Flutter,
by using out of the box functionality.

Firstly download and install Flutter from the official Flutter website (https://flutter.dev) by following 
the installation instructions specific to your operating system (as an example, for Windows you would need 
to register library in the system's PATH variable in Windows).

Choose an IDE for Flutter development, such as Android Studio, Visual Studio Code, or IntelliJ IDEA, or any other, 
and install the Flutter and Dart plugins/extensions for your chosen IDE. As an example, for VS Code steps would be the
next:
\begin{itemize}
    \item Open Visual Studio Code.
    \item Go to the Extensions view by clicking on the square icon on the left sidebar or pressing Ctrl+Shift+X.
    \item Search for "Flutter" in the Extensions marketplace and click on the "Flutter" extension by Dart Code.
    \item Click on the "Install" button to install the Flutter extension.
    \item Search for "Dart" in the Extensions marketplace and click on the "Dart" extension by Dart Code.
    \item Click on the "Install" button to install the Dart extension.
\end{itemize}

After that IDE will ask regarding the Flutter SDK folder, it is the directory where the Flutter software 
development kit (SDK) is installed on your computer. It contains all the necessary files, tools, and dependencies 
required for Flutter development. By default, the Flutter SDK folder is located in different paths depending on 
your operating system:

\noindent Windows: C:/flutter\\
macOS: /Users/<username>/flutter\\
Linux: /home/<username>/flutter\\
... or the place where you've put it by your own a step before.
\\
\noindent Inside the Flutter SDK folder, you'll find various directories and files, including:

\noindent bin: Contains the Flutter command-line tools, such as flutter, dart, and other utilities.\\
cache: Stores downloaded packages and other cached files.\\
doc: Documentation files for Flutter.\\
examples: Code examples and sample projects provided by the Flutter team.\\
packages: Core Flutter packages and dependencies.\\
version: A text file indicating the version of the Flutter SDK.

It's important to note that when using Flutter with an integrated development environment (IDE) 
like Visual Studio Code or Android Studio, you'll need to configure the IDE to point to the Flutter 
SDK folder so that it can access the necessary tools and libraries for Flutter development.

Open your IDE and choose the option to create a new Flutter project. As an example, for VS Code steps would be 
the next:
\begin{itemize}
    \item Open the Command Palette by pressing Ctrl+Shift+P (or Cmd+Shift+P on macOS).
    \item Type "Flutter: New Project" and select the "Flutter: New Project" command from the list.
    \item Enter a name for your project and choose a directory to create the project in.
    \item Wait for Visual Studio Code to create the project structure.
\end{itemize}

Open the project in your IDE and navigate to the lib directory.
Now we may start modifying the `main.dart` file to customize the project. This is where the entry point of the project
is located.

% \markdownInput{./first-steps/app-template.md}

\noindent Let's try the first Run!
\begin{itemize}
    \item Open the terminal in Visual Studio Code by clicking on "View" in the top menu and selecting "Terminal" 
    or by pressing Ctrl+ backtick ().
    \item In the terminal, navigate to the project directory by using the cd command followed by the project 
    directory path.
    \item Run the project using the command `flutter run`. Make sure you have a device (physical or emulator) 
    connected.
\end{itemize}

To debug the app, set breakpoints in your code by clicking on the left margin of the desired line. 
Then, use the command `flutter run --debug` to launch the app in debug mode. VS Code will show the output of the app 
in the terminal and provide debugging features like breakpoints, variable inspection, and stepping through code. 
Additionally, you can use the VS Code debugger toolbar to control the app execution.

After running the project, the Flutter app will launch on the connected device or emulator, and we will see the 
default Flutter app template.

\img{first-steps/app-template}{First run with autogenerated application}

Connect a physical device or start an emulator/simulator.
Use your IDE's run or debug command to launch the Flutter app.
The app will build and deploy to the connected device/emulator.

Start experimenting with Flutter widgets and explore different UI layouts.
Refer to the Flutter documentation (https://flutter.dev/docs) and online resources for learning more about Flutter 
development. Try adding features, integrating APIs, and building more complex functionality to expand your knowledge.

Flutter offers a rich set of widgets and a vibrant community, so don't hesitate to explore and ask questions when 
needed. Happy coding!