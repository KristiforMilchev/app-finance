% Copyright 2023 The terCAD team. All rights reserved.
% Use of this content is governed by a CC BY-NC-ND 4.0 license that can be found in the LICENSE file.

\subsection{[TBD] First Steps}
\markboth{Implementing Core Functionality}{First Steps}

We're starting from knowing nothing neither about Flutter, nor Dart.

The better way to start is to start from a template, so, let's create a sample project with Flutter,
by using out of the box functionality.

Firstly download and install Flutter from the official Flutter website (https://flutter.dev) by following 
the installation instructions specific to your operating system (as an example, for Windows you would need 
to register library in the system's PATH variable in Windows).

Choose an IDE for Flutter development, such as Android Studio, Visual Studio Code, or IntelliJ IDEA, or any other, 
and install the Flutter and Dart plugins/extensions for your chosen IDE. As an example, for VS Code steps would be the
next:
\begin{itemize}
    \item Open Visual Studio Code.
    \item Go to the Extensions view by clicking on the square icon on the left sidebar or pressing Ctrl+Shift+X.
    \item Search for "Flutter" in the Extensions marketplace and click on the "Flutter" extension by Dart Code.
    \item Click on the "Install" button to install the Flutter extension.
    \item Search for "Dart" in the Extensions marketplace and click on the "Dart" extension by Dart Code.
    \item Click on the "Install" button to install the Dart extension.
\end{itemize}

After that IDE will ask regarding the Flutter SDK folder, it is the directory where the Flutter software 
development kit (SDK) is installed on your computer. It contains all the necessary files, tools, and dependencies 
required for Flutter development. By default, the Flutter SDK folder is located in different paths depending on 
your operating system:

\noindent Windows: C:/flutter\\
macOS: /Users/<username>/flutter\\
Linux: /home/<username>/flutter\\
... or the place where you've put it by your own a step before.
\\
\noindent Inside the Flutter SDK folder, you'll find various directories and files, including:

\noindent bin: Contains the Flutter command-line tools, such as flutter, dart, and other utilities.\\
cache: Stores downloaded packages and other cached files.\\
doc: Documentation files for Flutter.\\
examples: Code examples and sample projects provided by the Flutter team.\\
packages: Core Flutter packages and dependencies.\\
version: A text file indicating the version of the Flutter SDK.

It's important to note that when using Flutter with an integrated development environment (IDE) 
like Visual Studio Code or Android Studio, you'll need to configure the IDE to point to the Flutter 
SDK folder so that it can access the necessary tools and libraries for Flutter development.

Open your IDE and choose the option to create a new Flutter project. As an example, for VS Code steps would be 
the next:
\begin{itemize}
    \item Open the Command Palette by pressing Ctrl+Shift+P (or Cmd+Shift+P on macOS).
    \item Type "Flutter: New Project" and select the "Flutter: New Project" command from the list.
    \item Enter a name for your project and choose a directory to create the project in.
    \item Wait for Visual Studio Code to create the project structure.
\end{itemize}

Open the project in your IDE and navigate to the lib directory.
Now we may start modifying the `main.dart` file to customize the project. This is where the entry point of the project
is located.

% \markdownInput{./first-steps/app-template.md}

\noindent Let's try the first Run!
\begin{itemize}
    \item Open the terminal in Visual Studio Code by clicking on "View" in the top menu and selecting "Terminal" 
    or by pressing Ctrl+ backtick ().
    \item In the terminal, navigate to the project directory by using the cd command followed by the project 
    directory path.
    \item Run the project using the command `flutter run`. Make sure you have a device (physical or emulator) 
    connected.
\end{itemize}

To debug the app, set breakpoints in your code by clicking on the left margin of the desired line. 
Then, use the command `flutter run --debug` to launch the app in debug mode. VS Code will show the output of the app 
in the terminal and provide debugging features like breakpoints, variable inspection, and stepping through code. 
Additionally, you can use the VS Code debugger toolbar to control the app execution.

After running the project, the Flutter app will launch on the connected device or emulator, and we will see the 
default Flutter app template.

\img{first-steps/app-template}{First run with autogenerated application}

Connect a physical device or start an emulator/simulator.
Use your IDE's run or debug command to launch the Flutter app.
The app will build and deploy to the connected device/emulator.

To run Flutter with the Android SDK, you need to make sure that you have the Android SDK installed on your machine and 
properly configured. Here are the steps to set up Flutter with the Android SDK:

Install Android Studio: Download and install Android Studio from the official website 
(https://developer.android.com/studio). This will install the Android SDK along with other necessary tools.

Set up Android SDK in Android Studio: Launch Android Studio and go through the initial setup wizard. When 
prompted to choose the type of setup, select "Standard" and follow the instructions to install the necessary components.

Set up Android device or emulator: Connect a physical Android device to your computer using a USB cable or set 
up an Android emulator in Android Studio. Make sure the device/emulator is recognized by your machine.

Install Flutter: Download and install the Flutter SDK from the official Flutter website (https://flutter.dev). 
Extract the downloaded archive to a desired location on your machine.

Set up Flutter in your environment: Add the Flutter binary path to your system's PATH variable. The exact steps 
for setting up the PATH variable may vary depending on your operating system. Refer to the official Flutter 
documentation for detailed instructions (https://flutter.dev/docs/get-started/install).

Verify Flutter installation: Open a terminal or command prompt and run the following command to verify that 
Flutter is properly installed: flutter doctor

This command will check your environment and display a report with any missing dependencies or configuration issues. 
Make sure to resolve any reported issues before proceeding.

Run a Flutter app on Android: Once you have set up the Android SDK and Flutter, you can run a Flutter app on an 
Android device or emulator. In your Flutter project directory, open a terminal or command prompt and run the 
following command: flutter run.

This command will build the Flutter app and launch it on the connected Android device or emulator. Make sure the 
device/emulator is already running or connected before running the command.

To set up an Android emulator in Android Studio, follow these steps:

Open Android Studio: Launch Android Studio on your computer.

Open AVD Manager: Click on the "Configure" button in the welcome screen, or go to "File" -> "Settings" (on Windows/Linux) 
or "Android Studio" -> "Preferences" (on macOS) to open the settings. In the settings window, navigate to 
"Appearance \& Behavior" -> "System Settings" -> "Android SDK".

Install emulator system image: In the "Android SDK" settings, click on the "SDK Platforms" tab. Here, you will 
see a list of available Android versions. Check the box next to the desired Android version(s) for which you want
 to create an emulator. Click "Apply" to install the selected system image(s).

Open AVD Manager: Once the system image(s) are installed, click on the "SDK Tools" tab in the "Android SDK" settings. 
Check the box next to "Android Emulator" and click "Apply" to install the emulator component.

Open AVD Manager: After installing the emulator component, go to "Tools" -> "AVD Manager" in the main toolbar to 
open the Android Virtual Device (AVD) Manager.

Create a new virtual device: In the AVD Manager, click on the "Create Virtual Device" button. This will open the 
"Select Hardware" window.

Choose a device definition: In the "Select Hardware" window, choose a device definition that matches the 
characteristics of the emulator you want to create. You can select a predefined device from the list, or click 
on "New Hardware Profile" to create a custom device profile. Click "Next" to proceed.

Select a system image: In the "System Image" window, select the desired Android version and select a system image 
from the list. The list will only show the system images that you have installed in step 3. Click "Next" to continue.

Configure emulator options: In the "Emulator Options" window, you can customize additional options for the 
emulator, such as its name, orientation, and initial scale. Make the desired selections and click "Finish" 
to create the emulator.

Launch the emulator: Once the emulator is created, it will appear in the AVD Manager list. Click on the green 
play button next to the emulator to launch it.

Wait for the emulator to start: The emulator will take some time to start up. You will see a progress bar 
indicating the startup progress. Once the emulator is fully started, you can use it to test and run your 
Android applications.

Start experimenting with Flutter widgets and explore different UI layouts.
Refer to the Flutter documentation (https://flutter.dev/docs) and online resources for learning more about Flutter 
development. Try adding features, integrating APIs, and building more complex functionality to expand your knowledge.

Flutter offers a rich set of widgets and a vibrant community, so don't hesitate to explore and ask questions when 
needed. Happy coding!