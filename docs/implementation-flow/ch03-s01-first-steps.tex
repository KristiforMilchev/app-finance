% Copyright 2023 The terCAD team. All rights reserved.
% Use of this content is governed by a CC BY-NC-ND 4.0 license that can be found in the LICENSE file.

\subsection{Configuring Development Environment}
\markboth{Implementing Core Functionality}{Configuring Development Environment}

Assume, our first steps we're doing from knowing nothing neither about Flutter, nor Dart. So, the better way to start 
from that position is to use some templates (and Flutter kindly provides us such an option).

But, initially, we do need to download and install Flutter (\href{https://flutter.dev}{https://flutter.dev}) by 
following the installation instructions specific to operating system that is used (as an example, for Windows it 
would be needed to register library directory location in the system's \q{PATH}-variable after downloading its sources, 
\cref{img:fs-windows-path}).

\img{first-steps/windows-path-variable}{Windows 10 \q{PATH}-variable location}{img:fs-windows-path}

The choice of a better Integrated Development Environment (IDE) for Flutter development is something negligible, widely
proposed Visual Studio Code, Android Studio, IntelliJ IDEA; but the truth is that nobody cares, what was used (even VIM,
Notepad, or nano), if the work is done. Nonetheless, further discussions in the book would be relevant to Visual 
Studio Code (\href{https://code.visualstudio.com/}{https://code.visualstudio.com/}) usage in "how-to configure IDE" 
examples if nobody minds.

With Visual Studio Code, let's start from Extensions Marketplace usage by installing "Flutter" and "Dart" extensions, 
right after that IDE will ask regarding the Flutter Software Development Kit (SDK) folder (it's the directory where 
the Flutter was installed / downloaded into).

Inside the Flutter SDK folder, we may find a various directories and files, including:

\begin{itemize}
  \item bin -- Flutter command-line tools, such as \q{flutter}, \q{dart}, and other utilities;
  \item cache -- cached files and downloaded packages;
  \item doc -- documentation;
  \item examples -- code examples and sample projects;
  \item packages -- core Flutter packages and dependencies.
\end{itemize}

After those steps, in Visual Studio Code from Command Palette (by pressing [Ctrl + Shift + P], or [Cmd + Shift + P] 
on macOS) will become available "Flutter: New Project"-command, by clicking on which prompt will ask regarding project 
name and location to generate a basic structure. Going farther, let's open the project in IDE and frustrate a minute 
about a wide variety of different scaring files that's been generated. Breathe... we do need only one of them by 
now, that's located in the \q{lib}-directory. File \q{main.dart} is the project's central orbit, the entry point of our 
application.

\begin{lstlisting}
// ./lib/main.dart
import 'package:flutter/material.dart';
// Entry point
void main() {
  runApp(const MyApp());
}
// 'Stateless' means missing any saved states of MyApp
class MyApp extends StatelessWidget {
  // 'super' - parent class, 'key' - widget unique identifier
  const MyApp({super.key}); // Class constructor
  
  // '@override' - replace implementation from parent class
  @override
  Widget build(BuildContext context) { // Returns 'Widget' instance
    return MaterialApp(
      title: 'Flutter Demo',
      // Application theme definition
      theme: ThemeData(
        // Basic color scheme
        colorScheme: ColorScheme.fromSeed(seedColor: Colors.deepPurple),
        // Configuration of Widgets' styles
        useMaterial3: true,
      ),
      // Main page definition
      home: const MyHomePage(title: 'Flutter Demo Home Page'),
    );
  }
}
// 'Stateful' to re-render widget by changed internal states
class MyHomePage extends StatefulWidget {
  // 'required' means that 'title' is a mandatory for constructor on line 25
  const MyHomePage({super.key, required this.title});
  // 'final' - cannot be changed afterwards
  final String title;
  // Wrap an object for a state management
  @override
  _MyHomePageState createState() => _MyHomePageState();
}
// Underscore is a convention for internal classes, methods, and variables
class _MyHomePageState extends State<MyHomePage> {
  int _counter = 0;
  // Internal method to update '_counter' state
  void _incrementCounter() {
    // 'setState' - a special method to save state
    // ... and re-trigger 'build'-method
    setState(() {
      _counter++;
    });
  }
  // Mandatory method for any Widget-based classes
  @override
  Widget build(BuildContext context) {
    return Scaffold(
      appBar: AppBar(
        backgroundColor: Theme.of(context).colorScheme.inversePrimary,
        title: Text(widget.title),
      ),
      body: Center(
        child: Column(
          mainAxisAlignment: MainAxisAlignment.center,
          children: <Widget>[
            // Usage of 'const' is recommended to improve performance
            // ... for any Widget that's not having a dynamic parameters
            const Text(
                'You have pushed the button this many times:',
            ),
            Text(
              '$_counter', // '$' - for a variable interpolation
              style: Theme.of(context).textTheme.headlineMedium,
            ),
          ],
        ),
      ),
      floatingActionButton: FloatingActionButton(
        onPressed: _incrementCounter,
        tooltip: 'Increment',
        child: const Icon(Icons.add),
      ),
    );
  }
}
\end{lstlisting}

Let's see results of a compilation (\cref{img:fs-app}) by triggering \q{flutter run}-command from a terminal of the 
project directory (as an example, for Visual Studio Code by clicking on "View" in the top menu, selecting "Terminal"; 
short key [Ctrl + backtick]). If no emulators are enabled, it may ask to clarify the type of execution.

Valuable tip for any development process is to activate a debug mode, that can be done simply by adding \q{--debug} to 
the previous console command (sample, \q{flutter run -d windows --debug} to run project on Windows in debug mode). Then, 
the process is similar to any other programming language: set breakpoints in a code by clicking in IDE on the left 
margin of the desired line, use the command, inspect variables and their states by stepping through the code. Instead of
any debug mode it can be simply used \q{print}-function to plot anything into console. By taking into account that we 
might know exactly nothing about Flutter (and, possibly, any other programming languages), let's return to that 
statement when our application (and we) will become more mature... too early without any business logic on board.

\img{first-steps/app-template}{First run with autogenerated application}{img:fs-app}

What we can do in addition, is to emulate a real device usage. That's achievable with the Android SDK by downloading 
(\href{https://developer.android.com/studio}{https://developer.android.com/studio}), installing, and going through 
the initial wizard setup (nothing special by now, choose everything by default). One additional step is to create a 
virtual device (\cref{img:fs-android})... a couple of clicks and we're done (\cref{img:fs-create}).

\img{first-steps/android-studio}{Android Studio start page}{img:fs-android}

\img{first-steps/android-studio-create}{Android Studio choose and create Emuator}{img:fs-create}

Once we've set up Android SDK and triggered "Start" for the emulator, Flutter will recognize its availability from
an execution \q{flutter run}. Flutter app will be build and launched on the connected Android emulator (or even real 
device, connected via USB; check manuals if needed).\\
\\

\noindent Let's dive into the coding!