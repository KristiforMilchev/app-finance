% Copyright 2023 The terCAD team. All rights reserved.
% Use of this content is governed by a CC BY-NC-ND 4.0 license that can be found in the LICENSE file.

\subsection{Configuring Development Environment}
\markboth{Implementing Core Functionality}{Configuring Development Environment}

Our first steps we're doing from knowing nothing neither about Flutter, nor Dart. So, the better way to start from that
position is to use some templates, and Flutter kindly gives us that option.

But, initially, we do need to download and install Flutter (\href{https://flutter.dev}{https://flutter.dev}) by 
following the installation instructions specific to operating system that is used (as an example, for Windows it 
would needed to register library directory location in the system's PATH variable after downloading its sources).

The choice of a better Integrated Development Environment (IDE) for Flutter development is something negligible, widely
proposed Android Studio, Visual Studio Code, IntelliJ IDEA; but the truth is that nobody will care if you're using VIM,
Notepad, or nano if the work is done. Nonetheless, further discussions in the book would be relevant to Visual Studio 
Code (\href{https://code.visualstudio.com/}{https://code.visualstudio.com/}) usage in how-to examples if nobody minds.

By using Visual Studio Code, it's better not to forget open Extensions Marketplace and install "Flutter" and 
"Dart" extensions from there, then IDE will ask regarding the Flutter SDK folder (it's the directory 
where the Flutter software development kit [SDK] is installed, a step above we've been downloading to).
\\
\noindent P.S. Inside the Flutter SDK folder, we may find a various directories and files, including:

\begin{lstlisting}
bin // Contains the Flutter command-line tools, such as flutter, dart, and other utilities.
cache // Stores downloaded packages and other cached files.
doc // (!) Documentation files for Flutter.
examples // (!) Code examples and sample projects provided by the Flutter team.
packages // Core Flutter packages and dependencies.
version // A text file indicating the version of the Flutter SDK.
\end{lstlisting}

After those steps, in Visual Studio Code from Command Palette (by pressing [Ctrl + Shift + P], or [Cmd + Shift + P] 
on macOS) will become available "Flutter: New Project" command, by clicking on which prompt will ask project name 
and location to generate a basic structure. Going farther, let's open the project in IDE and frustrate a minute 
about a wide variety of different scaring files that's been generated. Breathe... we do need only one of them by 
now that's located in the `lib`-directory. File `main.dart` is the project's central orbit, the entry point of our 
application.

\begin{lstlisting}
import 'package:flutter/material.dart';

// Entry point
void main() {
  runApp(const MyApp());
}

class MyApp extends StatelessWidget {
  // `super` - parent class, `key` - widget unique identifier
  const MyApp({super.key}); // Class constructor
  
  // `@override` - replace implementation from parent class
  @override
  Widget build(BuildContext context) { // Returns `Widget` class based Object
    return MaterialApp(
      title: 'Flutter Demo',
      // Application theme definition
      theme: ThemeData(
        // Basic color scheme
        colorScheme: ColorScheme.fromSeed(seedColor: Colors.deepPurple),
        // Configuration of Widgets' styles
        useMaterial3: true,
      ),
      // Main page definition
      home: const MyHomePage(title: 'Flutter Demo Home Page'),
    );
  }
}

class MyHomePage extends StatefulWidget {
  // `required` means that `title` is mandatory for constructor on line 25
  const MyHomePage({super.key, required this.title});
  // `final` - cannot be changed afterwards
  final String title;

  @override
  State<MyHomePage> createState() => _MyHomePageState();
}

// Underscore is a convention for internal classes, methods, and variables
class _MyHomePageState extends State<MyHomePage> {
  int _counter = 0;

  void _incrementCounter() {
    // `setState` - special method of `State` to re-draw Widgets
    setState(() {
      _counter++;
    });
  }

  @override
  Widget build(BuildContext context) {
    return Scaffold(
      appBar: AppBar(
        backgroundColor: Theme.of(context).colorScheme.inversePrimary,
        title: Text(widget.title),
      ),
      body: Center(
        child: Column(
          mainAxisAlignment: MainAxisAlignment.center,
          children: <Widget>[
            // Usage of `const` is recommended to improve performance
            const Text(
                'You have pushed the button this many times:',
            ),
            // But `const` cannot be used for a dynamic content
            Text(
              '$_counter', // `$` - for a variable interpolation
              style: Theme.of(context).textTheme.headlineMedium,
            ),
          ],
        ),
      ),
      floatingActionButton: FloatingActionButton(
        onPressed: _incrementCounter,
        tooltip: 'Increment',
        child: const Icon(Icons.add),
      ),
    );
  }
}
\end{lstlisting}

Let's try together the first run (!) (\cref{img:fs-app}) by triggering the command `flutter run` from a terminal of the 
project directory (as an example, for Visual Studio Code by clicking on "View" in the top menu, selecting "Terminal"; 
short key [Ctrl + backtick]). If no emulators are enabled, it may ask to clarify the type of execution, just follow 
instructions there.

Valuable tip for any development process is to activate a debug mode, that can be done simply by adding ` --debug` to 
the previous console command (sample, `flutter run -d windows --debug` to run project on Windows in debug mode). Then, 
the process is similar to any other programming language: set breakpoints in a code by clicking in IDE on the left 
margin of the desired line, use the command, inspect variables and their states by stepping through the code. 
By taking into account that we know exactly nothing about Flutter (and, possibly, any other programming languages 
at all), let's return when out application and we'll become more mature... too early without any business logic on board.

\img{first-steps/app-template}{First run with autogenerated application}{img:fs-app}

What we can do in addition, is to emulate a real device usage. That's achievable with the Android SDK by downloading 
(\href{https://developer.android.com/studio}{https://developer.android.com/studio}), installing, and going through 
the initial wizard setup (nothing special by now, choose everything by default). One additional step is to create a 
virtual device (\cref{img:fs-android})... a couple of clicks and we're done (\cref{img:fs-create}).

\img{first-steps/android-studio}{Android Studio start page}{img:fs-android}

\img{first-steps/android-studio-create}{Android Studio choose and create Emuator}{img:fs-create}

Once we've set up the Android SDK and the emulator is triggered to "Start", Flutter will recognize it automatically from
a known to us execution `flutter run`. It will build the Flutter app and launch it on the connected Android 
emulator (or even real device, connected via USB; check manuals if needed).
\\
\\
Let's dive into the coding!