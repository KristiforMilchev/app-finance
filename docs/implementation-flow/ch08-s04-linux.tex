% Copyright 2023 The terCAD team. All rights reserved.
% Use of this content is governed by a CC BY-NC-ND 4.0 license that can be found in the LICENSE file.

\subsubsection{Linux Snap Store}

While the world of Linux distributions can appear complex due to the sheer variety available, there is a unified 
solution to simplify the process: snaps. A snap is essentially a bundled package containing one or more applications 
along with all their dependencies. What's remarkable about snaps is their ability to run consistently across a wide 
array of Linux distributions, without requiring any modifications. These snaps are conveniently discoverable and 
installable from the Snap Store (\href{https://snapcraft.io}{https://snapcraft.io}).

To get started, we should visit the page \href{https://snapcraft.io/snaps}{https://snapcraft.io/snaps} to register 
our application by simply clicking the "Register a snap name" button, with options to choose between "private" or 
"public" application availability. A notable feature is the integration of Snap Store with GitHub account at 
\href{https://snapcraft.io/fingrom/builds}{https://snapcraft.io/fingrom/builds} to simplify the build and deploy 
procedures (both can be done directly from that page). If it's needed a control over the application's distribution, 
whether through Github Actions or manual processes, we may refer to the instructions on configuring and building 
Flutter projects provided at 
\href{https://snapcraft.io/docs/flutter-applications}{https://snapcraft.io/docs/flutter-applications}.
in that case, we will utilize the \q{snapcraft}-application (it can be installed and used on various systems, including 
different Linux distributions, macOS, and Windows):

\begin{lstlisting}[language=terminal]
# For Linux with snap-support
$ sudo snap install snapcraft --classic 

## Install Virtual Machine Manager
# For Github Actions "uses: canonical/setup-lxd"
$ sudo snap install lxd # required by snapcraft
$ sudo adduser $USER lxd # grant permissions
$ newgrp lxd # apply changes
$ sudo lxd waitready # revise state
$ sudo lxd init --auto # set up the LXD server

# Login to Snap Store
$ snapcraft login
Enter your Ubuntu One e-mail address and password.
Email: ...
Password: ...
Login successful 

# Retrieve Developer ID
$ snapcraft whoami
id: ...

## Export credentials for CI/CD
$ snapcraft export-login --snaps=fingrom --acls package_access,package_push,package_update,package_release credentials.txt

# Generate snapcraft.yaml
$ snapcraft init

## Build Package
$ snapcraft

# Test generated package
$ sudo snap install my-snap-name_0.1_amd64.snap --devmode

## Release Package
$ snapcraft release
\end{lstlisting}

\noindent The configuration file for the Snapcraft tool, which provides instructions on how to build the \q{snap}-package:

\begin{lstlisting}[language=yaml]
# ./snapcraft.yaml
name: fingrom
base: core22 # span, based on Ubuntu 22.04 
version: 1.0.0+1 # to be replaced by CI/CD
summary: Platform-agnostic financial accounting app
description: |
  ... description details ...

# Type of the build
grade: stable # a stable version
confinement: strict # enforces strict isolation

# To recognize from a command line
apps:
  fingrom:
    command: fingrom
    # Grant access to the application
    # From: https://snapcraft.io/docs/supported-interfaces
    plugs: [home, network, network-bind, removable-media]
    # Use basic libraries (alternative to 'stage-packages') 
    extensions: [gnome]

parts:
  app-finance:
    source: .
    # To trigger: 'flutter linux' build
    plugin: flutter
    flutter-target: lib/main.dart
    # Dependencies to external packages
    build-packages: [libgtk-3-dev, ninja-build]
    stage-packages: [libgtk-3-0]
\end{lstlisting}

\noindent \q{Parts}-section defines the parts that make up our \q{snap}-package. In our case, there's only one part 
called app-finance. It specifies the source code location (in this directory), the plugin to use (in this case, the 
Flutter plugin), the target for building your Flutter app (usually the main Dart file), and any additional build 
packages or dependencies required.

Finally, the pipeline for the \q{snap}-package build and distribution would be the next \issue{209}{} (\cref{img:d-snap}):

\begin{lstlisting}[language=yaml]
# ./snapcraft.yaml
- name: Install LXD
  if: matrix.target == 'Linux'
  uses: canonical/setup-lxd@v0.1.1

- name: Build Snap
  if: matrix.target == 'Linux'
  run: |
    sudo snap install snapcraft --classic
    snapcraft --verbose

- name: Publish Snap
  if: matrix.target == 'Linux'
  env:
    SNAPCRAFT_STORE_CREDENTIALS: ${{ secrets.CREDENTIALS }}
    # Trick with root-access to propagate credentials
    # Otherwise, error: Cannot login with 'SNA...' set.
    run: sudo --preserve-env=SNAPCRAFT_STORE_CREDENTIALS snapcraft upload *.snap --release=latest/stable
\end{lstlisting}

\img{distributing/linux-discover}{KDE Neon - Discover: Availability check}{img:d-snap}
