% Copyright 2023 The terCAD team. All rights reserved.
% Use of this content is governed by a CC BY-NC-ND 4.0 license that can be found in the LICENSE file.

\subsection{Supporting Accessibility}
\markboth{Optimizing}{Supporting Accessibility}

Before delving into Flutter's accessibility capabilities, it's valuable to comprehend the significance of accessibility. 
Accessibility involves the design and development of applications that are usable by all individuals, including those 
with disabilities such as visual, auditory, motor, or cognitive impairments. Providing equal access to information 
and services for all users is a matter of ethical and social responsibility. By ensuring that our app is accessible, 
we broaden its user base to include a more diverse audience. Lastly, accessibility often constitutes as a legal 
obligation in many regions.

The \q{Semantics} and \q{Tooltip} widgets serve as the initial building blocks for enhancing accessibility. They play a 
critical role in assisting screen readers by providing context to users with visual impairments. It can be implemented 
in a way of wrapping button by duplicating a written title to a label of it (some elements, as \q{FloatingActionButton} 
contains the label in a scope of properties):

\begin{lstlisting}
// ./lib/widgets/wrapper/elevate_button_widget.dart
class ElevatedButtonWidget extends StatelessWidget {
  Widget build(BuildContext context) {
    final colorScheme = context.colorScheme;
    return Semantics(
      label: text,
      child: SizedBox(
        width: double.infinity,
        child: ElevatedButton(
          style: ButtonStyle(
            shape: MaterialStateProperty.resolveWith((states) => 
                const ContinuousRectangleBorder()),
            // Respond to user actions, such as mouse hover
            backgroundColor: 
              MaterialStateProperty.resolveWith<Color>((states) {
                if (states.contains(MaterialState.hovered)) {
                  return hoveredColor ?? 
                      colorScheme.onSecondaryContainer;
                }
                return backgroundColor ?? colorScheme.secondary;
              },
            )),
          onPressed: onPressed,
          child: Text(
            text,
            style: TextStyle(
            color: textColor ?? colorScheme.inversePrimary,
            shadows: const [],
// ... closing brackets
\end{lstlisting}

\noindent Regardless of its role as a canvas for creating user interfaces, Flutter excels in ensuring the accessibility 
of widgets, allowing them to seamlessly interact with screen readers like TalkBack on Android and VoiceOver on iOS. In 
this context, the semantic tree in Flutter closely aligns with the widget tree. Flutter integrates with the 
accessibility APIs of the underlying platforms, which means it can effectively communicate with screen readers, 
ensuring a cohesive and inclusive experience for all users. This emphasis on accessibility isn't limited solely to text 
recognition, by extending experience to effectively manage the focus for users who navigate through the app using 
keyboards or voice commands.

Basic navigation can be further enhanced with keyboard shortcuts provided through the use of the \q{Listener}-widget. 
This feature is particularly valuable for users who rely on keyboard navigation (either for accessibility, or from 
desktop usage perspective). This not only adheres to accessibility guidelines but also makes our application more 
inclusive and user-friendly:

\begin{table}[h!]
  \begin{tabular}{ |p{9cm}||l|  }
    \hline
    Description & Shortcut\\
    \hline
    Open / Close the Navigation Drawer &  \key{Shift} + \key{Enter} \\
    Navigate Up                        &  \key{up} \\
    Navigate Down                      &  \key{down} \\
    Open Selected                      &  \key{Enter} \\
    Zoom In                            &  \key{Ctrl} + \key{+} \\
    Zoom In (with mouse)               &  \key{Ctrl} + \key{scroll down} \\
    Zoom Out                           &  \key{Ctrl} + \key{-} \\
    Zoom Out (with mouse)              &  \key{Ctrl} + \key{scroll up} \\
    Reset Zoom                         &  \key{Ctrl} + \key{0} \\
    Add new Transaction                &  \key{Ctrl} + \key{N} \\
    \hline
  \end{tabular}
  \caption{Shortcuts in the application} \label{tb:shortcuts}
\end{table}

\begin{lstlisting}
// ./lib/widgets/wrapper/input_controller_wrapper.dart
class InputControllerWrapper extends StatefulWidget { 
  /* ... */
}
class InputControllerWrapperState extends State<InputControllerWrapper> {
  @override
  Widget build(BuildContext context) {
    // Obtain the current zoom value from 'AppZoom'-provider
    zoom = Provider.of<AppZoom>(context, listen: false);
    // Wrap the widget with listeners
    return Listener(
      onPointerSignal: _onPointerSignal, // Listen for mouse input
      child: RawKeyboardListener(
        focusNode: focus,
        onKey: _onKeyPressed, // Listen for keyboard input
        child: widget.child,
      ),
    );
// ... other stuff
\end{lstlisting}
