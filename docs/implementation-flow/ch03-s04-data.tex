% Copyright 2023 The terCAD team. All rights reserved.
% Use of this content is governed by a CC BY-NC-ND 4.0 license that can be found in the LICENSE file.

\subsection{Defining Storage}
\markboth{Prototyping}{Defining Storage}


\subsubsection{Using Shared Preferences}

To use shared preferences (store / retrieve string values) in Flutter, we should install the proper package via
\q{flutter pub add shared\_preferences} command. To operate with preferences it would be better to use mixin --
a powerful way in Flutter to reuse code across multiple classes by combining the members of the mixin with the 
classes that use it (a composition without inheritance).

A mixin is defined by using the \q{mixin}-keyword followed by its name and a set of members, such as properties, methods, 
and getters/setters. It cannot be instantiated on its own; instead, it is meant to be mixed into other classes 
using the \q{with}-keyword (it inherits all the members from the mixin). This allows the class to access and use the 
properties and methods defined in the mixin as if they were part of its own implementation.

\begin{lstlisting}
// ./lib/_mixins/shared_preferences_mixin.dart
mixin SharedPreferencesMixin {
  Future<void> setPreference(String key, String value) async {
    SharedPreferences pref = await SharedPreferences.getInstance();
    await pref.setString(key, value);
  }

  Future<String?> getPreference(String key) async {
    SharedPreferences pref = await SharedPreferences.getInstance();
    return pref.getString(key);
  }
}
\end{lstlisting}

\noindent And now we can apply it to our class(es):

\begin{lstlisting}
// ./lib/widgets/bill/expenses_tab.dart
class ExpensesTabState<T extends ExpensesTab> extends State<T>
    with SharedPreferencesMixin {(*@ \stopnumber @*)

  // ... properties and constructor
  (*@ \startnumber{10} @*)
  @override
  void initState() { // We cannot make \q{initState} async
    account = widget.account;
    super.initState();
    // Waiting for the value, then apply via \q{setState}
    getPreference('account')
        .then((value) => setState(() => account ??= value));
  }

  void updateStorage() {
    // On \q{save}-action store preferences
    setPreference(prefAccount, account ?? '');
// ... other code
\end{lstlisting}


\subsubsection{Streaming Big Files}

Since our transaction log can be counted in millions of millions, the better way to restore data structures from a file
is to use streaming approach.

\begin{lstlisting}
// ./lib/_classes/data/transaction_log.dart
class TransactionLog {
  // Get or Create file
  static Future<File> get _logFle async {
    final path = await getApplicationDocumentsDirectory();
    var file = File('${path.absolute.path}/app-finance.log');
    if (!(await file.exists())) {
      await file.create(recursive: true);
    }
    return file;
  }

  static void save(dynamic content) async {
    // Under the hood \q{content.toString()} evaluates
    (await _logFle).writeAsString("$content\n", mode: FileMode.append);
  }

  static Future<bool> load(AppData store) async {
    Stream<String> lines = (await _logFle)
        .openRead()
        .transform(utf8.decoder)
        .transform(const LineSplitter());
    try {
      await for (var line in lines) {
        var obj = json.decode(line);
        // ... restore object [explained further]
      }
      return true;
    } catch (e) {
      return false;
    }
  }
}
\end{lstlisting}

And adjust our constructor for AppData as follows:

\begin{lstlisting}
class AppData extends ChangeNotifier {
  bool isLoading = false;

  AppData() : super() {
    isLoading = true;
    TransactionLog.load(this)
        .then((success) => notifyListeners())
        .then((success) => isLoading = false);
  }
// ... other stuff
\end{lstlisting}


\subsubsection{Streaming Local Storage}

File usage is limited by systems where our application can be used as installed instance. To support web browsers
we may rely on Local Storage. Instead of using \q{dart:html}, let's proceed with known \q{shared\_preferences} (\cref{img:mn-preferences}).

\begin{lstlisting}
import 'package:flutter/foundation.dart' show kIsWeb;

static void save(dynamic content) async {
  if (kIsWeb) {
    await TransactionLog().setPreference('log$increment', content.toString());
    increment++;
  } else {
    (await _logFle).writeAsString("$line\n", mode: FileMode.append);
  }
}

static Stream<String> _loadWeb() async* {
  // Some transactions might be lost or deleted
  int attempts = 0;
  do {
    int i = increment + attempts;
    var line = await TransactionLog().getPreference('log$i');
    if (line == null) {
      attempts++;
    } else {
      increment += attempts + 1;
      attempts = 0;
    }
    yield line ?? '';
    // Adding retrial approach
  } while (attempts < 10);
}

static Future<bool> load(AppData store) async {
  Stream<String> lines;
  if (kIsWeb) {
    lines = _loadWeb();
  } else {
    lines = (await _logFle)
        .openRead()
        .transform(utf8.decoder)
        .transform(const LineSplitter());
  }
  // ... other stuff
}
\end{lstlisting}

\img{prototyping/local-storage-web}{Usage of Local Storage by Application}{img:mn-preferences}


\subsubsection{Using Dynamic Structures}

To update object properties dynamically in Flutter/Dart, we can use the object's setter methods.

\begin{lstlisting}
class AppData {
  int? prop1;
  int? prop2;
}

void main() {
  AppData data = AppData();
  Map<String, dynamic> properties = {'prop1': 1, 'prop2': 2};

  properties.forEach((key, value) {
    if (data.hasOwnProperty(key)) {
      data.setProp(key, value);
    }
  });
}

extension ObjectExtension on Object {
  bool hasOwnProperty(String propertyName) {
    return this.runtimeType
        .declarations
        .any((declaration) => declaration.name == propertyName);
  }

  void setProp(String propertyName, dynamic value) {
    (this as dynamic)[propertyName] = value;
  }
}
\end{lstlisting}

To convert an object to a string representation and then back to an object, it can be used the toString() 
and fromJson() methods. 

\begin{lstlisting}
class MyClass {
  String name;
  DateTime timestamp;
  // Constructor
  MyClass(this.timestamp, this.name);
  // Props to simple built-in types (String, double, int, bool)
  Map<String, dynamic> toJson() => {
    'name': name,
    'timestamp': timestamp.toIso8601String(),
  };
  // By using 'dart:convert' transform \q{Map}-object
  String toString() {
    return json.encode(toJson());
  }
  // NOTE: in factory it cannot be used class' methods
  factory MyClass.fromJson(String jsonString) {
    Map<String, dynamic> json = jsonDecode(jsonString);
    return MyClass(DateTime.parse(json['timestamp']), json['name']);
  }
}

void main() {
  MyClass myObject = MyClass(DateTime.now(), 'Sample');
  // Convert object to a string representation
  String jsonString = myObject.toString();
  print(jsonString); // {"timestamp": 123..., "name": "John"}
  // Convert string back to an object
  MyClass newObj = MyClass.fromJson(jsonString);
  print(newObj.timestamp.toIso8601String()); // '2023-10-01T...'
  print(newObj.name); // 'Sample'
}
\end{lstlisting}

In the above example, the MyClass object has a custom toString() method that converts the object's properties to a 
JSON-like string representation. The fromJson() factory method takes a JSON string, parses it using jsonDecode(), 
and creates a new MyClass object with the extracted values.

By calling toString() on the object, you can convert it to a string representation. Then, by using fromJson() with the 
string, you can recreate the object from the string representation. This approach allows you to serialize and 
deserialize objects for storage or transmission purposes.

So, finally our method to restore objects from their representation in log-file would be the next:

\begin{lstlisting}[firstnumber=54]
// ./lib/_classes/data/transaction_log.dart
static void init(AppData store, String type, Map<String, dynamic> data) {
  final typeToClass = {
    'GoalAppData': (data) => GoalAppData.fromJson(data),
    'AccountAppData': (data) => AccountAppData.fromJson(data),
    'BillAppData': (data) => BillAppData.fromJson(data),
    'BudgetAppData': (data) => BudgetAppData.fromJson(data),
    'CurrencyAppData': (data) => CurrencyAppData.fromJson(data),
  };
  final obj = typeToClass[type];
  if (obj != null) {
    final el = obj(data);
    store.update(el.getType(), el.uuid ?? '', el, true);
  }
}
\end{lstlisting}


\subsubsection{Adding Multiple Providers}

In case of being aligned with SOLID principles, it would be better to preserve our classes small with a single 
responsibility. So, for the case of handling theme mode (system, light, dark) it can be created a separate notifier:

\begin{lstlisting}
class AppTheme extends ValueNotifier<ThemeMode> with SharedPreferencesMixin {
  AppTheme(ThemeMode value) : super(value) {
    // Get from Shared Preferences current state of value
    getPreference(prefTheme).then((val) {
      if (val != null) {
        _set(val);
      }
    });
  }

  _set(String val) {
    int? idx = int.tryParse(val);
    if (idx != null) {
      // Trigger notification by value change
      value = ThemeMode.values[idx];
      notifyListeners();
    }
  }

  updateState(String value) {
    setPreference(prefTheme, value).then((_) {
      _set(value);
    });
  }
}
\end{lstlisting}

\noindent The simples way to observe value changes can be done via \q{context.watch}:

\begin{lstlisting}
void main() {
  runApp(
    // Clarify that we have more than one provider
    MultiProvider(
      providers: [
        ChangeNotifierProvider<AppData>(
          create: (_) => AppData(),
        ),
        ChangeNotifierProvider<AppTheme>(
          create: (_) => AppTheme(ThemeMode.system),
        ),
      ],
      child: MyApp(platform: platform),
    ),
  );(*@ \stopnumber @*)
}

class MyAppState extends State<MyApp> {
  Widget build(BuildContext context) {
    return MaterialApp(
      themeMode: context.watch<AppTheme>().value,
// ... other code
\end{lstlisting}

\noindent And to update the state:

\begin{lstlisting}
Provider.of<AppTheme>(context, listen: false).updateState(value);
\end{lstlisting}

\noindent The only this that the usage of asynchronous \q{getPreference} leads to a not needed complexity, and 
additional computational efforts:

\begin{lstlisting}[language=terminal]
User I/Choreographer( 4431): Skipped 244 frames!  The application may be doing too much work on its main thread.
\end{lstlisting}

\noindent While we've understood that, we can easily fix the problem by its single initialization during
application start:

\begin{lstlisting}
// ./lib/main.dart
void main() async {
  SharedPreferencesMixin.pref = await SharedPreferences.getInstance();(*@ \stopnumber @*)

// ./lib/_mixins/shared_preferences_mixin.dart
mixin SharedPreferencesMixin {
  static late SharedPreferences pref;
\end{lstlisting}

But we should be very accurate with what we're placing into a static variable, since, as an example, 
caching \q{AdaptiveWindowType} for our \q{ThemeHelper} leads to an infinite loop of re-rendering widgets:
\href{https://github.com/lyskouski/app-finance/issues/112}{https://github.com/lyskouski/app-finance/issues/112}.

Unwanted loop of rendering might also happen on the usage of \q{key: UniqueKey()} for \q{StatelessWidget}, it can cause 
the widget to be rebuilt whenever the widget tree is rebuilt since a new UniqueKey will be generated for the widget.
