% Copyright 2023 The terCAD team. All rights reserved.
% Use of this content is governed by a CC BY-NC-ND 4.0 license that can be found in the LICENSE file.

\subsection{[TBD] Adding Marginal Notes of Flutter usage}
\markboth{Implementing Core Functionality}{Adding Marginal Notes of Flutter usage}

[TBD]

\subsubsection{Mixing `PageView`' and `PageController`}

Now we're going to add three pages controlled by swiping left and right for adding new bills, we'll use a combination 
of PageView and PageController. 

\begin{lstlisting}
class Page1 extends StatelessWidget { /* ... */ }
class Page2 extends StatelessWidget { /* ... */ }
class Page3 extends StatelessWidget { /* ... */ }

class MyApp extends StatelessWidget {
  final PageController _pageController = PageController(initialPage: 0);

  @override
  Widget build(BuildContext context) {
    return MaterialApp(
      home: Scaffold(
        body: GestureDetector(
          onHorizontalDragEnd: (DragEndDetails details) {
            if (details.primaryVelocity! > 0) { // Swiped right
              _pageController.previousPage(
                duration: Duration(milliseconds: 500),
                curve: Curves.ease,
              );
            } else if (details.primaryVelocity! < 0) { // Swiped left
              _pageController.nextPage(
                duration: Duration(milliseconds: 500),
                curve: Curves.ease,
              );
            }
          },
          child: PageView(
            controller: _pageController,
            children: [Page1(), Page2(), Page3()],
          ),
        ),
      ),
    );
  }
}
\end{lstlisting}

If we do want to click on tabs instead of swiping, then it can be changed to:

\begin{lstlisting}
class MyApp extends StatelessWidget {
  final PageController _pageController = PageController(initialPage: 0);

  @override
  Widget build(BuildContext context) {
    return MaterialApp(
      home: DefaultTabController(
        length: 3,
        child: Scaffold(
          appBar: AppBar(
            title: Text('Swiping Pages'),
            bottom: TabBar(
              tabs: [
                Tab(text: 'Page 1'),
                Tab(text: 'Page 2'),
                Tab(text: 'Page 3'),
              ],
            ),
          ),
          body: TabBarView(
            children: [
              Page1(),
              Page2(),
              Page3(),
            ],
          ),
        ),
      ),
    );
  }
}
\end{lstlisting}

By loving a reach User Interface flow, let's combine all together:

\begin{lstlisting}
class MyApp extends StatelessWidget {
  final int tabCount = 3;
  int tabIndex = 1;
  PageController? pageController;
  TabController? tabController;

  @override
  void initState() {
    super.initState();
    pageController = PageController(initialPage: tabIndex);
    tabController = TabController(
      length: tabCount,
      vsync: const _VSync(),
      initialIndex: tabIndex,
    );
  }

  @override
  void dispose() {
    pageController?.dispose();
    tabController?.dispose();
    super.dispose();
  }

  void switchTab(int newIndex) {
    setState(() {
      tabIndex = newIndex;
      tabController?.animateTo(newIndex);
      pageController?.animateToPage(
        newIndex,
        duration: Duration(milliseconds: 300),
        curve: Curves.ease,
      );
    });
  }

  @override
  Widget build(BuildContext context) {
    return MaterialApp(
      home: GestureDetector(
        onHorizontalDragEnd: (DragEndDetails details) {
          if (details.primaryVelocity! > 0) {
            switchTab(widget.tabIndex - 1);
          } else if (details.primaryVelocity! < 0) {
            switchTab(widget.tabIndex + 1);
          }
        },
        child: Scaffold(
          appBar: TabBar(
              controller: tabController,
              onTap: switchTab,
              tabs: [
                Tab(text: 'Page 1'),
                Tab(text: 'Page 2'),
                Tab(text: 'Page 3'),
              ],
            ),
          body: PageView(
            controller: pageController,
            onPageChanged: switchTab,
            children: [
              Page1(),
              Page2(),
              Page3(),
            ],
          ),
        ),
      ),
    );
  }
}

class _VSync implements TickerProvider {
  const _VSync();

  @override
  Ticker createTicker(TickerCallback onTick) {
    return Ticker(onTick);
  }
}

class Page1 extends StatelessWidget { /* ... */ }

class Page2 extends StatelessWidget { /* ... */ }

class Page3 extends StatelessWidget { /* ... */ }
\end{lstlisting}

By combining both solutions, some irritating behavior happens, - we cannot tab from `Page1' to `Page3' 
(more than one position) since `pageController` after the first movement will send an update via `onPageChanged`. 
Hopefully, we do have a solution for that by introducing delayed trigger to proceed with switching till the chosen tab:

\begin{lstlisting}
Future<void> delaySwitchTab(int delay, int newIndex) async {
  await Future.delayed(Duration(milliseconds: delay));
  switchTab(newIndex);
}

void switchTab(int newIndex) {
  if (newIndex < 0 || newIndex >= widget.tabCount) {
    return;
  }
  setState(() {
    const delay = 300;
    // Saving current state for the check after 
    final currIndex = tabIndex;
    tabIndex = newIndex;
    tabController?.animateTo(newIndex);
    pageController?.animateToPage(
      newIndex,
      duration: const Duration(milliseconds: delay),
      curve: Curves.ease,
    );
    // Verify that the difference is more than one
    if ((currIndex - newIndex).abs() > 1) {
      delaySwitchTab(delay, newIndex);
    }
  });
}
\end{lstlisting}


\subsubsection{Adding `AppBar` as dots to `TabBar`}

In some cases we do not need a fully defined navigation bar (with titles and icons), just dots to identify that
current block can be swiped or changed by clicking on any of dots. To implement them it can be used either existing 
external components (and added to the page as Widget), or by defining `indicator`-property of `TabBar`-widget with
custom implementation of `Decoration`-class. Let's paint our dots:

\begin{lstlisting}
// ./lib/widgets/_wrappers/dots_indicator_decoration.dart
import 'package:flutter/material.dart';

class DotsIndicatorDecoration extends Decoration {(*@ \stopnumber @*)

  // ... properties and constructor
  (*@ \startnumber{10} @*)
  @override
  BoxPainter createBoxPainter([VoidCallback? onChanged]) {
    return _CustomTabIndicatorPainter(
      // ... properties
      onChanged: onChanged,
    );
  }
}

class _CustomTabIndicatorPainter extends BoxPainter {(*@ \stopnumber @*)

  // ... properties and constructor
  (*@ \startnumber{33} @*)
  @override
  void paint(Canvas canvas, Offset offset, ImageConfiguration configuration) {
    if (itemCount <= 1) { // Skip for a single tab
      return;
    }
    // Take from PageController an actual page, otherwise - initial
    final activeIndex = controller.page?.round() ?? controller.initialPage;
    // dotSize, spacing, color - initialized properties
    final active = Paint()..color = color;
    final inactive = Paint()..color = color.withOpacity(0.3);
    for (int i = 0; i < itemCount; i++) {
      double xPos = spacing + i * (dotSize + spacing);
      // Pu cycle a little below the baseline
      double yPos = spacing * 0.6; 
      if (i == activeIndex) {
        canvas.drawCircle(Offset(xPos, yPos), dotSize / 2, active);
      } else {
        canvas.drawCircle(Offset(xPos, yPos), dotSize / 2, inactive);
      }
    }
  }
}
\end{lstlisting}

\noindent Additionally, we might hide the usage of our custom indicator by extending from `TabBar`:

\begin{lstlisting}
// ./lib/widgets/_wrappers/dots_tab_bar_widget.dart
import 'package:app_finance/widgets/_wrappers/dots_indicator_decoration.dart';
import 'package:flutter/material.dart';

class DotsTabBarWidget extends TabBar {
  final TabController tabController;
  final PageController pageController;
  final List<Widget> tabList;
  final double indent; // Indentation between cycles
  final double width; // MediaQuery.of(context).size.width
  final Color color; // Color for active cycle

  const DotsTabBarWidget({
    super.key,
    required this.tabController,
    required this.pageController,
    required this.tabList,
    required this.indent,
    required this.width,
    required this.color,
    onTap,
  }) : super(
          controller: tabController,
          mouseCursor: SystemMouseCursors.click,
          onTap: onTap,
          // hook, since `tabs` is required
          tabs: tabList, // getter overrides the flow
        );
  // Convert children of `PageView` to a clickable area 
  @override
  get tabs =>
      tabList.map((tab) => SizedBox(width: indent, height: indent)).toList();
  // Make our dots centered on the page
  @override
  get padding => EdgeInsets.symmetric(
      horizontal: (width - tabList.length * 2 * indent) / 2);
  // Apply custom decorator to draw cycles
  @override
  get indicator => DotsIndicatorDecoration(
        controller: pageController,
        itemCount: tabList.length,
        color: color,
        dotSize: indent,
      );
}
\end{lstlisting}

\subsubsection{Using Shared Preferences}

To use shared preferences (store / retrieve string values) in Flutter, we should install the proper package via
`flutter pub add shared\_preferences` command. To operate with preferences it would be better to use mixin --
a powerful way in Flutter to reuse code across multiple classes by combining the members of the mixin with the 
classes that use it (a composition without inheritance).

A mixin is defined by using the `mixin`-keyword followed by its name and a set of members, such as properties, methods, 
and getters/setters. It cannot be instantiated on its own; instead, it is meant to be mixed into other classes 
using the `with`-keyword (it inherits all the members from the mixin). This allows the class to access and use the 
properties and methods defined in the mixin as if they were part of its own implementation.

\begin{lstlisting}
// ./lib/_mixins/shared_preferences_mixin.dart
mixin SharedPreferencesMixin {
  Future<void> setPreference(String key, String value) async {
    SharedPreferences pref = await SharedPreferences.getInstance();
    await pref.setString(key, value);
  }

  Future<String?> getPreference(String key) async {
    SharedPreferences pref = await SharedPreferences.getInstance();
    return pref.getString(key);
  }
}
\end{lstlisting}

\noindent And now we can apply it to our class(es):

\begin{lstlisting}
// ./lib/widgets/bill/expenses_tab.dart
class ExpensesTabState<T extends ExpensesTab> extends State<T>
    with SharedPreferencesMixin {(*@ \stopnumber @*)

  // ... properties and constructor
  (*@ \startnumber{10} @*)
  @override
  void initState() { // We cannot make `initState` async
    account = widget.account;
    super.initState();
    // Waiting for the value, then apply via `setState`
    getPreference('account')
        .then((value) => setState(() => account ??= value));
  }

  void updateStorage() {
    // On `save`-action store preferences
    setPreference(prefAccount, account ?? '');
// ... other code
\end{lstlisting}


\subsubsection{Streaming Big Files}

Since our transaction log can be counted in millions of millions, the better way to restore data structures from a file
is to use streaming approach.

\begin{lstlisting}
// ./lib/_classes/data/transaction_log.dart
class TransactionLog {
  // Get or Create file
  static Future<File> get _logFle async {
    final path = await getApplicationDocumentsDirectory();
    var file = File('${path.absolute.path}/app-finance.log');
    if (!(await file.exists())) {
      await file.create(recursive: true);
    }
    return file;
  }

  static void save(dynamic content) async {
    // Under the hood `content.toString()` evaluates
    (await _logFle).writeAsString("$content\n", mode: FileMode.append);
  }

  static Future<bool> load(AppData store) async {
    Stream<String> lines = (await _logFle)
        .openRead()
        .transform(utf8.decoder)
        .transform(const LineSplitter());
    try {
      await for (var line in lines) {
        var obj = json.decode(line);
        // ... restore object [explained further]
      }
      return true;
    } catch (e) {
      return false;
    }
  }
}
\end{lstlisting}

And adjust our constructor for AppData as follows:

\begin{lstlisting}
class AppData extends ChangeNotifier {
  bool isLoading = false;

  AppData() : super() {
    isLoading = true;
    TransactionLog.load(this)
        .then((success) => notifyListeners())
        .then((success) => isLoading = false);
  }
// ... other stuff
\end{lstlisting}


\subsubsection{Streaming Local Storage}

File usage is limited by systems where our application can be used as installed instance. To support web browsers
we may rely on Local Storage. Instead of using `dart:html`, let's proceed with known `shared\_preferences` (\cref{img:mn-preferences}).

\begin{lstlisting}
import 'package:flutter/foundation.dart' show kIsWeb;

static void save(dynamic content) async {
  if (kIsWeb) {
    await TransactionLog().setPreference('log$increment', content.toString());
    increment++;
  } else {
    (await _logFle).writeAsString("$line\n", mode: FileMode.append);
  }
}

static Stream<String> _loadWeb() async* {
  // Some transactions might be lost or deleted
  int attempts = 0;
  do {
    int i = increment + attempts;
    var line = await TransactionLog().getPreference('log$i');
    if (line == null) {
      attempts++;
    } else {
      increment += attempts + 1;
      attempts = 0;
    }
    yield line ?? '';
    // Adding retrial approach
  } while (attempts < 10);
}

static Future<bool> load(AppData store) async {
  Stream<String> lines;
  if (kIsWeb) {
    lines = _loadWeb();
  } else {
    lines = (await _logFle)
        .openRead()
        .transform(utf8.decoder)
        .transform(const LineSplitter());
  }
  // ... other stuff
}
\end{lstlisting}

\img{prototyping/local-storage-web}{Usage of Local Storage by Application}{img:img:mn-preferences}


\subsubsection{Using Dynamic Structures}

To update object properties dynamically in Flutter/Dart, we can use the object's setter methods.

\begin{lstlisting}
class AppData {
  int? prop1;
  int? prop2;
}

void main() {
  AppData data = AppData();
  Map<String, dynamic> properties = {'prop1': 1, 'prop2': 2};

  properties.forEach((key, value) {
    if (data.hasOwnProperty(key)) {
      data.setProp(key, value);
    }
  });
}

extension ObjectExtension on Object {
  bool hasOwnProperty(String propertyName) {
    return this.runtimeType
        .declarations
        .any((declaration) => declaration.name == propertyName);
  }

  void setProp(String propertyName, dynamic value) {
    (this as dynamic)[propertyName] = value;
  }
}
\end{lstlisting}

To convert an object to a string representation and then back to an object, it can be used the toString() 
and fromJson() methods. 

\begin{lstlisting}
class MyClass {
  String name;
  DateTime timestamp;
  // Constructor
  MyClass(this.timestamp, this.name);
  // Props to simple built-in types (String, double, int, bool)
  Map<String, dynamic> toJson() => {
    'name': name,
    'timestamp': timestamp.toIso8601String(),
  };
  // By using 'dart:convert' transform `Map`-object
  String toString() {
    return json.encode(toJson());
  }
  // NOTE: in factory it cannot be used class' methods
  factory MyClass.fromJson(String jsonString) {
    Map<String, dynamic> json = jsonDecode(jsonString);
    return MyClass(DateTime.parse(json['timestamp']), json['name']);
  }
}

void main() {
  MyClass myObject = MyClass(DateTime.now(), 'Sample');
  // Convert object to a string representation
  String jsonString = myObject.toString();
  print(jsonString); // Output: '{"timestamp": 123..., "name": "John"}'
  // Convert string back to an object
  MyClass newObj = MyClass.fromJson(jsonString);
  print(newObj.timestamp.toIso8601String()); // Output: 2023-...
  print(newObj.name); // Output: Sample
}
\end{lstlisting}

In the above example, the MyClass object has a custom toString() method that converts the object's properties to a 
JSON-like string representation. The fromJson() factory method takes a JSON string, parses it using jsonDecode(), 
and creates a new MyClass object with the extracted values.

By calling toString() on the object, you can convert it to a string representation. Then, by using fromJson() with the 
string, you can recreate the object from the string representation. This approach allows you to serialize and 
deserialize objects for storage or transmission purposes.

So, finally our method to restore objects from their representation in log-file would be the next:

\begin{lstlisting}[firstnumber=54]
// ./lib/_classes/data/transaction_log.dart
static void init(AppData store, String type, Map<String, dynamic> data) {
  final typeToClass = {
    'GoalAppData': (data) => GoalAppData.fromJson(data),
    'AccountAppData': (data) => AccountAppData.fromJson(data),
    'BillAppData': (data) => BillAppData.fromJson(data),
    'BudgetAppData': (data) => BudgetAppData.fromJson(data),
    'CurrencyAppData': (data) => CurrencyAppData.fromJson(data),
  };
  final obj = typeToClass[type];
  if (obj != null) {
    final el = obj(data);
    store.update(el.getType(), el.uuid ?? '', el, true);
  }
}
\end{lstlisting}



\subsubsection{Using Encryption}

To "disable" (at least make harder) capability to change log files directly, may be used control sum hash key:

\begin{lstlisting}
// ./lib/_classes/data/transaction_log.dart
import 'package:crypto/crypto.dart';
class TransactionLog {
  static String getHash(Map<String, dynamic> data) {
    return md5.convert(utf8.encode(data.toString())).toString();
  }

  static Future<bool> load() async {
    // ... other stuff
    await for (var line in lines) {
      var obj = json.decode(line);
      if (getHash(obj['data']) != obj['type']['hash']) {
        continue; // Corrupted data... skip
      }
\end{lstlisting}

Additionally, to protect user data we should apply mechanics of lines encryption:

\begin{lstlisting}
// ./lib/_classes/data/transaction_log.dart
import 'package:encrypt/encrypt.dart';
class TransactionLog {
  static Encrypter get salt =>
      Encrypter(AES(Key.fromUtf8('32symbols-salt-for-encryption...')));

  static IV get code => IV.fromLength(8);(*@ \stopnumber @*)

  (*@ \startnumber{49} @*)
  // to encrypt content
    content = salt.encrypt(content, iv: code).base64;
  // to decrypt content
    content = salt.decrypt64(content, iv: code);
\end{lstlisting}


\subsubsection{Android: loading screen and icons}

For Android to change background of loading screen:

\begin{lstlisting}[language=xml]
<!-- android/app/src/main/AndroidManifest.xml -->
<manifest xmlns:android="http://schemas.android.com/apk/res/android">
  <application
      android:background="@color/mainColor"

<!-- android/app/src/main/res/drawable/launch_background.xml -->
<!-- android/app/src/main/res/drawable-v21/launch_background.xml -->
<layer-list xmlns:android="http://schemas.android.com/apk/res/android">
    <item android:drawable="@color/mainColor" />

<!-- android/app/src/main/res/values/colors.xml -->
<!-- android/app/src/main/res/values-night/colors.xml -->
<resources>
    <color name="mainColor">#912391</color>
</resources>
\end{lstlisting}

To set icon with background:

\begin{lstlisting}[language=xml]
<!-- android/app/src/main/AndroidManifest.xml -->
<manifest xmlns:android="http://schemas.android.com/apk/res/android">
  <application
    android:icon="@mipmap/ic_launcher"
    android:roundIcon="@mipmap/ic_launcher_round"

<!-- android/app/src/main/res/mipmap-anydpi-v26/ic_launcher.xml -->
<!-- android/app/src/main/res/mipmap-anydpi-v26/ic_launcher_round.xml -->
<?xml version="1.0" encoding="utf-8"?>
<adaptive-icon xmlns:android="http://schemas.android.com/apk/res/android">
    <background android:drawable="@drawable/ic_launcher_background" />
    <foreground android:drawable="@drawable/ic_launcher_foreground" />
    <monochrome android:drawable="@drawable/ic_launcher_foreground" />
</adaptive-icon>

<!-- android/app/src/main/res/drawable/ic_launcher_background.xml -->
<vector
    xmlns:android="http://schemas.android.com/apk/res/android"
    android:width="108dp" android:height="108dp"
    android:viewportWidth="108" android:viewportHeight="108">
    <path android:fillColor="#912391" android:pathData="M0,0h108v108h-108z" />
</vector>

<!-- android/app/src/main/res/drawable-v24/ic_launcher_foreground.xml -->
<vector><!-- ... vector image --></vector>
\end{lstlisting}

And it has to be created `ic\_launcher.png` and `ic\_launcher\_round.png` icons for all other formats (mipmap-hdpi, 
mipmap-mdpi, ...) to support less than Android SDK 26 versions.
