% Copyright 2023 The terCAD team. All rights reserved.
% Use of this content is governed by a CC BY-NC-ND 4.0 license that can be found in the LICENSE file.

\subsection{[TBD] Adding Marginal Notes of Flutter usage}
\markboth{Implementing Core Functionality}{Adding Marginal Notes of Flutter usage}

[TBD]

\subsubsection{Mixing `PageView`' and `PageController`}

Now we're going to add three pages controlled by swiping left and right for adding new bills, we'll use a combination 
of PageView and PageController. 

\begin{lstlisting}
class Page1 extends StatelessWidget { /* ... */ }
class Page2 extends StatelessWidget { /* ... */ }
class Page3 extends StatelessWidget { /* ... */ }

class MyApp extends StatelessWidget {
  final PageController _pageController = PageController(initialPage: 0);

  @override
  Widget build(BuildContext context) {
    return MaterialApp(
      home: Scaffold(
        body: GestureDetector(
          onHorizontalDragEnd: (DragEndDetails details) {
            if (details.primaryVelocity! > 0) { // Swiped right
              _pageController.previousPage(
                duration: Duration(milliseconds: 500),
                curve: Curves.ease,
              );
            } else if (details.primaryVelocity! < 0) { // Swiped left
              _pageController.nextPage(
                duration: Duration(milliseconds: 500),
                curve: Curves.ease,
              );
            }
          },
          child: PageView(
            controller: _pageController,
            children: [Page1(), Page2(), Page3()],
          ),
        ),
      ),
    );
  }
}
\end{lstlisting}

If we do want to click on tabs instead of swiping, then it can be changed to:

\begin{lstlisting}
class MyApp extends StatelessWidget {
  final PageController _pageController = PageController(initialPage: 0);

  @override
  Widget build(BuildContext context) {
    return MaterialApp(
      home: DefaultTabController(
        length: 3,
        child: Scaffold(
          appBar: AppBar(
            title: Text('Swiping Pages'),
            bottom: TabBar(
              tabs: [
                Tab(text: 'Page 1'),
                Tab(text: 'Page 2'),
                Tab(text: 'Page 3'),
              ],
            ),
          ),
          body: TabBarView(
            children: [
              Page1(),
              Page2(),
              Page3(),
            ],
          ),
        ),
      ),
    );
  }
}
\end{lstlisting}

By loving a reach User Interface flow, let's combine all together:

\begin{lstlisting}
class MyApp extends StatelessWidget {
  final int tabCount = 3;
  int tabIndex = 1;
  PageController? pageController;
  TabController? tabController;

  @override
  void initState() {
    super.initState();
    pageController = PageController(initialPage: tabIndex);
    tabController = TabController(
      length: tabCount,
      vsync: const _VSync(),
      initialIndex: tabIndex,
    );
  }

  @override
  void dispose() {
    pageController?.dispose();
    tabController?.dispose();
    super.dispose();
  }

  void switchTab(int newIndex) {
    setState(() {
      tabIndex = newIndex;
      tabController?.animateTo(newIndex);
      pageController?.animateToPage(
        newIndex,
        duration: Duration(milliseconds: 300),
        curve: Curves.ease,
      );
    });
  }

  @override
  Widget build(BuildContext context) {
    return MaterialApp(
      home: GestureDetector(
        onHorizontalDragEnd: (DragEndDetails details) {
          if (details.primaryVelocity! > 0) {
            switchTab(widget.tabIndex - 1);
          } else if (details.primaryVelocity! < 0) {
            switchTab(widget.tabIndex + 1);
          }
        },
        child: Scaffold(
          appBar: TabBar(
              controller: tabController,
              onTap: switchTab,
              tabs: [
                Tab(text: 'Page 1'),
                Tab(text: 'Page 2'),
                Tab(text: 'Page 3'),
              ],
            ),
          body: PageView(
            controller: pageController,
            onPageChanged: switchTab,
            children: [
              Page1(),
              Page2(),
              Page3(),
            ],
          ),
        ),
      ),
    );
  }
}

class _VSync implements TickerProvider {
  const _VSync();

  @override
  Ticker createTicker(TickerCallback onTick) {
    return Ticker(onTick);
  }
}

class Page1 extends StatelessWidget { /* ... */ }

class Page2 extends StatelessWidget { /* ... */ }

class Page3 extends StatelessWidget { /* ... */ }
\end{lstlisting}

By combining both solutions, some irritating behavior happens, - we cannot tab from `Page1' to `Page3' 
(more than one position) since `pageController` after the first movement will send an update via `onPageChanged`. 
Hopefully, we do have a solution for that by introducing delayed trigger to proceed with switching till the chosen tab:

\begin{lstlisting}
Future<void> delaySwitchTab(int delay, int newIndex) async {
  await Future.delayed(Duration(milliseconds: delay));
  switchTab(newIndex);
}

void switchTab(int newIndex) {
  if (newIndex < 0 || newIndex >= widget.tabCount) {
    return;
  }
  setState(() {
    const delay = 300;
    // Saving current state for the check after 
    final currIndex = tabIndex;
    tabIndex = newIndex;
    tabController?.animateTo(newIndex);
    pageController?.animateToPage(
      newIndex,
      duration: const Duration(milliseconds: delay),
      curve: Curves.ease,
    );
    // Verify that the difference is more than one
    if ((currIndex - newIndex).abs() > 1) {
      delaySwitchTab(delay, newIndex);
    }
  });
}
\end{lstlisting}


\subsubsection{Adding `AppBar` as dots to `TabBar`}

In some cases we do not need a fully defined navigation bar (with titles and icons), just dots to identify that
current block can be swiped or changed by clicking on any of dots. To implement them it can be used either existing 
external components (and added to the page as Widget), or by defining `indicator`-property of `TabBar`-widget with
custom implementation of `Decoration`-class. Let's paint our dots:

\begin{lstlisting}
// ./lib/widgets/_wrappers/dots_indicator_decoration.dart
import 'package:flutter/material.dart';

class DotsIndicatorDecoration extends Decoration {(*@ \stopnumber @*)

  // ... properties and constructor
  (*@ \startnumber{10} @*)
  @override
  BoxPainter createBoxPainter([VoidCallback? onChanged]) {
    return _CustomTabIndicatorPainter(
      // ... properties
      onChanged: onChanged,
    );
  }
}

class _CustomTabIndicatorPainter extends BoxPainter {(*@ \stopnumber @*)

  // ... properties and constructor
  (*@ \startnumber{33} @*)
  @override
  void paint(Canvas canvas, Offset offset, ImageConfiguration configuration) {
    if (itemCount <= 1) { // Skip for a single tab
      return;
    }
    // Take from PageController an actual page, otherwise - initial
    final activeIndex = controller.page?.round() ?? controller.initialPage;
    // dotSize, spacing, color - initialized properties
    final active = Paint()..color = color;
    final inactive = Paint()..color = color.withOpacity(0.3);
    for (int i = 0; i < itemCount; i++) {
      double xPos = spacing + i * (dotSize + spacing);
      // Pu cycle a little below the baseline
      double yPos = spacing * 0.6; 
      if (i == activeIndex) {
        canvas.drawCircle(Offset(xPos, yPos), dotSize / 2, active);
      } else {
        canvas.drawCircle(Offset(xPos, yPos), dotSize / 2, inactive);
      }
    }
  }
}
\end{lstlisting}

\noindent Additionally, we might hide the usage of our custom indicator by extending from `TabBar`:

\begin{lstlisting}
// ./lib/widgets/_wrappers/dots_tab_bar_widget.dart
import 'package:app_finance/widgets/_wrappers/dots_indicator_decoration.dart';
import 'package:flutter/material.dart';

class DotsTabBarWidget extends TabBar {
  final TabController tabController;
  final PageController pageController;
  final List<Widget> tabList;
  final double indent; // Indentation between cycles
  final double width; // MediaQuery.of(context).size.width
  final Color color; // Color for active cycle

  const DotsTabBarWidget({
    super.key,
    required this.tabController,
    required this.pageController,
    required this.tabList,
    required this.indent,
    required this.width,
    required this.color,
    onTap,
  }) : super(
          controller: tabController,
          mouseCursor: SystemMouseCursors.click,
          onTap: onTap,
          // hook, since `tabs` is required
          tabs: tabList, // getter overrides the flow
        );
  // Convert children of `PageView` to a clickable area 
  @override
  get tabs =>
      tabList.map((tab) => SizedBox(width: indent, height: indent)).toList();
  // Make our dots centered on the page
  @override
  get padding => EdgeInsets.symmetric(
      horizontal: (width - tabList.length * 2 * indent) / 2);
  // Apply custom decorator to draw cycles
  @override
  get indicator => DotsIndicatorDecoration(
        controller: pageController,
        itemCount: tabList.length,
        color: color,
        dotSize: indent,
      );
}
\end{lstlisting}
