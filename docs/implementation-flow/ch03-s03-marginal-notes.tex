% Copyright 2023 The terCAD team. All rights reserved.
% Use of this content is governed by a CC BY-NC-ND 4.0 license that can be found in the LICENSE file.

\subsection{[TBD] Adding Marginal Notes of Flutter usage}
\markboth{Implementing Core Functionality}{Adding Marginal Notes of Flutter usage}

[TBD]

\subsubsection{Applying Getters and Setters for Class Variables}

By migrating our data structures to class objects it can be used additionally getter and setter for class variables.
That will help us to cover, as an example, `description`-attribute for Budget structures by showing 
`Spent Amount / Total Budget`, and for `details`-attribute - `Left Amount`.

\begin{lstlisting}
class MyClass {
  String _name; // Private variable

  // Getter for name
  String get name => _name;

  // Setter for name
  set name(String value) {
    _name = value;
  }
}

void main() {
  MyClass myObject = MyClass();

  // Using the setter
  myObject.name = 'John';

  // Using the getter
  print(myObject.name); // Output: John
}
\end{lstlisting}

In the example above, the `MyClass`-class has a private variable `\_name`. The getter for `\_name` is defined 
using the get keyword, and the setter is defined using the set keyword. The getter returns the value of 
`\_name`, and the setter sets the value of `\_name` to the provided value. By using that approach we can control 
the access to class variables and perform additional logic or validation when getting or setting their values.

One more question, is how to set `DateTime.now()` inside a constructor as default value, since, by defining it inside
the constructor, an error would be taken: The default value of an optional parameter must be constant.

So, to set a default value for a DateTime variable as the current date and time, we can use the `DateTime.now()` method 
during the initialization of the variable.

\begin{lstlisting}
class MyClass {
    DateTime _createdAt;

    MyClass({DateTime? createdAt})
        : _createdAt = createdAt ?? DateTime.now();

    DateTime get createdAt => _createdAt;
}
\end{lstlisting}


\subsubsection{Using cast to resolve dynamic structures}

Migration our Bills to classes construction has lead to an error on .firstWhere when Account of Budget not found. 
To resolve that we have to cast our getter (otherwise it would be an error: type '() => dynamic' is not a subtype 
of type '(() => BudgetAppData)?' of 'orElse') and add orElse operator as following:

\begin{lstlisting}
  dynamic getByUuid(AppDataType property, String uuid) {
    var scope = (_data[property] as Map)['list'];
    return scope.cast<dynamic>() // cast to cover different types, 
    .firstWhere(
      (item) => item.uuid == uuid,
      orElse: () => null, // Return null if not found
    );
  }
\end{lstlisting}


\subsubsection{Mixing `PageView`' and `PageController`}

Now we're going to add three pages controlled by swiping left and right for adding new bills, we'll use a combination 
of PageView and PageController. 

\begin{lstlisting}
class Page1 extends StatelessWidget {
  @override
  Widget build(BuildContext context) {
    return Container(
      color: Colors.blue,
      child: Center(
        child: Text('Page 1'),
      ),
    );
  }
}

class Page2 extends StatelessWidget {
  @override
  Widget build(BuildContext context) {
    return Container(
      color: Colors.green,
      child: Center(
        child: Text('Page 2'),
      ),
    );
  }
}

class Page3 extends StatelessWidget {
  @override
  Widget build(BuildContext context) {
    return Container(
      color: Colors.orange,
      child: Center(
        child: Text('Page 3'),
      ),
    );
  }
}

class MyApp extends StatelessWidget {
  final PageController _pageController = PageController(initialPage: 0);

  @override
  Widget build(BuildContext context) {
    return MaterialApp(
      home: Scaffold(
        body: GestureDetector(
          onHorizontalDragEnd: (DragEndDetails details) {
            if (details.primaryVelocity! > 0) {
              // Swiped right
              _pageController.previousPage(
                duration: Duration(milliseconds: 500),
                curve: Curves.ease,
              );
            } else if (details.primaryVelocity! < 0) {
              // Swiped left
              _pageController.nextPage(
                duration: Duration(milliseconds: 500),
                curve: Curves.ease,
              );
            }
          },
          child: PageView(
            controller: _pageController,
            children: [
              Page1(),
              Page2(),
              Page3(),
            ],
          ),
        ),
      ),
    );
  }
}
\end{lstlisting}

To operate with visible tabs it can be changed to:

\begin{lstlisting}
class MyApp extends StatelessWidget {
  final PageController _pageController = PageController(initialPage: 0);

  @override
  Widget build(BuildContext context) {
    return MaterialApp(
      home: DefaultTabController(
        length: 3,
        child: Scaffold(
          appBar: AppBar(
            title: Text('Swiping Pages'),
            bottom: TabBar(
              tabs: [
                Tab(text: 'Page 1'),
                Tab(text: 'Page 2'),
                Tab(text: 'Page 3'),
              ],
            ),
          ),
          body: TabBarView(
            children: [
              Page1(),
              Page2(),
              Page3(),
            ],
          ),
        ),
      ),
    );
  }
}
\end{lstlisting}

Let's combine all together:

\begin{lstlisting}
class MyApp extends StatelessWidget {
  final int tabCount = 3;
  int tabIndex = 1;
  PageController? pageController;
  TabController? tabController;

  @override
  void initState() {
    super.initState();
    pageController = PageController(initialPage: tabIndex);
    tabController = TabController(
      length: tabCount,
      vsync: const _VSync(),
      initialIndex: tabIndex,
    );
  }

  @override
  void dispose() {
    pageController?.dispose();
    tabController?.dispose();
    super.dispose();
  }

  void switchTab(int newIndex) {
    setState(() {
      tabIndex = newIndex;
      tabController?.animateTo(newIndex);
      pageController?.animateToPage(
        newIndex,
        duration: Duration(milliseconds: 300),
        curve: Curves.ease,
      );
    });
  }


  @override
  Widget build(BuildContext context) {
    return MaterialApp(
      home: GestureDetector(
        onHorizontalDragEnd: (DragEndDetails details) {
          if (details.primaryVelocity! > 0) {
            switchTab(widget.tabIndex - 1);
          } else if (details.primaryVelocity! < 0) {
            switchTab(widget.tabIndex + 1);
          }
        },
        child: Scaffold(
          appBar: TabBar(
              controller: tabController,
              onTap: switchTab,
              tabs: [
                Tab(text: 'Page 1'),
                Tab(text: 'Page 2'),
                Tab(text: 'Page 3'),
              ],
            ),
          body: PageView(
            controller: pageController,
            onPageChanged: switchTab,
            children: [
              Page1(),
              Page2(),
              Page3(),
            ],
          ),
        ),
      ),
    );
  }
}

class _VSync implements TickerProvider {
  const _VSync();

  @override
  Ticker createTicker(TickerCallback onTick) {
    return Ticker(onTick);
  }
}

class Page1 extends StatelessWidget {
  @override
  Widget build(BuildContext context) {
    return Container(
      child: Center(
        child: Text('Page 1'),
      ),
    );
  }
}

class Page2 extends StatelessWidget {
  @override
  Widget build(BuildContext context) {
    return Container(
      child: Center(
        child: Text('Page 2'),
      ),
    );
  }
}

class Page3 extends StatelessWidget {
  @override
  Widget build(BuildContext context) {
    return Container(
      child: Center(
        child: Text('Page 3'),
      ),
    );
  }
}
\end{lstlisting}

By combining both solutions, some irritating behavior is happened, - we cannot tab from `Income' to `Transfer' 
(more than one position) since `pageController` after the first movement will send an update via `onPageChanged`. 
Hopefully, we do have a solution for that by introducing delayed trigger to proceed with switching to a chosen tab:

\begin{lstlisting}
Future<void> delaySwitchTab(int delay, int newIndex) async {
  await Future.delayed(Duration(milliseconds: delay));
  switchTab(newIndex);
}

void switchTab(int newIndex) {
  if (newIndex < 0 || newIndex >= widget.tabCount) {
    return;
  }
  setState(() {
    const delay = 300;
    final currIndex = widget.tabIndex; // Saving current state for the check after 
    widget.tabIndex = newIndex;
    widget.tabController?.animateTo(newIndex);
    widget.pageController?.animateToPage(
      newIndex,
      duration: const Duration(milliseconds: delay),
      curve: Curves.ease,
    );
    // Verify that the difference is more than one
    if ((currIndex - newIndex).abs() > 1) {
      delaySwitchTab(delay, newIndex);
    }
  });
}
\end{lstlisting}


\subsubsection{Asserting Functions Definition}

While defining functions as arguments for widgets, it can be used an expectation of that function structure

\begin{lstlisting}
typedef SetViewFunction = String Function(Currency input);

class CurrencySelector extends StatelessWidget {
  SetViewFunction? setView;

  CurrencySelector({
    Key? key,
    this.setView,
  }) {
    setView ??= setDefaultView;
  }

  String setDefaultView(Currency input) {
    // Default implementation
  }
}
\end{lstlisting}