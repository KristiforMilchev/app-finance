% Copyright 2023 The terCAD team. All rights reserved.
% Use of this content is governed by a CC BY-NC-ND 4.0 license that can be found in the LICENSE file.

\subsection{Microsoft Store}
\markboth{Distributing}{Microsoft Store}

To distribute a Windows app through the Microsoft Store, we should create a Microsoft Partner Center account
(\href{https://partner.microsoft.com}{https://partner.microsoft.com}). By going through the registration form, it 
would be needed to pay the initial fee (roughly 80 Euro with TAX). The registration would be a prerequisite for 
creating the distributable MSIX-build, since a few properties for the build should be aligned with the information on 
our Partner Center (\cref{img:d-windows}). 

The most straightforward method for generating an MSIX-installer for our project is by utilizing the 
\q{msix}-package \issue{209}{}:

\begin{lstlisting}[language=terminal]
$ flutter pub add --dev msix
\end{lstlisting}

\begin{lstlisting}[language=yaml]
# ./pubspec.yaml
msix_config:
  display_name: Fingrom MSIX
  publisher: CN=78144012-EC6A-4BE8-97BF-A392EF55482E
  publisher_display_name: terCAD
  identity_name: terCAD.FingromMSIX
  logo_path: web/icons/Icon-512.png
  capabilities: internetClient # permissions
  store: true # distributable package
\end{lstlisting}

\img{distributing/win-store}{Microsoft Store -- Product Identity}{img:d-windows}

\noindent As a tip, MS Store requires to use all zeros for the version definition:

\begin{lstlisting}[language=terminal]
[X] Package acceptance validation error: Apps are not allowed
 to have a Version with a revision number other than zero 
 specified in the app manifest.
\end{lstlisting}

\begin{lstlisting}[language=terminal]
# Prepare Windows release
$ flutter config --enable-windows-desktop
$ flutter build -v windows \
  --build-name=${{ version }} \
  --build-number=${{ build_number }} \
  --release

# Generate the MSIX-installer
$ dart run msix:create \
  --version 0.0.0.0	\
  --output-path ${{ matrix.build_path }} \
  --build-windows false \
  --capabilities internetClient
\end{lstlisting}

\noindent MSIX installers should be signed with a certificate, to ensure that app installs and updates come from 
trustworthy sources. But in case of publishing to the Microsoft Store, the installation package will be signed 
automatically by the store. In addition, we may utilize App Center (\href{https://appcenter.ms}{https://appcenter.ms}) 
as an early access platform for the users; it contains own Analytics and Crashes that can be used across multiple 
platforms.

To streamline the distribution process, we might use MS Store CLI (command line interface), a tool designed to 
interface with the Microsoft Store submission API for publishing Windows apps to the Microsoft Store. To get started 
with this integration, we'll need to establish a connection between our Microsoft Partner Center account and an Azure 
AD application. We have to retrieve essential information like Tenant Name, Tenant ID, Client ID, and Client Secret. 

We would start from creating a new tenant at 
\href{https://partner.microsoft.com/en-us/dashboard/account/v3/tenantmanagement}{https://partner.microsoft.com/en-us/dashboard/account/v3/tenantmanagement},
by clicking on "Create", and completing the necessary information. Next, we'll access our Azure AD account using the 
new tenant credentials via \href{https://entra.microsoft.com}{https://entra.microsoft.com}. Within Azure AD, by 
navigating to "Applications", we'll proceed with a new registration (as a note, the "Redirect URI"-field might be blank).
Under the "Certificates \& secrets" tab, we can generate a new certificate. This certificate will be utilized by our 
CI/CD pipeline, especially "Value"-field within this certificate represents our "Client Secret" for the CLI. To finalize 
the configuration, we'll visit the User Management 
(\href{https://partner.microsoft.com/en-us/dashboard/account/v3/usermanagement}{https://partner.microsoft.com/en-us/dashboard/account/v3/usermanagement})
to associate the created "MS Entra application", and grant a "Development" access to enable package uploads. 


\subsection{[TBD] Linux Snap Store}
\markboth{Distributing}{[TBD] Linux Snap Store}

While the world of Linux distributions can appear complex due to the sheer variety available, there is a unified 
solution to simplify the process: snaps. A snap is essentially a bundled package containing one or more applications 
along with all their dependencies. What's remarkable about snaps is their ability to run consistently across a wide 
array of Linux distributions, without requiring any modifications. These snaps are conveniently discoverable and 
installable from the Snap Store (\href{https://snapcraft.io}{https://snapcraft.io}).

To get started, we should visit the page \href{https://snapcraft.io/snaps}{https://snapcraft.io/snaps} to register 
our application by simply clicking the "Register a snap name" button, with options to choose between "private" or 
"public" application availability. A valuable feature to highlight is available on the page at 
\href{https://snapcraft.io/fingrom/builds}{https://snapcraft.io/fingrom/builds} to link our GitHub account. 
This integration allows for the building and distribution of our application directly from this page, 
streamlining the deployment process. 
If it's needed a control over the application's distribution, whether through Github Actions or manual processes, we may
refer to the instructions on configuring and building Flutter projects provided at 
\href{https://snapcraft.io/docs/flutter-applications}{https://snapcraft.io/docs/flutter-applications}.
So, we will utilize the \q{snapcraft}-application (it can be installed and used on various systems, including different 
Linux distributions, macOS, and Windows):

\begin{lstlisting}[language=terminal]
# For Linux with snap-support
$ sudo snap install snapcraft --classic 

## Install Virtual Machine Manager
# For Github Actions "uses: canonical/setup-lxd"
$ sudo snap install lxd # required by snapcraft
$ sudo adduser $USER lxd # grant permissions
$ newgrp lxd # apply changes
$ sudo lxd waitready # revise state
$ sudo lxd init --auto # set up the LXD server

# Login to Snap Store
$ snapcraft login
Enter your Ubuntu One e-mail address and password.
Email: ...
Password: ...
Login successful 

# Retrieve Developer ID
$ snapcraft whoami
id: ...

## Export credentials for CI/CD
$ snapcraft export-login --snaps=fingrom --acls package_access,package_push,package_update,package_release credentials.txt

# Generate snapcraft.yaml
$ snapcraft init

## Build Package
$ snapcraft

# Test generated package
$ sudo snap install my-snap-name_0.1_amd64.snap --devmode

## Release Package
$ snapcraft release
\end{lstlisting}

\noindent The configuration file for the Snapcraft tool, which provides instructions on how to build the Snap package:

\begin{lstlisting}[language=yaml]
# ./snapcraft.yaml
name: fingrom
base: core22 # span, based on Ubuntu 22.04 
version: 1.0.0+1 # to be replaced by CI/CD
summary: Platform-agnostic financial accounting app
description: |
  ... description details ...

# Type of the build
grade: stable # a stable version
confinement: strict # enforces strict isolation

parts:
  app-finance:
    source: .
    # To trigger: 'flutter linux' build
    plugin: flutter
    flutter-target: lib/main.dart
    # Dependencies to external packages
    build-packages: [libgtk-3-dev, ninja-build]
\end{lstlisting}

\noindent \q{Parts}-section defines the parts that make up our Snap package. In our case, there's only one part called 
app-finance. It specifies the source code location (in this directory), the plugin to use (in this case, the Flutter 
plugin), the target for building your Flutter app (usually the main Dart file), and any additional build packages or 
dependencies required.

Finally, the pipeline for \q{snap}-packages distribution would be the next:

\begin{lstlisting}[language=yaml]
# ./snapcraft.yaml
- name: Install LXD
  if: matrix.target == 'Linux'
  uses: canonical/setup-lxd@v0.1.1

- name: Build Snap
  if: matrix.target == 'Linux'
  run: |
    sudo snap install snapcraft --classic
    snapcraft --verbose

- name: Publish Snap
  if: matrix.target == 'Linux'
  env:
    SNAPCRAFT_STORE_CREDENTIALS: ${{ secrets.CREDENTIALS }}
    run: |
      sudo --preserve-env=SNAPCRAFT_STORE_CREDENTIALS snapcraft upload *.snap --release=latest/stable
\end{lstlisting}
