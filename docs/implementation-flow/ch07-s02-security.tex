% Copyright 2023 The terCAD team. All rights reserved.
% Use of this content is governed by a CC BY-NC-ND 4.0 license that can be found in the LICENSE file.

\subsection{Securing Information}
\markboth{Productionizing}{Securing Information}

Securing information is essential for financial accounting, as well as for any application. Ensuring the privacy 
of this information is not only ethically necessary but also legally mandated. Many regions have stringent data 
protection laws that require businesses to secure financial data, such as GDPR in Europe and HIPAA in the United States.
Failing to do so can result in severe fines and legal consequences.

When users entrust their financial information to an application, they expect that their data will be kept secure. 
Building and maintaining trust is vital in the financial sector, and a data breach can irreparably damage a company's 
reputation. So, a financial institution or application that demonstrates strong security practices can use this as a 
competitive advantage.

To achieve information security, applications must implement robust cybersecurity measures, including encryption, 
access controls, and regular security audits. Ensuring the app security is a continuous endeavor that requires a 
proactive approach.


\subsubsection{Storing API keys}
\markboth{Productionizing}{Storing API keys}

Every API requires passing a secret key with each request to validate the identity. These keys should remain 
confidential (as not be a part of the \q{git} history). Even detaching credentials into a separate \q{.env}-file  
susceptible to leaks. The most secure way is to store secrets as environment variables during the compilation 
(\q{build}) or evaluation (\q{run}):

\begin{lstlisting}[language=terminal]
flutter build --dart-define==KEY=SECRET \
    --dart-define==KEY2=SECRET2 \
    --dart-define==KEY3=SECRET3
\end{lstlisting}

\noindent To get the API key from an environment:

\begin{lstlisting}
String? apiKey = String.fromEnvironment('KEY', defaultValue: null);
\end{lstlisting}

\noindent For handling dynamic values, consider employing the \q{flutter\_secure\_storage}-package to securely store 
sensitive data through encrypted shared preferences (\emph{Keychain for iOS and macOS, AES encryption for Android and 
Windows, libsecret for Linux, and WebCrypto for Web}):

\begin{lstlisting}
final storage = new FlutterSecureStorage();
await storage.write(key: key, value: value);
String value = await storage.read(key: key);
\end{lstlisting}


\subsubsection{Encrypting Storage}
\markboth{Productionizing}{Encrypting Storage}

Encryption is a critical process in the realm of cybersecurity, as it involves transforming a plain text into its 
encoded format, bolstering the confidentiality, integrity, and overall security of sensitive information. The process 
of "hiding" information can be one-way direction (by taking \q{hash}-key from a value, and using it for a comparison) 
or bidirectional (decrypt with a cryptographic key). For the last, there are two primary types of encryption: symmetric 
(AES, 3-DES, SNOW) and asymmetric (RSA, Elliptic curve cryptography) encryption, which is also referred to as public 
key encryption. Symmetric encryption relies on a single shared key, which all parties involved use for both encrypting 
and decrypting data. In contrast, asymmetric encryption, often referred to as public-key encryption, employs a pair of 
keys: one for encryption and another distinct key for decryption purposes.

To "disable" (at least make harder) capability to change financial transactions directly, it may be used a sum control 
\q{hash}-key \issue{58}{}:

\begin{lstlisting}
// ./lib/_classes/data/transaction_log.dart
import 'package:crypto/crypto.dart';

class TransactionLog {
  static String getHash(Map<String, dynamic> data) {
    return md5.convert(utf8.encode(data.toString())).toString();
  }

  static Future<bool> load() async {
    // ... other stuff
    await for (var line in lines) {
      var obj = json.decode(line);
      if (getHash(obj['data']) != obj['type']['hash']) {
        continue; // Corrupted data... skip
      }
\end{lstlisting}

\noindent Additionally, to protect user data we should apply mechanics of lines encryption:

\begin{lstlisting}
// ./lib/_classes/data/transaction_log.dart
import 'package:encrypt/encrypt.dart';

class TransactionLog {
  static Encrypter get salt =>
      Encrypter(AES(Key.fromUtf8('32symbols-salt-for-encryption...')));

  static IV get code => IV.fromLength(8);(*@ \stopnumber @*)

  (*@ \startnumber{49} @*)
  // to encrypt content
  content = salt.encrypt(content, iv: code).base64;
  // to decrypt content
  content = salt.decrypt64(content, iv: code);
\end{lstlisting}

\noindent As a note, the usage of any external library for encryption (and not only) might be relevant to additional 
risks, as a failed upgrade ("dependency shock" \cite{Inki23}) or a breakable change (in case of \q{encrypt}-package, a 
failed decryption after a migration from 5.0.1 to 5.0.2 -- 
\href{https://github.com/leocavalcante/encrypt/issues/314}{https://github.com/leocavalcante/encrypt/issues/314}).
So, it's better to be chosen a built-in functionality even with an additional complexity; in our case, through the 
\q{dart:crypto}-library, which includes hashing, symmetric and asymmetric encryption.
