% Copyright 2023 The terCAD team. All rights reserved.
% Use of this content is governed by a CC BY-NC-ND 4.0 license that can be found in the LICENSE file.

\subsubsection{Google Play}

The process is started from Play Console (\href{https://play.google.com/console}{https://play.google.com/console}) 
via "Select All apps", then by clicking on "Create app". We begin by selecting a default language and typing a 
title of our application. This title is how it will be known to users on the Google Play Store. As a tip, it can be 
updated later if inspiration strikes. The choice between "game" or "app" affect the categorization, also changeable 
afterwards.

We need our release build to be signed (\emph{\q{keytool}-command might not be recognized, it's a part of Java SDK; so,
use the full path of its location instead of a simple command}):

\begin{lstlisting}[language=terminal]
# For Linux and macOS
$ keytool -genkey -v -keystore ~/key.jks -keyalg RSA -keysize 2048 \
    -validity 10000 -alias key

# For Windows
$ keytool -genkey -v -keystore c:\Users\USER_NAME\key.jks -storetype \
    JKS -keyalg RSA -keysize 2048 -validity 10000 -alias key
\end{lstlisting}

\noindent From "App signing preferences" in Google Play Console we have to download a public key with Play encrypt 
private key (PEPK) tool to generate and upload \q{.zip}-file:

\begin{lstlisting}[language=terminal]
$ java -jar pepk.jar --keystore=key.jks --alias=key \
    --output=google-play_output.zip --include-cert \
    --rsa-aes-encryption \
    --encryption-key-path=./google-play_public_key.pem 

Enter password for store 'key.jks':
Enter password for key 'key':
\end{lstlisting}

\noindent Then, it's a time to sign the bundle by adding keychain preferences (as a warning, \q{key.properties}-file 
should never be a part of the repository):

\begin{lstlisting}
// ./android/key.properties
key.password=/* key.jks password */
key.file=/* location of key.jks */
\end{lstlisting}

\noindent And adjust the android build-config of our application:\\

{
\xpretocmd{\lstlisting}{\vspace{-12pt}}{}{}
\begin{lstlisting}
// ./android/app/build.gradle
def localProperties = new Properties()
\end{lstlisting}
\begin{lstlisting}[firstnumber=3, backgroundcolor=\color{backgreen}]
(*@\kdiff{+}@*)
(*@\kdiff{+}@*)def store = rootProject.file('key.properties')
(*@\kdiff{+}@*)if (store.exists()) {
(*@\kdiff{+}@*)  store.withReader("UTF-8"){reader -> localProperties.load(reader)}
(*@\kdiff{+}@*)}
(*@\kdiff{+}@*)
\end{lstlisting}
\begin{lstlisting}[firstnumber=9]
// ... other code
\end{lstlisting}
\begin{lstlisting}[firstnumber=63, backgroundcolor=\color{backgreen}]
(*@\kdiff{+}@*)signingConfigs {
(*@\kdiff{+}@*) release {
(*@\kdiff{+}@*)  keyAlias 'key'
(*@\kdiff{+}@*)  keyPassword localProperties.getProperty('key.password')
(*@\kdiff{+}@*)  storeFile file(localProperties.getProperty('key.file'))
(*@\kdiff{+}@*)  storePassword localProperties.getProperty('key.password')
(*@\kdiff{+}@*) }
(*@\kdiff{+}@*)}
\end{lstlisting}
\begin{lstlisting}[firstnumber=71]
buildTypes {
  release {
\end{lstlisting}
\begin{lstlisting}[firstnumber=73, backgroundcolor=\color{backred}]
(*@\kdiff{-}@*)    signingConfig signingConfigs.debug
\end{lstlisting}
\begin{lstlisting}[firstnumber=73, backgroundcolor=\color{backgreen}]
(*@\kdiff{+}@*)    signingConfig signingConfigs.release
\end{lstlisting}
\begin{lstlisting}[firstnumber=74]
  }
}
\end{lstlisting}
}


\subsubsection{Galaxy Store}

The process of distribution to Galaxy marketplace is started from a Samsung Galaxy Store Seller Portal:
\href{https://seller.samsungapps.com}{https://seller.samsungapps.com}.
After a registration form we'll asked to request "Commercial Seller Status" (from 
\href{https://seller.samsungapps.com/member/getSellerDetail.as}{https://seller.samsungapps.com/member/getSellerDetail.as})
to be able to distribute Android applications. That would require a D-U-N-S Number (a unique nine-digit identifier for 
businesses) for the case of payable app or in-app purchases. It can be taken from the page 
\href{https://www.dnb.com/duns/get-a-duns.html}{ https://www.dnb.com/duns/get-a-duns.html}
from a multiple options, one of them -- via Apple Store 
(\href{https://developer.apple.com/enroll/duns-lookup/}{https://developer.apple.com/enroll/duns-lookup/}).
That request might take up to 5 business days to receive the number from D\&B (\href{https://dnb.com}{https://dnb.com})
(and up to 15 days in case of the data inconcistency, as missed address -- 
\href{https://support.dnb.com/?CUST=APPLEDEV}{https://support.dnb.com/?CUST=APPLEDEV}). 

Then we would need to follow instructions \issue{284}{} from 
\href{https://developer.samsung.com/galaxy-store/galaxy-store-developer-api.html}{https://developer.samsung.com/galaxy-store/galaxy-store-developer-api.html}.
Also, we might encounter the error "The registered binaries do not meet the category conditions for Galaxy Specials" 
after submitting, it's needed to update \q{manifest.xml}-file by adding the appropriate permission:

\begin{lstlisting}[language=xml]
<uses-permission android:name="com.samsung.android.providers.context.permission.WRITE_USE_APP_FEATURE_SURVEY"/>
\end{lstlisting}

\subsubsection{Huawei Gallery}

The marketplace link for distribution is \href{https://developer.huawei.com/}{https://developer.huawei.com/}.
We may proceed with a registration either as individuals, or business 
(\href{https://developer.huawei.com/consumer/en/verified/company}{https://developer.huawei.com/consumer/en/verified/company}).
The identity verification will be completed within 1 to 2 working days.

It would also be needed to upload the signature key (that we've created for Google Play earlier) with a 
personal encryption key (can be taken from the "App Signature" page):

\begin{lstlisting}[language=terminal]
$ java -jar pepk.jar --keystore key.jks --alias key \
  --output sign.zip --encryptionkey {from "App Signing" page} \
  --include-cert
\end{lstlisting}

\img{distributing/huawei-gallery-new}{Huawei Gallery -- New App Creation}{img:d-huawei}

\noindent Huawei Gallery does not provide any mechanics to upload the package automatically. So, we should navigate to 
"My App" in the Huawei Developer Console, and choose the application we've created. Then fill out the mandatory fields, 
including compatibility, localization, and categorization, on the "App Information" screen. Afterward, we should click 
the "Next"-button and proceed to the "Version Information"-step. On the "Version Information"-screen, we provide details 
for various fields such as Country/Region for release, Open testing, Payment information, In-app purchases, Content 
rating, Privacy statement, Privacy tags, Copyright information, For reviewer, and Release. Then, we upload the apk or 
app bundle of our project to the app version \issue{286}{}. With these steps completed, our app is now prepared for 
submission to Huawei AppGallery (\cref{img:d-huawei-review}).

\img{distributing/huawei-gallery-review}{Huawei Gallery -- App Distribution}{img:d-huawei-review}

\noindent To distribute payable applications we have to fulfil merchant information (as bank account details, TAX ID, 
and sales tax registration status [for Canada]):
\href{https://developer.huawei.com/consumer/en/console/setting/merchant}{https://developer.huawei.com/consumer/en/console/setting/merchant}.
