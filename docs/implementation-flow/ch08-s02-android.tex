% Copyright 2023 The terCAD team. All rights reserved.
% Use of this content is governed by a CC BY-NC-ND 4.0 license that can be found in the LICENSE file.

\subsection{Google Play}
\markboth{Distributing}{Google Play}

The process is started from Play Console (\href{https://play.google.com/console}{https://play.google.com/console}) 
via "Select All apps", then by clicking on "Create app". We begin by selecting a default language and typing a 
title of our application. This title is how it will be known to users on the Google Play Store. As a tip, it can be 
updated later if inspiration strikes. The choice between "game" or "app" affect the categorization, also changeable 
afterwards.

We need our release build to be signed (\emph{\q{keytool}-command might not be recognized, it's a part of Java SDK; so,
use the full path of its location instead of a simple command}):

\begin{lstlisting}[language=terminal]
# For Linux and macOS
keytool -genkey -v -keystore ~/key.jks -keyalg RSA -keysize 2048 \
    -validity 10000 -alias key

# For Windows
keytool -genkey -v -keystore c:\Users\USER_NAME\key.jks -storetype \
    JKS -keyalg RSA -keysize 2048 -validity 10000 -alias key
\end{lstlisting}

\noindent From "App signing preferences" in Google Play Console we have to download a public key with Play encrypt 
private key (PEPK) tool to generate and upload \q{.zip}-file:

\begin{lstlisting}[language=terminal]
$ java -jar pepk.jar --keystore=key.jks --alias=key \
    --output=google-play_output.zip --include-cert \
    --rsa-aes-encryption \
    --encryption-key-path=./google-play_public_key.pem 

Enter password for store 'key.jks':
Enter password for key 'key':
\end{lstlisting}

\noindent Then, it's a time to sign the bundle by adding keychain preferences (as a warning, \q{key.properties}-file 
should never be a part of the repository):

\begin{lstlisting}
// ./android/key.properties
key.password=/* key.jks password */
key.file=/* location of key.jks */
\end{lstlisting}

\noindent And adjust the android build-config of our application:\\

{
\xpretocmd{\lstlisting}{\vspace{-12pt}}{}{}
\begin{lstlisting}
// ./android/app/build.gradle
def localProperties = new Properties()
\end{lstlisting}
\begin{lstlisting}[firstnumber=3, backgroundcolor=\color{backgreen}]
(*@\kdiff{+}@*)
(*@\kdiff{+}@*)def store = rootProject.file('key.properties')
(*@\kdiff{+}@*)if (store.exists()) {
(*@\kdiff{+}@*)  store.withReader("UTF-8"){reader -> localProperties.load(reader)}
(*@\kdiff{+}@*)}
(*@\kdiff{+}@*)
\end{lstlisting}
\begin{lstlisting}[firstnumber=9]
// ... other code
\end{lstlisting}
\begin{lstlisting}[firstnumber=63, backgroundcolor=\color{backgreen}]
(*@\kdiff{+}@*)signingConfigs {
(*@\kdiff{+}@*) release {
(*@\kdiff{+}@*)  keyAlias 'key'
(*@\kdiff{+}@*)  keyPassword localProperties.getProperty('key.password')
(*@\kdiff{+}@*)  storeFile file(localProperties.getProperty('key.file'))
(*@\kdiff{+}@*)  storePassword localProperties.getProperty('key.password')
(*@\kdiff{+}@*) }
(*@\kdiff{+}@*)}
\end{lstlisting}
\begin{lstlisting}[firstnumber=71]
buildTypes {
  release {
\end{lstlisting}
\begin{lstlisting}[firstnumber=73, backgroundcolor=\color{backred}]
(*@\kdiff{-}@*)    signingConfig signingConfigs.debug
\end{lstlisting}
\begin{lstlisting}[firstnumber=73, backgroundcolor=\color{backgreen}]
(*@\kdiff{+}@*)    signingConfig signingConfigs.release
\end{lstlisting}
\begin{lstlisting}[firstnumber=74]
  }
}
\end{lstlisting}
}
