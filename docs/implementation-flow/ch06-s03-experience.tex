% Copyright 2023 The terCAD team. All rights reserved.
% Use of this content is governed by a CC BY-NC-ND 4.0 license that can be found in the LICENSE file.

\subsection{Asserting User Experience}
\markboth{Optimizing UI/UX Flow}{Asserting User Experience}

Users have a distinct expectation regarding how an application should behave in general (summarized "know-how"), and 
covering these expectations might be beneficial by asserting that applications is aligned with user needs. That approach
is mostly known as User-Centric Design (UCD) \cite{Stil16}. It entails understanding the needs, goals, and pain points 
of the target audience by conducting user research, creating user personas, and gathering feedback, designers can 
craft interfaces that resonate with users and fulfill their expectations.

\paragraph{Intuitive Navigation} Asserting user experience involves designing clear and logical navigation paths,
minimizing complexity, and ensuring that users can seamlessly move through the application's features and content.

As an example, it can be an additional actions' accessability on an element by swiping (\cref{u-swipe}). In our case, 
that helps to access \q{Edit}-form or \q{Delete} item without going through the multiple navigation steps. Such behavior 
can be achieved by the usage of \q{flutter\_swipe\_action\_cell}-component as a wrapper of our Widget (\issue{206}{}).

\img{uiux/swipe-actions}{Swipe Actions on Cell}{img:u-swipe}


\paragraph{Platform Conventions} Consistency in design elements, such as button placement, color schemes, and 
typography, helps users feel at ease and reduces cognitive load.


\paragraph{Users Feedback} Collecting user input, analyzing user behavior through analytics, and conducting usability 
testing all contribute to iterative improvements that align the application with evolving user expectations.


\paragraph{Personalization} Tailored content recommendations or customizable settings, enable applications to assert 
user experience by delivering a more individualized and engaging interaction.
