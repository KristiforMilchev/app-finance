% Copyright 2023 The terCAD team. All rights reserved.
% Use of this content is governed by a CC BY-NC-ND 4.0 license that can be found in the LICENSE file.

\subsection{Unifying Stylistic}
\markboth{Optimizing}{Unifying Stylistic}


\subsubsection{Branding Palette}

A Branding Palette, often referred to as a Color Palette, is a thoughtfully curated selection of colors that reflect the 
brand's personality, values, and objectives. It declares Primary (dominate the visual identity) and Secondary 
(complement the primary colors) Colors, Typography (fonts and typography styles), and Imagery (guidelines for images 
and icons selection or creation). Mostly, the palette is included into a Brand Book \cite{Geyr16}, also known as a 
Brand Style Guide or Brand Guidelines, that provides in-depth guidance on how a brand's visual identity should be 
applied consistently across all touchpoints. It serves as a reference manual for designers, marketers, and anyone 
involved in representing the brand.

In a context of Flutter applications, the palette is defined for \q{MaterialApp}-widget via \q{colorScheme} and 
\q{floatingActionButtonTheme} properties, and, instead of fully declared color scheme, we might take a default one
by overriding valuable for us colors: 

\begin{lstlisting}
// ./lib/main.dart
final palette = context.watch<AppPalette>().value;
final text = CustomTextTheme.textTheme(Theme.of(context));
return MaterialApp(
  theme: ThemeData(
    colorScheme: const ColorScheme.light().withCustom(),
    textTheme: text,
    floatingActionButtonTheme: const FloatingActionButtonThemeData().withCustom(Brightness.light),
    datePickerTheme: DatePickerTheme.of(context).withCustom(palette, text, Brightness.light),
    timePickerTheme: TimePickerTheme.of(context).withCustom(palette, text, Brightness.light),
    // ... other options
  ),
  darkTheme: ThemeData(/* same options with 'Brightness.dark' */),
\end{lstlisting}

\noindent Where \q{withCustom}-method is a part of our extension for \q{ColorScheme} and other types:

\begin{lstlisting}
// ./lib/_classes/herald/app_palette.dart
class AppPalette extends ValueNotifier<String> {
  // Get current palette from shared preferences
  static get state => AppPreferences.get(AppPreferences.prefColor) ?? AppColors.colorApp;
  // Keep its light mode (since can be configured by user)
  static get light => AppPreferences.get(AppPreferences.prefPalette) ?? AppDefaultColors().toString();
  // Keep its dark mode (since can be configured by user)
  static get dark => AppPreferences.get(AppPreferences.prefPaletteDark) ?? AppDarkColors().toString();

  AppPalette() : super(state);
  // Change mode type
  Future<void> setMode(String newValue) async {
    if (newValue != value) {
      value = newValue;
      await AppPreferences.set(AppPreferences.prefColor, value);
      notifyListeners();
    }
  }
  // Change color preferences
  Future<void> set(AppDefaultColors light, AppDefaultColors dark) async {
    await AppPreferences.set(AppPreferences.prefPalette, light.toString());
    await AppPreferences.set(AppPreferences.prefPaletteDark, dark.toString());
    notifyListeners();
  }(*@ \stopnumber @*)
}

// ./lib/_configs/custom_color_scheme.dart
class AppColors {
  late AppDefaultColors palette;
  // Choose palette from brightness condition
  AppColors(Brightness brightness) {
    if (brightness == Brightness.dark) {
      palette = AppDarkColors();
    } else {
      palette = AppDefaultColors();
    }
  }
}
// Light mode
class AppDefaultColors {
  Color get primary => const Color(0xff912391);
  // ... other options
}
// Dark mode
class AppDarkColors implements AppDefaultColors {
  // ... other options
}

extension CustomColorScheme on ColorScheme {
  ColorScheme withCustom() {
    final palette = AppColors(brightness).palette;
    return copyWith(
      primary: palette.primary,
      //... other options
    );
  }
}

extension CustomButtonTheme on FloatingActionButtonThemeData {
  FloatingActionButtonThemeData withCustom(Brightness brightness) {
    final palette = AppColors(brightness).palette;
    return copyWith(
      backgroundColor: palette.inversePrimary,
      foregroundColor: palette.onButton,
    );
  }
}
\end{lstlisting}


\subsubsection{Declaring Typography}

We may customize not only the palette but a text declaration, and even extend \q{TextTheme} by our own types:

\begin{lstlisting}
extension MyTextTheme on TextTheme {
  TextStyle get numberLarge => TextStyle(
    fontSize: 48,
    fontWeight: FontWeight.bold,
    // to align color set across all declarations
    color: headline1?.color,
  );
}
\end{lstlisting}

\noindent In addition, it might be used \q{copyWith}-method to update only a few properties (as a font size) while 
keeping others intact: 

\begin{lstlisting}
Theme.of(context).textTheme.copyWith(
  headline1: baseTheme.headline1!.copyWith(fontSize: 32),
  headline2: baseTheme.headline2!.copyWith(fontSize: 24),
);
\end{lstlisting}


\subsubsection{Setting Icons}

Setting application icons in Flutter for multiple platforms (Android, iOS, macOS, Windows, and Linux) involves 
preparing icon images in various sizes to cover different device resolutions and platform requirements:

\begin{itemize}
  \item Android: mipmap-hdpi, mipmap-mdpi, mipmap-xhdpi, mipmap-xxhdpi, mipmap-xxxhdpi
  \item iOS: AppIcon.appiconset
  \item macOS: Images.xcassets
  \item Windows: 16x16, 32x32, 48x48, 256x256
  \item Linux: 16x16, 32x32, 48x48, 256x256
\end{itemize}

\noindent So, we need to replace all the images, that's been generated by Flutter during a project creation. But there 
are a few additional steps needed for Android and Linux.

To set an icon with a background for \q{Android}:

\begin{lstlisting}[language=xml]
<!-- android/app/src/main/AndroidManifest.xml -->
<manifest xmlns:android="http://schemas.android.com/apk/res/android">
  <application
    android:icon="@mipmap/ic_launcher"
    android:roundIcon="@mipmap/ic_launcher_round"

<!-- android/app/src/main/res/mipmap-anydpi-v26/ic_launcher.xml -->
<!-- android/app/src/main/res/mipmap-anydpi-v26/ic_launcher_round.xml -->
<?xml version="1.0" encoding="utf-8"?>
<adaptive-icon xmlns:android="http://schemas.android.com/apk/res/android">
    <background android:drawable="@drawable/ic_launcher_background" />
    <foreground android:drawable="@drawable/ic_launcher_foreground" />
    <monochrome android:drawable="@drawable/ic_launcher_foreground" />
</adaptive-icon>

<!-- android/app/src/main/res/drawable/ic_launcher_background.xml -->
<vector
    xmlns:android="http://schemas.android.com/apk/res/android"
    android:width="108dp" android:height="108dp"
    android:viewportWidth="108" android:viewportHeight="108">
    <path android:fillColor="#912391" android:pathData="M0,0h108v108h-108z" />
</vector>

<!-- android/app/src/main/res/drawable-v24/ic_launcher_foreground.xml -->
<vector><!-- ... vector image --></vector>
\end{lstlisting}

\noindent And it has to be created \q{ic\_launcher.png} and \q{ic\_launcher\_round.png} icons for all other formats 
(mipmap-hdpi, mipmap-mdpi, ...) to support less than Android SDK 26 versions. Additionally, it can be changed 
background of loading screen:

\begin{lstlisting}[language=xml]
<!-- android/app/src/main/AndroidManifest.xml -->
<manifest xmlns:android="http://schemas.android.com/apk/res/android">
  <application
      android:background="@color/mainColor"

<!-- android/app/src/main/res/drawable/launch_background.xml -->
<!-- android/app/src/main/res/drawable-v21/launch_background.xml -->
<layer-list xmlns:android="http://schemas.android.com/apk/res/android">
    <item android:drawable="@color/mainColor" />

<!-- android/app/src/main/res/values/colors.xml -->
<!-- android/app/src/main/res/values-night/colors.xml -->
<resources>
    <color name="mainColor">#912391</color>
</resources>
\end{lstlisting}

\noindent For \q{Linux} the icons are declared from \q{my\_application.cc}-file before the line 
\q{gtk\_widget\_show(GTK\_WIDGET(window))}:

\begin{lstlisting}
// ./linux/my_application.cc
if (g_file_test("assets", G_FILE_TEST_IS_DIR)) {
  // For debug mode
  gtk_window_set_icon_from_file(window, "assets/icon.png", NULL); 
} else {
  // For release mode
  gtk_window_set_icon_from_file(window, "data/flutter_assets/assets/icon.png", NULL);
}
\end{lstlisting}
