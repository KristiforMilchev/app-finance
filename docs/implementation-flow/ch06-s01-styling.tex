% Copyright 2023 The terCAD team. All rights reserved.
% Use of this content is governed by a CC BY-NC-ND 4.0 license that can be found in the LICENSE file.

\subsection{Branding Palette}
\markboth{Optimizing UI/UX Flow}{Branding Palette}

A Branding Palette, often referred to as a Color Palette, is a thoughtfully curated selection of colors that reflect the 
brand's personality, values, and objectives. It declares Primary (dominate the visual identity) and Secondary 
(complement the primary colors) Colors, Typography (fonts and typography styles), and Imagery (guidelines for images 
and icons selection or creation). Mostly, the palette is included into a Brand Book \cite{Geyr16}, also known as a 
Brand Style Guide or Brand Guidelines, that provides in-depth guidance on how a brand's visual identity should be 
applied consistently across all touchpoints. It serves as a reference manual for designers, marketers, and anyone 
involved in representing the brand.

In a context of Flutter applications, the palette is defined for \q{MaterialApp}-widget via \q{colorScheme} and 
\q{floatingActionButtonTheme} properties, and, instead of fully declared color scheme, we might take a default one
by overriding valuable for us colors: 

\begin{lstlisting}
// ./lib/main.dart
final palette = context.watch<AppPalette>().value;
final text = CustomTextTheme.textTheme(Theme.of(context));
return MaterialApp(
  theme: ThemeData(
    colorScheme: const ColorScheme.light().withCustom(),
    textTheme: text,
    floatingActionButtonTheme: const FloatingActionButtonThemeData().withCustom(Brightness.light),
    datePickerTheme: DatePickerTheme.of(context).withCustom(palette, text, Brightness.light),
    timePickerTheme: TimePickerTheme.of(context).withCustom(palette, text, Brightness.light),
    // ... other options
  ),
  darkTheme: ThemeData(/* same options with 'Brightness.dark' */),
\end{lstlisting}

\noindent Where \q{withCustom}-method is a part of our extension for \q{ColorScheme} and other types:

\begin{lstlisting}
// ./lib/_classes/herald/app_palette.dart
class AppPalette extends ValueNotifier<String> {
  // Get current palette from shared preferences
  static get state => AppPreferences.get(AppPreferences.prefColor) ?? AppColors.colorApp;
  // Keep its light mode (since can be configured by user)
  static get light => AppPreferences.get(AppPreferences.prefPalette) ?? AppDefaultColors().toString();
  // Keep its dark mode (since can be configured by user)
  static get dark => AppPreferences.get(AppPreferences.prefPaletteDark) ?? AppDarkColors().toString();

  AppPalette() : super(state);
  // Change mode type
  Future<void> setMode(String newValue) async {
    if (newValue != value) {
      value = newValue;
      await AppPreferences.set(AppPreferences.prefColor, value);
      notifyListeners();
    }
  }
  // Change color preferences
  Future<void> set(AppDefaultColors light, AppDefaultColors dark) async {
    await AppPreferences.set(AppPreferences.prefPalette, light.toString());
    await AppPreferences.set(AppPreferences.prefPaletteDark, dark.toString());
    notifyListeners();
  }
}

// ./lib/_configs/custom_color_scheme.dart
class AppColors {
  late AppDefaultColors palette;
  // Choose palette from brightness condition
  AppColors(Brightness brightness) {
    if (brightness == Brightness.dark) {
      palette = AppDarkColors();
    } else {
      palette = AppDefaultColors();
    }
  }
}
// Light mode
class AppDefaultColors {
  Color get primary => const Color(0xff912391);
  // ... other options
}
// Dark mode
class AppDarkColors implements AppDefaultColors {
  // ... other options
}

extension CustomColorScheme on ColorScheme {
  ColorScheme withCustom() {
    final palette = AppColors(brightness).palette;
    return copyWith(
      primary: palette.primary,
      //... other options
    );
  }
}

extension CustomButtonTheme on FloatingActionButtonThemeData {
  FloatingActionButtonThemeData withCustom(Brightness brightness) {
    final palette = AppColors(brightness).palette;
    return copyWith(
      backgroundColor: palette.inversePrimary,
      foregroundColor: palette.onButton,
    );
  }
}
\end{lstlisting}
