% Copyright 2023 The terCAD team. All rights reserved.
% Use of this content is governed by a CC BY-NC-ND 4.0 license that can be found in the LICENSE file.

\subsection{Automating Quality Gates}
\markboth{Defining Quality Gates}{Automating Quality Gates}

Writing tests is not sufficient without a defined / automated continues integration. To achieve a robust quality check, 
we need to implement a plethora of automations that gradually alleviate the need for a manual intervention. Futhermore, 
Flutter provides a couple of commands to check the code itself:

\begin{itemize}
  \item flutter analyze -- static analysis;
  \item dart format -- apply rules from `analysis\_options.yaml`, described on the page:
  \href{https://dart.dev/effective-dart/style\#formatting}{https://dart.dev/effective-dart/style\#formatting};
  \item dart fix -- check points of an improvement and optimisation.
\end{itemize}


\subsubsection{Creating Git Hooks}

Someone might think that additional checks before `git commit` or `git push` will slowdown the development process, 
contrary that minimizes pushes back and forward to resolve failures on CI/CD build procedure. For a really big projects
that might take hours in contrast with seconds on a local environment. Certainly, the automation process must be 
carried out accurately and diligently; otherwise, it will become a constant source of frustration.

Implementation of git hooks is mostly a preparation of `bash`-scripts for execution, so, let's discuss here 
`pre-push`-file (since `pre-commit`-file contains almost the same logic for another scope of commands).

\begin{lstlisting}[language=bash]
#!/bin/bash
status=0
# Trigger tests
flutter test 
# Take an exit code from lastly executed command
status_test=$? 
# Verify command results
if [ $status_test -ne 0 ]; then
  echo "[x] flutter test - failed."
  status=1
else
  echo "[+] flutter test - passed."
fi
# Other than zero means a failure
exit $status
\end{lstlisting}

\noindent And we'll use Grinder (\ref{a-grinder}) to install our hooks:

\begin{lstlisting}
// ./tool/grind.dart
import 'dart:io';
import 'package:path/path.dart' as path;
import 'package:grinder/grinder.dart';

main(args) => grind(args);

@Task('Install Git Hooks')
installGitHooks() {
  final currDir = Directory('./');
  final hookDir = Directory('./.git/hooks');
  final hookNames = ['pre-commit', 'pre-push'];
  for (final name in hookNames) {
    log('Applying: $name');
    final sourceFile = File(path.join(currDir.absolute.path, name));
    sourceFile.copySync(path.join(hookDir.absolute.path, name));
  }
  log('Git Hooks applied!');
}
\end{lstlisting}

\noindent After that, by using a command line, we may check what's been done:

\begin{lstlisting}[language=bash]
> dart run grinder -h
Dart workflows, automated.

Usage: grinder [options] [<tasks>...]

Global options:
  --no-color           Whether to use terminal colors.
  --version            Reports the version of this tool.
  -h, --help           Print this usage information.

Available tasks:
  install-git-hooks    Install Git Hooks

> dart run grinder install-git-hooks
  grinder running install-git-hooks
  
  install-git-hooks
    Applying: pre-commit
    Applying: pre-push
    Git Hooks applied!
  
  finished in 0.0 seconds

> git add .
> git commit -m "Sample Commit"
Computing fixes in app-finance (dry run)...
Nothing to fix!
Formatted 56 files (0 changed) in 0.42 seconds.

[+] dart fix - passed.
[+] dart format - passed.
Sample Commit
 3 files changed, 65 insertions(+), 90 deletions(-)

> git push
# ... logs from tests
[+] flutter test - passed.
# ... other logs from git
\end{lstlisting}


\subsubsection{Adding Grinder Tasks} \label{a-grinder}

Grinder is a task runner for Dart, helping to define and automate common project workflows, as said on its main page
\href{https://pub.dev/packages/grinder}{https://pub.dev/packages/grinder}. 
It can be added to the project by `flutter pub add grinder --dev`. Then, `dart run grinder:init`-command would help to
generate a skeleton of the code structure (if missing).

As one of the project workflow automation it can be mentioned a localization task 
(\href{https://docs.flutter.dev/accessibility-and-localization/internationalization}{https://docs.flutter.dev/accessibility-and-localization/internationalization}). 
Just to recap, in Flutter it's used Application Resource Bundle(s) (`.arb`-files) for internationalizing apps. 
So, not to create a mess in those files, it's better to have an automation to rearrange our labels alphabetically
(in `/lib/l10n/app\_*.arb`-files). Lately, the automation can be also extended by checking consistency (enter missing 
labels across the `.arb`-files), exporting labels to `.scv`-file for a simplification of a translation process, and 
importing them from `.scv`-file back to `.arb`-files.

\begin{lstlisting}
// ./tool/grind.dart
import './localization.dart' as locale;

@Task('Update Translations by sorting values alphabetically')
sortTranslations() {
  // Get all additional arguments for the command
  TaskArgs args = context.invocation.arguments;
  // Run: dart run grinder sort-translations --quiet
  bool isQuiet = args.getFlag('quiet'); // Returns `true` if set
  bool isChanged = locale.sortArbKeys('./lib/l10n');
  if (isChanged && !isQuiet) {
    fail('Changes detected'); // (!) to handle failure
  }
}
\end{lstlisting}

\begin{lstlisting}
// ./tool/localization.dart
import 'dart:convert';
import 'dart:io';
import 'package:grinder/grinder.dart';

bool sortArbKeys(String path) {
  log('Checking $path'); // Add to console output the text
  final arbDir = Directory(path);
  bool isChanged = false;
  // Check, that folder is not missing
  if (!arbDir.existsSync()) {
    log('Error: Directory not found');
    return true;
  }
  // Loop per each file there
  for (var file in arbDir.listSync()) {
    // Take only .arb-files
    if (file is File && file.path.endsWith('.arb')) {
      log('- ${file.path}');
      // `|=` is a boolean OR assignment
      isChanged |= sortArbFileKeys(file);
    }
  }
  log(isChanged ? 'Labels reordered' : 'Nothing was changed');
  return isChanged;
}

bool sortArbFileKeys(File file) {
  // Load data from file
  final jsonContent = file.readAsStringSync();
  final arbMap = json.decode(jsonContent) as Map<String, dynamic>;
  // Sort labels
  final entries = arbMap.entries.toList();
  entries.sort((a, b) {
    // `@key` should go after `key`
    final aKey = a.key.startsWith('@') ? a.key.substring(1) : a.key;
    final bKey = b.key.startsWith('@') ? b.key.substring(1) : b.key;
    if (aKey == bKey) {
      if (a.key.startsWith('@')) {
        return 1;
      }
      if (b.key.startsWith('@')) {
        return -1;
      }
    }
    return aKey.compareTo(bKey);
  });
  final sortedArbMap = Map.fromEntries(entries);
  // Write back to the file with preserved indentation
  const encoder = JsonEncoder.withIndent('    ');
  var jsonOutputContent = encoder.convert(sortedArbMap);
  file.writeAsStringSync(jsonOutputContent);
  return jsonOutputContent != jsonContent; // Check if changed
}
\end{lstlisting}

\newpage
\subsubsection{Running GitHub Workflows}

Since we've done with our automation for the local environment, let's go further and define quality gates on GitHub 
repository.

\begin{lstlisting}[language=yaml]
## ./.github/workflows/dart.yml
# Name would be shown in Actions-tab
name: Flutter/Dart Quality Gates 
# Trigger options
on:
  push: # Trigger after a merge
    branches: [ "main" ] # limit to main-branch
  pull_request: # Trigger on pull-request
    branches: [ "main" ] # if the target is main-branch

jobs:
  build:
    runs-on: ubuntu-latest # Environment for execution

    steps:
      - uses: actions/checkout@v3 # Checkout from repository
      - uses: subosito/flutter-action@v2 # Install Flutter/Dart
        with:
          channel: 'stable'
      - run: flutter --version # Show version in logs

      - name: Install Dependencies
        run: flutter pub get

      - name: Verify Formatting
        run: dart format --output=none --set-exit-if-changed .

      - name: Check Localizations ordering
        run: dart run grinder sort-translations

# Disabled by having too many failures by now )) 
#      # Consider passing '--fatal-infos' to be strict
#      - name: Analyze Project Source
#        run: flutter analyze

      - name: Run tests
        run: flutter test
\end{lstlisting}

\noindent By preparing a pull-request we would see next representative validations flow 
(\cref{img:pt-github}, \cref{img:pt-github-details})):

\img{prototyping/github-check}{GitHub Toolbar on pull-request}{img:pt-github}

\img{prototyping/github-check-details}{GitHub Workflow Actions details}{img:pt-github-details}


\subsubsection{Commending with Badges} \label{a-badges}

One of the ways to enhance the current state is by acknowledging and appreciating our efforts through Badges of Workflow 
Status and Code Coverage. These badges serve as a testament to our commitment to quality and efficiency, motivating us 
to continually improve and achieve even greater heights.

\begin{lstlisting}[language=bash]
## ./README.md
# Notation of an image with tooltip "Build Status" 
![Build Status](https://github.com/{user}/{repo}/actions/workflows/{workflow-name}.yml/badge.svg?branch=main)
# Notation of an image with a link to our repository
![Tests Coverage](https://{user}.github.io/{repo}/coverage_badge.svg)](https://github.com/{user}/{repo})
\end{lstlisting}

`Build Status` is supported out of the box (OOTB) by GitHub, just it's needed to replace `\{user\}` (by account name
on GitHub), `\{repo\}` (repository name), and `\{workflow-name\}` (name of the file in `.github/workflows`-folder).

`Test Coverage` is needed to be cooked (there are many ways to do that and none of them is universal), and we would 
start from creating `gh-pages` in our repository (drop hooks from `.git/hooks`-folder if they've been enabled):

\begin{lstlisting}[language=bash]
> git switch --orphan gh-pages
> git commit --allow-empty -m "Initial commit"
> git push -u origin gh-pages
> git checkout main
\end{lstlisting}

\noindent That will create for us a special branch that is accessible via \`https://\{user\}.github.io/\{repo\}\`-link. 
Then we need to extend our GitHub Workflow by adding a section with uploading artifacts:

\begin{lstlisting}[language=yaml]
## ./.github/workflows/dart.yml
# Updating previously created step by a conditional execution
- name: Run tests
  run: |
    if [[ "${{ github.ref }}" == "refs/heads/main" ]]; then
        flutter test --coverage
        dart run grinder coverage-badge # New Grinder task to generate `.svg'-file with coverage
    else
        flutter test
    fi
# New section to upload artifacts
- name: Update Coverage Badge
  # Get default branch variable and compare with current
  if: github.ref == format('refs/heads/{0}', github.event.repository.default_branch)
  uses: peaceiris/actions-gh-pages@v3 # Special action to upload artifacts
  with:
    github_token: ${{ secrets.GITHUB_TOKEN }} # Generated automatically by GitHub
    publish_dir: ./coverage # Folder to upload
\end{lstlisting}

\img{prototyping/badges}{Generated README.md for the repository}{img:pt-badge}

Everything is done (\cref{img:pt-badge})... almost. Coverage 46\% is unbelievable for written tests. The problem is 
that untouched (by tests) files are not included into the report. And, additionally, widget tests cover most of the 
files without checking them. So, let's separate type of tests' execution, and provide additional Grinder task to 
touch all our files for the coverage report.

\begin{lstlisting}[language=yaml]
## ./.github/workflows/dart.yml
- name: Run tests
  run: |
    if [[ "${{ github.ref }}" == "refs/heads/main" ]]; then
      dart run grinder full-coverage
      flutter test --coverage test/unit
      flutter test test/widget
      dart run grinder coverage-badge
    else
      flutter test
    fi
\end{lstlisting}

\begin{lstlisting}
// ./tool/coverage.dart
import 'dart:io';
void scanDirectory(Directory directory, List<String> files) {
  directory.listSync(recursive: true).forEach((entity) {
    if (entity is File && // Check that it's a file
        entity.path.endsWith('.dart') && // Include only Dart
        !entity.path.endsWith('.g.dart')) { // Code generation
      files.add(entity.absolute.path
        .replaceAll(root.absolute.path, '') // Cut out root prefix
        .replaceAll('\\', '/')); // Fix for Windows' systems
    } else if (entity is Directory) {
      scanDirectory(entity, files); // Recursively scan further
    }
  });
}
\end{lstlisting}

\begin{lstlisting}
// ./tool/grind.dart
import './coverage.dart' as coverage;
@Task('Generate file with all lib/**.dart-files included')
fullCoverage() {
  List<String> files = [];
  String content = "// AUTOGENERATED BY `dart run grinder full-coverage` \n";
  final rootFolder = Directory('${Directory.current.path}/lib');
  coverage.scanDirectory(rootFolder, rootFolder, files);
  for (var file in files) {
    content += "import 'package:app_finance$file';\n";
  }
  content += "void main() {}\n";
  File(path.join(Directory.current.path, 'test/_coverage.dart')).writeAsStringSync(content);
}
\end{lstlisting}

\begin{lstlisting}[language=bash]
## .gitignore
# Coverage report
coverage/
test/_coverage.dart
\end{lstlisting}

\noindent 10\% Coverage... here we are. Slowly and progressively, we will correct this assessment but not immediately.
Do remember that we're on an early stage and nothing is neither stable, nor finalized yet; but tests are a significant
investment into the whole application quality and to speed-up the implementation cycles (by skipping a manual 
verification each time for covered components).
