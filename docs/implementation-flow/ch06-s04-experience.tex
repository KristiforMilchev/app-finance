% Copyright 2023 The terCAD team. All rights reserved.
% Use of this content is governed by a CC BY-NC-ND 4.0 license that can be found in the LICENSE file.

\subsection{[TBD] Asserting User Experience}
\markboth{Optimizing UI/UX Flow}{Asserting User Experience}

Users have a distinct expectation regarding how an application should behave in general (summarized "know-how"), and 
covering these expectations might be beneficial by asserting that applications is aligned with user needs. That approach
is mostly known as User-Centric Design (UCD) \cite{Stil16}. It entails understanding the needs, goals, and pain points 
of the target audience by conducting user research, creating user personas, and gathering feedback, designers can 
craft interfaces that resonate with users and fulfill their expectations.


\subsubsection{Building Navigation} 

Asserting user experience involves designing clear and logical navigation paths,
minimizing complexity, and ensuring that users can seamlessly move through the application's features and content.

As an example, it can be an additional actions' accessability on an element by swiping (\cref{img:u-swipe}). In our case, 
that helps to access \q{Edit}-form or \q{Delete} item without going through the multiple navigation steps. Such behavior 
can be achieved by the usage of \q{flutter\_swipe\_action\_cell}-component as a wrapper of our Widget (\issue{206}{}).

\img{uiux/swipe-actions}{Swipe Actions on Cell}{img:u-swipe}


\subsubsection{Following Conventions} 

Consistency in design elements, such as button placement, color schemes, and typography, helps users feel at ease and 
reduces cognitive load.


\subsubsection{Adding Localizations} 

Adapting to various languages and regions makes the application more accessible and inclusive to a global audience. 
That includes translating text, adjusting layouts for different languages, and incorporating region-specific content 
or functionalities.

Localization can be enabled from a configurational \q{pubspec.yaml}-file with an additional declaration in 
\q{l10n.yaml}-file: 

\begin{lstlisting}[language=yaml]
# ./pubspec.yaml
dependencies:
  flutter_localizations:
    sdk: flutter

# ./l10n.yaml
arb-dir: lib/l10n
template-arb-file: app_en.arb
output-localization-file: app_localization.dart
preferred-supported-locales:
  - en
  - be
\end{lstlisting}

\noindent \q{MaterialApp}-widget should be extended by \q{locale}-properties \issue{7}{}:

\begin{lstlisting}
// Autogenerated package from '.arb'-files
import 'package:flutter_gen/gen_l10n/app_localization.dart';
// Main application builder
Widget build(BuildContext context) => MaterialApp(
  localizationsDelegates: AppLocalizations.localizationsDelegates,
  // List is generated based on the '.arb'-files availability
  supportedLocales: AppLocalizations.supportedLocales,
  // 'AppLocale' extends from 'ValueNotifier'
  locale: context.watch<AppLocale>().value,
  // ... other options
);
\end{lstlisting}

\noindent Rich text can be stored in \q{assets}-folder as \q{.md}-files and integrated into application by using 
\q{flutter\_markdown}-package:

\begin{lstlisting}
FutureBuilder(
  future: DefaultAssetBundle.of(context).loadString('./assets/l10n/file.md'),
  builder: (ContextBuilder context, AsyncSnapshot<String> snapshot) {
    if (snapshot.hasData) {
      return Markdown(data: snapshot.data!);
    }
    return Container();
  },
)
\end{lstlisting}

\noindent Localization encompasses more than just translating text; it involves adapting content and design for 
different regions and cultures. For instance, consider adjusting the layout by rearranging elements from left-to-right
to right-to-left (as a reposition of a left-hand navigation bar to the right) for languages like Arabic, Hebrew, and 
Japanese. Additionally, color choices can vary, with colder colors preferred in Europe and warmer colors in Latin 
America. These details enhance the user experience for diverse audiences \cite{Rein14}.


\subsubsection{Personalizing View}

Tailored content recommendations or customizable settings, enable applications to assert user experience by delivering 
a more individualized and engaging interaction.


\subsubsection{Looping Feedback} 

Collecting user input, analyzing user behavior through analytics, and conducting usability testing all contribute to 
iterative improvements that align the application with evolving user expectations.
