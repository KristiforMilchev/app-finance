% Copyright 2023 The terCAD team. All rights reserved.
% Use of this content is governed by a CC BY-NC-ND 4.0 license that can be found in the LICENSE file.

\subsection{Benchmarking Prototype}
\markboth{Unleashing Features}{Benchmarking Prototype}

Before adding functionality in the form of muscles to the created prototype skeleton, we need to verify its reliability.
Restructuring the fundamental concepts of the application in the future would not only pose a considerable challenge 
but also entail a substantial effort and potential complications.


\subsubsection{Providing Integration Tests}

Unit tests (\ref{ut-unit}) and widget tests (\ref{widget-tests}) serve as valuable tools for assessing isolated classes, 
functions, or widgets. However, not all of the problems can be tackled by them. Integration tests are used to identify 
systemic flaws (data corruption, concurrency problems, miscommunication between services, etc.) that might not be 
evident in unit tests by verifying a synergy of individual assets, validating the application as a whole. 
Integration tests are designed to reflect the real-time performance of an application on an actual device or platform. 
In conclusion, they provide a vital link in the testing hierarchy by validating a collocation of various components 
within an application. In such a way integration tests simulate end-to-end user workflows that we've implemented and 
discussed earlier -- \ref{t-gherkin}.

Integration tests in Flutter can be written by using \q{integration\_test}-package, \q{flutter\_driver}-package would 
help us to evaluate our tests on real / virtual devices and environments and track the timeline of tests execution 
(both packages are provided by the SDK):

\begin{lstlisting}[language=yaml]
## ./pubspec.yaml
dev_dependencies:
  integration_test:
    sdk: flutter
  flutter_driver: 
    sdk: flutter
\end{lstlisting}

\noindent The implementation's deference from a widget test is in a usage of the next code line, that enables tests 
execution on a physical device or platform:
\begin{lstlisting}
IntegrationTestWidgetsFlutterBinding.ensureInitialized();
\end{lstlisting}


\subsubsection{Doing Performance Testing}

Performance testing is a type of software testing designed to evaluate the speed, responsiveness, stability, and 
overall performance of an application under different conditions. It involves subjecting the application to 
simulated workloads and stress scenarios to assess how it behaves in terms of speed, scalability, and resource usage. 
Performance testing ensures that the software can handle the expected load without degradation in performance.

By simulating different levels of user traffic, performance testing helps determine the application's scalability by
assessing resources utilization (CPU, memory, network bandwidth, and other parameters), and identify performance 
bottlenecks, such as slow database queries, inefficient code, or network latency, and address these issues before 
they will impact users.

The detailed information about performance testing can be taken from the International Software Testing Qualifications 
Board (ISTQB) or the Software Engineering Institute (SEI), while here we'll highlight only their types definition 
(\cite{Ian15}, \cite{Sag16}, \cite{Sag23}):
\begin{itemize}
  \item Load Testing: Evaluates how an application performs under expected load conditions. It helps determine the 
  application's response time, resource utilization, and overall stability.

  \item Stress Testing: Pushes the application to its limits by subjecting it to extreme conditions, such as excessive 
  user loads or resource scarcity. It aims to identify the breaking point and understand how the application recovers 
  from failures.

  \item Endurance Testing: Assesses the application's performance over an extended period to identify issues related to 
  memory leaks, resource exhaustion, or gradual degradation in performance.

  \item Spike Testing: Simulates sudden spikes in user traffic to assess how the application responds to rapid changes
  in load. This helps uncover bottlenecks and issues related to sudden surges in demand.

  \item Volume Testing: Focuses on testing the application's performance with large volumes of data, such as a high 
  number of records in a database. It helps identify scalability and performance issues associated with data volume.
\end{itemize}

\noindent Back to our process, it would be used the next command to evaluate performance tests:

\begin{lstlisting}[language=bash]
# Precondition for Web profiling
chromedriver --port=4444
# Launch tests
flutter drive \
  --driver=test_driver/perf_driver.dart \
  --target=test/performance/name_of_test.dart \
  --profile
\end{lstlisting}

The \q{--profile}-option enables the application compilation in "profile mode" that helps the benchmark results to be
closer to what will be experienced by end users. By running on a mobile device or emulator it's proposed to use 
\q{--no-dds}-parameter in addition, that will disable unaccessible Dart Development Service (DDS). The \q{--target} 
declares the scope of test executions while \q{--driver}-option does track the outcomes. The driver configuration can be
taken from \href{https://docs.flutter.dev/cookbook/testing/integration/profiling}{https://docs.flutter.dev/cookbook/testing/integration/profiling}:

\begin{lstlisting}
// ./test_driver/perf_driver.dart
import 'package:flutter_driver/flutter_driver.dart' as driver;
import 'package:integration_test/integration_test_driver.dart';

Future<void> main() {
  return integrationDriver(
    responseDataCallback: (data) async {
      if (data != null) {
        final timeline = driver.Timeline.fromJson(data['timeline']);
        final summary = driver.TimelineSummary.summarize(timeline);
        await summary.writeTimelineToFile(
          'timeline',
          pretty: true,
          includeSummary: true,
          destinationDirectory: './coverage/',
        );
      }
    },
  );
}
\end{lstlisting}

\noindent Since it's a Widget Tests'-based approach (\ref{widget-tests}, \ref{t-gherkin}), we'll accent only on the 
usage of \q{traceAction}-method to store time-based metrics:

\begin{lstlisting}
// ./test/performance/load/creation_test.dart
void main() {
  final binding = IntegrationTestWidgetsFlutterBinding.ensureInitialized();
  testWidgets('Cover Starting Page', (WidgetTester tester) async {
    await binding.traceAction(() async {
        // ... other steps
        final amountField = find.byWidgetPredicate((widget) {
          return widget is TextField && widget.decoration?.hintText == 'Set Balance';
        });
        await tester.ensureVisible(amountField);
        await tester.tap(amountField);
        // In profiling mode some delay is needed:
        await tester.pumpAndSettle(const Duration(seconds: 1));
        // await tester.pump();
        await tester.enterText(amountField, '1000');
        await tester.pumpAndSettle();
        expect(find.text('1000'), findsOneWidget);
        // ... other steps
      },
      reportKey: 'timeline',
    );
  });
}
\end{lstlisting}

\noindent Generated file \q{timeline.timeline.json} can be traced by \q{chrome://tracing/} in Google Chrome browser 
(\cref{img:perf-chrome-tracing}):

\img{features/perf-chrome-tracing}{Google Chrome -- performance trace}{img:perf-chrome-tracing}

\noindent The \q{timeline.timeline\_summary.json}-file can be opened in IDE as a native \q{JSON}-file and analyzed 
manually a performance of the application. For example, the value of \q{average\_frame\_build\_time\_millis}-parameter 
is recommended to be below 16 milliseconds to ensure that the app runs at 60 frames per second without glitches. Other 
parameters are widely described on the page --
\href{https://api.flutter.dev/flutter/flutter\_driver/TimelineSummary/summaryJson.html}{https://api.flutter.dev/flutter/flutter\_driver/TimelineSummary}.


\paragraph{Load Testing}
Check response time and resource utilization for the first run (Initial Setup) by creating account and budget 
category:

\begin{lstlisting}[language=cucumber]
@start
Feature: Verify Initial Flow
  Scenario: Applying basic configuration through the start pages
    Given I am firstly opened the app
    Then I can see "Initial Setup" component
    When I tap "Save to Storage (Go Next)" button
    Then I can see "Acknowledge (Go Next)" component
    When I tap "Acknowledge (Go Next)" button
    Then I can see "Create new Account" component
    When I tap on "first" of "ListSelector" field
    And I tap "Bank Account" element
    And I enter "New Account" to "Enter Account Identifier" text field
    And I enter "1000" to "Set Balance" text field
    And I tap "Create new Account" button
    Then I can see "Create new Budget Category" component
    When I enter "New Budget" to "Enter Budget Category Name" text field
    And I enter "1000" to "Set Balance" text field
    When I tap "Create new Budget Category" button
    Then I can see "Accounts, total" component
\end{lstlisting}

\noindent And, what we've identified from our first tests execution is a degraded \q{frame build}-parameter 
(\cref{tb:frame-build}) that affects our frames per second (FPS) by generating only 37 frames instead of 60:\\

\begin{table}[h!]
  \begin{tabular}{ |p{6.8cm}||r|r|r|  }
    \hline
    \multicolumn{4}{|c|}{Frame Build Time, in milliseconds} \\
    \hline
    Type of state & Cold Start & Retrial & With Data\\
    \hline
    average          &  26.00 &  24.28 &  29.65 \\
    90th percentile  &  47.20 &  43.38 &  70.33 \\
    99th percentile  & 158.31 & 159.41 & 198.03 \\
    \hline
  \end{tabular}
  \caption{Performance Test Results for Feature "Verify Initial Flow"} \label{tb:frame-build}
\end{table}

\img{features/perf-slow-frame}{Performance Monitor in Visual Studio Code}{img:perf-slow-frame}

\noindent This issue (\cref{img:perf-slow-frame}) pertains to a compilation jank in animations due to shaders 
calculation (a code snippets executed on a graphics processing unit [GPU] to render a sequence of draw commands). 
Their pre-compilation strategy mitigates the compilation-related disruptions during subsequent animations, and improves 
frames per second rendering. To run the app with \q{--cache-sksl} turned on to capture shaders in SkSL:

\begin{lstlisting}[language=bash]
flutter run --profile --cache-sksl --purge-persistent-cache
\end{lstlisting}

\noindent Warm-up shaders in Skia Shader Language (SkSL) format for an application build:

\begin{lstlisting}[language=bash]
# Capture shaders in Skia Shader Language (SkSL) format into a file
flutter drive --profile --cache-sksl --write-sksl-on-exit sksl.json -t test_driver/warm_up.dart
# Build app with SkSL warm-up
flutter build ios --bundle-sksl-path sksl.json
\end{lstlisting}

\begin{lstlisting}
// ./test_driver/warm_up.dart
import 'package:integration_test/integration_test_driver.dart';
Future<void> main() {
  return integrationDriver();(*@ \stopnumber @*)
}

// ./test_driver/warm_up_test.dart
Future<void> main() async {
  IntegrationTestWidgetsFlutterBinding.ensureInitialized();
  SharedPreferencesMixin.pref = await SharedPreferences.getInstance();

  testWidgets('Warm-up', (WidgetTester tester) async {
    await tester.pumpWidget(MultiProvider(
      providers: [
        ChangeNotifierProvider<AppData>(
          create: (_) => AppData(),
        ),
        ChangeNotifierProvider<AppTheme>(
          create: (_) => AppTheme(ThemeMode.system),
        ),
      ],
      child: const MyApp(),
    ));
    await tester.pumpAndSettle(const Duration(seconds: 3));
  });
}
\end{lstlisting}

\noindent Finally, we've taken \q{56 FPS (average)} as an outcome from that tunning.


\paragraph{Stress Testing}
Check initial load (a time before the enabled interaction) with a huge transaction log history (32Mb, 128Mb, 
512Mb, 2Gb).


\paragraph{Endurance Testing}
Check response time and resource utilization by adding different types of data within a different time 
periods (15 minutes, an hour, 4 hours, 16 hours).


\paragraph{Spike Testing}
Postponed till the enabled synchronization between different devices.


\paragraph{Volume Testing}
Combine reporting of "Load Testing" with data from "Stress Testing".
