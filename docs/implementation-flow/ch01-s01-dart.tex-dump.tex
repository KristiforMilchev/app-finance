\subsection{[TBD] Dart}

Question: Flutter check that variable is a function

Answer: In Flutter, you can use the is operator to check if a variable is a function. 

\begin{lstlisting}
void main() {
  var variable = () {
    // Function body
  };

  if (variable is Function) {
    print('Variable is a function');
  } else {
    print('Variable is not a function');
  }
}
\end{lstlisting}

In this example, the is operator is used to check if the variable is of type Function. If it is, the condition 
variable is Function will evaluate to true, indicating that the variable is a function. Otherwise, it will 
evaluate to false, indicating that the variable is not a function.


Question: flutter default value in constructor

Answer: In Flutter, you can assign default values to constructor parameters using the = operator. 

\begin{lstlisting}
class MyClass {
  final int myValue;

  MyClass({this.myValue = 10});
}
\end{lstlisting}

In the example above, the MyClass constructor has a parameter myValue with a default value of 10. 
If no value is provided for myValue when creating an instance of MyClass, it will default to 10.

You can then create an instance of MyClass with or without providing a value for myValue:

\begin{lstlisting}
// Using the default value
MyClass instance1 = MyClass();
print(instance1.myValue); // Output: 10

// Providing a custom value
MyClass instance2 = MyClass(myValue: 20);
print(instance2.myValue); // Output: 20    
\end{lstlisting}

In the first case, where no value is provided for myValue, it takes on the default value of 10. In the second case, 
a custom value of 20 is provided, which overrides the default value.


Question: Operating with Dates
As an example, first business logic
if endDate <= currentDate then return 1
if startDate = currentDate then return 0
else return related float

\begin{lstlisting}
double calculateValue(DateTime startDate, DateTime endDate, DateTime currentDate) {
  if (endDate.isBefore(currentDate) || endDate.isAtSameMomentAs(currentDate)) {
    return 1.0;
  } else if (startDate.isAtSameMomentAs(currentDate)) {
    return 0.0;
  } else {
    double totalDays = endDate.difference(startDate).inDays.toDouble();
    double currentDays = currentDate.difference(startDate).inDays.toDouble();
    return currentDays / totalDays;
  }
}
\end{lstlisting}

In this example, the calculateValue function takes three DateTime parameters: startDate, endDate, and currentDate.

If endDate is less than or equal to currentDate, it means the end date has passed or is the same as 
the current date, so we return 1.0.
If startDate is the same as currentDate, we return 0.0.
Otherwise, we calculate the float value based on the ratio of the number of days between startDate and 
currentDate to the total number of days between startDate and endDate.

To get the current date in Flutter, you can use the DateTime class along with the DateTime.now() method. 