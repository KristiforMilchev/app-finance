% Copyright 2023 The terCAD team. All rights reserved.
% Use of this content is governed by a CC BY-NC-ND 4.0 license that can be found in the LICENSE file.

\subsection{Aggregating External Sources}
\markboth{Unleashing Features}{Aggregating External Sources}

In the realm of any application, the concept of "Seamless Migration" stands as a pivotal attribute that can 
significantly elevate user experience. And it's imperative to provide convenient options to seamlessly import their 
data \emph{(a broad spectrum of financial information, ranging from transaction histories to detailed account 
statements)} from various external sources (actually, existing market competitors).

The crux of this functionality lies in the establishment of integration channels with renowned financial institutions. 
By enabling users to link their bank accounts, credit cards, or other financial platforms, our application can 
orchestrate an automated and systematic import of transactional data. The architecture of this integration hinges 
on the utilization of APIs and secure connection protocols that guarantee the confidentiality of sensitive financial 
information.

The implications of such seamless data import is appreciated in an era characterized by the urgency of time and the 
demand for instantaneous solutions. And it reinforces the notion that our application isn't just another financial 
tool, but a resource that is acutely attuned to users' needs, providing practical solutions that simplify their 
financial management endeavors. So, it's a strategic initiative that encapsulates the essence of user-centric design. 


\subsubsection{Importing Comma-Separated Values (CSV)}

A CSV (comma-separated values) file is a text-based file format distinguished by its specific structure that facilitates 
the storage of data in a tabular format. This format is particularly suited for representing data in rows and columns, 
lending itself to various applications, ranging from simple data storage to complex data analysis.

And that format is the simples in its usage to export from external resources financial data, and import it into our 
application by using \q{csv}-package \issue{35}{a425bbd}:

\begin{lstlisting}
String content = await importFile(ext);
// splitter - '\n' or '\r\n'
final result = CsvToListConverter(eol: splitter).convert(content);
\end{lstlisting}


\subsubsection{Parsing Quicken Data (QIF)}

QIF files stand as plain text documents that adhere to a structure of "tag-value" pairs. The architecture is 
characterized by a distinct format wherein each line commences with a solitary character "tag" that is promptly 
succeeded by its corresponding "value," which persists until the culmination of that line. This format provides a 
straightforward and human-readable way to encapsulate financial data within a text-based paradigm, facilitating 
easier comprehension and interpretation
(\href{https://en.wikipedia.org/wiki/Quicken_Interchange_Format}{https://en.wikipedia.org/wiki/Quicken\_Interchange\_Format}):

\begin{lstlisting}[language=terminal]
!Type:Bank # "!" - a section of records
D7/02/84   # "D" - date
T-500      # "T" - total amount of the transaction
N1234      # "N" - transaction identifier
C*         # "C": "*" - reconciled, "X" - cleared
M          # "M" - transaction memo
PJohn S.   # "P" - payee
L[Visa]/   # "L" - category line
^          # "^" - end of record
\end{lstlisting}

\noindent Armed with these foundational details, it becomes feasible to construct a parser for QIF files 
\issue{189}{133ee9f}:

\begin{lstlisting}
FileScope _parseQif(String fileContent, [String splitter = '\n']) {
  FileScope result = [];
  final scope = fileContent.split(splitter);
  int idx = 1;
  Map<String, int> mapping = {
    'N': 0, 'T': 1, 'P': 2, 'L': 3, 'D': 4
  };
  for (int i = 0; i < scope.length; i++) {
    if (scope[i].isEmpty) {
      continue;
    }
    final key = scope[i].substring(0, 1);
    final value = scope[i].substring(1);
    if (key == '^') {
      idx++;
      result.add(List<dynamic>.filled(header.length, null));
      continue;
    }
    int? pos = mapping[key];
    if (pos != null) {
      result[idx][pos] = value;
    }
  }
  return result;
}
\end{lstlisting}


\subsubsection{Fulfilling Financial Exchange Data (OFX)}

Open Financial Exchange (OFX) files serve as the conduit through which financial data is seamlessly shared between 
software applications and financial institutions. This dynamic format plays a pivotal role in streamlining the complex 
and intricate interactions that underpin modern financial management. The scope of this exchange varies from intricate 
transaction details to comprehensive account information, even extending to encompass bill payments.

One of the defining attributes of the OFX file format is its compatibility across a spectrum of software applications. 
This format engenders a unified language that bridges the diverse software platforms utilized by both individuals and 
financial institutions. Through this universal language, financial data flows seamlessly and amplifies the efficiency 
of financial interactions across the digital domain. The data stored inside the OFX files is based on the Standard 
Generalized Markup Language (SGML; ISO 8879:1986) standard:

\begin{lstlisting}[language=xml]
<?xml version="1.0" encoding="UTF-8" standalone="no"?>
<OFX>
<BANKMSGSRSV1>
<STMTTRNRS>
  <STMTRS>
    <BANKTRANLIST>
      <STMTTRN>
        <TRNTYPE>DEBIT</TRNTYPE>
        <DTPOSTED>***4332</DTPOSTED>
        <TRNAMT>-100.00</TRNAMT>
        <FITID>1234232</FITID>
        <NAME>Market Store</NAME>
      </STMTTRN>
\end{lstlisting}

\noindent That's why we may use \q{xml}-package to parse content and import all needed data \issue{189}{7bd7eea}:

\begin{lstlisting}
// @var String content
XmlDocument.parse(content);
\end{lstlisting}


\subsubsection{[TBD] Supporting Banking Protocols}

Financial Transaction Services (FinTS) protocol, previously recognized as HBCI (Home Banking Computer Interface), stands 
as a testament to technological advancement in the realm of digital finance. FinTS emerges as a bank-independent 
interface, meticulously crafted to cater to the distinct needs of German banking institutions and their clientele;
enabling them to seamlessly engage in statement downloads, initiate bank transfers, and facilitate direct debits.

[TBD]

\subsubsection{[TBD] Parsing Notifications (Android)}

[TBD]