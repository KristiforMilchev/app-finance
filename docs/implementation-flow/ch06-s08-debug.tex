% Copyright 2023 The terCAD team. All rights reserved.
% Use of this content is governed by a CC BY-NC-ND 4.0 license that can be found in the LICENSE file.

\subsection{Debugging}
\markboth{Optimizing}{Debugging}

Any application may encounter bugs and issues during the development process that might be hardly understandable 
(\cref{img:o-error}, \issue{421}{}).

\img{features/crash-stacktrace}{Firebase Chrashlytics: Failure report}{img:o-error}

\noindent One of the simplest yet effective methods for debugging in Flutter is logging. By strategically placing logger 
statements in our code, we may trace the flow of execution and monitor variable changes. For the case of distributed 
applications, it should be used loggers with external (to application) storage, as Sentry
(\href{https://pub.dev/packages/sentry\_flutter}{https://pub.dev/packages/sentry\_flutter}):

\begin{lstlisting}
// Initialization
runZonedGuarded(() async {
  Future<void> main() async {
    await SentryFlutter.init(
      (o) => o.dsn = 'https://{NAME}@sentry.io/{DSN}'
    );
  }
  runApp(MyApp());
}, (ex, trace) async {
  await Sentry.captureException(ex, stackTrace: trace);
});
// ... in the code
Sentry.captureMessage(
  'A message with some info',
  withScope: (scope) {
    scope.setTag('variable', 'state');
  },
);
\end{lstlisting}

\noindent Once the issue has been successfully replicated locally, we can strategically place breakpoints in our code 
to halt execution at specific points. By default, when executing the \q{flutter run}-command, the application runs in 
debug mode. This enables us to scrutinize variables and navigate through the code step by step, facilitating a more 
granular inspection of the codebase.

In the case of utilizing a web distribution, it becomes necessary to find a means to examine Web Progressive Applications 
(WPA). The majority of mobile browsers facilitate remote debugging through their developer tools. For instance, 
in Chrome, we may use \href{chrome://inspect\#devices}{chrome://inspect\#devices} to enable access to a phone that is 
connected via a USB. Underneath is would be shown a list of recognized devices (\cref{img:o-devtool}). By clicking on 
the respective device, the inspect panel is activated, allowing for detailed examination and debugging.

\img{features/devtool-devices}{Google Chrome: Devices Inspector}{img:o-devtool}

\noindent The phone might not be recognized after being connected. In that case we should turn on the "USB debugging".
As for Android, those options are available on \q{Developer options} from \q{Settings}-menu after clicking 7 times 
on \q{Build Number} (from "About phone"-submenu).

For the case of "\emph{Pending authentication: please accept debugging session on the device}", consider experimenting 
with various USB cables or connecting to different ports on a machine. Additionally, we should attempt a sequence of 
command prompts to manage the Android Debug Bridge (ADB): start from \q{adb kill-server}, followed by 
\q{adb start-server}, and finally \q{adb devices} to verify and check the status of connections. This sequence might 
help resolve authentication-related issues during the debugging session.

An additional tip is to modify the USB connection type to \q{MIDI} instead of file transfers. Simply slide down the 
main header to access the configuration settings and make the necessary adjustment.

~

\noindent All of those things were needed to transform our initial error (\cref{img:o-error}) into a more clear trace 
(\cref{img:o-error-details}, 
\href{https://github.com/flutter/flutter/issues/138880}{https://github.com/flutter/flutter/issues/138880}):

\img{features/crash-stacktrace-details}{Google Chrome: Failure detection}{img:o-error-details}

\noindent The encountered problem surfaced exclusively in the context of Progressive Web App (PWA) development, 
specifically tied to the corruption of transliterations. This issue arises (in Flutter 3.16.0 with Dart 3.2.0) because 
the Dart code undergoes a two-step compilation process: first, it is compiled to optimized JavaScript for deployment, 
and subsequently, this JavaScript is JIT-compiled (Just-In-Time) by the web platform to a native code while the 
application is in runtime. The complexity of this compilation pipeline introduces a potential source of errors as above.

Previously, there was a brief mention of the option to build with Web Assembly (\ref{deploy-web}). Web Assembly is a 
low-level format that resides closer to the abstraction level of machine code compared to JavaScript. This proximity 
results in enhanced startup times and a more predictable execution flow. Opting for Web Assembly eliminates the need 
for the two-step compilation process, making it a potential solution amidst various uncertainties. However, it's worth 
noting that Web Assembly, introduced in 2020, remains experimental and introduces its set of considerations and 
potential trade-offs.
