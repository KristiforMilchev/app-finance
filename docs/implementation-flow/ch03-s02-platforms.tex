% Copyright 2023 The terCAD team. All rights reserved.
% Use of this content is governed by a CC BY-NC-ND 4.0 license that can be found in the LICENSE file.

\newpage
\subsection{Unlocking Multi-Platform}
\markboth{Prototyping}{Unlocking Multi-Platform}

Flutter is designed to work seamlessly across all environments. However, there are a certain scenarios, such as file
handling, where its behavior may deviate from being entirely platform-agnostic. Independently of that, there are two
ways to brake that limitation:

\begin{lstlisting}
// Conditional Imports
import 'package:file_picker_web/file_picker_web.dart' if (dart.library.io) 'package:file_picker/file_picker.dart';
// ... and Exports
export 'src/connector_none.dart' // Stub implementation
  if (dart.library.io) 'src/connector_io.dart' // dart:io
  if (dart.library.html) 'src/connector_html.dart'; // dart:html
// By using postponed initialization: await heavy.loadLibrary();
import 'package:{heavy}/{heavy}.dart' deferred as heavy;
\end{lstlisting}

\noindent Platform channels is used as a communicational bridge to inject a platform-specific code, like Swift/Obj-C 
for iOS or Java/Kotlin for Android:

\begin{lstlisting}
// To create a Flutter Method Channel
const platform = MethodChannel('channelName');
// To call the native code
try {
  final result = await platform.invokeMethod('methodName', params);
  // ... handle the result
} on PlatformException catch (e) { /* handle errors */ }
\end{lstlisting}