% Copyright 2023 The terCAD team. All rights reserved.
% Use of this content is governed by a CC BY-NC-ND 4.0 license that can be found in the LICENSE file.

\subsection{Configuring Deployment}
\markboth{Defining Quality Gates}{Configuring Deployment}


\subsubsection{Building Web Package} \label{deploy-web}

Creating a web package is handled by \q{flutter build -v web --release} command. It'll be more strict to our code, and 
that might lead to additional failures as "non-constant instances of IconData".

\begin{lstlisting}
// ./lib/_classes/data/goal_app_data.dart
factory GoalAppData.fromJson(Map<String, dynamic> json) {
  return GoalAppData(
    icon: json['icon'] != null
      ? IconData(json['icon'], fontFamily: 'MaterialIcons') // Error
      : null,
// ... other code
\end{lstlisting}

\noindent To fix this issue, we need to ensure that all instances of IconData are used in a constant context, that can 
be partially solved by the next modification:
\begin{lstlisting}
// ./lib/_mixins/formatter_mixin.dart
mixin FormatterMixin {
  static final Map<String, IconData> _cache = {};

  static IconData getIconFromString(int icon) {
    if (_cache.containsKey(icon)) {
      return _cache[icon]!;
    } else {
      const String fontFamily = 'MaterialIcons';
      return IconData(icon, fontFamily: fontFamily);
    }(*@ \stopnumber @*)
  }

// ./lib/_classes/data/goal_app_data.dart
factory GoalAppData.fromJson(Map<String, dynamic> json) {
  return GoalAppData(
    icon: json['icon'] != null
      ? FormatterMixin.getIconFromString(json['icon'])
      : null,
\end{lstlisting}

Now, when we'll call \q{getIconFromString} with a dynamic icon name, it will create the \q{IconData}-instance for that name 
if it doesn't already exist and cache it in the \q{\_cache}-map. If the same icon name is used again in the future, 
it will return the cached \q{IconData}-instance, allowing tree shaking to work correctly. But we still have to suppress 
the error itself because, when we're dealing with dynamic values that cannot be statically analyzed at compile time, 
tree shaking is impossible:

\begin{lstlisting}[language=bash]
flutter build -v web --release --no-tree-shake-icons
\end{lstlisting}

Another problem is that our production code is obfuscated. That means that \q{AccountAppData} will be converted to 
something like \q{minified:iP}; and our restoring procedure from transactions is going to fail. By a first assumption,
it can be changed "magic strings" to a \q{runtimeType}-property of classes:

\begin{lstlisting}
// ./lib/_classes/data/transaction_log.dart
  static void init(AppData store, String type, Map<String, dynamic> data) {
    final goal = GoalAppData(title: '', initial: 0.0).runtimeType.toString();
    final account = AccountAppData(title: '', type: '').runtimeType.toString();
    final bill = BillAppData(title: '', account: '', category: '').runtimeType.toString();
    final budget = BudgetAppData(title: '').runtimeType.toString();
    final currency = CurrencyAppData(title: '').runtimeType.toString();
    final typeToClass = {
      goal: (data) => GoalAppData.fromJson(data),
      account: (data) => AccountAppData.fromJson(data),
      bill: (data) => BillAppData.fromJson(data),
      budget: (data) => BudgetAppData.fromJson(data),
      currency: (data) => CurrencyAppData.fromJson(data),
    };
// ... other code
\end{lstlisting}

\noindent The main problem with that solution is that it cannot be portaged to other platforms since each type of the 
build (even within one platform) will use different runtime notation for the same class. So, the right way would be 
to adjust the conversion to JSON:

\begin{lstlisting}
// ./lib/_classes/data/abstract_app_data.dart
  Map<String, Map<String, dynamic>> toFile() {
    var data = {...toJson()};
    return {
      'type': {
\end{lstlisting}
{
\xpretocmd{\lstlisting}{\vspace{-12pt}}{}{}
\begin{lstlisting}[firstnumber=6, backgroundcolor=\color{backred}]
(*@\kdiff{-}@*)        'name': runtimeType.toString(),
\end{lstlisting}
\begin{lstlisting}[firstnumber=6, backgroundcolor=\color{backgreen}]
(*@\kdiff{+}@*)        'name': getClassName(),
\end{lstlisting}
\begin{lstlisting}[firstnumber=7]
        'hash': TransactionLog.getHash(data),
      },
      'data': data,
    };
  }
  // Abstract, has to be implemented by each class
  String getClassName();
\end{lstlisting}
}

\noindent That's done.

Finally, we cannot ignore anymore a notification "Incorrect use of ParentDataWidget" while it has become a crucial for 
the production type of the build, and may lead to a greyed-out output instead of rendered widgets.\\
\\

Regarding the deployment, while we are following a Serverless approach, which means there is no any dedicated server to 
deploy into, we can improve usage of our GitHub Pages (\ref{a-badges}):

\begin{lstlisting}[language=yaml]
name: Web Deployment

on:
  push:
    branches:
      - main

jobs:
  build:
    runs-on: ubuntu-latest

    steps:
      - uses: actions/checkout@v3
      - uses: subosito/flutter-action@v2
        with:
          channel: 'stable'
          cache: true
      - run: flutter --version

      - name: Install Dependencies
        run: flutter pub get

      - name: Build Web Package
        run: flutter build -v web --release --no-tree-shake-icons --base-href="/app-finance/"

      - name: Add Coverage Report to the Package
        run: cp -R ./coverage ./build/web/coverage

      - name: Update GitHub Pages 
        uses: peaceiris/actions-gh-pages@v3
        with:
          github_token: ${{ secrets.GITHUB_TOKEN }}
          publish_dir: ./build/web
\end{lstlisting}

\noindent A valuable point here is to update \q{base-href} (line 24) for our Web package. It has to be aligned with 
GitHub Pages notation:

\begin{lstlisting}[language=bash]
https://<account-name>.github.io/<repository-name>/
\end{lstlisting}

Additionally, we may buy a domain name and configure GitHub Pages to be viewed there (detailed instructions provided by
GitHub: \href{https://docs.github.com/en/pages/configuring-a-custom-domain-for-your-github-pages-site/managing-a-custom-domain-for-your-github-pages-site}{https://docs.github.com/en/pages/configuring-a-custom-domain-for-your-github-pages-site/managing-a-custom-domain-for-your-github-pages-site}).


\subsubsection{Preparing Releases}

Things, that can be automated, has to be automated, and Release generation is not an exception. Let's generate Release
page by setting a new tag (as an example, \q{git tag v1.0.1}, \q{git push origin v1.0.1}):

\begin{lstlisting}[language=yaml]
name: Project Release Artifacts
# Trigger Action by new tag recognition
on:
  push:
    tags:
      - 'v*' # Filter by pattern
# Run jobs
jobs:
  release:
    name: Create Release Page
    runs-on: ubuntu-latest
    outputs:
      version: ${{ steps.tag.outputs.name }}
      upload_url: ${{ steps.create_release.outputs.upload_url }}
      build_number: ${{ steps.build_number.outputs.build_number }}
    steps:
      - uses: actions/checkout@v3
        # Next argument is required to retrieve full git-history
        with:
          fetch-depth: 0
      - uses: subosito/flutter-action@v2
        with:
          channel: 'stable'
          cache: true
      - run: flutter pub get

      - name: Get Tag Name
        id: tag
        run: echo "name=${GITHUB_REF#refs/tags/v}" >> $GITHUB_OUTPUT
      # Generate notes from git-history between tags
      - name: Prepare Release Notes
        run: |
          # Approach to handle multiline content 
          dart run grinder release-notes --tag=v${{ steps.tag.outputs.name }} --output=release_notes.log
          echo "RELEASE_NOTES<<EOF" >> $GITHUB_ENV
          while IFS= read -r line; do
            echo "$line" >> $GITHUB_ENV
          done < "release_notes.log"
          echo "EOF" >> $GITHUB_ENV
          rm release_notes.log
      # Build number as totals of commits
      - name: Set Build Number
        id: build_number
        run: echo "build_number=$(git rev-list --count HEAD)" >> $GITHUB_OUTPUT

      - name: Create Release Page
        id: create_release
        uses: actions/create-release@v1
        env:
          GITHUB_TOKEN: ${{ secrets.GITHUB_TOKEN }}
        with:
          tag_name: v${{ steps.tag.outputs.name }}
          release_name: Release ${{ steps.tag.outputs.name }}
          body: ${{ env.RELEASE_NOTES }}
          # Prevent its publicity before a manual check
          draft: true
          prerelease: false
\end{lstlisting}


\subsubsection{Generating Artifacts}

We've prepared a release page, but what can be done in addition, is to generate artifacts and attach them to our release
page. Let's do that and generate desktop runners for Macos, Linux, and Windows:

\begin{lstlisting}[language=yaml]
jobs:
  release:
    # ... shown earlier
  build:
    name: Create ${{ matrix.target }} build
    runs-on: ${{ matrix.os }}
    needs: release
    strategy:
      fail-fast: false
      matrix:
        target: [macOS, Windows, Linux]
        include:
          - os: macos-latest
            target: macOS
            build_path: build/macos/Build/Products/Release
            asset_extension: .zip
            asset_content_type: application/zip
          - os: windows-latest
            target: Windows
            build_path: build/windows/runner/Release
            asset_extension: .zip
            asset_content_type: application/zip
          - os: ubuntu-latest
            target: Linux
            build_path: build/linux/x64/release/bundle
            asset_extension: .tar.gz
            asset_content_type: application/gzip

    steps:
      - uses: actions/checkout@v3
      - uses: subosito/flutter-action@v2
        with:
          channel: 'stable'
          cache: true
      - run: flutter pub get

      - name: Run Windows Build 
        if: matrix.target == 'Windows'
        run: |
          flutter config --enable-windows-desktop
          flutter build -v windows --build-name=${{ needs.release.outputs.version }} --build-number=${{ needs.release.outputs.build_number }} --release

      - name: Compress Windows Package
        if: matrix.target == 'Windows'
        run: |
          compress-archive -Path * -DestinationPath ${env:GITHUB_WORKSPACE}/fingrom_${{ matrix.target }}${{ matrix.asset_extension }}
        working-directory: ${{ matrix.build_path }}

# ... the same steps for Linux '-v linux' and Macos '-v macos'

      - name: Upload ${{ matrix.target }} Artifact
        id: upload_release_asset
        uses: actions/upload-release-asset@latest
        env:
          GITHUB_TOKEN: ${{ secrets.GITHUB_TOKEN }}
        with:
          upload_url: ${{ needs.release.outputs.upload_url }}
          asset_path: ./fingrom_${{ matrix.target }}${{ matrix.asset_extension }}
          asset_name: fingrom_${{ matrix.target }}${{ matrix.asset_extension }}
          asset_content_type: ${{ matrix.asset_content_type }}
\end{lstlisting}




