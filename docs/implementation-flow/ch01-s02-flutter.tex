% Copyright 2023 The terCAD team. All rights reserved.
% Use of this content is governed by a CC BY-NC-ND 4.0 license that can be found in the LICENSE file.

\subsection{Flutter}
\markboth{Bootcamping}{Flutter}

Flutter is an open-source framework developed by Google for building natively compiled applications for mobile, web, and 
desktop from a single codebase by using \q{Dart}-language. The framework provides a comprehensive set of libraries and 
tools for building UIs, handling user input, and accessing device features. Flutter's UI is constructed by using widgets, 
that are the building blocks of Flutter apps, and everything, from a simple button to a complex layout, is a widget
(including animations and layout controls).

\paragraph{Engine} At the core of Flutter is the C/C++ engine, known as Skia (open-source 2D-graphics library, developed 
and maintained by Google), that provides a set of low-level APIs for rendering graphics (support for both vector and 
bitmap graphics to draw shapes, text, and images) and low-level platform services. Skia includes anti-aliasing 
capabilities, which smooth out edges and lines to create visually pleasing graphics. This engine provides interfaces 
with the platform-specific code to interact with device hardware. It bridges the gap between the Flutter framework, 
written in Dart, and the platform-specific code, allowing Flutter apps to access device features and services.

\paragraph{Engine Embedder}

The Flutter Engine is window toolkit agnostic, meaning a capability to build Flutter embedders on one of the platforms 
not supported out of the box. That enables creation of custom solutions with the power and flexibility of Flutter for 
a distribution to anywhere, including smart devices, cars, and more (as \q{raspi-flutter} for \q{Raspberry Pi}).

\paragraph{Platform Channels} Device-specific features and APIs are accessible through what are known as platform channels, 
which facilitate the integration of a native functionality. Platform Channels enables bi-directional asynchronous 
communication between Dart and the native code of the host platform. It's possible to invoke methods on that channel 
from either the Dart or native side (for example, a native method to access the device's camera). Bi-directional 
communication is covered by a data serialization and deserialization process that ensures that it is in a format that 
both can understand.

\paragraph{Plugins} Flutter plugins provide a way to encapsulate and interact with platform-specific code, by using 
\q{Dart}-language as a bridge between Flutter and the native code, accessing hardware features, third-party libraries, 
and platform-specific APIs.

\subsubsection{Considering Diversity}

Out of the box (OOTB), Flutter offers a vast array of widgets that greatly facilitate the creation of diverse and 
complex user interfaces. These widgets are designed to cater a wide range of application needs, from crafting simple 
layouts to developing intricate and feature-rich interfaces. This comprehensive selection of widgets significantly 
accelerates the development process, allowing us to focus more on reaching the functional needs and less on building UI 
components from scratch.

\begin{itemize}
  \item \q{SizedBox} for a size allocation, \q{Container} extends it by styling as padding, margin, and background color.

  \item \q{Row} and \q{Column} arrange child widgets horizontally and vertically accordingly, and \q{DataTable}  
  mixes both.

  \item \q{ListView} for creating scrollable lists of widgets

  \item \q{Stack} allows to overlay widgets on top of one another.

  \item \q{AppBar} enables a navigation on top of the page, \q{BottomNavigationBar} is specific for the bottom 
  layer, for the sidebar it can be used \q{NavigationRail}, \q{Drawer} -- to implement navigation drawers, and 
  \q{SnackBar} for the popping up content.

  \item \q{FlatButton}, \q{RaisedButton}, \q{FloatingActionButton} -- different buttons with customizable visual feedback.

  \item \q{TextField}-widget offers input fields where users can enter text, numbers, or other data. Meanwhile,
  \q{SearchAnchor} manages selections from a list of options.
  
  \item \q{Card} -- for creating material design cards with a consistent and customizable elevation effect.

  \item \q{PageRouteBuilder} enables custom page transitions and animations when navigating between screens.

  \item \q{FutureBuilder} simplifies working with asynchronous operations by rebuilding itself when the future state is 
  resolved.

  \item \q{InkWell} provides a Material-style ripple effect for making widgets tappable and interactive.
\end{itemize}

\noindent And, a lot of others...


\newpage
\subsubsection{Distinguishing Stateless and Stateful}

A \q{StatelessWidget} in Flutter represents interfaces with an immutable state (once created), requiring a new instance 
for updates; providing suitability for static content like text, icons, or images; and performance advantages due to 
their lack of mutable state, making them generally more memory and processing-efficient.

Where as \q{StatefulWidget} is capable of changing its properties and appearance over a time. It consists of two classes: 
immutable \q{Widget}-class and a corresponding mutable \q{State}-class. Key attributes encompass mutable state stored 
within the associated \q{State}-object and can be changed by the \q{setState}-method. Suits for crafting interactive 
and dynamic user interfaces such as forms and animations. It's worth noting that \q{StatefulWidget} may demand more 
memory due to a potentially frequent rebuilds, but performance itself can be improved by \q{RepaintBoundary}-widget 
usage that tears off (isolate) the render of bounded widgets from predecessor objects rendering.

\begin{lstlisting}
class LoadingWidget extends StatefulWidget {
  const LoadingWidget({super.key});
  @override
  LoadingWidgetState createState() => LoadingWidgetState();
}
class LoadingWidgetState extends State<LoadingWidget> with TickerProviderStateMixin {
  late AnimationController _controller = AnimationController(
      vsync: this,
      duration: const Duration(seconds: 2),
    )..repeat();

  @override
  void dispose() {
    _controller.dispose();
    super.dispose();
  }
  @override
  Widget build(BuildContext context) => AnimatedBuilder(
      animation: _controller,
      builder: (context, child) {
        return CircularProgressIndicator(
          value: _controller.value,
          color: context.colorScheme.inversePrimary,
          strokeWidth: 4,
        );
      },
    );
}
\end{lstlisting}


\newpage
\subsubsection{Asserting in Tests}

In case of any test (unit, widget, or integration) creating for Flutter applications, the ultimate objective is to 
confirm that the actual outcomes align with the anticipated results.

\begin{lstlisting}
// Find and check Widget by its text
expect(find.text('text'), findsNothing);
// ... by Widget's type
expect(find.byType(DemoPage), findsOneWidget);
expect(find.byType(Container), findsWidgets);
\end{lstlisting}

\noindent Flutter helps to automate accessibility checks (that can be extended by inheriting 
\q{AccessibilityGuideline}-class):

\begin{lstlisting}
testWidgets('Verify Guidelines', (tester) async {
  await tester.pumpWidget(MyApp()); // Initialize Widget
  // Verify a tappable area (min 44x44 for iOS, 48x48 - Android)
  await expectLater(tester, 
      meetsGuideline(iOSTapTargetGuideline));
  await expectLater(tester, 
      meetsGuideline(androidTapTargetGuideline));
  // WCAG text contrast requirements
  await expectLater(tester, 
      meetsGuideline(textContrastGuideline));
  // Enforces all navigation elements to have a label
  await expectLater(tester,
      meetsGuideline(labeledTapTargetGuideline));
}
\end{lstlisting}

\noindent In the context of a complex layout, we may employ an image as a means to verify the rendered output (as a 
warning, fonts generation is a platforms specific even with \q{FlutterTest}-font, that is used if font family isn't 
specified or not available; check \ref{golden-image} for details):

\begin{lstlisting}
testWidgets('Compare with Image', (tester) async {
  await tester.pumpWidget(widget);
  final check = find.byType(MyApp);
  // By using a stored image file
  await expectLater(check, matchesGoldenFile('result.png'));
  // Compare with generated image
  final recorder = PictureRecorder();
  final imageCanvas = Canvas(recorder);
  imageCanvas.drawCircle(Offset.zero, 20.0, Paint());
  Picture image = recorder.endRecording();
  await expectLater(check, matchesReferenceImage(image));
});
\end{lstlisting}
