% Copyright 2023 The terCAD team. All rights reserved.
% Use of this content is governed by a CC BY-NC-ND 4.0 license that can be found in the LICENSE file.

\subsection{Handling Instabilities}
\markboth{Optimizing}{Handling Instabilities}

A rich interface may road us to an area of instabilities...

By example, an error \emph{"Scaffold.geometryOf() must only be accessed during the paint phase"} (with ongoing 
\emph{"'package:flutter/src/rendering/mouse\_tracker.dart': Failed assertion: ... '!\_debugDuringDeviceUpdate': 
is not true"}) leads to a fully nonfunctional state  of the application (by not responding to any of interaction).
Sometimes we might be not alone with the problematic area; as for the error above, it's been highlighted to Flutter 
SDK team a few times (issues \href{https://github.com/flutter/flutter/issues/76325}{\#76325},
\href{https://github.com/flutter/flutter/issues/108363}{\#108363}, and
\href{https://github.com/flutter/flutter/issues/126513}{\#126513}), but closed without being resolved simply by not 
declared steps to reproduce it constantly.

\img{uiux/bottom-bar}{Application Bottom Bar for small screens}{img:u-bottom-bar}

In our case (\cref{img:u-bottom-bar}) the problem is relevant to the usage of \q{FloatingActionButtonLocation.centerDocked} 
with enabled \q{bottomNavigationBar} for \q{Scaffold}, and appears by navigating back and forth between pages. The issue 
is related to how Flutter renders under the hood, and there are no declared steps to resolve it. Each time, the 
solution may require a unique approach:

\begin{lstlisting}
// Awaiting for the route transition to be completed
await route.completed;

// Deactivate inactive routes to be remained in memory
MaterialPageRoute(maintainState: false, ... 
      // or contrary, has to be "true"

// Wrap lists content and declare scrolling
ListView.builder(
  scrollDirection: Axis.vertical,
  shrinkWrap: true,
\end{lstlisting}

\noindent Sometimes the problem cannot be resolved in the application itself, and we have to change somehow used 
external components. We may either create an issue 
(\href{https://github.com/flutter/flutter/issues/134408}{https://github.com/flutter/flutter/issues/134408}) 
with an assumption that it will be taken into account and fixed sometime, or create a patch by our own 
(\href{https://github.com/flutter/flutter/pull/134409}{https://github.com/flutter/flutter/pull/134409}). In cases, when 
the fix is not universal (or we do not have a time for) and relevant only for us, external component should be forked 
(\href{https://github.com/lyskouski/flutter}{https://github.com/lyskouski/flutter}) and re-targeted for the application
via \q{pubspec.yaml}, or, in case of Flutter itself, by a local binding (as for Linux -- 
\href{https://docs.flutter.dev/get-started/install/linux#install-flutter-manually}{https://docs.flutter.dev/get-started/install/linux\#install-flutter-manually}).\\
\\

\noindent The main idea is not to let the problem take its course. Neither users, nor superiors would care that the 
problem falls outside the scope of our application or responsibilities. The provided impetus have to reach the 
target with our full assistance every step of the way.
