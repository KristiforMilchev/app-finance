% Copyright 2023 The terCAD team. All rights reserved.
% Use of this content is governed by a CC BY-NC-ND 4.0 license that can be found in the LICENSE file.

\newpage
\subsection{Relying Versatility}
\markboth{Optimizing}{Relying Versatility}

Basically, responsive design relies on fluid layouts and media queries to adapt content dynamically to different screen 
sizes within a single codebase. Adaptive design, on the other hand, involves creating a specific layout per different 
devices or screen sizes. Each approach has its strengths and weaknesses, and the choice between them depends on many 
factors. But ideally it would be better to mix both solutions: widgets itself should be responsive, pages (aggregative 
widgets, called by routes) -- adaptive.


\subsubsection{Packaging Components}

Embracing a grid layout as part of an adaptive design strategy is a pivotal step toward creating a versatile and 
user-centric digital experience (\cref{img:u-grid}). Grids provide the structure needed to maintain visual consistency, 
prioritize content, and seamlessly adapt to diverse screen sizes.

And for that case let's dive into an external package creation. From IDE, by pressing \key{F1} -- \q{Flutter: New 
Project} -- \q{Package}, we will generate a package template (only "lib" and "test"-folders plus configuration files). 
Inside \q{lib}-folder the file, named identically to our package, works as an entry point for the package
(\href{https://pub.dev/packages/flutter_grid_layout}{https://pub.dev/packages/flutter\_grid\_layout}):

\begin{lstlisting}
// ./lib/flutter_grid_layout.dart
library flutter_grid_layout;
// 'src'-folder for internal files that 
// might not be accessible without export
export 'src/grid_container.dart';
export 'src/grid_item.dart';
\end{lstlisting}

\img{uiux/cssgrid-flexbox}{Representation of Grid Layout \cite{Tayl18}}{img:u-grid}

\noindent The implementation has a resemblance to the previously created \q{RowWidget} by its functionality of 
converting relative values into density-independent pixels (DIP). \q{GridItem}-widget serves as an encapsulating 
entity for the original widget, delineating its placement within a grid matrix; while \q{GridContainer}-widget covers
all requisite calculations for repositioning:

\begin{lstlisting}
// ./lib/src/grid_item.dart
class GridItem extends StatelessWidget {
  final Size start;
  final Size end;
  final int zIndex;
  final Widget child;

  const GridItem({/* ... */});


  @override
  Widget build(BuildContext context) {
    return child;
  }(*@ \stopnumber @*)
}

// ./lib/src/grid_container.dart
class GridContainer extends StatelessWidget {
  final List<double?> columns;
  final List<double?> rows;
  final List<GridItem> children;

  const GridContainer({/* ... */});

  // Calculate actual DIP based on relative values
  List<double> _calc(double size, List<double?> scope) {
    // ... implementation similar to RowWidget
    // Sample, 0.5 -> 50%, 10 -> 10 DIP, null -> auto
    scope.insert(0, 0.0); // leading zero
    return scope.cast(); // null-safety
  }

  // Count length from zero-point
  List<double> _scale(List<double> scope) =>
      scope.asMap().entries.map((entry) => 
          scope.sublist(0, entry.key + 1).fold(0.0, 
              (v, e) => v + e)).toList();
  List<double> _calcWidth(width) => _calc(width, rows);
  List<double> _calcHeight(height) => _calc(height, columns);
  @override
  Widget build(BuildContext context) {
    if (columns.isEmpty || rows.isEmpty || children.isEmpty) {
      return const SizedBox();
    }
    children.sort((a, b) => a.zIndex.compareTo(b.zIndex));
    return LayoutBuilder(builder: (context, constraints) {
      final width = _scale(_calcWidth(constraints.maxWidth));
      final height = _scale(_calcHeight(constraints.maxHeight));
      return Stack(
        children: List<Widget>.generate(children.length, (index) {
          final item = children[index];
          // Calculate actual Widget size
          final itemWidth = width[item.end.width.toInt()] - 
              width[item.start.width.toInt()];
          final itemHeight = height[item.end.height.toInt()] - 
              height[item.start.height.toInt()];
          // Bound by Container
          return Container(
            // Shift by starting position
            margin: EdgeInsets.only(
              left: width[item.start.width.toInt()],
              top: height[item.start.height.toInt()],
            ),
            // Expand to the full area, alike in CSS:
            // > justify-self: stretch;
            constraints: BoxConstraints(
              maxWidth: itemWidth,   minWidth: itemWidth,
              maxHeight: itemHeight, minHeight: itemHeight,
            ),
            width: itemWidth,
            height: itemHeight,
            child: item.child,
          );
        }));
    });
  }
}
\end{lstlisting}

\noindent By using \q{flutter pub publish}-command our package will become available on 
\href{https://pub.dev}{https://pub.dev} and can be aggregated into the main project by 
\q{flutter pub add flutter\_grid\_layout}-command evaluation.


\subsubsection{Extending Behavior}

Responsiveness, by being a paramount in engaging users across a multitude of devices, is covered by a couple of design 
concepts \cite{Frai22}: fluid layouts, media queries, and content prioritization. The concept of "Fluid Layouts" is in 
a usage of relative units like percentages for widths and heights, allowing content to dynamically expand or contract 
based on the available screen real estate. "Media Queries" are used to apply rules based on characteristics such as 
screen size and orientation. Whereas "Content Prioritization" entails thoughtful content ordering and reposition;
based on screen configuration, less critical elements can gracefully adapt, be positioned differently, or even hidden.

\q{RowWidget} describes responsiveness behavior \issue{37}{abd9308}, \issue{185}{e1e7385} when we set a direct or 
relative (values \q{0.0 ... 0.9(9)} are converted into \q{0 ... 99.9(9)\%} of an available width) size for a couple of 
components, and left nullable size (alike, \q{Spacer}-widget) for others. In that case the text section will be hidden 
for the width less than 40 density-independent pixels (DIP):

\begin{lstlisting}
RowWidget(
  chunk: [null, 40],
  children: [
    [
      Text(
        'Sample text with a long description', 
        maxLines: 1,
        overflow: TextOverflow.ellipsis,
      ),
    ],
    [
      FloatingActionButton(
        child: Icon(Icons.add),
      ),
    ],
  ],
)
\end{lstlisting}

\noindent On other hand, \q{GridLayer} manages \issue{158}{} different strategies of a widgets' composition: 

\begin{lstlisting}
// ./lib/routes/home_page.dart
GridLayer(
  padding: indent,
  crossAxisCount: countWidth,
  strategy:
    switch (countWidth) {
    // Rows with a single component per each
    4 => [
      [2], [3], [1], [0]
    ],
    // Three rows
    3 => [
      [2], [3], [0, 1]
    ],
    // Two rows with a column in each
    2 => [
      [2, 3], [0, 1]
    ],
    // Single column
    _ => [
      [0, 1, 2, 3]
    ]
  },
  children: [
    // Hide widget on a portrait mode
    matrix.getHeightCount(constraints) > 3
        ? goalWidget
        : ThemeHelper.emptyBox,
    billWidget,
    accountWidget,
    budgetWidget,
  ],
);
\end{lstlisting}
\begin{lstlisting}
// ./lib/widgets/_wrappers/grid_layer.dart
class GridLayer extends StatelessWidget {
  // Indent between sections
  final double padding;
  // Number of rows
  final int crossAxisCount;

  // List<Widget | Widget Function>
  final List<dynamic> children;

  // Representation strategy
  final List<dynamic> strategy;

  const GridLayer({
    super.key,
    required this.padding,
    required this.crossAxisCount,
    required this.children,
    required this.strategy,
  });

  @override
  Widget build(BuildContext context) {
    // Item can be callable to avoid unnecessary data aggregation
    // if the widget is not a part of all strategies
    fnItem(int index) => children[index] is Function 
        ? children[index]() 
        : children[index];
    // Convert widget indexes into their representation
    fnList(List<dynamic> scope) => scope.map((e) => e is List 
        ? fnList(e).cast<Widget>().toList() 
        : fnItem(e));
    return Padding(
      padding: EdgeInsets.only(left: padding, right: padding),
      child: strategy.length > 1
        ? RowWidget(
            indent: padding,
            maxWidth: ThemeHelper.getWidth(context, 2),
            chunk: List.filled(crossAxisCount, null),
            children: fnList(strategy).cast<List<Widget>>().toList()
          )
        : Column(
            children: fnList(strategy.first).cast<Widget>().toList()
          )
    );
  }
}
\end{lstlisting}
