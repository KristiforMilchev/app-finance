% Copyright 2023 The terCAD team. All rights reserved.
% Use of this content is governed by a CC BY-NC-ND 4.0 license that can be found in the LICENSE file.

\subsection{Applying Adaptive \& Responsive Design}
\markboth{Optimizing UI/UX Flow}{Applying Adaptive \& Responsive Design}

Basically, responsive design relies on fluid layouts and media queries to adapt content dynamically to different screen 
sizes within a single codebase. Adaptive design, on the other hand, involves creating a specific layout per different 
devices or screen sizes. Each approach has its strengths and weaknesses, and the choice between them depends on many 
factors. But ideally it would be better to mix both solutions: widgets itself should be responsive, pages (aggregative 
widgets, called by routes) -- adaptive.

\q{RowWidget} describes responsiveness behavior \issue{37}{abd9308}, \issue{185}{e1e7385} when we set a direct size 
for a couple of components, and left nullable size (alike, \q{Spacer}-widget) for others. In that case the text section 
will be hidden in case of width less than 40 density-independent pixels (DIP):

\begin{lstlisting}
RowWidget(
  chunk: [null, 40],
  children: [
    [
      Text(
        'Sample text with a long description', 
        maxLines: 1,
        overflow: TextOverflow.ellipsis,
      ),
    ],
    [
      FloatingActionButton(
        child: Icon(Icons.add),
      ),
    ],
  ],
)
\end{lstlisting}

\noindent On other hand, \q{GridLayer} manages \issue{158}{} different strategies of a widgets' composition: 

\begin{lstlisting}
// ./lib/routes/home_page.dart
GridLayer(
  padding: indent,
  crossAxisCount: countWidth,
  strategy: switch (countWidth) {
    // Rows with a single component per each
    4 => [
      [2], [3], [1], [0]
    ],
    // Three rows
    3 => [
      [2], [3], [0, 1]
    ],
    // Two rows with a column in each
    2 => [
      [2, 3], [0, 1]
    ],
    // Single column
    _ => [
      [0, 1, 2, 3]
    ]
  },
  children: [
    // Hide widget on a portrait mode
    matrix.getHeightCount(constraints) > 3
        ? goalWidget
        : ThemeHelper.emptyBox,
    billWidget,
    accountWidget,
    budgetWidget,
  ],(*@ \stopnumber @*)
);

// ./lib/widgets/_wrappers/grid_layer.dart
class GridLayer extends StatelessWidget {
  // Indent between sections
  final double padding;
  // Number of rows
  final int crossAxisCount;
  // List<Widget | Widget Function>
  final List<dynamic> children;
  // Representation strategy
  final List<dynamic> strategy;

  const GridLayer({
    super.key,
    required this.padding,
    required this.crossAxisCount,
    required this.children,
    required this.strategy,
  });

  @override
  Widget build(BuildContext context) {
    // Item can be callable to avoid unnecessary data aggregation
    // if the widget is not a part of all strategies
    fnItem(int index) => children[index] is Function ? children[index]() : children[index];
    // Convert widget indexes into their representation
    fnList(List<dynamic> scope) => scope.map((e) => e is List ? fnList(e).cast<Widget>().toList() : fnItem(e));
    return Padding(
      padding: EdgeInsets.only(left: padding, right: padding),
      child: strategy.length > 1
        ? RowWidget(
            indent: padding,
            maxWidth: ThemeHelper.getWidth(context, 2),
            chunk: List.filled(crossAxisCount, null),
            children: fnList(strategy).cast<List<Widget>>().toList()
          )
        : Column(
            children: fnList(strategy.first).cast<Widget>().toList()
          )
    );
  }
}
\end{lstlisting}
