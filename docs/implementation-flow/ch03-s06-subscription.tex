% Copyright 2023 The terCAD team. All rights reserved.
% Use of this content is governed by a CC BY-NC-ND 4.0 license that can be found in the LICENSE file.

\subsection{Activating Sponsorship}
\markboth{Prototyping}{Activating Sponsorship}

It might be thought that we're going to use a Freemium business model (portmanteau of "free" and "premium"), which is 
a pricing strategy combining free basic features with paid access to premium functionality. But we've broken it due to 
open sourcing the solution. So, whenever any limitation will appear, the application might be forked and "lost" 
for us. Nonetheless, monetization is vital because, as we attract more users, the application demands greater attention 
and resources.

That's why, the subscription is not just about making money; it's about sustaining (the costs of development, 
maintenance, and support) and growing while delivering value to our users. It allows us to keep pace with competitors 
and invest in innovation.

And here, the Donation-Based Crowdfunding is coming into play. It involves collecting small sums of money from a large 
number of people. Unlike other forms of crowdfunding, such as equity or reward-based, this method doesn't promise 
backers financial returns or specific rewards. Instead, people contribute to the project out of goodwill and a desire 
to support our ideas and objectives. It allows to raise funds from individuals who genuinely resonate with the 
product vision.

\subsubsection{Adding Custom References}

For Windows, Linux, and Web releases the simplest way of adding a subscription would be to provide links to external 
services, that're specialized on crowdfunding initiatives:

\begin{itemize}
  \item  \href{https://github.com/sponsors}{https://github.com/sponsors}
  \item  \href{https://patreon.com}{https://patreon.com}
  \item  \href{https://donorbox.org}{https://donorbox.org}
  \item  \href{https://paypal.me}{https://paypal.me}
  \item  and many others...
\end{itemize}

\img{first-steps/subscription-links}{List of external subscriptions}{img:fs-links}

\noindent Implementation of external links \issue{20}{} is possible by \q{url\_launcher}-package (\cref{img:fs-links}):

\begin{lstlisting}
import 'package:url_launcher/url_launcher.dart';

class OtherWidget extends StatelessWidget {
  const OtherWidget({super.key});

  Future<void> _launchURL(String url) =>
    launchUrl(Uri.parse(url), mode: LaunchMode.externalApplication);

  Widget build(BuildContext context) {
    return ElevatedButton(
      onPressed: () => _launchURL(
          'https://buymeacoffee.com/lyskouski'),
      child: Text('Buy me a coffee'),
    ),
  }
}
\end{lstlisting}


\subsubsection{Enabling Purchases}

In-app purchases are required for mobile app stores since it cannot be used any third-party systems to handle payments
by the terms and conditions of marketplaces. As, to work with Google and Apple subscriptions we need to be registered 
there as a trusted developers / companies. Google Play Developer account 
(\href{https://play.google.com/console}{https://play.google.com/console}; set up and create a subscription product in 
the "Store Presence" section) would costs us 25 Euro at once, Apple Store Developer account 
(\href{https://developer.apple.com}{https://developer.apple.com}; via Apple Developer account create App ID and 
configure the in-app purchase settings) -- roughly 100 Euro per year. Don't forget about other marketplaces as 
Galaxy and Huawei Store.

There would be a couple of steps to verify our identity, add bank account, sign documents (tax forms) and agreements.
As an example, for Apple Store:
\begin{enumerate}
  \item Activate paid apps on page:\\
  \href{https://appstoreconnect.apple.com/agreements}{https://appstoreconnect.apple.com/agreements}
  \item Register Bundle ID:\\
  \href{https://developer.apple.com/account/resources/identifiers/bundleId/add/bundle}{https://developer.apple.com/.../add/bundle}
  \item Fulfill app definition:\\
  \href{https://appstoreconnect.apple.com/apps}{https://appstoreconnect.apple.com/apps}
  \item Via "Features" $\rightarrow$ "In-App Purchases" [Create] and "Subscriptions" [Create] declare costs
\end{enumerate}

\noindent As a remark, we might enroll "Small-Business Support" to minimize marketplace deductions:
\href{https://developer.apple.com/app-store/small-business-program/enroll/}{https://developer.apple.com/app-store/small-business-program/enroll/}.

In-app purchases are classified based on their usage: consumables (can be used multiple times), non-consumables 
(one-time use), and subscriptions (recurring payments). And three different ways to integrate them into Flutter 
application: \q{in\_app\_purchase}, \q{flutter\_inapp\_purchase}, \q{purchases\_flutter}. 

As for the first one, the CodeLabs provides a step-by-step manual in details:
\href{https://codelabs.developers.google.com/codelabs/flutter-in-app-purchases}{https://codelabs.developers.google.com/codelabs/flutter-in-app-purchases}.

\img{first-steps/subscription}{Apple TestFlight -- Subscription check \issue{100}{}}{img:d-apple-review}

\noindent The potential downside of using the \q{in\_app\_purchase}-package is that it might result in build failures 
for \q{macOS}. This could occur due to the requirement for a minimum macOS version of 10.15 for deployment:

\begin{lstlisting}
// ./macos/Runner.xcodeproj/project.pbxproj
\end{lstlisting}
{
\xpretocmd{\lstlisting}{\vspace{-12pt}}{}{}
\begin{lstlisting}[firstnumber=2, backgroundcolor=\color{backred}]
(*@\kdiff{-}@*)   MACOSX_DEPLOYMENT_TARGET = <any other version>;
\end{lstlisting}
\begin{lstlisting}[firstnumber=2, backgroundcolor=\color{backgreen}]
(*@\kdiff{+}@*)   MACOSX_DEPLOYMENT_TARGET = 10.15;
\end{lstlisting}
}

\noindent And, it still won't work without \q{./macos/Podfile}-file defined, hopefully with a universal content 
\issue{100}{05329b3}.