% Copyright 2023 The terCAD team. All rights reserved.
% Use of this content is governed by a CC BY-NC-ND 4.0 license that can be found in the LICENSE file.

\subsection{[TBD] Refactoring by Principles}
\markboth{Unleashing Features}{Refactoring by Principles}

Having scrutinized the application's "skeleton" encompassing scalability and fault tolerance (\ref{benchmark}), it's 
now an opportune juncture to delve into its "nervous system" -- ensuring alignment with architectural and code style 
principles.\\
\\

\noindent Architectural Principles:

\begin{itemize}
  \item Architectural modularity \cite{Rich20} emphasizes breaking down a software system into distinct modules or 
  components. Each module should have a clear and well-defined purpose, promoting easier development, testing, and 
  maintenance.

  \item Layered architecture \cite{Rich22} divides a system into logical layers, each responsible for specific tasks. 
  This separation enhances code organization and facilitates the isolation of concerns.

  \item Separation of Concerns (SoC) advocates segregating different aspects of functionality to prevent overlap. By 
  example, Clean Architecture \cite{Mart18}, as an architectural pattern, introduces a distinctive approach to code 
  structuring, fostering a meticulously organized framework that seamlessly aligns with software development principles. 
  At its core, Clean Architecture reimagines the arrangement of code into distinct layers, each bearing a clearly 
  defined and specialized responsibility. These layers are demarcated by robust boundaries that serve to insulate 
  inner layers from the influences of outer ones. This strategic isolation promotes a coherent structure that safeguards 
  the integrity of the system, even as it evolves and undergoes modifications over time. This minimizes code 
  entanglement, making it easier to modify, debug, and maintain individual components and layers.

  \item Decoupled systems \cite{Kass05} have minimal dependencies between components. This flexibility enables easier 
  upgrades, replacements, and integrations without causing ripple effects.
\end{itemize}

\noindent Code Style Principles \cite{Mart22}:

\begin{itemize}
  \item Consistent code style promotes uniformity in formatting, naming conventions, and overall structure. 
  This aids in enhancing code readability and collaboration among developers.

  \item The DRY (Don't Repeat Yourself) principle advocates avoiding duplications in code.

  \item Simplicity is key (KISS). Complex solutions are harder to understand and maintain. Embrace simplicity unless a 
  more complex approach is justified.

  \item SOLID Principles, as an acronym, representing a set of five design principles: Single Responsibility, 
  Open-Closed, Liskov Substitution, Interface Segregation, and Dependency Inversion.
\end{itemize}

\noindent We'll not dive into details of architectural and code style principles (check references above), and even 
won't discuss made changes \issue{159}{}, by reminding only the semaphore approach (\ref{ut-fail}) that declares 
refactoring only after a significant code coverage. Refactoring as a process of rearchitecting \cite{Chec23} previously
made decisions, to restate, revise and improve them by visualizing in a clean code. 

In the context of extreme programming, the methodology includes a provision for pair programming \cite{Ligu19}. This 
involves two programmers collaborating closely at a single workstation, communicating actively throughout the process 
of writing each line of code. Delighting to restate and extremize that practice to next flow: the first developer 
composing tests, followed by the second developer undertaking the implementation, and the third developer 
performing an ultimate refactoring; and they're changing those "hats" \cite{Bono17} continuously.
