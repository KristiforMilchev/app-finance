% Copyright 2023 The terCAD team. All rights reserved.
% Use of this content is governed by a CC BY-NC-ND 4.0 license that can be found in the LICENSE file.

In the realm of transient concepts, many individuals tend to become perplexed. The truth is, nearly every idea we 
utilize is an amalgamation of numerous insights contributed by diverse individuals spanning a considerable span of 
time. Pioneering ideas entirely from the ground up often proves to be a futile endeavor. Instead, the optimal approach 
involves assimilating the finest aspects of what has already been established and then enhancing upon that foundation
(\cite{Azar22}, \cite{Page19}, \cite{Bara18}, \cite{Kleo12}, \cite{Thag12}, \cite{John11}).

There can be different strategies for the product creation \cite{Lomb17} but we're not going to concentrate on that,
and declare a Disruptive Product with an initial phase in three months (since it's a free-spot time allocation
for a single developer, me; approximately 200 hours). Disruptive strategy is stand out by providing a simple, 
cost-effective solutions to customers' challenges. 

For instance, when Netflix was introduced, it didn't immediately disrupt the market due to the lack of instant 
gratification for customers who were used to buying movies from Blockbuster stores. However, as Netflix 
improved its service, reducing movie delivery time and eventually introducing online streaming, it ultimately led 
to the downfall of Blockbuster stores \cite{Eby17}.

\paragraph{Project Assumption} In today's fast-paced world, managing personal finances has become increasingly 
important. Whether it's tracking expenses, setting budgets, or monitoring investments, individuals are seeking for an 
efficient and intuitive solution to keep their financial lives in order (take control of finances).


\subsection{Simplifying User Interface}
\markboth{Conceptualizing}{Simplifying User Interface}

One of the fundamental principle in designing an application is a simplicity. The application should have a clean and 
intuitive interface, allowing users to effortlessly navigate through various sections and perform tasks. We should avoid 
overwhelming users with unnecessary complexities and focus on providing a streamlined experience that caters their 
specific needs. The main goal is to create the application that can be easily understood and used by a wide range of 
users, regardless of their level of technical expertise.

A \textbf{Minimalistic Design} philosophy guides this endeavor. By keeping the user interface uncluttered and avoiding 
excessive information or visual elements, the application becomes visually appealing and organized. White space, clear 
typography, and a consistent color scheme contribute to a visually pleasing and organized interface.
Furthermore, clear navigation is essential. The navigation within the application should be intuitive and 
straightforward. This involves using descriptive labels for navigation elements like tabs, menus, and buttons. 
The logical organization of different sections and features ensures that users can find what they need without 
confusion or frustration. By maintaining the consistent placement of buttons, menus, and interactive elements, users 
can develop a mental model of the application interface. This allows them to predict how different actions and 
interactions will unfold based on their previous experiences within the app.

\textbf{Task-Oriented Design} (TOD) places specific tasks and goals at the forefront. The application is designed 
around these tasks, providing clear and easily accessible options for the most commonly performed scenarios.
For instance, when it comes to Smart Speakers, we shift from a Graphical User Interface (GUI) to a Voice User Interface 
(VUI). The overarching objective of Task-Oriented Design is to create a solution that maintains a cohesive product 
identity across various platforms while tailoring features to suit specific capabilities of each device type.
Within the framework of Task-Oriented Design, the application seamlessly integrates contextual assistance and guidance. 
This encompasses the deployment of tooltips, on-screen cues, and informative messages, strategically delivering 
pertinent information when it is needed. Efficient data entry is facilitated through a range of functionalities, 
including auto-suggestions, pre-filled forms coupled with input assistance validation, and intelligent categorization 
through the application of smart defaults. Application feedback reassures users that their inputs have been successfully 
registered and processed. This adds to the overall user confidence in the application.

By observing how users interact with the application and identifying pain points, the design can be refined based on 
their input. The iterative approach allows addressing usability issues and continually enhancing the user experience. 
So, the user testing and iteration further refine the application, ultimately creating an interface that caters to a 
diverse user base.

Last but not least, the application should be accessible to users with disabilities, by following \textbf{Accessibility 
Guidelines and Standards}. This ensures that individuals with visual impairments, motor limitations, or other 
accessibility needs can use the application. Options for adjusting font sizes, color contrasts, and support for 
screen readers are essential elements of this inclusivity. It's worth noting that approximately 15\% of the global 
population faces some form of disability, and regrettably, over 80\% of websites and applications fall short in 
addressing their needs \cite{Worl11}. While achieving the full compliance might be challenging (\emph{targeted 
automated testing frameworks can catch less than 25\% of accessibility issues}, \cite{Univ22}), adherence to these 
principles remains of paramount importance.

\img{concept/apple-calendar}{Task-Oriented Design -- Apple Calendar \cite{Thal20}}{img:fs-calendar}

\subsection{Hierarching Information}
\markboth{Conceptualizing}{Hierarching Information}

A well-designed financial accounting application must establish a clear information hierarchy. It's imperative to place 
critical financial data, such as account balances, transactions, and budgets, in the front and center, ensuring that 
users can access these key details with ease. That reduces a cognitive load by making it easier to understand the status 
and make informed decisions.

To construct the information effectively we need to identify the most crucial data that users should readily see and 
rely on, and prominently display it on the application's main screen. By keeping things organized and user-friendly, 
we should categorize and group data logically. This involves utilizing clear headings and visual cues to differentiate 
between various categories, simplify data retrieval. So, let's harnesses the power of a \textbf{Visual Hierarchy}. 
It guides users' attention to important details by employing tactics such as larger fonts, contrasting colors, and 
meticulous typography. 

Utilization of a \textbf{Progressive Disclosure} is started from offering users a high-level overview, with the 
flexibility to delve deeper for more detailed information by using intuitive gestures and interactions. This approach 
ensures that users can access the level of detail that aligns with their needs without being overwhelmed with complexity. 
This is achieved by maintaining a consistent layout and organizational structure across different screens and sections 
of the application without relearning the interface. Our objective is to streamline data entry, minimize user effort, 
and enhance user understanding. By offering the contextual information and explanations where needed, we aid users in 
understanding the significance of different terms or calculations, complex concepts or unfamiliar jargon. We ensure 
that labels accurately represent the actions or functions they perform and complement these labels with appropriate 
icons for quick recognition and comprehension. This entire user-friendly interface is made complete by its adaptability 
to various screen sizes and orientations that users may employ.


\subsection{Onboarding}
\markboth{Conceptualizing}{Onboarding}

The onboarding process in a financial accounting application is a critical step in ensuring user engagement and 
long-term retention. Its primary goal is to guide users through the initial setup and familiarize them with the 
key features of the application. A well-designed onboarding process boosts user confidence, minimizes friction, and 
lays the foundation for lasting engagement and satisfaction.

The journey begins with a concise yet compelling introduction, clearly outlining the application's value proposition 
and highlighting its distinctive features. This initial introduction is crucial in setting the stage for what users 
can expect. Following the introduction, a guided setup process is introduced. This process is thoughtfully divided 
into manageable stages, with each step presented one at a time. Explanatory text and visual cues accompany each step, 
ensuring users can easily navigate through the setup process. Each onboarding step is complemented by clear 
calls-to-action, providing users with explicit instructions on what they need to do next. These intuitive prompts 
make it simple for users to complete each task. Indicator panels are thoughtfully integrated to help users track 
their progress through the onboarding process. These visual cues reassure users that they're making headway and help 
manage their expectations regarding the remaining steps. The motivation and engagement are boosted through quick wins 
and rewards, fostering a positive user experience that encourages users to continue using the application.

As the onboarding process concludes, users are invited to provide feedback. This open channel for user input, questions, 
and issue reporting is invaluable for identifying areas of improvement and enhancing the evolving onboarding experience.

By incorporating all of these elements into the onboarding process, the application aims to enhance user understanding, 
satisfaction, and long-term engagement, setting the stage for a successful and user-friendly experience.


\subsection{Personalizing Options}
\markboth{Conceptualizing}{Personalizing Options}

Empowering users to tailor the application to their unique requirements, needs, and preferences is at the heart of 
personalization options. One aspect of this personalization involves accommodating diverse users by providing support 
for multiple languages (and currencies). This ensures that users feel comfortable (a data interpretation) and fosters 
a sense of ownership.

Customization extends to categories, allowing users to create and personalize data sections. These tags help improve 
data accuracy and relevance. Users can define budget limits for specific categories, restating frequencies like monthly 
or weekly to allocate funds effectively. The convenience of backing up their data and syncing it across multiple 
devices adds a further layer of personalization. Moreover, users can benefit from the ability to export their data in 
various formats and create customizable report templates.

Customizable \textbf{Dashboards} offer the flexibility to select the main page's content. Users can choose what 
information and widgets they want to see at a glance, arranging them according to their priorities. The ability to 
choose from various color schemes, fonts, and layouts further enhances the tailored experience. Customization in the 
application goes beyond layout changes. It extends to notifications, allowing users to personalize their experience 
by setting preferences for progress tracking, reminders, and insights. These tailored notifications are designed to 
keep users motivated and on track toward achieving their unique financial goals. Whether it's saving for a dream 
vacation, paying off debt, or planning for retirement, users can define their aspirations. The beauty of this feature 
lies in the granular control it provides over notification preferences, ensuring users stay informed without being 
overwhelmed, making their financial journey both effective and personalized.


\subsection{Securing Information}
\markboth{Conceptualizing}{Securing Information}

Securing financial data is a top priority in the design of any application. We should understand the sensitive nature 
of the information users entrust us with and are committed to ensuring its safety. Financial data is highly sensitive, 
and users need the assurance that their information is secure (such as the usage of end-to-end encryption and 
two-factor authentication). By prioritizing secure data handling practices, we may establish trust with users, mitigate 
risks, and safeguard a sensitive information. That can be made even more transparent and trustful by a regular security 
audits. These audits help us identify vulnerabilities and weaknesses in our application's infrastructure and codebase. 
That helps us to guarantee prompt correction of any known (identified) vulnerabilities.

Our application should employ robust \textbf{encryption} techniques to protect user data both in transit and analyses. 
This means that sensitive information, like account credentials, transaction details, and personal identifiers, is 
encrypted using industry-standard algorithms. Encryption adds an extra layer of protection and ensures that even if 
data is compromised, it remains unintelligible to unauthorized parties.

\textbf{Multi-Factor Authentication} (MFA) adds another level of confidence to user identity verification by requiring
a combination of something the user knows (a password), something they have (like a token or mobile device), or 
something they are (biometrics). With these stringent authentication procedures, users can trust that their accounts 
are well-protected. 

By implementing \textbf{Role-Based Access Control} (RBAC), we can precisely restrict access to sensitive data and critical 
functionalities according to predefined permissions. This ensures that only authorized personnel can interact with 
specific parts of the application, enhancing security and data protection. That allows us to meticulously manage access 
permissions by assigning roles to individuals. The robust secure storage mechanism as data backups will mitigate the 
risks in scenarios such as system failures.

Finally, \textbf{Privacy Policy} should be transparent and explained in details to the users. It needs to be clear with
regards to the types of data we collect, its purpose, and who can access it. We maintain transparency in how we secure 
user information and respond to security incidents, with user-friendly channels for reporting concerns. This commitment 
ensures that users are well-informed about their data and can trust our unwavering dedication to data security. User 
consent is an essential aspect of our data handling process. We should be explicit about how user information is 
safeguarded. The protocols for addressing security incidents are easily accessible for users to voice any concerns. So, 
the transparency is essential in conveying how the user information is protected and security incidents are managed.


\subsection{Visualizing Data}
\markboth{Conceptualizing}{Visualizing Data}

A visual representation of data empower users to grasp information more effectively and gain better control over their 
financial situation.

We should start from creating a captivating overview \textbf{Dashboard} that provides a high-level summary of the 
information through charts, graphs, and key performance indicators (KPIs). These visuals illustrate essential data, 
such as account balances, income versus expenses, savings progress, or net worth.  By incorporating interactive features 
as zooming, panning, and filtering, we enable users to explore data in-depth. Visual indicators, such as progress bars 
and thermometers, are valuable tools for tracking progress toward goals and budget adherence. 

Metrics help to effectively compare different financial metrics or periods using side-by-side visualizations like bar 
charts, line graphs, or stacked area charts. With that users gain clarity on their spending habits, stay within 
their financial targets, and quickly identify areas of focus or anomalies within their data.


\subsection{Integrating Services}
\markboth{Conceptualizing}{Integrating Services}

To provide users with a comprehensive financial tracking experience, the application can be integrated with financial 
institutions and services. This integration enables users to seamlessly connect their bank accounts, credit cards, and 
investment platforms, facilitating the automatic import of transaction data. Synchronization features ensure real-time 
updates and effortless reconciliation between the application and external financial sources. By implementing secure 
industry-standard protocols like Open Banking APIs, the application establishes a reliable and secure connection with 
financial institutions. Regular updates of account balances and transaction information provide users with real-time 
visibility into their financial standing, supporting various account types, from checking and savings accounts to 
credit cards and investment accounts. 

Stringent data security measures, including encryption and secure data transmission, along with compliance with data 
protection regulations such as GDPR or CCPA, ensure user data remains safe. Whereas a robust error handling mechanism 
addresses connectivity issues, promptly notifying users of any encountered problems and offering clear instructions for 
resolution. 

Users should be able to do transactions as verified or reconciled once cross-checked with their official bank statements, 
enhancing data accuracy and aiding in the identification of discrepancies or missing transactions. This comprehensive 
integration sets the stage for a holistic and effective financial tracking experience.


\subsection{Shaking the Market}
\markboth{Conceptualizing}{Shaking the Market}

The application has a potential to stand out in the market by identifying areas where users can reduce costs, 
providing valuable recommendations, and highlighting opportunities for optimizing their budgets. This capability 
sets it apart by actively assisting users in managing their finances more efficiently.

For instance, the Automated Expense Categorization feature streamlines the categorization process by automatically 
assigning categories like groceries, transportation, entertainment, bills, and more, based on transaction data. This 
eliminates the need for manual categorization, saving users significant time and effort and ensuring accurate expense 
tracking.

Real-time Expense Insights provide users with a comprehensive view of their spending patterns through visualizations, 
charts, and graphs. These tools highlight monthly spending trends, category breakdowns, and comparisons with previous 
periods. Incorporating forecasting and predictive analytics into these visualizations assists users in planning and 
decision-making. Trend lines project future income, expenses, or savings based on historical data, allowing users to 
anticipate future financial scenarios and make informed choices.

The \textbf{Intelligent Notifications} feature harnesses the power of artificial intelligence (AI) to offer users a 
personalized and proactive financial management experience. It identifies anomalies, like unexpected high expenses, 
and can even detect irregularities that might indicate potential issues, such as unauthorized charges. AI should offer 
actionable financial insights by suggesting adjustments (potential cost-cutting measures) to users' budget based on 
their spending patterns.


Furthermore, the application with employed \textbf{Optical Character Recognition} (OCR) technology may extract data 
from receipts. This means that users can simply take a photo of a receipt, and the application will automatically 
extract key information, making expense tracking even more user-friendly and efficient.


\subsection{Enhancing Continually}
\markboth{Conceptualizing}{Enhancing Continually}

Encouraging a culture of continuous improvement is vital to maintaining user loyalty and satisfaction. Actively seek 
feedback from users to identify areas for enhancement and regularly update the application to improve the overall user 
experience. User feedback serves as a valuable source of insights for enhancing functionality, usability, and the 
application's overall user experience. This iterative approach ensures that the application evolves and remains 
pertinent in a dynamically changing landscape, ultimately leading to increased user satisfaction and loyalty.

Diverse channels are provided for users to share their feedback, suggestions, and concerns, including in-app feedback 
forms, email support, community forums, or social media channels. Users can easily voice their opinions and are 
actively encouraged to provide feedback on their experience with the application, and even upvote others' ideas in a 
feature request system. That can be improved by periodic surveys and questionnaires conducted to gather insights 
from users, specifically focusing on their satisfaction level, pain points, desired features, and overall user 
experience. This feedback is instrumental in identifying areas for improvement and prioritizing development efforts 
based on user needs and preferences.

Analytics tools might help to track a user behavior within the application, monitoring their navigation through 
different screens, identifying drop-off points, and tracking usage patterns. On the other hand, \textbf{Usability 
Studies}, which encompass a diverse group of users, provide valuable insights. These studies enable the observation of 
user interactions with the application, gather feedback on specific features or workflows, and are instrumental in 
pinpointing areas where users may encounter challenges or confusion. Together, these approaches create a comprehensive 
understanding of the user experience, helping to shape continuous improvements in the application. Also, it might  
engage enthusiastic users to gather feedback on upcoming features or major updates. Beta testers provide valuable 
insights, uncover edge cases, and help identify issues before a wider release.

A seamless bug reporting process is an essential component of any application. Reported issues should be promptly 
acknowledged, and transparent communication should be upheld throughout the resolution process. This commitment to 
user feedback and continuous improvement ensures the development of a dynamic and \textbf{User-Centric Application}, 
placing users at the core of the improvement journey. In addition, release notes or in-app notifications serve as clear 
channels for communicating updates, enhancements, and bug fixes, further demonstrating the commitment to active
listening to users.\\
\\

\noindent In the end, our mission is clear: create a financial accounting application that is intuitive, efficient, 
and inclusive. Through thoughtful design and unwavering attention to detail, we're shaping an experience that empowers 
users to effortlessly manage their finances while ensuring that no one is left behind.
