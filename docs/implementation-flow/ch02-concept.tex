% Copyright 2023 The terCAD team. All rights reserved.
% Use of this content is governed by a CC BY-NC-ND 4.0 license that can be found in the LICENSE file.

In the realm of transient concepts, many individuals tend to become perplexed. The truth is, nearly every idea we 
utilize is an amalgamation of numerous insights contributed by diverse individuals spanning a considerable span of 
time. Pioneering ideas entirely from the ground up often proves to be a futile endeavor. Instead, the optimal approach 
involves assimilating the finest aspects of what has already been established and then enhancing upon that foundation
(\cite{John11}, \cite{Azar22}, \cite{Page19}, \cite{Bara18}, \cite{Kleo12}, \cite{Thag12}).

There can be different strategies for the product creation \cite{Lomb17} but we're not going to concentrate on that,
and declare a Disruptive Product with an initial phase in three months (since it's a free-spot time allocation
for a single developer, me; approximately 200 hours). Disruptive strategy is stand out by providing a simple, 
cost-effective solutions to customers' challenges. 

For instance, when Netflix was introduced, it didn't immediately disrupt the market due to the lack of instant 
gratification for customers who were used to buying movies from Blockbuster stores. However, as Netflix 
improved its service, reducing movie delivery time and eventually introducing online streaming, it ultimately led 
to the downfall of Blockbuster stores \cite{Eby17}.

\paragraph{Project Assumption} In today's fast-paced world, managing personal finances has become increasingly 
important. Whether it's tracking expenses, setting budgets, or monitoring investments, individuals are seeking for an 
efficient and intuitive solution to keep their financial lives in order (take control of finances).


\subsection{Simplifying User Interface}
\markboth{Conceptualizing}{Simplifying User Interface}

One of the fundamental principle in designing an application is a simplicity. The application should have a clean and 
intuitive interface, allowing users to effortlessly navigate through various sections and perform tasks. We should avoid 
overwhelming users with unnecessary complexities and focus on providing a streamlined experience that caters to 
their specific needs. Since the main goal is to create the application that can be easily understood and used by a wide 
range of users, regardless of their level of technical expertise.

A \textbf{Minimalistic Design} philosophy guides this endeavor. By keeping the user interface uncluttered and avoiding 
excessive information or visual elements, the application becomes visually appealing and organized. White space, clear 
typography, and a consistent color scheme contribute to a visually pleasing and organized interface.
Furthermore, clear navigation is essential. The navigation within the application should be intuitive and 
straightforward. This involves using descriptive labels for navigation elements like tabs, menus, and buttons. 
The logical organization of different sections and features ensures that users can find what they need without 
confusion or frustration. By maintaining the consistent placement of buttons, menus, and interactive elements, users 
can develop a mental model of the application interface. This allows them to predict how different actions and 
interactions will unfold based on their previous experiences within the app.

\textbf{Task-Oriented Design} (TOD) places specific tasks and goals at the forefront. The application is designed 
around these tasks, providing clear and easily accessible options for the most commonly performed scenarios.
For instance, when it comes to Smart Speakers, we shift from a Graphical User Interface (GUI) to a Voice User Interface 
(VUI). The overarching objective of Task-Oriented Design is to create a solution that maintains a cohesive product 
identity across various platforms while tailoring features to suit the specific capabilities of each device type.
Within the framework of Task-Oriented Design, the application seamlessly integrates contextual assistance and guidance. 
This encompasses the deployment of tooltips, on-screen cues, and informative messages, strategically delivering 
pertinent information precisely when it is needed. Efficient data entry is facilitated through a range of 
functionalities, including auto-suggestions, pre-filled forms coupled with input assistance validation, and 
intelligent categorization through the application of smart defaults. Where as an application feedback reassures users 
that their inputs have been successfully registered and processed. This adds to the overall user confidence in the 
application.

By observing how users interact with the application and identifying pain points, the design can be refined based on 
their input. The iterative approach allows for addressing usability issues and continually enhancing the user 
experience. So, the user testing and iteration further refine the application, ultimately creating an interface that 
caters to a diverse user base.

Last but not least, the application should be accessible to users with disabilities, by following \textbf{Accessibility 
Guidelines and Standards}. This ensures that individuals with visual impairments, motor limitations, or other 
accessibility needs can use the application. Options for adjusting font sizes, color contrasts, and support for 
screen readers are essential elements of this inclusivity. It's worth noting that approximately 15\% of the global 
population faces some form of disability, and regrettably, over 80\% of websites and applications fall short in 
addressing their needs \cite{Worl11}. While achieving the full compliance might be challenging (\emph{targeted 
automated testing frameworks can catch less than 25\% of accessibility issues}, \cite{Univ22}), adherence to these 
principles remains of paramount importance.

\img{concept/apple-calendar}{Task-Oriented Design -- Apple Calendar \cite{Thal20}}{img:fs-windows-path}

\subsection{Hierarching Information}
\markboth{Conceptualizing}{Hierarching Information}

A well-designed financial accounting application must establish a clear information hierarchy. It's imperative to place 
critical financial data, such as account balances, transactions, and budgets, front and center, ensuring that users can 
access these key details with ease. That reduces a cognitive load by making it easier to understand the status and make 
informed decisions.

To construct the information effectively we need to identify the most crucial data that users should readily see and 
rely on, and prominently display it on the application's main screen. By keeping things organized and user-friendly, 
we should categorize and group data logically. This involves utilizing clear headings and visual cues to differentiate 
between various categories, making data retrieval a breeze. So, let's harnesses the power of a \textbf{Visual Hierarchy}. 
It guides users' attention to important details by employing tactics such as larger fonts, contrasting colors, and 
meticulous typography. 

Utilization of a \textbf{Progressive Disclosure} is started from offering users a high-level overview, with the 
flexibility to delve deeper for more detailed information by using intuitive gestures and interactions. This approach 
ensures that users can access the level of detail that aligns with their needs without being overwhelmed by complexity. 
This is achieved by maintaining a consistent layout and organizational structure across different screens and sections 
of the application without relearning the interface. Our objective is to streamline data entry, minimize user effort, 
and enhance user understanding. By offering the contextual information and explanations where needed, we aid users in 
understanding the significance of different terms or calculations, complex concepts or unfamiliar jargon. We ensure 
that labels accurately represent the actions or functions they perform and complement these labels with appropriate 
icons for quick recognition and comprehension. This entire user-friendly interface is made complete by its adaptability 
to various screen sizes and orientations that users may employ.


\subsection{Onboarding}
\markboth{Conceptualizing}{Onboarding}

The onboarding process in a financial accounting application is a critical step in ensuring user engagement and 
long-term retention. Its primary goal is to guide users through the initial setup and familiarize them with the 
key features of the application. A well-designed onboarding process boosts user confidence, minimizes friction, and 
lays the foundation for lasting engagement and satisfaction.

The journey begins with a concise yet compelling introduction, clearly outlining the application's value proposition 
and highlighting its distinctive features. This initial introduction is crucial in setting the stage for what users 
can expect. Following the introduction, a guided setup process is introduced. This process is thoughtfully divided 
into manageable stages, with each step presented one at a time. Explanatory text and visual cues accompany each step, 
ensuring users can easily navigate through the setup process. Each onboarding step is complemented by clear 
calls-to-action, providing users with explicit instructions on what they need to do next. These intuitive prompts 
make it simple for users to complete each task. Progress indicators are thoughtfully integrated to help users track 
their progress through the onboarding process. These visual cues reassure users that they're making headway and help 
manage their expectations regarding the remaining steps. The motivation and engagement are boosted through quick wins 
and rewards, fostering a positive user experience that encourages users to continue using the application.

As the onboarding process concludes, users are invited to provide feedback. This open channel for user input, questions, 
and issue reporting is invaluable for identifying areas of improvement and enhancing the evolving onboarding experience.

By incorporating all of these elements into the onboarding process, the application aims to enhance user understanding, 
satisfaction, and long-term engagement, setting the stage for a successful and user-friendly experience.


\subsection{Personalizing Options}
\markboth{Conceptualizing}{Personalizing Options}

Empowering users to tailor the application to their unique requirements, needs, and preferences is at the heart of 
personalization options. One aspect of this personalization involves accommodating diverse users by providing support 
for multiple languages (and currencies). This ensures that users feel comfortable (a data interpretation) and fosters 
a sense of ownership.

Customization extends to categories, allowing users to create and personalize data sections. These tags help improve 
data accuracy and relevance. Users can define budget limits for specific categories, restating frequencies like monthly 
or weekly to allocate funds effectively. The convenience of backing up their data and syncing it across multiple 
devices adds a further layer of personalization. Moreover, users can benefit from the ability to export their data in 
various formats and create customizable report templates.

Customizable \textbf{Dashboards} offer the flexibility to select the main page's content. Users can choose what 
information and widgets they want to see at a glance, arranging them according to their priorities. The ability to 
choose from various color schemes, fonts, and layouts further enhances the tailored experience. Customization in the 
application goes beyond layout changes. It extends to notifications, allowing users to personalize their experience 
by setting preferences for progress tracking, reminders, and insights. These tailored notifications are designed to 
keep users motivated and on track toward achieving their unique financial goals. Whether it's saving for a dream 
vacation, paying off debt, or planning for retirement, users can define their aspirations. The beauty of this feature 
lies in the granular control it provides over notification preferences, ensuring users stay informed without being 
overwhelmed, making their financial journey both effective and personalized.


\subsection{Securing Information}
\markboth{Conceptualizing}{Securing Information}

Securing financial data is a top priority in the design of any application. We should understand the sensitive nature 
of the information users entrust us with and are committed to ensuring its safety. Financial data is highly sensitive, 
and users need the assurance that their information is secure (such as the usage of end-to-end encryption and 
two-factor authentication). By prioritizing secure data handling practices, we may establish trust with users, mitigate 
risks, and safeguard a sensitive information. That can be made even more transparent and trustful by a regular security 
audits. These audits help us identify vulnerabilities and weaknesses in our application's infrastructure and codebase. 
That helps us to guarantee prompt correction of any known (identified) vulnerabilities.

Our application should employ robust \textbf{encryption} techniques to protect user data both in transit and at rest. 
This means that sensitive information, like account credentials, transaction details, and personal identifiers, is 
encrypted using industry-standard algorithms. Encryption adds an extra layer of protection and ensures that even if 
data is compromised, it remains unintelligible to unauthorized parties.

\textbf{Multi-Factor Authentication} (MFA) adds another level of confidence to user identity verification by requiring
a combination of something the user knows (a password), something they have (like a token or mobile device), or 
something they are (biometrics). With these stringent authentication procedures, users can trust that their accounts 
are well-protected. 

By implementing \textbf{Role-Based Access Control} (RBAC), we can precisely restrict access to sensitive data and critical 
functionalities according to predefined permissions. This ensures that only authorized personnel can interact with 
specific parts of the application, enhancing security and data protection. That allows us to meticulously manage access 
permissions by assigning roles to individuals. With that we have to establish robust secure storage mechanisms to 
safeguard user data, as data backups to mitigate the risks in scenarios such as system failures.

Finally, \textbf{Privacy Policy} should be transparent and explained in details to users with information about the 
types of data we collect, its purpose, and who can access it. Obtaining explicit user consent for data collection and 
processing is integral to our approach. We maintain transparency in how we secure user information and respond to 
security incidents, with user-friendly channels for reporting concerns. This commitment ensures that users are 
well-informed about their data and can trust our unwavering dedication to data security. Our privacy policy is 
characterized by transparency and detail, where users receive information about the categories of data we collect, 
its utilization, and the parties with access. User consent is an essential aspect of our data handling process, and 
we should be explicit about how user information is safeguarded and our protocols for addressing security incidents, 
providing easily accessible channels for users to voice any concerns. Transparency is essential in conveying how user 
information is protected and security incidents are managed.



\subsection{Visualizing Data}
\markboth{Conceptualizing}{Visualizing Data}

Well-designed visualization enhances a data comprehension, engagement, and the overall experience, by making it easier 
for users to take a control over the data clarity.

\paragraph{Overview Dashboards}
Create visually appealing and informative overview dashboards that provides a high-level summary of the information. 
Use charts, graphs, and key performance indicators (KPIs) to present data such as account balances, income versus 
expenses, savings progress, or net worth. Utilize color-coded icons, symbols, or illustrations to indicate transaction 
types (income, expense, transfer) or payment methods (cash, credit card). Consider displaying transaction details such 
as date, category, amount, and account in a concise and visually appealing manner. Include interactive features like 
zooming or panning by allowing users to explore data in details, and options to choose the visual representation in
accordance with to preferences.

\paragraph{Progress Tracking}
Use progress bars, thermometers, or visual indicators to track users' progress toward their goals and expectations. 
Represent the budgeted amount versus actual expenses visually to provide users with a clear understanding of their 
spending habits and help them stay within the targets.

\paragraph{Forecasting Analytics}
Incorporate forecasting and predictive analytics into visualizations to assist users in planning and decision-making.
Use trend lines to project future income, expenses, or savings based on historical data. Predictive analytics can help 
users anticipate future financial scenarios and make informed financial choices.

\paragraph{Interactive Filters}
Enable drill-down functionality to delve into detailed information by interacting with specific data points or 
categories.

\paragraph{Comparative Analysis}
Enable users to compare different financial metrics or periods side by side using visualizations like bar charts, 
line graphs, or stacked area charts.

\paragraph{Heat Maps}
Utilize heat maps or color schemes to highlight data patterns or variations. Heat maps can help users quickly identify 
areas of focus or anomalies.


\subsection{Applying Accessibility}
\markboth{Conceptualizing}{Applying Accessibility}

Cross-platform accessibility ensures users can access their information anytime, anywhere, regardless of the devices 
they prefer to use.

\paragraph{Responsive Design}
Implement a responsive design approach to ensure that the application adapts and optimizes its layout and user 
interface based on the screen size and resolution of the device being used.

\paragraph{Native Mobile Apps}
Develop native mobile applications for popular platforms such as iOS and Android. Native apps provide a tailored 
user experience, leveraging the platform-specific features, performance optimizations, and user interface guidelines. 
Native apps can offer enhanced performance, offline capabilities, and seamless integration with device functionalities, 
providing users with a more optimized experience on their respective mobile platforms.

\paragraph{Web-Based Application}
Build a web-based version of the financial accounting application that can be accessed through web browsers on 
different devices. Ensure compatibility across major browsers, including Google Chrome, Mozilla Firefox, Safari, 
and Microsoft Edge. Design the web application to be user-friendly and intuitive, offering a consistent experience 
regardless of the operating system or device.

\paragraph{Data Synchronization}
Implement synchronization mechanisms that automatically update data in real-time, ensuring users have the latest 
information available on all their devices.

\paragraph{Consistent User Experience}
Ensure that the core functionalities, features, and navigation patterns remain consistent regardless of the platform or 
device being used. This allows users to familiarize themselves with the application quickly and reduces the learning 
curve when switching between devices.

\paragraph{Device-Specific Features}
Leverage device-specific features and capabilities to enhance the user experience on different platforms. For 
example, on mobile devices, leverage features like camera integration for scanning receipts, push notifications 
for transaction alerts, or biometric authentication.

\paragraph{Accessibility Standards}
Implement features such as adjustable font sizes, high contrast modes, screen reader compatibility, and keyboard 
navigation support. By making the application accessible, we may cater to a wider range of users.


\subsection{Shaking the Market}
\markboth{Conceptualizing}{Shaking the Market}

By incorporating the AI-powered Assistant, the application might differentiate itself in the market. That is supposed 
to simplify and automate the expense tracking process, delivers actionable insights, and empowers users to make 
smarter financial decisions. 

\paragraph{Automated Expense Categorization}
Automatically categorize expenses based on transaction data, eliminating the need for manual categorization. 
It analyzes transaction descriptions, amounts, and patterns to intelligently assign categories such as groceries, 
transportation, entertainment, bills, and more. Users can save a significant time and effort in manually categorizing 
each expense, making expense tracking effortless and accurate.

\paragraph{Receipt Scanning and OCR}
Use optical character recognition (OCR) technology to scan and extract data from receipts. Users can simply take a photo 
of a receipt, and the application will automatically extract key information.

\paragraph{Real-time Expense Insights}
Provide real-time insights and analytics on users' spending patterns. It offers visualizations, charts, and graphs that 
highlight monthly spending trends, category breakdowns, and comparisons with previous periods.

\paragraph{Intelligent Notifications}
AI analyzes users' income, expenses, and goals by providing proactive notifications to manage finances effectively, and 
avoid overspending.

\paragraph{Savings Suggestions}
It can be potentially identified areas where users can cut costs, offers recommendations and highlights opportunities 
for optimizing their budgets.


\subsection{Integrating Services}
\markboth{Conceptualizing}{Integrating Services}

To provide users with a comprehensive financial tracking experience, integrate the application with financial 
institutions and services. Enable users to connect their bank accounts, credit cards, and investment platforms 
to automatically import transaction data. Implement synchronization features to ensure real-time updates and 
seamless reconciliation between the application and external financial sources. This integration streamlines 
the process of tracking transactions, monitoring account balances, and gaining insights into their overall financial 
well-being.

\paragraph{Account Connectivity}
Implement secure and industry-standard protocols such as Open Banking APIs to establish a secure connection between 
the application and the financial institution's systems. Regularly update account balances and transaction information 
from connected financial institutions. Implement automated processes to fetch and update account balances, ensuring 
users have real-time visibility into their financial standing.

\paragraph{Multiple Account Types}
Support integration with a wide range of financial accounts, including checking accounts, savings accounts, credit 
cards, investment accounts, and loans. This enables users to centralize their financial data within a single 
application.

\paragraph{Data Security}
Adhere to industry-standard security practices, including encryption, secure data transmission, and compliance with 
data protection regulations such as GDPR or CCPA.

\paragraph{Error Handling}
Implement robust error handling mechanisms to handle potential connectivity issues or errors that may occur during the 
integration process. Notify users of any issues encountered during data retrieval, and provide clear instructions on 
how to resolve the issue.

\paragraph{Reconciliation}
Enable users to mark transactions as verified or reconciled once they have cross-checked them with their official bank 
statements. This ensures data accuracy and helps users identify any discrepancies or missing transactions.

\paragraph{Continuous Enhancements}
Stay informed about industry changes and ensure the application remains compatible with the latest standards and 
protocols.


\subsection{Improving Continuously}
\markboth{Conceptualizing}{Improving Continuously}

Encourage users to provide feedback and actively seek their input to identify areas for improvement. Regularly update 
the application by enhancing the user experience. This ongoing commitment to improvement will foster user loyalty and 
satisfaction. User feedback becomes a valuable source of insights and inspiration for enhancing the application's 
functionality, usability, and overall user experience. This iterative approach ensures that the application evolves 
and remains relevant in a rapidly changing landscape, ultimately leading to increased user satisfaction and loyalty.

\paragraph{Feedback Channels}
Provide various channels for users to share their feedback, suggestions, and concerns. Offer options such as in-app 
feedback forms, email support, community forums, or social media channels. Make it easy for users to voice their 
opinions (a feature request system where users can submit and upvote ideas) and actively encourage them to provide 
feedback on their experience with the application.

\paragraph{Surveys and Questionnaires}
Conduct periodic surveys or questionnaires to gather insights from users. Ask specific questions about their 
satisfaction level, pain points, desired features, and overall user experience. Use this feedback to identify 
areas for improvement and prioritize development efforts based on user needs and preferences.

\paragraph{Usability Studies}
Conduct user testing sessions and usability studies with a diverse group of users. Observe their interactions with 
the application, gather feedback on specific features or workflows, and identify areas where users face challenges 
or confusion.

\paragraph{Analyze User Behavior}
Utilize analytics tools to track user behavior within the application. Monitor how users navigate through different 
screens, identify drop-off points, and track usage patterns.

\paragraph{Issue Resolution}
Provide a seamless process for users to report bugs, issues, or technical glitches they encounter. Ensure that 
reported issues are acknowledged promptly, and transparent communication is maintained throughout the resolution 
process.

\paragraph{Release Notes}
Clearly communicate updates, enhancements, and bug fixes to users through release notes or in-app notifications. 
Transparent communication demonstrates your commitment to listening others.

\paragraph{Early Access Programs}
Engage enthusiastic users in beta testing or early access programs to gather feedback on upcoming features or 
major updates. Beta testers can provide valuable insights, uncover edge cases, and help identify any issues 
before a wider release.

\paragraph{Continuous Adaptation}
Foster a culture of continuous learning and adaptation. Encourage regular retrospectives to reflect on the development 
process and areas for improvement. Embrace an agile mindset by seeking opportunities to iterate and refine the 
application.\\
\\

\noindent In the end, our mission is clear: create a financial accounting application that is intuitive, efficient, 
and inclusive. Through thoughtful design and unwavering attention to detail, we're shaping an experience that empowers 
users to effortlessly manage their finances while ensuring that no one is left behind.
