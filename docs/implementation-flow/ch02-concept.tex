% Copyright 2023 The terCAD team. All rights reserved.
% Use of this content is governed by a CC BY-NC-ND 4.0 license that can be found in the LICENSE file.

In the realm of transient concepts, many individuals tend to become perplexed. The truth is, nearly every idea we 
utilize is an amalgamation of numerous insights contributed by diverse individuals spanning a considerable span of 
time. Pioneering ideas entirely from the ground up often proves to be a futile endeavor. Instead, the optimal approach 
involves assimilating the finest aspects of what has already been established and then enhancing upon that foundation
(\cite{John11}, \cite{Azar22}, \cite{Page19}, \cite{Bara18}, \cite{Kleo12}, \cite{Thag12}).

There can be a different strategy for the product creation \cite{Lomb17} but we're not going to concentrate on that,
and declare a Disruptive Product with an initial phase in three months (since it's a free-spot time allocation
for a single developer, me; approximately 200 hours). Disruptive strategy is stand out by providing a simple, 
cost-effective solutions to customers' challenges. 

For instance, when Netflix was introduced, it didn't immediately disrupt the market due to the lack of instant 
gratification for customers who were used to buying movies from Blockbuster stores. However, as Netflix 
improved its service, reducing movie delivery time and eventually introducing online streaming, it ultimately led 
to the downfall of Blockbuster stores \cite{Eby17}.

\paragraph{Project Assumption} In today's fast-paced world, managing personal finances has become increasingly 
important. Whether it's tracking expenses, setting budgets, or monitoring investments, individuals are seeking for an 
efficient and intuitive solution to keep their financial lives in order (take control of finances).


\subsection{Simplifying User Interface}
\markboth{Conceptualizing}{Simplifying User Interface}

One of the fundamental principle in designing an application is a simplicity. The application should have a clean and 
intuitive interface, allowing users to effortlessly navigate through various sections and perform tasks. We should avoid 
overwhelming users with unnecessary complexities and focus on providing a streamlined experience that caters to 
their specific needs. Since the main goal is to create the application that can be easily understood and used by a wide 
range of users, regardless of their level of technical expertise.

\paragraph{Minimalistic Design}
Adopt a minimalistic design approach by keeping the user interface clean and uncluttered. 
Avoid overwhelming users with excessive information or visual elements. Use ample white space, clear typography, 
and a consistent color scheme to create a visually pleasing and organized interface.

\paragraph{Clear Navigation}
Ensure that the application's navigation is intuitive and straightforward. Use clear and 
descriptive labels for navigation elements, such as tabs, menus, and buttons. Organize different sections and 
features logically, making it easy for users to find what they need without confusion or frustration.

\paragraph{Consistent User Flow}
Keep consistent the placement of buttons, menus, and other interactive elements to "develop" a mental model of 
the application interface. Users should be able to predict how different actions and interactions will unfold based 
on their previous experiences within the app. 

\paragraph{Task-oriented Design}
Understand the specific tasks and goals to design the application around them. Provide clear and easily accessible 
options for the most commonly performed scenarios.

\paragraph{Contextual Help and Guidance}
Incorporate contextual help and guidance within the application to assist users in understanding its features and 
functionalities. Use tooltips, on-screen prompts, and informative messages to provide relevant information at the 
right moment. This helps users to make informed decisions and ensures they are aware of how to use the application 
effectively.

\paragraph{Streamlined Data Entry}
Make data entry as effortless as possible. Implement features such as auto-suggestions, 
pre-filled forms, and smart categorization to minimize manual input and reduce the chances of errors. Utilize 
input validation and intelligent defaults to guide users and speed up the data entry process.

\paragraph{Feedback and Confirmation}
Provide visual feedback and confirmation to users when they perform actions or operations within the application. This 
reassures users that their inputs have been registered and processed successfully. 

\paragraph{User Testing and Iteration}
Observe how users interact with the application, identify pain points, and refine the design based on their input. 
Iteratively improve the interface to address usability issues and enhance the overall user experience.

\paragraph{Accessibility Considerations}
Ensure that the application is accessible to users with disabilities. Follow 
accessibility guidelines and standards to make the application usable by individuals with visual impairments, 
motor limitations, or other accessibility needs. Provide options for adjusting font sizes, color contrasts, and 
support for screen readers.

\subsection{Hierarching Information}
\markboth{Conceptualizing}{Hierarching Information}

A well-designed financial accounting application must establish a clear information hierarchy. Prioritize essential 
financial data such as account balances, transactions, and budgets, ensuring they are easily accessible to users. 
A visual cues like color coding and icons help to guide users' attention and make it easier to caught the status at a 
glance. That reduces a cognitive load by making it easier to understand the financial status and make informed decisions.

\paragraph{Prioritize Key Information}
Identify the most important data that users need to see at a glance. Place this information prominently on the main 
dashboard (home screen) of the application.

\paragraph{Categorize and Group Data}
Organize data into logical categories and groups. Use clear headings and visual cues, such as dividers or 
section headers, to distinguish different categories.

\paragraph{Use Visual Hierarchy Techniques}
Utilize visual hierarchy techniques to guide users' attention and emphasize important information. Use larger and 
bolder fonts, contrasting colors, and appropriate typography to make critical data stand out.

\paragraph{Progressive Disclosure}
Implement progressive disclosure to manage complex or detailed information. Start with a high-level overview of the 
data and provide options to drill down for more detailed insights.

\paragraph{Consistent Layout and Organization}
Maintain a consistent layout and organization of information across different screens and sections of the application. 
Users should be able to navigate between screens and find information easily without having to relearn the interface. 
Consistency in the placement of elements such as menus, filters, and search bars enhances user familiarity and 
efficiency.

\paragraph{Filter and Search Functionality}
Enable users to filter and search for specific data based on their needs. Incorporate a robust search functionality 
that allows users to find anything quickly.

\paragraph{Contextual Information}
Provide a contextual information and explanations where necessary to help users understand the meaning and relevance of 
different terms or calculations. Use tooltips or inline help text to clarify any complex concepts or jargon that might 
be unfamiliar to users.

\paragraph{Clear Labels and Icons}
Use clear and concise labels for different elements within the application. Ensure that labels accurately represent the 
actions or functions they perform. Supplement labels with appropriate icons to aid in quick recognition and comprehension.

\paragraph{Gestures and Interactions}
Design intuitive gestures and interactions that enable users to access additional information or perform actions 
effortlessly.

\paragraph{Responsive Design}
Consider the various screen sizes and orientations that users might utilize. Ensure that the 
application's information hierarchy adapts seamlessly to different screen resolutions and orientations, providing a 
consistent and optimal experience across devices.


\subsection{Onboarding}
\markboth{Conceptualizing}{Onboarding}

The onboarding process sets the stage for user engagement and retention by guiding through the initial setup and 
familiarizing with the application's key features. That helps to explain how the application can assist in achieving 
users' goals. A well-designed onboarding process increases user confidence, reduces friction, and sets the stage for 
long-term engagement and satisfaction.

\paragraph{Concise Introduction}
Provide a concise and compelling introduction to the application. Clearly communicate the value proposition and benefits 
users can expect from using it. Highlight key features that differentiate the app from others in the market.

\paragraph{Guided Setup Process}
Design a guided setup process that walks users through the initial steps of getting started with the application. Break 
down the setup process into manageable stages or tasks, presenting one step at a time. Each step should be clearly 
defined and accompanied by explanatory text or visual cues to guide users through the setup.

\paragraph{Personalization Options}
Allow users to customize their experience by setting preferences, such as currency and date format, or language. 
Personalization enhances user engagement and creates a sense of ownership over the app.

\paragraph{Sample Data and Interactive Tutorials}
Consider providing users with sample data or interactive tutorials that demonstrate how to use the application 
effectively to give them a hands-on experience before entering their own data.

\paragraph{Clear Calls-to-Action}
Provide explicit instructions for each step, specifying what users need to do to proceed. Use intuitive buttons or 
visual cues to guide users and make it easy for them to complete each onboarding task.

\paragraph{Progress Indicators}
Incorporate progress indicators to give users a sense of their advancement through the onboarding process. 
This visual feedback assures users that they are making progress and helps manage their expectations regarding 
the remaining steps.

\paragraph{Contextual Help and Tooltips}
Explain unfamiliar terms, provide additional information on features, or clarify any potential confusion. Contextual 
help ensures users feel supported and informed as they navigate through the onboarding experience.

\paragraph{Quick Wins and Rewards}
Include quick wins and rewards during onboarding to motivate and engage users. This helps build a positive user 
experience and encourages users to continue using the app.

\paragraph{Seamless Data Import}
If applicable, provide options for users to seamlessly import their financial data from external 
sources. Enable integration with popular financial institutions, allowing users to connect their bank accounts, credit 
cards, or other financial platforms to automatically import transaction data.

\paragraph{Onboarding Feedback}
Gather feedback from users during or at the end of the onboarding process. Offer a way for users 
to provide input, ask questions, or report any issues they may encounter. This feedback can help identify areas for 
improvement and ensure a continuously evolving onboarding experience.


\subsection{Personalizing Options}
\markboth{Conceptualizing}{Personalizing Options}

By offering personalization and customization features, we empower users to adapt the application to their unique 
requirements. Personalization fosters a sense of ownership and encourages to actively engage with the application.

\paragraph{Localization Preferences}
Support multiple languages and currencies to cater to a diverse user base. This ensures that users feel comfortable and 
can easily understand and interpret their data.

\paragraph{Flexible Categories}
Allow to create and customize sections (grouping tags) for the data to improve the accuracy and relevance.
Provide options to define budget limits for specific categories, and restate frequency (monthly, weekly, etc.) to 
allocate funds accordingly.

\paragraph{Customizable Dashboards}
Provide the ability to customize a main page by choosing information and widgets (in order and size) that's needed to 
be shown at a glance. Additionally it can be allowed a selection of different color schemes, fonts, and layouts 
that can be chosen from.

\paragraph{Notification Preferences}
Implement progress tracking, reminders, and insights to help users stay motivated and on track towards achieving their 
goals. Whether it's saving for a vacation, paying off debt, or investing for retirement, users should be able to define 
them. Personalize notification preferences by giving the control over which types of notifications users want to receive 
without feeling overwhelmed.

\paragraph{Data Export}
Offer export options in different formats and customizable report templates, that allows users to tailor the output to 
their specific needs. That can be enhanced by an ability to back up their data and sync it across multiple devices.


\subsection{Securing Information}
\markboth{Conceptualizing}{Securing Information}

Financial data is highly sensitive, and users need the assurance that their information is secure (such as the usage of 
end-to-end encryption and two-factor authentication). By prioritizing secure data handling practices, we may establish 
trust with users, mitigate risks, and safeguard a sensitive information.

\paragraph{Encryption}
Implement strong encryption techniques to safeguard users' data both in transit and at rest. Utilize 
industry-standard encryption algorithms to encrypt sensitive information such as account credentials, transaction 
details, and personal identifiers. Encryption adds an extra layer of protection and ensures that even if data is 
compromised, it remains unintelligible to unauthorized parties.

\paragraph{User Authentication}
Implement robust user authentication mechanisms to verify the identity of users accessing the application. Require strong 
and unique passwords, and consider implementing multi-factor authentication (MFA) for an added layer of security. MFA 
involves a combination of something the user knows (password), something they have (such as a token or mobile device), 
or something they are (biometrics).

\paragraph{Role-Based Access Control}
Implement role-based access control (RBAC) to restrict access to a sensitive data and functionalities based on 
permissions.

\paragraph{Security Audits}
Conduct security audits to identify and address any vulnerabilities or weaknesses in the application's infrastructure 
and codebase. Stay updated with security best practices and patches for the technologies and frameworks used in the 
application. Promptly apply security updates and fixes to ensure the application is protected against known 
vulnerabilities.

\paragraph{Data Transmission}
Ensure that data transmitted between the application and its servers is protected using secure protocols. 

\paragraph{Data Minimization}
Practice data minimization by collecting and storing only the necessary information required for 
the application's functionality. Minimize the collection of sensitive data, and regularly review and purge outdated 
or no longer needed data. This reduces the risk of exposure and potential harm in the event of a data breach.

\paragraph{Secure Storage}
Implement secure storage mechanisms to protect user data within the application's infrastructure. Perform backups of a 
data to prevent a loss in the event of system failures, disasters, or other unforeseen circumstances.

\paragraph{Privacy Policy and Transparency}
Clearly communicate via application's privacy policy and data handling practices to users. Inform them about the types 
of data collected, how it is used, and who has access to it. Obtain explicit consent from users for collecting and 
processing their data. Be transparent about how you protect user information and handle security incidents, providing 
clear channels for users to report any concerns.


\subsection{Visualizing Data}
\markboth{Conceptualizing}{Visualizing Data}

Well-designed visualization enhances a data comprehension, engagement, and the overall experience, by making it easier 
for users to take a control over the data clarity.

\paragraph{Overview Dashboards}
Create visually appealing and informative overview dashboards that provides a high-level summary of the information. 
Use charts, graphs, and key performance indicators (KPIs) to present data such as account balances, income versus 
expenses, savings progress, or net worth. Utilize color-coded icons, symbols, or illustrations to indicate transaction 
types (income, expense, transfer) or payment methods (cash, credit card). Consider displaying transaction details such 
as date, category, amount, and account in a concise and visually appealing manner. Include interactive features like 
zooming or panning by allowing users to explore data in details, and options to choose the visual representation in
accordance with to preferences.

\paragraph{Progress Tracking}
Use progress bars, thermometers, or visual indicators to track users' progress toward their goals and expectations. 
Represent the budgeted amount versus actual expenses visually to provide users with a clear understanding of their 
spending habits and help them stay within the targets.

\paragraph{Forecasting Analytics}
Incorporate forecasting and predictive analytics into visualizations to assist users in planning and decision-making.
Use trend lines to project future income, expenses, or savings based on historical data. Predictive analytics can help 
users anticipate future financial scenarios and make informed financial choices.

\paragraph{Interactive Filters}
Enable drill-down functionality to delve into detailed information by interacting with specific data points or 
categories.

\paragraph{Comparative Analysis}
Enable users to compare different financial metrics or periods side by side using visualizations like bar charts, 
line graphs, or stacked area charts.

\paragraph{Heat Maps}
Utilize heat maps or color schemes to highlight data patterns or variations. Heat maps can help users quickly identify 
areas of focus or anomalies.


\subsection{Applying Accessibility}
\markboth{Conceptualizing}{Applying Accessibility}

Cross-platform accessibility ensures users can access their information anytime, anywhere, regardless of the devices 
they prefer to use.

\paragraph{Responsive Design}
Implement a responsive design approach to ensure that the application adapts and optimizes its layout and user 
interface based on the screen size and resolution of the device being used.

\paragraph{Native Mobile Apps}
Develop native mobile applications for popular platforms such as iOS and Android. Native apps provide a tailored 
user experience, leveraging the platform-specific features, performance optimizations, and user interface guidelines. 
Native apps can offer enhanced performance, offline capabilities, and seamless integration with device functionalities, 
providing users with a more optimized experience on their respective mobile platforms.

\paragraph{Web-Based Application}
Build a web-based version of the financial accounting application that can be accessed through web browsers on 
different devices. Ensure compatibility across major browsers, including Google Chrome, Mozilla Firefox, Safari, 
and Microsoft Edge. Design the web application to be user-friendly and intuitive, offering a consistent experience 
regardless of the operating system or device.

\paragraph{Data Synchronization}
Implement synchronization mechanisms that automatically update data in real-time, ensuring users have the latest 
information available on all their devices.

\paragraph{Consistent User Experience}
Ensure that the core functionalities, features, and navigation patterns remain consistent regardless of the platform or 
device being used. This allows users to familiarize themselves with the application quickly and reduces the learning 
curve when switching between devices.

\paragraph{Device-Specific Features}
Leverage device-specific features and capabilities to enhance the user experience on different platforms. For 
example, on mobile devices, leverage features like camera integration for scanning receipts, push notifications 
for transaction alerts, or biometric authentication.

\paragraph{Accessibility Standards}
Implement features such as adjustable font sizes, high contrast modes, screen reader compatibility, and keyboard 
navigation support. By making the application accessible, we may cater to a wider range of users.


\subsection{Shaking the Market}
\markboth{Conceptualizing}{Shaking the Market}

By incorporating the AI-powered Assistant, the application might differentiate itself in the market. That is supposed 
to simplify and automate the expense tracking process, delivers actionable insights, and empowers users to make 
smarter financial decisions. 

\paragraph{Automated Expense Categorization}
Automatically categorize expenses based on transaction data, eliminating the need for manual categorization. 
It analyzes transaction descriptions, amounts, and patterns to intelligently assign categories such as groceries, 
transportation, entertainment, bills, and more. Users can save a significant time and effort in manually categorizing 
each expense, making expense tracking effortless and accurate.

\paragraph{Receipt Scanning and OCR}
Use optical character recognition (OCR) technology to scan and extract data from receipts. Users can simply take a photo 
of a receipt, and the application will automatically extract key information.

\paragraph{Real-time Expense Insights}
Provide real-time insights and analytics on users' spending patterns. It offers visualizations, charts, and graphs that 
highlight monthly spending trends, category breakdowns, and comparisons with previous periods.

\paragraph{Intelligent Notifications}
AI analyzes users' income, expenses, and goals by providing proactive notifications to manage finances effectively, and 
avoid overspending.

\paragraph{Savings Suggestions}
It can be potentially identified areas where users can cut costs, offers recommendations and highlights opportunities 
for optimizing their budgets.


\subsection{Integrating Services}
\markboth{Conceptualizing}{Integrating Services}

To provide users with a comprehensive financial tracking experience, integrate the application with financial 
institutions and services. Enable users to connect their bank accounts, credit cards, and investment platforms 
to automatically import transaction data. Implement synchronization features to ensure real-time updates and 
seamless reconciliation between the application and external financial sources. This integration streamlines 
the process of tracking transactions, monitoring account balances, and gaining insights into their overall financial 
well-being.

\paragraph{Account Connectivity}
Implement secure and industry-standard protocols such as Open Banking APIs to establish a secure connection between 
the application and the financial institution's systems. Regularly update account balances and transaction information 
from connected financial institutions. Implement automated processes to fetch and update account balances, ensuring 
users have real-time visibility into their financial standing.

\paragraph{Multiple Account Types}
Support integration with a wide range of financial accounts, including checking accounts, savings accounts, credit 
cards, investment accounts, and loans. This enables users to centralize their financial data within a single 
application.

\paragraph{Data Security}
Adhere to industry-standard security practices, including encryption, secure data transmission, and compliance with 
data protection regulations such as GDPR or CCPA.

\paragraph{Error Handling}
Implement robust error handling mechanisms to handle potential connectivity issues or errors that may occur during the 
integration process. Notify users of any issues encountered during data retrieval, and provide clear instructions on 
how to resolve the issue.

\paragraph{Reconciliation}
Enable users to mark transactions as verified or reconciled once they have cross-checked them with their official bank 
statements. This ensures data accuracy and helps users identify any discrepancies or missing transactions.

\paragraph{Continuous Enhancements}
Stay informed about industry changes and ensure the application remains compatible with the latest standards and 
protocols.


\subsection{Improving Continuously}
\markboth{Conceptualizing}{Improving Continuously}

Encourage users to provide feedback and actively seek their input to identify areas for improvement. Regularly update 
the application by enhancing the user experience. This ongoing commitment to improvement will foster user loyalty and 
satisfaction. User feedback becomes a valuable source of insights and inspiration for enhancing the application's 
functionality, usability, and overall user experience. This iterative approach ensures that the application evolves 
and remains relevant in a rapidly changing landscape, ultimately leading to increased user satisfaction and loyalty.

\paragraph{Feedback Channels}
Provide various channels for users to share their feedback, suggestions, and concerns. Offer options such as in-app 
feedback forms, email support, community forums, or social media channels. Make it easy for users to voice their 
opinions (a feature request system where users can submit and upvote ideas) and actively encourage them to provide 
feedback on their experience with the application.

\paragraph{Surveys and Questionnaires}
Conduct periodic surveys or questionnaires to gather insights from users. Ask specific questions about their 
satisfaction level, pain points, desired features, and overall user experience. Use this feedback to identify 
areas for improvement and prioritize development efforts based on user needs and preferences.

\paragraph{Usability Studies}
Conduct user testing sessions and usability studies with a diverse group of users. Observe their interactions with 
the application, gather feedback on specific features or workflows, and identify areas where users face challenges 
or confusion.

\paragraph{Analyze User Behavior}
Utilize analytics tools to track user behavior within the application. Monitor how users navigate through different 
screens, identify drop-off points, and track usage patterns.

\paragraph{Issue Resolution}
Provide a seamless process for users to report bugs, issues, or technical glitches they encounter. Ensure that 
reported issues are acknowledged promptly, and transparent communication is maintained throughout the resolution 
process.

\paragraph{Release Notes}
Clearly communicate updates, enhancements, and bug fixes to users through release notes or in-app notifications. 
Transparent communication demonstrates your commitment to listening others.

\paragraph{Early Access Programs}
Engage enthusiastic users in beta testing or early access programs to gather feedback on upcoming features or 
major updates. Beta testers can provide valuable insights, uncover edge cases, and help identify any issues 
before a wider release.

\paragraph{Continuous Adaptation}
Foster a culture of continuous learning and adaptation. Encourage regular retrospectives to reflect on the development 
process and areas for improvement. Embrace an agile mindset by seeking opportunities to iterate and refine the 
application.
