% Copyright 2023 The terCAD team. All rights reserved.
% Use of this content is governed by a CC BY-NC-ND 4.0 license that can be found in the LICENSE file.

\markboth{Productionizing}{Productionizing}

Transitioning from developing an application to deploying it into a production environment requires careful planning 
and consideration. As, it's valuable to ensure that it's is well-structured and follows best practices 
\ref{refactoring}; implement a comprehensive testing strategy, including all layers (unit, widget, and integration 
tests, \ref{quality}) with a forecast \ref{benchmark} and usability \ref{usability} analysis; add logging and crash 
reporting tools for diagnosing issues in a production environment \ref{telemetry}. Productionizing the application 
involves much more than writing code. It requires careful planning, attention to security, scalability, and a focus 
on user feedback and ongoing maintenance. 

Let's dive into the most controversial topic like planning, as it boils over from time to time (\emph{as, by James O. 
McKinsey' publication regarding the productivity measurement of development teams \cite{McKi23}}). On one hand, we 
must commit to delivering the next scope of features as promised to our stakeholders and customers, while on the other, 
we need to maintain a buffer to mitigate potential risks. Hence, a productivity measurement has become a key indicator 
for capacity allocation. The issue at hand is that software engineering entails more than just coding; it involves 
making architectural decisions, conducting testing, performing security analysis, monitoring performance, and other 
not mentioned but valuable activities. Believe that the key to success lies in ensuring two-way transparency. 

Distrusting employees (as, fearing insider leaks) create a cycle of mistrust that turns the development process into a 
black box. So, rather than simply focusing on planning a specific N-week iteration or a broader quarterly increment, 
it's essential to take a long-term perspective, visualizing and making decisions that span decades. This approach 
emphasizes the importance of strategic thinking and future-proofing solutions. It means considering how the software 
will evolve and adapt over the years, not just in the short term. This transformation shifts our focus from "mere" 
performance monitoring to a stability measurement \cite{Heal23}, that might significantly enhance product delivery 
predictability.

