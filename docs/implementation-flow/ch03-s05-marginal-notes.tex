% Copyright 2023 The terCAD team. All rights reserved.
% Use of this content is governed by a CC BY-NC-ND 4.0 license that can be found in the LICENSE file.

\subsection{Enriching Interface}

\subsubsection{Splitting Pages}

Now that we've implemented forms for managing Accounts and Budgets, the next step is to introduce the functionality 
for adding Bills \issue{18}{}. This addition involves creating three separate tabs, specifically for Income, Expenses, 
and Transfers \issue{3}{}. We can accomplish this by utilizing the \q{PageView}-widget:

\begin{lstlisting}
class Page1 extends StatelessWidget { /* ... */ }
class Page2 extends StatelessWidget { /* ... */ }
class Page3 extends StatelessWidget { /* ... */ }

class MyApp extends StatelessWidget {
  final _pageController = PageController(initialPage: 0);
  @override
  Widget build(BuildContext context) {
    return MaterialApp(
      home: Scaffold(
        body: GestureDetector(
          onHorizontalDragEnd: (DragEndDetails details) {
            // Swipe right
            if (details.primaryVelocity! > 0) { 
              _pageController.previousPage(
                duration: Duration(milliseconds: 500),
                curve: Curves.ease,
              );
            // Swipe left
            } else if (details.primaryVelocity! < 0) { 
              _pageController.nextPage(
                duration: Duration(milliseconds: 500),
                curve: Curves.ease,
              );
            }
          },
          child: PageView(
            controller: _pageController,
            children: [Page1(), Page2(), Page3()],
          ),
        ),
      ),
    );
  }
}
\end{lstlisting}

\noindent To enable switching between tabs by tapping instead of swiping, we should make following adjustments:

\begin{lstlisting}
class MyApp extends StatelessWidget {
  final _pageController = PageController(initialPage: 0);

  @override
  Widget build(BuildContext context) {
    return MaterialApp(
      home: DefaultTabController(
        length: 3,
        child: Scaffold(
          appBar: AppBar(
            title: Text('Swiping Pages'),
            bottom: TabBar(
              tabs: [
                Tab(text: 'Page 1'),
                Tab(text: 'Page 2'),
                Tab(text: 'Page 3'),
              ],
            ),
          ),
          body: TabBarView(
            children: [Page1(), Page2(), Page3()],
          ),
        ),
      ),
    );
  }
}
\end{lstlisting}

\noindent It's a time to combine both solutions into one:

\begin{lstlisting}
class MyApp extends StatelessWidget with TickerProviderStateMixin {
  final int tabCount = 3;
  int tabIndex = 1;

  PageController? pageController;
  TabController? tabController;

  @override
  void initState() {
    super.initState();
    pageController = PageController(initialPage: tabIndex);
    tabController = TabController(
      length: tabCount,
      vsync: this, // covered by mixin
      initialIndex: tabIndex,
    );
  }

  @override
  void dispose() {
    pageController?.dispose();
    tabController?.dispose();
    super.dispose();
  }

  void switchTab(int newIndex) {
    setState(() {
      tabIndex = newIndex;
      tabController?.animateTo(newIndex);
      pageController?.animateToPage(
        newIndex,
        duration: Duration(milliseconds: 300),
        curve: Curves.ease,
      );
    });
  }

  @override
  Widget build(BuildContext context) {
    return MaterialApp(
      home: GestureDetector(
        onHorizontalDragEnd: (DragEndDetails details) {
          if (details.primaryVelocity! > 0) {
            switchTab(widget.tabIndex - 1);
          } else if (details.primaryVelocity! < 0) {
            switchTab(widget.tabIndex + 1);
          }
        },
        child: Scaffold(
          appBar: TabBar(
              controller: tabController,
              onTap: switchTab,
              tabs: [
                Tab(text: 'Page 1'),
                Tab(text: 'Page 2'),
                Tab(text: 'Page 3'),
              ],
            ),
          body: PageView(
            controller: pageController,
            onPageChanged: switchTab,
            children: [Page1(), Page2(), Page3()],
          ),
        ),
      ),
    );
  }
}
\end{lstlisting}

\noindent By combining both solutions, we encounter an issue -- we can't navigate from \q{Page1} to \q{Page3} directly 
(skipping more than one position) because the \q{pageController} triggers an update through \q{onPageChanged} after the 
first movement. Fortunately, we can resolve this by introducing a delayed trigger to enable switching to the selected 
tab:

\begin{lstlisting}
Future<void> delaySwitchTab(int delay, int newIndex) async {
  await Future.delayed(Duration(milliseconds: delay));
  switchTab(newIndex);
}

void switchTab(int newIndex) {
  if (newIndex < 0 || newIndex >= widget.tabCount) {
    return;
  }
  setState(() {
    const delay = 300;
    // Saving current state for the check after 
    final currIndex = tabIndex;
    tabIndex = newIndex;
    tabController?.animateTo(newIndex);
    pageController?.animateToPage(
      newIndex,
      duration: const Duration(milliseconds: delay),
      curve: Curves.ease,
    );
    // Verify that the difference is more than one
    if ((currIndex - newIndex).abs() > 1) {
      delaySwitchTab(delay, newIndex);
    }
  });
}
\end{lstlisting}

\noindent As we might be thinking, this solution might seem overly complex for the seemingly simple task of supporting 
both swiping and tapping for tabs. And we would be right in our doubts, the whole implementation can be replaced by 
\q{TabBarView}-widget:

\begin{lstlisting}
Column(
  children: [
    TabBar.secondary(
      controller: tabController,
      tabs: [ /* ... */ ],
    ),
    Expanded(
      child: TabBarView(
        controller: tabController,
        children: [Page1(), Page2(), Page3()],
      )),
  ]);
\end{lstlisting}

\subsubsection{Sprinkling Breadcrumbs}

In certain scenarios, like when we want to display Goals on the main page (\cref{img:prototype}), we don't require a 
fully structured navigation bar with titles and icons. Instead, we simply need dots to indicate that the current block 
can be swiped or changed by clicking on any of these dots. Implementing these dots can be achieved by either using 
existing external components (adding them to the page as widgets) or by customizing the indicator property of the 
\q{TabBar}-widget with a custom implementation of the \q{Decoration}-class:

\begin{lstlisting}
// ./lib/widgets/_wrappers/dots_indicator_decoration.dart
class DotsIndicatorDecoration extends Decoration {
  // ... properties and constructor definition

  @override
  BoxPainter createBoxPainter([VoidCallback? onChanged]) {
    return _CustomTabIndicatorPainter(
      // ... properties
      onChanged: onChanged,
    );
  }
}

class _CustomTabIndicatorPainter extends BoxPainter {
  // ... properties and constructor definition
  @override
  void paint(Canvas canvas, Offset offset, ImageConfiguration configuration) {
    if (itemCount <= 1) { // Skip for a single tab
      return;
    }
    // Take from PageController an actual page, otherwise - initial
    final activeIndex = controller.page?.round() ?? 
        controller.initialPage;
    // dotSize, spacing, color - initialized properties
    final active = Paint()..color = color;
    final inactive = Paint()..color = color.withOpacity(0.3);
    for (int i = 0; i < itemCount; i++) {
      double xPos = spacing + i * (dotSize + spacing);
      // Put the cycle a little below the baseline
      double yPos = spacing * 0.6; 
      if (i == activeIndex) {
        canvas.drawCircle(Offset(xPos, yPos), dotSize/2, active);
      } else {
        canvas.drawCircle(Offset(xPos, yPos), dotSize/2, inactive);
      }
    }
  }
}
\end{lstlisting}

\noindent Additionally, we might simplify the usage by extending our widget from \q{TabBar}:

\begin{lstlisting}
// ./lib/widgets/_wrappers/dots_tab_bar_widget.dart
class DotsTabBarWidget extends TabBar {
  final TabController tabController;
  final PageController pageController;
  final List<Widget> tabList;
  final double indent; // Indentation between cycles
  final double width; // MediaQuery.of(context).size.width
  final Color color; // Color for active cycle

  const DotsTabBarWidget({
    super.key,
    required this.tabController,
    required this.pageController,
    required this.tabList,
    required this.indent,
    required this.width,
    required this.color,
    onTap,
  }) : super(
          controller: tabController,
          mouseCursor: SystemMouseCursors.click,
          onTap: onTap,
          // hook, since 'tabs' is required
          tabs: tabList, // getter overrides the flow
        );
  // Convert children of 'PageView' to a clickable area 
  @override
  get tabs =>
      tabList.map((tab) => SizedBox(width: indent, height: indent)).toList();
  // Make our dots centered on the page
  @override
  get padding => EdgeInsets.symmetric(
      horizontal: (width - tabList.length * 2 * indent) / 2);
  // Apply custom decorator to draw cycles
  @override
  get indicator => DotsIndicatorDecoration(
        controller: pageController,
        itemCount: tabList.length,
        color: color,
        dotSize: indent,
      );
}
\end{lstlisting}
