% Copyright 2023 The terCAD team. All rights reserved.
% Use of this content is governed by a CC BY-NC-ND 4.0 license that can be found in the LICENSE file.

\subsection{[TBD] Prototyping}
\markboth{Implementing Core Functionality}{Prototyping}

Question: how to apply MainAxisAlignment.spaceBetween only for the last item

Answer: In Flutter, you can apply MainAxisAlignment.spaceBetween to only the last item in a row or column by 
wrapping the last item with an additional container or widget.

\begin{lstlisting}
Row(
  mainAxisAlignment: MainAxisAlignment.spaceBetween,
  children: [
    Row(
        children: [
            Container(
            child: Text("Left 1"),
            ),
            Container(
            child: Text("Left 2"),
            ),
        ]
    ),
    Container(
      child: Text("Right 3"),
    ),
  ],
)
\end{lstlisting}


Question: how to take max available width for the widget

Answer: To take the maximum available width for a widget in Flutter, you can use the Expanded widget or 
the Flexible widget. Both widgets allow a child widget to expand and take up all the remaining available 
space within a row or column.

\begin{lstlisting}
Expanded(
    child: Container(
        color: Colors.red,
        child: Text("Widget 2"),
    ),
),
\end{lstlisting}

Widget is wrapped in an Expanded widget, allowing it to take up the maximum available width within the row.

The Flexible widget can be used instead of Expanded. The behavior is similar, allowing the 
child widget to expand and take up the maximum available width within the row. Both the Expanded and Flexible 
widgets can be used within a Row or Column to control the width or height, respectively, of the child widgets.


Question: How to plot vertical progress bar

Answer: To plot a vertical progress bar in Flutter, you can use the LinearProgressIndicator widget and rotate 
it vertically using the Transform widget. 

\begin{lstlisting}
import 'package:flutter/material.dart';

class VerticalProgressBar extends StatelessWidget {
  final double value;
  final Color color;
  final double width;
  final double height;

  VerticalProgressBar({required this.value, this.color = Colors.blue, this.width = 10.0, this.height = 100.0});

  @override
  Widget build(BuildContext context) {
    return Container(
      width: width,
      height: height,
      child: Transform.rotate(
        angle: -90 * 3.14 / 180, // Rotate the progress bar vertically
        child: LinearProgressIndicator(
          value: value,
          backgroundColor: Colors.grey,
          valueColor: AlwaysStoppedAnimation<Color>(color),
        ),
      ),
    );
  }
}
\end{lstlisting}

In the example above, we define a VerticalProgressBar widget that accepts the value, color, width, and height as 
parameters. The value represents the progress value between 0.0 and 1.0, color specifies the color of the 
progress bar, and width and height define the dimensions of the progress bar.

Inside the build method, we create a container with the specified width and height. Within the container, 
we use the Transform.rotate widget to rotate the child widget, which is the LinearProgressIndicator, 
by -90 degrees (counter-clockwise) to make it vertical. The LinearProgressIndicator is set up with the 
provided value, backgroundColor, and valueColor properties.

To use the VerticalProgressBar widget in your application, simply instantiate it and provide the necessary parameters:

\begin{lstlisting}
VerticalProgressBar(
  value: 0.75, // Example value
  color: Colors.blue, // Optional, default is blue
  width: 10.0, // Optional, default width is 10.0
  height: 100.0, // Optional, default height is 100.0
)
\end{lstlisting}


Question: how to retrieve max height for nested Container

Answer: To retrieve the maximum height for nested Container widgets in Flutter, you can use the LayoutBuilder 
widget. The LayoutBuilder widget provides the constraints of its parent widget, which allows you to obtain 
the maximum available height for its child widgets.

\begin{lstlisting}
LayoutBuilder(
  builder: (BuildContext context, BoxConstraints constraints) {
    final maxHeight = constraints.maxHeight;

    return Container(
      color: Colors.blue,
      height: maxHeight,
      child: Column(
        children: [
          Container(
            color: Colors.red,
            height: maxHeight * 0.5, // Example height based on a percentage of the maximum height
          ),
          Container(
            color: Colors.green,
            height: maxHeight * 0.3, // Example height based on a percentage of the maximum height
          ),
          Container(
            color: Colors.yellow,
            height: maxHeight * 0.2, // Example height based on a percentage of the maximum height
          ),
        ],
      ),
    );
  },
);
\end{lstlisting}

In the example above, we wrap the nested Container widgets with a LayoutBuilder widget. Inside the builder function, 
we retrieve the maxHeight from the constraints provided by the LayoutBuilder. The maxHeight represents the 
maximum available height within the parent widget.

We then use the maxHeight value to set the desired heights of the nested Container widgets. In this example, 
we set the heights to be a percentage of the maximum height.

By using the LayoutBuilder widget, you can dynamically calculate and assign the maximum height to the nested 
Container widgets based on the available space within their parent widget.
\\
That leads to our first error: The following assertion was thrown during performLayout(): 
BoxConstraints forces an infinite width.

Use the LayoutBuilder widget along with the ConstrainedBox widget.

\begin{lstlisting}
    LayoutBuilder(
        builder: (BuildContext context, BoxConstraints constraints) {
          final maxHeight = constraints.maxHeight;
      
          return Container(
            color: Colors.blue,
            child: Column(
              children: [
                ConstrainedBox(
                  constraints: BoxConstraints(
                    maxHeight: maxHeight * 0.5, // Example height based on a percentage of the maximum height
                  ),
                  child: Container(
                    color: Colors.red,
                  ),
                ),
                ConstrainedBox(
                  constraints: BoxConstraints(
                    maxHeight: maxHeight * 0.3, // Example height based on a percentage of the maximum height
                  ),
                  child: Container(
                    color: Colors.green,
                  ),
                ),
                ConstrainedBox(
                  constraints: BoxConstraints(
                    maxHeight: maxHeight * 0.2, // Example height based on a percentage of the maximum height
                  ),
                  child: Container(
                    color: Colors.yellow,
                  ),
                ),
              ],
            ),
          );
        },
      );      
\end{lstlisting}

In this updated example, we use the LayoutBuilder widget to retrieve the maxHeight from the constraints provided by 
its parent. We then use the maxHeight value to set the maximum height of each Container widget by wrapping 
them with ConstrainedBox widgets.

Each ConstrainedBox widget sets the maxHeight of the nested Container based on a percentage of the maximum height. 
Adjust the percentage values as needed to achieve the desired layout.

That leads to error: The following assertion was thrown during performLayout():
RenderFlex children have non-zero flex but incoming width constraints are unbounded.
When a row is in a parent that does not provide a finite width constraint, for example if it is in a
horizontal scrollable, it will try to shrink-wrap its children along the horizontal axis. Setting a
flex on a child (e.g. using Expanded) indicates that the child is to expand to fill the remaining
space in the horizontal direction.

To address the error you encountered, you can use a combination of Expanded and FractionallySizedBox widgets. 

\begin{lstlisting}
LayoutBuilder(
  builder: (BuildContext context, BoxConstraints constraints) {
    final maxHeight = constraints.maxHeight;

    return Container(
      color: Colors.blue,
      child: Column(
        children: [
          Expanded(
            child: FractionallySizedBox(
              alignment: Alignment.topCenter,
              heightFactor: 0.5, // Example height factor based on a percentage of the maximum height
              child: Container(
                color: Colors.red,
              ),
            ),
          ),
          Expanded(
            child: FractionallySizedBox(
              alignment: Alignment.topCenter,
              heightFactor: 0.3, // Example height factor based on a percentage of the maximum height
              child: Container(
                color: Colors.green,
              ),
            ),
          ),
          Expanded(
            child: FractionallySizedBox(
              alignment: Alignment.topCenter,
              heightFactor: 0.2, // Example height factor based on a percentage of the maximum height
              child: Container(
                color: Colors.yellow,
              ),
            ),
          ),
        ],
      ),
    );
  },
);
\end{lstlisting}

In this updated example, we use the LayoutBuilder widget to retrieve the maxHeight from the constraints provided 
by its parent. We then use the maxHeight value to set the height of each Container widget by wrapping them 
with FractionallySizedBox widgets.

Each FractionallySizedBox widget sets the heightFactor of the nested Container based on a percentage of the 
maximum height. The alignment property is set to Alignment.topCenter to align the content at the top.

Using the combination of Expanded and FractionallySizedBox, the nested Container widgets will expand to occupy 
the available height based on the provided height factors.


Question: flutter default value in constructor

Answer: In Flutter, you can assign default values to constructor parameters using the = operator. 

\begin{lstlisting}
class MyClass {
  final int myValue;

  MyClass({this.myValue = 10});
}
\end{lstlisting}

In the example above, the MyClass constructor has a parameter myValue with a default value of 10. 
If no value is provided for myValue when creating an instance of MyClass, it will default to 10.

You can then create an instance of MyClass with or without providing a value for myValue:

\begin{lstlisting}
// Using the default value
MyClass instance1 = MyClass();
print(instance1.myValue); // Output: 10

// Providing a custom value
MyClass instance2 = MyClass(myValue: 20);
print(instance2.myValue); // Output: 20    
\end{lstlisting}

In the first case, where no value is provided for myValue, it takes on the default value of 10. In the second case, 
a custom value of 20 is provided, which overrides the default value.


Question: What's the element for indent in a list of items

Answer: If you want to add an indent or padding to each element in a list of items in Flutter, you can 
wrap each item with a Padding widget.

\begin{lstlisting}
ListView(
  children: myList.map((item) {
    return Padding(
      padding: EdgeInsets.symmetric(horizontal: 16.0),
      child: ListTile(
        title: Text(item),
      ),
    );
  }).toList(),
);
\end{lstlisting}

In the example above, we have a ListView with a list of items stored in the myList variable. We use the map 
method to transform each item into a Padding widget wrapped around a ListTile. The Padding widget provides 
horizontal padding of 16.0 on both sides.

By using the map method, we iterate over each item in the list and return a Padding widget with the desired 
padding applied. Finally, we convert the resulting list of Padding widgets to a list of children using the 
toList method.

This approach adds the specified horizontal padding to each item in the list, providing an indentation effect. 
Adjust the horizontal value in the EdgeInsets.symmetric constructor to control the amount of indentation you want.


Question: What's vertical intent component

Answer: If you want to add vertical spacing or indentation between components in Flutter, you can use the 
SizedBox or Container widget with a specified height. 

\begin{lstlisting}
Column(
    children: [
        Text('Component 1'),
        SizedBox(height: 16.0), // Vertical spacing of 16.0
        Text('Component 2'),
        SizedBox(height: 32.0), // Vertical spacing of 32.0
        Text('Component 3'),
    ],
    ) 
\end{lstlisting}

In the example above, we use a Column widget to arrange multiple components vertically. To add vertical spacing 
between components, we insert a SizedBox widget with a specified height property. The height value determines 
the amount of vertical spacing between components.

You can adjust the height value to control the amount of vertical spacing you want between the components. 
For example, 16.0 represents 16 logical pixels of spacing.

Alternatively, you can use a Container widget with a height property to achieve the same effect:

\begin{lstlisting}
Column(
  children: [
    Text('Component 1'),
    Container(height: 16.0), // Vertical spacing of 16.0
    Text('Component 2'),
    Container(height: 32.0), // Vertical spacing of 32.0
    Text('Component 3'),
  ],
)
\end{lstlisting}

Both approaches allow you to add vertical spacing or indentation between components in a Column or any other 
widget that arranges its children vertically.


Question: show dot on LinearProgressIndicator by reposition one element on top of another.

Answer: To reposition one element on top of another in Flutter, you can use a Stack widget and adjust the 
positioning of the child widgets using the Positioned widget.

\begin{lstlisting}
import 'package:flutter/material.dart';

void main() {
  runApp(MyApp());
}

class MyApp extends StatelessWidget {
  @override
  Widget build(BuildContext context) {
    return MaterialApp(
      home: Scaffold(
        appBar: AppBar(
          title: Text('Element Repositioning Demo'),
        ),
        body: Center(
          child: Stack(
            children: [
              Container(
                width: 200,
                height: 200,
                color: Colors.blue,
              ),
              Positioned(
                top: 100,
                left: 100,
                child: Container(
                  width: 100,
                  height: 100,
                  color: Colors.red,
                ),
              ),
            ],
          ),
        ),
      ),
    );
  }
}
\end{lstlisting}

In this example, we use a Stack widget to overlay two containers. The first container represents the background 
element and has a blue color. The second container represents the foreground element and has a red color.

We use the Positioned widget to reposition the second container on top of the first container. By specifying 
the top and left properties of the Positioned widget, we can adjust the position of the child widget relative 
to the stack.

In this case, the second container is positioned 100 pixels from the top and 100 pixels from the left, resulting 
in it being centered on top of the first container.

You can modify the width, height, colors, and positioning properties of the containers and adjust them 
according to your specific requirements.

By running this code in a Flutter project, you will see the second container positioned on top of the first 
container based on the specified top and left values.


it throw exception:
The following assertion was thrown while applying parent data.:
Incorrect use of ParentDataWidget.
The ParentDataWidget Positioned wants to apply ParentData of type StackParentData to a RenderObject,
which has been set up to accept ParentData of incompatible type FlexParentData.

The error you encountered suggests that the parent data provided to the Positioned widget is not compatible 
with the parent Stack widget.

To resolve this issue, you can use a Positioned widget inside a Positioned.fill widget instead of directly 
within the Stack widget.

\begin{lstlisting}
import 'package:flutter/material.dart';

void main() {
  runApp(MyApp());
}

class MyApp extends StatelessWidget {
  @override
  Widget build(BuildContext context) {
    return MaterialApp(
      home: Scaffold(
        appBar: AppBar(
          title: Text('Element Repositioning Demo'),
        ),
        body: Center(
          child: Stack(
            children: [
              Container(
                width: 200,
                height: 200,
                color: Colors.blue,
              ),
              Positioned.fill(
                child: Container(
                  alignment: Alignment.center,
                  child: Container(
                    width: 100,
                    height: 100,
                    color: Colors.red,
                  ),
                ),
              ),
            ],
          ),
        ),
      ),
    );
  }
}
\end{lstlisting}

In this updated example, we use a Positioned.fill widget as the parent for the second container. The Positioned.fill 
widget sizes itself to fill the available space within the Stack widget.

The inner container is aligned to the center using the alignment property of the Positioned.fill widget. 
The second container is then placed inside the Positioned.fill widget and sized with a width and height of 
100 pixels, and given a red color.

By wrapping the second container with Positioned.fill, it ensures that the second container will be positioned 
on top of the first container within the Stack widget, without causing any conflicts with the parent data.

You can modify the sizes, colors, and alignment properties according to your specific requirements. Running this 
updated code in a Flutter project should display the second container positioned on top of the first container.


If the previous approach didn't work for you, there is an alternative way to reposition elements using a Stack 
widget and the Transform.translate widget.

\begin{lstlisting}
import 'package:flutter/material.dart';

void main() {
  runApp(MyApp());
}

class MyApp extends StatelessWidget {
  @override
  Widget build(BuildContext context) {
    return MaterialApp(
      home: Scaffold(
        appBar: AppBar(
          title: Text('Element Repositioning Demo'),
        ),
        body: Center(
          child: Stack(
            children: [
              Container(
                width: 200,
                height: 200,
                color: Colors.blue,
              ),
              Transform.translate(
                offset: Offset(100, 100),
                child: Container(
                  width: 100,
                  height: 100,
                  color: Colors.red,
                ),
              ),
            ],
          ),
        ),
      ),
    );
  }
}
\end{lstlisting}

In this updated example, we use the Transform.translate widget to reposition the second container on top of the 
first container. The offset property of Transform.translate allows you to specify the horizontal and vertical 
displacement of the child widget.

In this case, we set the offset to Offset(100, 100), which moves the second container 100 pixels to the right and 
100 pixels down from its original position.

You can adjust the values of offset according to your desired positioning. Additionally, you can modify the sizes 
and colors of the containers to suit your needs.

Running this updated code in a Flutter project should display the second container positioned on top of the first 
container based on the specified offset values.


Question: show tooltip for container

Answer: To add a tooltip to a Container widget in Flutter, you can wrap the Container with a Tooltip widget. 
The Tooltip widget provides a tooltip when the user hovers over or long-presses on the widget.

\begin{lstlisting}
import 'package:flutter/material.dart';

void main() {
    runApp(MyApp());
}

class MyApp extends StatelessWidget {
    @override
    Widget build(BuildContext context) {
    return MaterialApp(
        home: Scaffold(
        appBar: AppBar(
            title: Text('Tooltip Demo'),
        ),
        body: Center(
            child: Tooltip(
            message: 'This is a tooltip',
            child: Container(
                width: 200,
                height: 200,
                color: Colors.blue,
            ),
            ),
        ),
        ),
    );
    }
}    
\end{lstlisting}

In this example, we wrap the Container widget with the Tooltip widget. The Tooltip widget requires a message parameter, 
which represents the text to be displayed as the tooltip.

By running this code in a Flutter project, you will see a blue container. When you hover over or long-press on 
the container, a tooltip with the message "This is a tooltip" will appear.

You can modify the message property of the Tooltip widget to display a different tooltip message. Additionally, 
you can adjust the size, color, and other properties of the Container according to your specific requirements.


Question: get full width for Offset

Answer: To get the full width of the screen in Flutter, you can use the MediaQuery class to access the 
device's screen size.

\begin{lstlisting}
import 'package:flutter/material.dart';

void main() {
  runApp(MyApp());
}

class MyApp extends StatelessWidget {
  @override
  Widget build(BuildContext context) {
    // Get the screen width
    double screenWidth = MediaQuery.of(context).size.width;

    return MaterialApp(
      home: Scaffold(
        appBar: AppBar(
          title: Text('Full Width Example'),
        ),
        body: Center(
          child: Container(
            width: screenWidth,
            height: 200,
            color: Colors.blue,
            child: Text(
              'Full Width',
              style: TextStyle(fontSize: 24, color: Colors.white),
              textAlign: TextAlign.center,
            ),
          ),
        ),
      ),
    );
  }
}
\end{lstlisting}

In this example, we retrieve the screen width using MediaQuery.of(context).size.width and store it in the screenWidth 
variable. Then, we use the screenWidth variable as the width of the Container widget to make it take up the full 
width of the screen.

You can access the screenWidth variable within your widget's build method and use it wherever you need to set 
the width dynamically.

By running this code in a Flutter project, you will see a blue container with a text "Full Width" centered 
horizontally and taking up the full width of the screen.

Remember to import the necessary dependencies (import 'package:flutter/material.dart';) and wrap your app with 
the appropriate MaterialApp and Scaffold widgets.


Question: First business logic
if endDate <= currentDate then return 1
if startDate = currentDate then return 0
else return related float

Answer: Based on your description, you can create a function that takes startDate, endDate, and currentDate as 
parameters and returns a float value based on the conditions you mentioned. Here's an example implementation:

\begin{lstlisting}
double calculateValue(DateTime startDate, DateTime endDate, DateTime currentDate) {
  if (endDate.isBefore(currentDate) || endDate.isAtSameMomentAs(currentDate)) {
    return 1.0;
  } else if (startDate.isAtSameMomentAs(currentDate)) {
    return 0.0;
  } else {
    double totalDays = endDate.difference(startDate).inDays.toDouble();
    double currentDays = currentDate.difference(startDate).inDays.toDouble();
    return currentDays / totalDays;
  }
}
\end{lstlisting}

In this example, the calculateValue function takes three DateTime parameters: startDate, endDate, and currentDate.

If endDate is less than or equal to currentDate, it means the end date has passed or is the same as 
the current date, so we return 1.0.
If startDate is the same as currentDate, we return 0.0.
Otherwise, we calculate the float value based on the ratio of the number of days between startDate and 
currentDate to the total number of days between startDate and endDate.

To get the current date in Flutter, you can use the DateTime class along with the DateTime.now() method. 

Question: debug by plotting into console values

Answer: To debug and plot values in the console in Flutter, you can use the print() function to display 
the values of variables or expressions. 

\begin{lstlisting}
void main() {
  double value = 3.14159;
  String name = 'John Doe';
  int count = 42;

  print('Value: $value');
  print('Name: $name');
  print('Count: $count');
}
\end{lstlisting}

In this example, we have a double variable named value, a String variable named name, and an int variable named count. 
We use the print() function to display their values in the console.

By running this code in a Flutter project, you will see the values of value, name, and count printed in the console 
output. The values will be displayed along with their respective variable names.

You can apply the print() function wherever you need to debug and plot values in your code. You can also combine 
variables and strings using string interpolation, as shown in the example, to format the output and provide 
additional context for the displayed values.


Question: Make content scrollable

Answer: To make the content of a Flutter widget scrollable, you can wrap it with a SingleChildScrollView widget. 
The SingleChildScrollView allows its child to scroll vertically when its content exceeds the available vertical space.

\begin{lstlisting}
import 'package:flutter/material.dart';

void main() {
  runApp(MyApp());
}

class MyApp extends StatelessWidget {
  @override
  Widget build(BuildContext context) {
    return MaterialApp(
      home: Scaffold(
        appBar: AppBar(
          title: Text('Scrollable Content Example'),
        ),
        body: SingleChildScrollView(
          child: Container(
            height: 1000, // Set a larger height to demonstrate scrolling
            color: Colors.yellow,
            child: Center(
              child: Text(
                'Scrollable Content',
                style: TextStyle(fontSize: 24),
              ),
            ),
          ),
        ),
      ),
    );
  }
}
\end{lstlisting}

In this example, we wrap the Container widget with SingleChildScrollView. The Container has a fixed height of 1000 to 
exceed the available vertical space, so it requires scrolling.

By running this code in a Flutter project, you will see a yellow container with the text "Scrollable Content" 
centered within it. If the container's height exceeds the available vertical space, you will be able to scroll 
vertically to see the entire content.

Feel free to adjust the size, color, and content of the Container widget according to your requirements. The 
SingleChildScrollView will enable scrolling when needed, allowing you to display scrollable content in your app.

Question: how to avoid bottom overflow error on screen

Answer: o avoid the "Bottom overflowed by..." error in Flutter, you can make use of widgets that provide built-in 
scrolling behavior, such as ListView, Column inside a SingleChildScrollView, or CustomScrollView.

\begin{lstlisting}
    import 'package:flutter/material.dart';

    void main() {
      runApp(MyApp());
    }
    
    class MyApp extends StatelessWidget {
      @override
      Widget build(BuildContext context) {
        return MaterialApp(
          home: Scaffold(
            appBar: AppBar(
              title: Text('Avoid Bottom Overflow Error'),
            ),
            body: ListView(
              children: <Widget>[
                // Add your widgets here
                // ...
              ],
            ),
          ),
        );
      }
    }    
\end{lstlisting}

In this example, instead of using a regular Container or Column as the body of the Scaffold, we use a ListView widget. 
The ListView automatically provides scrolling behavior when the content exceeds the available vertical space.

By using a ListView, you can add as many child widgets as needed, and if the content exceeds the screen's height, 
the list will become scrollable, preventing the bottom overflow error.

Alternatively, you can wrap a Column widget with a SingleChildScrollView if you need more flexibility in organizing 
your content vertically.

\begin{lstlisting}
body: SingleChildScrollView(
  child: Column(
    children: <Widget>[
      // Add your widgets here
      // ...
    ],
  ),
),
\end{lstlisting}

By using a SingleChildScrollView with a Column, the content inside the Column will be scrollable if it exceeds the 
available vertical space.

Question: how to hide ListView on small screen based on height

Answer: To hide a ListView on a small screen based on its height, you can use the MediaQuery widget to access the 
screen dimensions and conditionally render the ListView based on a specific height threshold.

\begin{lstlisting}
import 'package:flutter/material.dart';

void main() {
  runApp(MyApp());
}

class MyApp extends StatelessWidget {
  final double screenHeightThreshold = 600; // Adjust the threshold value as needed

  @override
  Widget build(BuildContext context) {
    return MaterialApp(
      home: Scaffold(
        appBar: AppBar(
          title: Text('Hide ListView Example'),
        ),
        body: LayoutBuilder(
          builder: (BuildContext context, BoxConstraints constraints) {
            double screenHeight = MediaQuery.of(context).size.height;

            if (screenHeight < screenHeightThreshold) {
              // Hide the ListView on small screens
              return Container();
            } else {
              // Show the ListView on large screens
              return ListView(
                children: <Widget>[
                  // Add your ListView children here
                  // ...
                ],
              );
            }
          },
        ),
      ),
    );
  }
}
\end{lstlisting}

In this example, we use the LayoutBuilder widget to access the available screen size. We retrieve the screen height 
using MediaQuery.of(context).size.height. We then compare the screenHeight with a screenHeightThreshold value 
(in this case, 600), and if the screenHeight is below the threshold, we return an empty Container widget, 
effectively hiding the ListView. If the screenHeight is equal to or greater than the threshold, we return 
the ListView with its children.

You can adjust the screenHeightThreshold value to match your desired screen height breakpoint where the ListView 
should be hidden. This approach allows you to conditionally render the ListView based on the screen height, 
providing a responsive and adaptable UI.


Question: how to hide ListView on small screen based on constraints sizes

Answer: To hide a ListView on a small screen based on its constraints sizes, you can use the LayoutBuilder widget to 
access the constraints and conditionally render the ListView based on specific size constraints.

\begin{lstlisting}
import 'package:flutter/material.dart';

void main() {
  runApp(MyApp());
}

class MyApp extends StatelessWidget {
  final double minWidthThreshold = 400; // Adjust the minimum width threshold
  final double minHeightThreshold = 600; // Adjust the minimum height threshold

  @override
  Widget build(BuildContext context) {
    return MaterialApp(
      home: Scaffold(
        appBar: AppBar(
          title: Text('Hide ListView Example'),
        ),
        body: LayoutBuilder(
          builder: (BuildContext context, BoxConstraints constraints) {
            double minWidth = constraints.minWidth;
            double minHeight = constraints.minHeight;

            if (minWidth < minWidthThreshold || minHeight < minHeightThreshold) {
              // Hide the ListView on small screens
              return Container();
            } else {
              // Show the ListView on large screens
              return ListView(
                children: <Widget>[
                  // Add your ListView children here
                  // ...
                ],
              );
            }
          },
        ),
      ),
    );
  }
}
\end{lstlisting}

In this example, we use the LayoutBuilder widget to access the constraints provided by its parent. We retrieve the 
minimum width and minimum height of the constraints using constraints.minWidth and constraints.minHeight. We then 
compare these values with the respective thresholds (minWidthThreshold and minHeightThreshold), and if either the 
minimum width or minimum height is below the threshold, we return an empty Container widget, effectively hiding 
the ListView. If both the minimum width and minimum height meet or exceed the thresholds, we return the ListView 
with its children.

You can adjust the minWidthThreshold and minHeightThreshold values to match your desired size constraints where 
the ListView should be hidden. This approach allows you to conditionally render the ListView based on the 
constraints sizes, providing a responsive and adaptable UI.

Question: Add button to appbar

Answer: To add a button to the AppBar in Flutter, you can use the actions property of the AppBar and provide a 
list of widgets that represent the buttons you want to add.

\begin{lstlisting}
import 'package:flutter/material.dart';

void main() {
  runApp(MyApp());
}

class MyApp extends StatelessWidget {
  @override
  Widget build(BuildContext context) {
    return MaterialApp(
      home: Scaffold(
        appBar: AppBar(
          title: Text('AppBar with Button'),
          actions: [
            IconButton(
              icon: Icon(Icons.settings),
              onPressed: () {
                // Add your button's onPressed logic here
                // ...
              },
            ),
          ],
        ),
        body: Center(
          child: Text('Content goes here'),
        ),
      ),
    );
  }
}
\end{lstlisting}

In this example, we add an IconButton to the actions property of the AppBar. The IconButton displays an icon (in this 
case, the "settings" icon) and invokes the onPressed callback when pressed. Inside the onPressed callback, you can 
define the logic that should be executed when the button is pressed.

You can customize the appearance and behavior of the button by using different types of widgets, such as IconButton, 
FlatButton, RaisedButton, or TextButton. Choose the widget that best suits your needs and customize it accordingly.

By running this code in a Flutter project, you will see an AppBar with a title and a button on the right side. When 
you tap the button, the onPressed callback will be invoked, allowing you to define the desired behavior for your button.


Question: add navigation menu

Answer: To add a navigation menu, you can use the Drawer widget in Flutter. The Drawer provides a panel that slides 
in from the left or right side of the screen to display navigation options or settings.

\begin{lstlisting}
import 'package:flutter/material.dart';

void main() {
  runApp(MyApp());
}

class MyApp extends StatelessWidget {
  void _navigateToPage(BuildContext context, String routeName) {
    Navigator.pop(context); // Close the drawer
    Navigator.pushNamed(context, routeName); // Navigate to the specified route
  }

  @override
  Widget build(BuildContext context) {
    return MaterialApp(
      home: Scaffold(
        appBar: AppBar(
          title: Text('Navigation Menu Example'),
        ),
        drawer: Drawer(
          child: ListView(
            padding: EdgeInsets.zero,
            children: [
              DrawerHeader(
                decoration: BoxDecoration(
                  color: Colors.blue,
                ),
                child: Text(
                  'Navigation Menu',
                  style: TextStyle(
                    fontSize: 24,
                    color: Colors.white,
                  ),
                ),
              ),
              ListTile(
                leading: Icon(Icons.home),
                title: Text('Home'),
                onTap: () => _navigateToPage(context, '/home'),
              ),
              ListTile(
                leading: Icon(Icons.settings),
                title: Text('Settings'),
                onTap: () => _navigateToPage(context, '/settings'),
              ),
            ],
          ),
        ),
        body: Center(
          child: Text('Content goes here'),
        ),
      ),
    );
  }
}
\end{lstlisting}

In this example, we create a Drawer and provide it as the drawer property of the Scaffold. Inside the Drawer, we use 
a ListView with a DrawerHeader and a list of ListTile widgets representing the navigation options.

Each ListTile represents a navigation item and is assigned an icon, title, and an onTap callback. In the onTap 
callback, we close the drawer using Navigator.pop(context) and navigate to the desired route using 
Navigator.pushNamed(context, routeName).

You can customize the appearance and behavior of the navigation menu by modifying the icons, titles, and onTap 
callbacks of the ListTile widgets. Additionally, you can add more navigation options or additional widgets inside 
the Drawer to suit your app's needs.

By running this code in a Flutter project, you will see an AppBar with a navigation menu icon (typically three 
horizontal lines or a menu icon). Tapping the menu icon will open the navigation menu, and tapping on a navigation 
item will close the drawer and navigate to the corresponding page.

Remember to define the routes for the different pages in your app using MaterialApp or Navigator so that the 
navigation can be properly handled.


Question: How to stylize Drawer

Answer: To stylize the Drawer in Flutter, you can customize various aspects such as the background color, text 
style, and item icons. 

\begin{lstlisting}
import 'package:flutter/material.dart';

void main() {
  runApp(MyApp());
}

class MyApp extends StatelessWidget {
  void _navigateToPage(BuildContext context, String routeName) {
    Navigator.pop(context); // Close the drawer
    Navigator.pushNamed(context, routeName); // Navigate to the specified route
  }

  @override
  Widget build(BuildContext context) {
    return MaterialApp(
      home: Scaffold(
        appBar: AppBar(
          title: Text('Stylized Drawer Example'),
        ),
        drawer: Drawer(
          child: Container(
            color: Colors.grey[900], // Set the background color
            child: ListView(
              padding: EdgeInsets.zero,
              children: [
                DrawerHeader(
                  decoration: BoxDecoration(
                    color: Colors.blue,
                  ),
                  child: Text(
                    'Navigation Menu',
                    style: TextStyle(
                      fontSize: 24,
                      color: Colors.white,
                    ),
                  ),
                ),
                ListTile(
                  leading: Icon(
                    Icons.home,
                    color: Colors.white, // Set the icon color
                  ),
                  title: Text(
                    'Home',
                    style: TextStyle(
                      fontSize: 16,
                      color: Colors.white, // Set the text color
                    ),
                  ),
                  onTap: () => _navigateToPage(context, '/home'),
                ),
                ListTile(
                  leading: Icon(
                    Icons.settings,
                    color: Colors.white, // Set the icon color
                  ),
                  title: Text(
                    'Settings',
                    style: TextStyle(
                      fontSize: 16,
                      color: Colors.white, // Set the text color
                    ),
                  ),
                  onTap: () => _navigateToPage(context, '/settings'),
                ),
              ],
            ),
          ),
        ),
        body: Center(
          child: Text('Content goes here'),
        ),
      ),
    );
  }
}
\end{lstlisting}

In this example, we make the following style modifications to the Drawer:

We set the background color of the Drawer container using the color property of the Container widget. In this case, 
we set it to Colors.grey[900].
We customize the text style of the navigation items by setting the fontSize and color properties of the Text widget.
We customize the icon color of the navigation items by setting the color property of the Icon widget.
We can also customize other aspects such as the decoration of the DrawerHeader, which sets the background color for
the header section.

Feel free to adjust these styles according to your preferences. You can modify the colors, font sizes, and other 
properties to match your app's design.

By running this code in a Flutter project, you will see a stylized Drawer with a customized background color, text 
style, and icon color.

Question: remove rounded corners for Drawer

Answer: To remove the rounded corners of the Drawer in Flutter and make it have straight edges, you can use the 
ShapeBorder property of the Drawer to define a custom shape without rounded corners.

\begin{lstlisting}
import 'package:flutter/material.dart';

void main() {
  runApp(MyApp());
}

class MyApp extends StatelessWidget {
  void _navigateToPage(BuildContext context, String routeName) {
    Navigator.pop(context); // Close the drawer
    Navigator.pushNamed(context, routeName); // Navigate to the specified route
  }

  @override
  Widget build(BuildContext context) {
    return MaterialApp(
      home: Scaffold(
        appBar: AppBar(
          title: Text('Drawer without Rounded Corners'),
        ),
        drawer: Drawer(
          elevation: 0, // Remove the elevation (shadow) of the drawer
          shape: Border.all(width: 0), // Set a border without any width
          child: Container(
            color: Colors.grey[900], // Set the background color
            child: ListView(
              padding: EdgeInsets.zero,
              children: [
                DrawerHeader(
                  decoration: BoxDecoration(
                    color: Colors.blue,
                  ),
                  child: Text(
                    'Navigation Menu',
                    style: TextStyle(
                      fontSize: 24,
                      color: Colors.white,
                    ),
                  ),
                ),
                ListTile(
                  leading: Icon(
                    Icons.home,
                    color: Colors.white, // Set the icon color
                  ),
                  title: Text(
                    'Home',
                    style: TextStyle(
                      fontSize: 16,
                      color: Colors.white, // Set the text color
                    ),
                  ),
                  onTap: () => _navigateToPage(context, '/home'),
                ),
                ListTile(
                  leading: Icon(
                    Icons.settings,
                    color: Colors.white, // Set the icon color
                  ),
                  title: Text(
                    'Settings',
                    style: TextStyle(
                      fontSize: 16,
                      color: Colors.white, // Set the text color
                    ),
                  ),
                  onTap: () => _navigateToPage(context, '/settings'),
                ),
              ],
            ),
          ),
        ),
        body: Center(
          child: Text('Content goes here'),
        ),
      ),
    );
  }
}
\end{lstlisting}

In this example, we modify the Drawer by setting the following properties:

We set the elevation property of the Drawer to 0 to remove the elevation (shadow) effect.
We set the shape property of the Drawer to a Border with zero width. This effectively removes the rounded 
corners and gives the drawer straight edges.
We wrap the Drawer content in a Container and set its background color using the color property.

By running this code in a Flutter project, you will see a Drawer without rounded corners and with a customized 
background color.


Question: Center vertically ListView

Answer: To center a ListView vertically in Flutter, you can wrap it with a Container and use the alignment property 
to align the contents vertically.

\begin{lstlisting}
import 'package:flutter/material.dart';

void main() {
  runApp(MyApp());
}

class MyApp extends StatelessWidget {
  List<String> items = [
    'Item 1',
    'Item 2',
    'Item 3',
    'Item 4',
    'Item 5',
  ];

  @override
  Widget build(BuildContext context) {
    return MaterialApp(
      home: Scaffold(
        appBar: AppBar(
          title: Text('Centered ListView'),
        ),
        body: Container(
          alignment: Alignment.center,
          child: ListView.builder(
            itemCount: items.length,
            itemBuilder: (context, index) {
              return ListTile(
                title: Text(items[index]),
              );
            },
          ),
        ),
      ),
    );
  }
}
\end{lstlisting}

In this example, we wrap the ListView.builder with a Container and set the alignment property to Alignment.center. 
This centers the ListView vertically within the available space.

You can customize the ListView by providing your own data and modifying the itemBuilder function to create the 
desired list items.

Run this code in a Flutter project to see the ListView centered vertically on the screen.


Question: get indent between children of ListView

Answer: To add indentation or spacing between children in a ListView, you can utilize the ListView.separated 
constructor instead of ListView.builder. The ListView.separated constructor allows you to specify a separator 
widget that will be inserted between each pair of children.

\begin{lstlisting}
import 'package:flutter/material.dart';

void main() {
  runApp(MyApp());
}

class MyApp extends StatelessWidget {
  List<String> items = [
    'Item 1',
    'Item 2',
    'Item 3',
    'Item 4',
    'Item 5',
  ];

  @override
  Widget build(BuildContext context) {
    return MaterialApp(
      home: Scaffold(
        appBar: AppBar(
          title: Text('Indented ListView'),
        ),
        body: Container(
          alignment: Alignment.center,
          child: ListView.separated(
            itemCount: items.length,
            separatorBuilder: (context, index) => SizedBox(height: 16.0),
            itemBuilder: (context, index) {
              return ListTile(
                title: Text(items[index]),
              );
            },
          ),
        ),
      ),
    );
  }
}
\end{lstlisting}

To add indentation or spacing between children in a ListView, you can utilize the ListView.separated constructor 
instead of ListView.builder. The ListView.separated constructor allows you to specify a separator widget that will 
be inserted between each pair of children.

Here's an example that demonstrates how to add indentation between ListTile widgets in a ListView:

\begin{lstlisting}
import 'package:flutter/material.dart';

void main() {
  runApp(MyApp());
}

class MyApp extends StatelessWidget {
  List<String> items = [
    'Item 1',
    'Item 2',
    'Item 3',
    'Item 4',
    'Item 5',
  ];

  @override
  Widget build(BuildContext context) {
    return MaterialApp(
      home: Scaffold(
        appBar: AppBar(
          title: Text('Indented ListView'),
        ),
        body: Container(
          alignment: Alignment.center,
          child: ListView.separated(
            itemCount: items.length,
            separatorBuilder: (context, index) => SizedBox(height: 16.0),
            itemBuilder: (context, index) {
              return ListTile(
                title: Text(items[index]),
              );
            },
          ),
        ),
      ),
    );
  }
}
\end{lstlisting}

In this example, we use ListView.separated instead of ListView.builder. We provide a separatorBuilder function that 
returns a SizedBox with a height of 16.0 between each pair of ListTile widgets.

Feel free to adjust the value of 16.0 to increase or decrease the indentation between the ListTile widgets according 
to your desired spacing.


Question: on hover for Drawer

Answer: To add a hover effect for the items in a Flutter Drawer, you can use the MouseRegion widget to detect mouse 
events and change the styling of the drawer items accordingly. 

\begin{lstlisting}
import 'package:flutter/material.dart';

void main() {
  runApp(MyApp());
}

class MyApp extends StatelessWidget {
  @override
  Widget build(BuildContext context) {
    return MaterialApp(
      home: Scaffold(
        appBar: AppBar(
          title: Text('Drawer with Hover Effect'),
        ),
        drawer: MyDrawer(),
        body: Center(
          child: Text('Content of the Page'),
        ),
      ),
    );
  }
}

class MyDrawer extends StatelessWidget {
  final List<String> drawerItems = [
    'Item 1',
    'Item 2',
    'Item 3',
  ];

  @override
  Widget build(BuildContext context) {
    return Drawer(
      child: ListView.builder(
        itemCount: drawerItems.length,
        itemBuilder: (context, index) {
          return MouseRegion(
            onEnter: (event) {
              // Handle mouse enter event
              // You can change the style of the drawer item here
            },
            onExit: (event) {
              // Handle mouse exit event
              // You can reset the style of the drawer item here
            },
            child: ListTile(
              title: Text(drawerItems[index]),
              onTap: () {
                // Handle item tap event
              },
            ),
          );
        },
      ),
    );
  }
}
\end{lstlisting}

In this example, we create a custom MyDrawer widget that extends StatelessWidget and represents the drawer content. 
Inside the build method of MyDrawer, we use a ListView.builder to build the list of drawer items dynamically.

For each drawer item, we wrap it with a MouseRegion widget to detect mouse enter and exit events. You can customize 
the styling of the drawer item based on these events.

Inside the MouseRegion, we have a ListTile widget representing each drawer item. You can handle the onTap event to 
perform specific actions when the item is clicked.

Question: highlight by hover

Answer: To add a hover effect to the items in a Flutter Drawer and highlight them when hovered, you can use the 
InkWell widget along with the onHover property.

\begin{lstlisting}
import 'package:flutter/material.dart';

void main() {
  runApp(MyApp());
}

class MyApp extends StatelessWidget {
  @override
  Widget build(BuildContext context) {
    return MaterialApp(
      home: Scaffold(
        appBar: AppBar(
          title: Text('Drawer with Hover Highlight'),
        ),
        drawer: MyDrawer(),
        body: Center(
          child: Text('Content of the Page'),
        ),
      ),
    );
  }
}

class MyDrawer extends StatefulWidget {
  @override
  _MyDrawerState createState() => _MyDrawerState();
}

class _MyDrawerState extends State<MyDrawer> {
  int selectedIndex = 0;

  List<String> drawerItems = [
    'Item 1',
    'Item 2',
    'Item 3',
  ];

  @override
  Widget build(BuildContext context) {
    return Drawer(
      child: ListView.builder(
        itemCount: drawerItems.length,
        itemBuilder: (context, index) {
          return InkWell(
            onTap: () {
              setState(() {
                selectedIndex = index;
              });
              // Handle item tap event
            },
            onHover: (isHovered) {
              if (isHovered) {
                setState(() {
                  selectedIndex = index;
                });
              }
            },
            child: Container(
              color: selectedIndex == index ? Colors.blue.withOpacity(0.2) : null,
              child: ListTile(
                title: Text(drawerItems[index]),
              ),
            ),
          );
        },
      ),
    );
  }
}
\end{lstlisting}

In this example, we have a MyDrawer widget that extends StatefulWidget to maintain the selected index state. Inside 
the build method, we create a Drawer and use a ListView.builder to dynamically build the drawer items.

For each item, we wrap it with an InkWell widget to handle tap and hover events. The onTap callback is triggered 
when the item is clicked, and the onHover callback is triggered when the item is hovered.

Inside the InkWell, we have a Container that applies a background color to the item when it is selected or hovered. 
You can modify the color and opacity to match your desired highlighting effect.

Question: how to identify horizontal layout

Answer: To identify whether a Flutter widget is using a horizontal layout, you can check the constraints and layout 
properties of the widget. Here are a few approaches to determine the horizontal layout:

Check the BoxConstraints: You can inspect the BoxConstraints of the widget to determine the available width and 
height. If the width constraint is greater than the height constraint, it indicates a horizontal layout.

\begin{lstlisting}
LayoutBuilder(
  builder: (context, constraints) {
    if (constraints.maxWidth > constraints.maxHeight) {
      // Horizontal layout
      return Container(
        color: Colors.red,
        width: constraints.maxWidth,
        height: constraints.maxHeight,
      );
    } else {
      // Vertical layout
      return Container(
        color: Colors.blue,
        width: constraints.maxWidth,
        height: constraints.maxHeight,
      );
    }
  },
)
\end{lstlisting}

In this example, the LayoutBuilder widget provides the constraints of its parent. By comparing the maxWidth and 
maxHeight, you can determine if the layout is horizontal or vertical.

Check the Orientation of the device: If you want to detect the device's orientation and identify a horizontal 
layout, you can use the MediaQuery to access the device's orientation. 

\begin{lstlisting}
if (MediaQuery.of(context).orientation == Orientation.landscape) {
    // Horizontal layout
    return Container(
        color: Colors.red,
        width: MediaQuery.of(context).size.width,
        height: MediaQuery.of(context).size.height,
    );
} else {
    // Vertical layout
    return Container(
        color: Colors.blue,
        width: MediaQuery.of(context).size.width,
        height: MediaQuery.of(context).size.height,
    );
}
\end{lstlisting}

In this example, the MediaQuery.of(context).orientation provides the current device orientation, and you can use it 
to conditionally render different layouts.

Question: Flutter Container overflow with dots

Answer: To create a container with overflow dots in Flutter, you can use the Text widget with the overflow 
property set to TextOverflow.ellipsis

\begin{lstlisting}
Container(
    width: 200,
    child: Text(
        'This is a long text that may overflow the container',
        overflow: TextOverflow.ellipsis,
    ),
)      
\end{lstlisting}

In this example, the Container has a fixed width of 200 pixels. The Text widget inside the Container contains a 
long text. By setting the overflow property of the Text widget to TextOverflow.ellipsis, the text will be truncated 
with an ellipsis (...) if it overflows the available width of the container.

If you want to display the overflow dots at the end of the text, you can wrap the Text widget with a Row and an 
Expanded widget. 

\begin{lstlisting}
Container(
  width: 200,
  child: Row(
    children: [
      Expanded(
        child: Text(
          'This is a long text that may overflow the container',
          overflow: TextOverflow.ellipsis,
        ),
      ),
    ],
  ),
)
\end{lstlisting}

In this updated example, the Text widget is wrapped with an Expanded widget, which allows the text to take up the 
remaining space within the Row. If the text overflows the available space, the ellipsis will be displayed at the 
end of the text.


Question: width half of screen

Answer: To set the width of a container to half of the screen in Flutter, you can use the MediaQuery to get the 
screen size and calculate the desired width. 

\begin{lstlisting}
Container(
  width: MediaQuery.of(context).size.width * 0.5,
)
\end{lstlisting}

In this example, MediaQuery.of(context).size.width retrieves the total width of the screen. By multiplying it with 0.5
 (half), you can set the width of the container to half of the screen width.

Make sure to use this code within a widget's build method and pass context as the parameter to access the current 
context.

Question: tooltip for container

Answer: To add a tooltip to a Flutter container, you can use the Tooltip widget as a parent to the container. 

\begin{lstlisting}
Tooltip(
    message: 'This is a tooltip',
    child: Container(
        width: 200,
        height: 50,
        color: Colors.blue,
        // other properties and child widgets
    ),
)      
\end{lstlisting}

In this example, the Tooltip widget wraps around the Container widget. The message property of the Tooltip widget 
defines the text to be displayed as the tooltip when the user hovers over or long-presses on the container.

You can customize the appearance of the tooltip by specifying additional properties for the Tooltip widget, such 
as preferBelow to control the placement of the tooltip below the container, verticalOffset to adjust the vertical 
position of the tooltip, and height to set a specific height for the tooltip.

By wrapping the container with the Tooltip widget, you can provide additional contextual information or descriptions 
to the user when interacting with the container in your Flutter application.

Question: how to extend TextTheme with own parameters

Answer: To extend the TextTheme in Flutter with your own custom parameters, you can create a new TextTheme instance 
and provide values for the desired properties.

\begin{lstlisting}
TextTheme myTextTheme = TextTheme(
  headline1: TextStyle(fontSize: 32, fontWeight: FontWeight.bold),
  headline2: TextStyle(fontSize: 24, fontWeight: FontWeight.bold),
  numberLarge: TextStyle(fontSize: 48, fontWeight: FontWeight.bold), // Custom property
);
\end{lstlisting}

In this example, we create a new TextTheme called myTextTheme. We define the headline1 and headline2 properties with 
custom font sizes and font weights. Additionally, we add a custom property called numberLarge with a larger font size.

However, please note that the error you mentioned (The named parameter 'numberLarge' isn't defined) occurs because 
numberLarge is not a built-in property of TextTheme. To define custom properties, you need to create your own 
extension of the TextTheme class.

Here's an example of how you can create a custom extension of TextTheme:

\begin{lstlisting}
extension MyTextTheme on TextTheme {
  TextStyle get numberLarge => TextStyle(fontSize: 48, fontWeight: FontWeight.bold);
}
\end{lstlisting}

By using an extension, you can define additional properties or methods for the TextTheme class. In this case, 
the numberLarge property is added to the TextTheme class, allowing you to access it directly from an instance of 
TextTheme.

You can then use your custom TextTheme and the extended properties in your app as follows:

\begin{lstlisting}
MaterialApp(
  theme: ThemeData(
    textTheme: myTextTheme,
  ),
  // other app configuration
)
\end{lstlisting}

By setting the textTheme property of your app's ThemeData to your custom TextTheme instance, you can apply the 
extended properties throughout your app.

To have an effect inside different files do not forget to include the theme:
\begin{lstlisting}
import 'package:app_finance/custom_text_theme.dart';
\end{lstlisting}

Otherwise your customizations won't be visible.


Question: will use theme dark and light modes?

Answer: to extend the TextTheme for both dark and light modes

\begin{lstlisting}
extension MyTextTheme on TextTheme {
  TextStyle get numberLarge => TextStyle(fontSize: 48, fontWeight: FontWeight.bold);
}

class MyApp extends StatelessWidget {
  @override
  Widget build(BuildContext context) {
    return MaterialApp(
      theme: ThemeData.light().copyWith(
        textTheme: TextTheme().apply(
          bodyColor: Colors.black, // Customize text color for light mode
        ),
      ),
      darkTheme: ThemeData.dark().copyWith(
        textTheme: TextTheme().apply(
          bodyColor: Colors.white, // Customize text color for dark mode
        ),
      ),
      themeMode: ThemeMode.system, // Use system theme mode (light or dark based on device settings)
      home: MyHomePage(),
    );
  }
}

class MyHomePage extends StatelessWidget {
  @override
  Widget build(BuildContext context) {
    return Scaffold(
      appBar: AppBar(
        title: Text(
          'My App',
          style: Theme.of(context).textTheme.headline6,
        ),
      ),
      body: Center(
        child: Text(
          'Hello, World!',
          style: Theme.of(context).textTheme.numberLarge,
        ),
      ),
    );
  }
}
\end{lstlisting}

In this example, we extend the TextTheme using the MyTextTheme extension. Then, in the MyApp widget, we configure the 
light and dark themes by customizing the text color for each mode using the bodyColor property. We set the themeMode 
to ThemeMode.system to follow the system's light or dark mode based on device settings.

Inside the MyHomePage widget, we use the Theme.of(context) to access the appropriate theme's TextTheme and apply 
the extended numberLarge style to the Text widget.

By using this approach, the text style defined in the numberLarge extension will automatically adapt to the chosen 
theme mode (light or dark) without needing separate extensions for each mode.


Question: TextTheme.copyWith to update only font size

Answer: To update only the font size of a TextTheme while keeping the other properties intact, you can use the 
copyWith method. 

\begin{lstlisting}
class CustomTextTheme {
    static TextTheme? createCustomTextTheme(TextTheme? baseTheme) {
        if (baseTheme == null) {
        return null;
        }
        
        return baseTheme.copyWith(
        headline1: baseTheme.headline1!.copyWith(fontSize: 32, fontWeight: FontWeight.bold),
        headline2: baseTheme.headline2!.copyWith(fontSize: 24, fontWeight: FontWeight.bold),
        // Add your custom text styles here
        );
    }
}     
\end{lstlisting}

In this updated example, the createCustomTextTheme function accepts a nullable baseTheme. It checks if the 
baseTheme is null and returns null in that case. Otherwise, it proceeds to create a modified copy of the 
baseTheme with your custom text styles applied.

To use this custom text theme in your app, you can pass CustomTextTheme.myTextTheme to the textTheme property of 
your app's ThemeData:

\begin{lstlisting}
MaterialApp(
    theme: ThemeData(
        textTheme: CustomTextTheme.createCustomTextTheme(Theme.of(context).textTheme),
    ),
    // Other app configurations
)   
\end{lstlisting}

By using this approach, you can generate a custom TextTheme by invoking the updated utility function that accepts a 
nullable TextTheme and returns a nullable TextTheme.

Question: How to resolve overflow on complex container

Answer: To resolve overflow issues on a complex container, you can use various techniques to handle the content 
overflow and ensure that it fits within the available space. Here are a few approaches you can consider:

Wrap the overflowing content with an OverflowBox widget: The OverflowBox widget allows you to specify how the 
child should behave when it overflows the parent's constraints. You can set the maxWidth and maxHeight properties 
of the OverflowBox to limit the size of the overflowing content.

Implement scrolling: If the container has a large amount of content, you can wrap it with a SingleChildScrollView, 
ListView, or CustomScrollView to provide scrolling functionality. This allows the user to scroll through the content 
and view it in its entirety.

Adjust the layout: Review the layout of your container and consider using techniques like Expanded, Flexible, or 
SizedBox with appropriate constraints to control the size and positioning of the child widgets. This can help 
distribute the available space more effectively and prevent overflow.

Use FittedBox or LayoutBuilder: If you have a specific widget that is causing the overflow, you can wrap it with a 
FittedBox widget to scale its content down to fit within the available space. Alternatively, you can use a 
LayoutBuilder to obtain the available space constraints and dynamically adjust the layout of the child widgets 
based on those constraints.

Remember to analyze your container's structure, review the content, and apply the appropriate techniques to handle 
the overflow effectively. It may require a combination of approaches depending on the complexity of the container.

Question: how to use OverflowBox 

Answer: To use the OverflowBox widget, you can wrap it around a child widget that may overflow its parent's constraints. 
The OverflowBox widget allows you to control how the child widget behaves when it exceeds the available space.

\begin{lstlisting}
OverflowBox(
  maxWidth: double.infinity, // Maximum width allowed for the child
  maxHeight: double.infinity, // Maximum height allowed for the child
  child: YourContentWidget(),
)
\end{lstlisting}

In this example, YourContentWidget() represents the child widget that might overflow its parent. By wrapping it with 
OverflowBox, you can set the maxWidth and maxHeight properties to limit the size of the child widget within the 
available space.

Note that setting double.infinity for maxWidth and maxHeight allows the child widget to take up the full available 
width and height, respectively. You can adjust these values according to your specific needs.

By using OverflowBox, you can prevent the child widget from exceeding the specified maximum dimensions, and it will 
be scaled down or clipped to fit within those bounds.

Remember to consider the overall layout and structure of your container, as well as the relationship between the child 
widget and its parent, to effectively handle the overflow and maintain a visually pleasing UI.

Question: check that nullable variable is not set

Answer: To check if a nullable variable is not set (i.e., null), you can use the null-aware operator ?. in Dart. 

\begin{lstlisting}
String? nullableVariable;

if (nullableVariable == null) {
  // Variable is not set
  print('Nullable variable is not set');
} else {
  // Variable is set
  print('Nullable variable is set: $nullableVariable');
}
\end{lstlisting}

In this code, we first declare a nullable variable nullableVariable of type String?. Then, we check if the variable 
is null using the == operator. If the variable is null, it means it is not set. Otherwise, if the variable has a 
non-null value, we can access and use it accordingly.

The null-aware operator ?. is used to safely access properties or invoke methods on nullable objects without causing 
a null reference exception. It short-circuits the expression and returns null if the object is null, otherwise, it 
continues with the property access or method invocation.

By using the == operator to check for null, you can determine if a nullable variable is not set and handle it 
accordingly in your code.


Question: how to use flutter pub run grinder

Answer: To use the flutter pub run grinder l10n command, you need to have the grinder package added as a dev 
dependency in your pubspec.yaml file. Here's how you can set it up:

Open your project's pubspec.yaml file.

Locate the dev\_dependencies section.

Add the grinder package as a dev dependency:

\begin{lstlisting}
dev_dependencies:
  grinder: ^1.0.0
\end{lstlisting}

Save the pubspec.yaml file.

Once you have the grinder package added, you can run the flutter pub run grinder l10n command to generate the 
localized resources for your Flutter app.

Note: The grinder package is a task runner for Dart, and it provides a convenient way to automate repetitive tasks 
in your project. In this case, it is used to generate localized resources based on the localization configuration 
in your project.

Make sure you have the necessary localization configuration set up in your Flutter project. Typically, this 
involves adding localization files (e.g., .arb files) and configuring them in your pubspec.yaml file.

After setting up the grinder package and the localization configuration, you can run the flutter pub run grinder 
l10n command in your project's root directory using a terminal or command prompt.

To install the grinder package from the console, you can use the following command:
dart pub global activate grinder

This command activates the grinder package globally on your system, making it available for use in any Dart project.

Make sure you have Dart SDK installed and added to your system's PATH environment variable before running the 
above command. You can verify the installation by running dart --version in the console, which should display the 
Dart SDK version if it is properly installed.

Once the installation is complete, you can use the grinder command in the console to run tasks defined in a 
Grindfile.dart script within your project.

For example, you can run grinder l10n to execute the l10n task defined in the Grindfile.dart script, assuming 
you have set up the Grindfile.dart with the necessary tasks.

Remember that grinder is a task runner for Dart, and you need to have a valid Grindfile.dart script with the 
required tasks defined in order to use it effectively.

dart run grinder:init - to generate grinder file with basic construction.

grinder show list of tasks: dart run grinder -h

Question: Flutter check that variable is a function

Answer: In Flutter, you can use the is operator to check if a variable is a function. 

\begin{lstlisting}
void main() {
  var variable = () {
    // Function body
  };

  if (variable is Function) {
    print('Variable is a function');
  } else {
    print('Variable is not a function');
  }
}
\end{lstlisting}

In this example, the is operator is used to check if the variable is of type Function. If it is, the condition 
variable is Function will evaluate to true, indicating that the variable is a function. Otherwise, it will 
evaluate to false, indicating that the variable is not a function.


Question: Flutter optional parameters to the function

Answer: In Flutter, you can define optional parameters in a function by enclosing them in square brackets 
([]) in the function declaration. 

\begin{lstlisting}
void greet(String name, {String? prefix, String? suffix}) {
  String message = 'Hello, $name';

  if (prefix != null) {
    message = '$prefix $message';
  }

  if (suffix != null) {
    message = '$message $suffix';
  }

  print(message);
}
\end{lstlisting}

n the greet function above, prefix and suffix are optional parameters. They are enclosed in square brackets 
in the function declaration. You can call the greet function and provide values for the optional parameters 
if needed, or omit them to use the default values. 

Question: Flutter set default locale if not found

\begin{lstlisting}
MaterialApp(
  ...
  localizationsDelegates: AppLocalizations.localizationsDelegates,
  supportedLocales: AppLocalizations.supportedLocales,
  ...
)
\end{lstlisting}

Question: how to create menu with three dots on right side

Answer: To create a menu with three dots on the right side of the app bar in Flutter, you can use the PopupMenuButton 
widget along with the PopupMenuEntry widgets.

\begin{lstlisting}
AppBar(
  title: Text('My App'),
  actions: [
    PopupMenuButton(
      itemBuilder: (BuildContext context) {
        return [
          PopupMenuItem(
            child: Text('Option 1'),
            value: 'option1',
          ),
          PopupMenuItem(
            child: Text('Option 2'),
            value: 'option2',
          ),
          PopupMenuItem(
            child: Text('Option 3'),
            value: 'option3',
          ),
        ];
      },
      onSelected: (value) {
        // Handle the selected option
        if (value == 'option1') {
          // Option 1 selected
        } else if (value == 'option2') {
          // Option 2 selected
        } else if (value == 'option3') {
          // Option 3 selected
        }
      },
      icon: Icon(Icons.more_vert),
    ),
  ],
)
\end{lstlisting}

In the above code, the PopupMenuButton widget is added to the actions list of the AppBar. The itemBuilder property 
is used to define the menu items as PopupMenuItem widgets. Each menu item has a child widget, which in this case 
is a Text widget, and a value associated with it. When a menu item is selected, the onSelected callback is 
triggered with the selected value.

The icon property of the PopupMenuButton widget is set to an Icon widget, in this case, the Icons.more\_vert 
icon, which represents the three dots.


Question: how to convert
\begin{lstlisting}
var menuList = [
  (
    name: 'Home',
    icon: Icons.home,
    route: homeRoute,
  ),
  (
    name: 'Goals',
    icon: Icons.star,
    route: homeRoute,
  )
];
\end{lstlisting}
to PopupMenuButton

To convert the menuList to a PopupMenuButton widget, you can map each item in the list to a PopupMenuItem widget. 

\begin{lstlisting}
PopupMenuButton(
  itemBuilder: (BuildContext context) {
    return menuList.map((menuItem) {
      return PopupMenuItem(
        child: Row(
          children: [
            Icon(menuItem['icon']),
            SizedBox(width: 8),
            Text(menuItem['name']),
          ],
        ),
        value: menuItem['route'],
      );
    }).toList();
  },
  onSelected: (value) {
    // Handle the selected route
    // You can navigate to the selected route using Navigator.pushNamed(context, value)
    Navigator.pushNamed(context, value);
  },
  icon: Icon(Icons.more_vert),
)
\end{lstlisting}

In this example, menuList is mapped to a list of PopupMenuItem widgets using the map method. Each PopupMenuItem 
contains a Row with an Icon and Text widget to display the icon and name from each item in the menuList. 
The value property of PopupMenuItem is set to the route value from each item.

When a menu item is selected, the onSelected callback is triggered with the selected route value. In this example,
Navigator.pushNamed is used to navigate to the selected route.


Question: Drawer icon change

Answer: To change the drawer icon in Flutter, you can use the IconButton widget along with the Scaffold widget. 

\begin{lstlisting}
Scaffold(
  appBar: AppBar(
    title: Text('My App'),
    leading: IconButton(
      icon: Icon(Icons.menu),
      onPressed: () {
        // Open the drawer
        Scaffold.of(context).openDrawer();
      },
    ),
  ),
  drawer: Drawer(
    // Drawer content goes here
  ),
  // Other body content of the scaffold
)
\end{lstlisting}

In this example, the leading property of the AppBar is set to an IconButton with the Icons.menu icon. When the 
icon is pressed, the onPressed callback is triggered, and the openDrawer() method is called to open the drawer.

Make sure to wrap the Scaffold widget with a Builder widget if you're using this code inside a function 
where the context is not directly available.

Question: set width of full width to FloatingActionButton

Answer: To set the width of a FloatingActionButton to the full width of the screen, you can wrap it with a Container 
widget and set the width property of the Container to double.infinity.

\begin{lstlisting}
Container(
  width: double.infinity,
  child: FloatingActionButton(
    onPressed: () {
      // Handle button press
    },
    child: Icon(Icons.add),
  ),
)
\end{lstlisting}

In this example, the Container widget has its width property set to double.infinity, which means it will take up the
full available width. The FloatingActionButton is the child of the Container.

By setting the width of the Container, you can control the width of the FloatingActionButton and make it extend 
to the full width of its parent widget, which in this case is the Container.

Question: center row

Answer: To center a Row widget horizontally, you can wrap it with an Align widget and set its alignment 
property to Alignment.center. Here's an example:

\begin{lstlisting}
Align(
  alignment: Alignment.center,
  child: Row(
    mainAxisAlignment: MainAxisAlignment.center,
    children: [
      // Add your widgets here
    ],
  ),
)
\end{lstlisting}

In this example, the Align widget is used to center the Row horizontally. The alignment property is set to 
Alignment.center, which positions the child Row at the center of the available space.

Inside the Row, you can add your desired widgets, such as Text, Icon, or any other widget that you want to display 
in a centered row.

By adjusting the mainAxisAlignment property of the Row, you can control how the children widgets are aligned within 
the row. Setting it to MainAxisAlignment.center will center the children horizontally within the row.

Question: Flutter Create Selector with list of icons

Question: Add Color Picker to Form

Question: get random MaterialColor

Answer: To get a random MaterialColor in Flutter, you can use the Colors.primaries property, which is a list of 
the standard material design colors defined by Flutter. You can select a random color from this list to create
a MaterialColor object.

\begin{lstlisting}
import 'package:flutter/material.dart';
import 'dart:math';

void main() {
  MaterialColor randomColor = getRandomMaterialColor();
  print(randomColor);
}

MaterialColor getRandomMaterialColor() {
  List<Color> colors = Colors.primaries;
  Random random = Random();
  Color randomColor = colors[random.nextInt(colors.length)];
  return MaterialColor(randomColor.value, {
    50: randomColor.shade50,
    100: randomColor.shade100,
    200: randomColor.shade200,
    300: randomColor.shade300,
    400: randomColor.shade400,
    500: randomColor.shade500,
    600: randomColor.shade600,
    700: randomColor.shade700,
    800: randomColor.shade800,
    900: randomColor.shade900,
  });
}
\end{lstlisting}

In this example, the getRandomMaterialColor function selects a random color from the Colors.primaries list using the 
Random class. It then creates a MaterialColor object using the selected color and returns it.

Note that the MaterialColor constructor requires shades of the color, so we create a Map with shade values ranging 
from 50 to 900 using the shade methods provided by the selected color.

Question: (id: 1, name: 'Title') - DropdownButton how to show text value value by select id

Answer: To display the text value in a DropdownButton based on the selected ID, you can use a combination of a list 
of items, a DropdownMenuItem, and the value and onChanged properties of the DropdownButton.

\begin{lstlisting}
import 'package:flutter/material.dart';

void main() {
  runApp(MyApp());
}

class Item {
  final int id;
  final String name;

  Item({required this.id, required this.name});
}

class MyApp extends StatelessWidget {
  final List<Item> items = [
    Item(id: 1, name: 'Title 1'),
    Item(id: 2, name: 'Title 2'),
    Item(id: 3, name: 'Title 3'),
  ];

  @override
  Widget build(BuildContext context) {
    return MaterialApp(
      home: Scaffold(
        appBar: AppBar(
          title: Text('DropdownButton Demo'),
        ),
        body: Center(
          child: DropdownButton<int>(
            value: 1, // Selected ID
            onChanged: (int? newValue) {
              // Update the selected ID here
              // You can use the newValue to perform any desired action
            },
            items: items.map((Item item) {
              return DropdownMenuItem<int>(
                value: item.id,
                child: Text(item.name),
              );
            }).toList(),
          ),
        ),
      ),
    );
  }
}
\end{lstlisting}

In this example, we have a list of Item objects, each containing an id and a name. The DropdownButton is configured 
with the items property, which maps each Item in the list to a DropdownMenuItem. The value property of the 
DropdownButton is set to the selected ID, which in this case is 1.

To handle the selection change, you can provide a callback to the onChanged property of the DropdownButton. 
In the example, the onChanged function is defined but doesn't perform any action. You can update this function 
to update the selected ID or perform any other desired action based on the selected ID.

Question: flutter get list of currencies

Question: value is not shown right after changed state

Answer: To ensure that the updated value is displayed in the TextFormField after the state has changed, you can make 
use of the ValueKey parameter and the key property of the TextFormField.

\begin{lstlisting}
class CurrencySelector extends StatelessWidget {
  TextEditingController _controller = TextEditingController();
  Function setState;
  Currency? value;

  CurrencySelector({
    Key? key,
    this.value,
    required this.setState,
  }) : super(key: key ?? ValueKey(value));

  String getValue(Currency? value) {
    return value != null ? '${value.symbol} - ${value.name} (${value.code})' : '';
  }

  @override
  Widget build(BuildContext context) {
    _controller.text = getValue(value);
    return TextFormField(
      key: ValueKey(value),
      controller: _controller,
      decoration: InputDecoration(
        filled: true,
        border: InputBorder.none,
        fillColor: Theme.of(context).colorScheme.inversePrimary.withOpacity(0.0),
        suffixIcon: GestureDetector(
          onTap: () {
            showCurrencyPicker(
              context: context,
              showFlag: true,
              showCurrencyName: true,
              showCurrencyCode: true,
              onSelect: (Currency currency) => setState(currency),
            );
            _controller.text = getValue(value);
          },
          child: const Icon(Icons.arrow_drop_down),
        ),
      ),
    );
  }
}
\end{lstlisting}

In this updated code, we set the key property of the TextFormField to ValueKey(value) using the null-aware operator ??. 
This ensures that when the state changes and the CurrencySelector is rebuilt, the TextFormField will receive a 
new key based on the updated value. This will trigger the TextFormField to update its value accordingly.

Question: Disable entering something into the field

To disable entering something into a TextFormField in Flutter, you can use the readOnly property. 

\begin{lstlisting}
TextFormField(
  readOnly: true,
),
\end{lstlisting}


By setting readOnly to true, the field becomes non-editable, and the user won't be able to enter any text into it. 
The field will still display the label and any existing value set in the controller.

Question: prevent from typing anything outside of definition TextInputType

Answer: To prevent users from typing anything outside the defined TextInputType, you can create a custom 
TextInputFormatter and attach it to the inputFormatters property of the TextFormField.

\begin{lstlisting}
class LimitedTextInputFormatter extends TextInputFormatter {
  final TextInputType textInputType;

  LimitedTextInputFormatter(this.textInputType);

  @override
  TextEditingValue formatEditUpdate(
      TextEditingValue oldValue, TextEditingValue newValue) {
    if (newValue.text.isNotEmpty &&
        !_isValidInput(newValue.text, textInputType)) {
      // Return the old value if the new value is not valid
      return oldValue;
    }
    return newValue;
  }

  bool _isValidInput(String text, TextInputType textInputType) {
    switch (textInputType) {
      case TextInputType.number:
        return double.tryParse(text) != null;
      case TextInputType.phone:
        return text.contains(RegExp(r'^[0-9+\-()\s]+$'));
      // Add more cases for other types if needed
      default:
        return true; // Allow any input for unsupported types
    }
  }
}
\end{lstlisting}

To use the LimitedTextInputFormatter in your TextFormField, you can pass it as an element of the inputFormatters list:

\begin{lstlisting}
TextFormField(
  keyboardType: textInputType,
  inputFormatters: [
    LimitedTextInputFormatter(textInputType),
  ],
  // Other properties...
)
\end{lstlisting}

In the above code, textInputType is the desired TextInputType for the TextFormField. The LimitedTextInputFormatter 
will validate the input based on the specified TextInputType and allow or disallow input accordingly.

FilteringTextInputFormatter.allow for double

To allow input of double values using FilteringTextInputFormatter.allow, you can define a pattern to allow specific 
characters in the input. Here's an example of how to allow input of double values using 
FilteringTextInputFormatter.allow:

\begin{lstlisting}
TextFormField(
  keyboardType: TextInputType.numberWithOptions(decimal: true),
  inputFormatters: [
    FilteringTextInputFormatter.allow(RegExp(r'^(\d+)?\.?\d{0,2}')),
  ],
  // Other properties...
)

In this example, RegExp(r'^(\d+)?\.?\d{0,2}') is used as the pattern to allow the following input:

    An optional sequence of digits (\d+).
    An optional decimal point (\.?).
    Up to two decimal places (\d{0,2}).
\end{lstlisting}

This pattern ensures that the input allows digits, an optional decimal point, and up to two decimal places.

By setting keyboardType to TextInputType.numberWithOptions(decimal: true), the keyboard will include a 
decimal point button for convenience.

Question: Scaffold.of(context).openDrawer() leads to error Scaffold.of() called with a context that does 
not contain a Scaffold.

The error "Scaffold.of() called with a context that does not contain a Scaffold" occurs when you try to access 
the Scaffold widget using the Scaffold.of() method but the BuildContext you provide does not have a Scaffold 
widget as an ancestor.

To resolve this error, make sure that you are calling Scaffold.of() within a context where a Scaffold widget 
exists. Here are a few possible solutions:

Wrap the widget that contains the Scaffold.of() call with a Builder widget:

\begin{lstlisting}
Builder(
  builder: (BuildContext context) {
    return YourWidget(
      onPressed: () {
        Scaffold.of(context).openDrawer();
      },
    );
  },
)
\end{lstlisting}

If the widget that contains the Scaffold.of() call is already inside a Builder widget, ensure that the BuildContext 
passed to Scaffold.of() is the correct one. Double-check that you are using the correct BuildContext in your code.

If you are trying to open the drawer from a different widget, you can consider using a callback or state management 
approach. Pass a callback function from the parent widget that owns the Scaffold to the child widget, and call that 
function to trigger the drawer opening.


Question: how to create dynamic routing?

Answer: To create dynamic routing in Flutter, you can use a combination of named routes and route parameters. 

\begin{lstlisting}
MaterialApp(
    // Other app configurations...
    
    onGenerateRoute: (settings) {
        if (settings.name!.startsWith("/app/finance/account/add/")) {
        // Extract the UUID from the route path
        final uuid = settings.name!.substring(25);
    
        return MaterialPageRoute(
            builder: (context) => HomePage(id: uuid),
        );
        }
    
        // Handle other routes if needed
    
        return null; // Return null if the route is not supported
    },
)     
\end{lstlisting}

In this example, we're using onGenerateRoute to handle routes that start with "/app/finance/account/add/". 
We extract the UUID from the route path by removing the common prefix, and then we create and return the 
appropriate widget, in this case, HomePage, with the extracted UUID.

Question: `[{id: 1}, {id: 2}]` how to find by id

Answer: To find an item in a list based on its ID, you can use the firstWhere method with a condition that 
matches the desired ID. 

\begin{lstlisting}
List<Map<String, dynamic>> items = [
  {'id': 1},
  {'id': 2},
];

int targetId = 2;

Map<String, dynamic>? foundItem = items.firstWhere((item) => item['id'] == targetId, orElse: () => null);

if (foundItem != null) {
  print('Item found: $foundItem');
} else {
  print('Item not found');
}
\end{lstlisting}

In this example, we have a list of maps items containing items with IDs. We want to find an item with ID 2. 
The firstWhere method iterates over the list and returns the first item that satisfies the condition 
item['id'] == targetId. If no item is found, the orElse argument is invoked, and we set foundItem to null. 
Finally, we check if foundItem is not null and print the result accordingly.

Error: For some reasons from AccountEditPageState widget.uuid is not accessible, and fails with an error The getter 
'uuid' isn't defined for the type 'AccountAddPage'.

\begin{lstlisting}
import 'package:flutter/material.dart';

class AccountEditPage extends AccountAddPage {
  String uuid;

  AccountEditPage({
    required this.uuid,
  }) : super();

  @override
  AccountEditPageState createState() => AccountEditPageState();
}

class AccountEditPageState extends AccountAddPageState<AccountEditPage> {
  String get uuid => widget.uuid; // ERROR: The getter 'uuid' isn't defined for the type 'AccountAddPage'.
}


class AccountAddPage extends AbstractPage {
  String? title;

  AccountAddPage({
    this.title,
  }) : super();

  @override
  AccountAddPageState createState() => AccountAddPageState();
}

class AccountAddPageState<T extends AccountAddPage> extends AbstractPageState<AccountAddPage> {
  String? get title => widget.title; // OK
}


abstract class AbstractPage<T> extends StatefulWidget {
  int selectedMenu = 0;

  AbstractPage() : super(key: UniqueKey());
}

abstract class AbstractPageState<T extends AbstractPage> extends State<T> {
  @override
  Widget build(BuildContext context) {
    return Text('test');
  }
}
\end{lstlisting}

Resolution - (widget as AccountEditPage).uuid


Question: do we have enum for Flutter

Answer: Yes, Flutter provides an enum class that you can use to define enumerations. Enums in Flutter work similarly to 
enums in other programming languages. You can use enums to define a set of named values, and each value represents 
a constant.

\begin{lstlisting}
enum Color {
  red,
  green,
  blue,
}

void main() {
  Color myColor = Color.red;

  switch (myColor) {
    case Color.red:
      print('Selected color: Red');
      break;
    case Color.green:
      print('Selected color: Green');
      break;
    case Color.blue:
      print('Selected color: Blue');
      break;
  }
}
\end{lstlisting}

In this example, we defined an enum called Color with three possible values: red, green, and blue. We created a 
variable myColor and assigned it the value Color.red. Then, using a switch statement, we checked the value of 
myColor and printed a corresponding message based on the selected color.

To validate function arguments using an enum in Flutter, you can define the function parameter with the enum type and 
use a switch statement to handle the different enum values.

\begin{lstlisting}
enum Status {
  active,
  inactive,
  pending,
}

void processStatus(Status status) {
  switch (status) {
    case Status.active:
      print('Processing active status');
      // Perform actions for active status
      break;
    case Status.inactive:
      print('Processing inactive status');
      // Perform actions for inactive status
      break;
    case Status.pending:
      print('Processing pending status');
      // Perform actions for pending status
      break;
  }
}

void main() {
  Status myStatus = Status.active;

  processStatus(myStatus);
}
\end{lstlisting}

In this example, we defined an enum called Status with three possible values: active, inactive, and pending. The 
processStatus function takes a parameter of type Status. Inside the function, we use a switch statement to 
handle each possible status value and perform the corresponding actions.

In the main function, we created a variable myStatus and assigned it the value Status.active. We then passed 
myStatus as an argument to the processStatus function, which will execute the appropriate code block based on 
the provided enum value.

By using enums in function parameters, you can enforce valid enum values and provide clear validation and handling 
for each possible value within the function.

you can use the toString() method of the AppAccountType enum to convert it to a string representation.

To convert a string representation back to an enum in Flutter, you can use the AppAccountType.values property and a 
loop or a lookup map to match the string with the corresponding enum value.

\begin{lstlisting}
enum AppAccountType {
  type1,
  type2,
  type3,
}

void main() {
  String accountTypeString = AppAccountType.type1.toString(); // Convert enum to string

  // Option 1: Use a loop to match the string with the enum value
  AppAccountType? accountType;
  for (var value in AppAccountType.values) {
    if (value.toString() == accountTypeString) {
      accountType = value;
      break;
    }
  }
  print('Converted enum value: $accountType');

  // Option 2: Use a lookup map to match the string with the enum value
  final accountTypeMap = {
    'type1': AppAccountType.type1,
    'type2': AppAccountType.type2,
    'type3': AppAccountType.type3,
  };
  AppAccountType? accountTypeFromMap = accountTypeMap[accountTypeString];
  print('Converted enum value from map: $accountTypeFromMap');
}
\end{lstlisting}

In this example, we start with a string representation of the AppAccountType.type1 enum value using 
AppAccountType.type1.toString().

Option 1 demonstrates how to match the string with the enum value using a loop. We iterate over the 
AppAccountType.values property and compare each enum's string representation with the desired string. 
If a match is found, we assign the corresponding enum value to the accountType variable.

Option 2 shows how to use a lookup map to match the string with the enum value. We define a map where the keys 
are the string representations of the enum values, and the values are the actual enum values. We then use the 
string representation as the key to retrieve the enum value from the map.

[<- to be fully refactored, and adjusted to context of development flow]
==========================================================================

By migrating our data structures to class objects it can be used additionally getter and setter for class variables.
That will help us to cover, as an example, `description`-attribute for Budget structures by showing 
`Spent Amount / Total Budget`, and for `details`-attribute - `Left Amount`.

\begin{lstlisting}
class MyClass {
  String _name; // Private variable

  // Getter for name
  String get name => _name;

  // Setter for name
  set name(String value) {
    _name = value;
  }
}

void main() {
  MyClass myObject = MyClass();

  // Using the setter
  myObject.name = 'John';

  // Using the getter
  print(myObject.name); // Output: John
}
\end{lstlisting}

In the example above, the `MyClass`-class has a private variable `\_name`. The getter for `\_name` is defined 
using the get keyword, and the setter is defined using the set keyword. The getter returns the value of 
`\_name`, and the setter sets the value of `\_name` to the provided value. By using that approach we can control 
the access to class variables and perform additional logic or validation when getting or setting their values.

One more question, is how to set `DateTime.now()` inside a constructor as default value, since, by defining it inside
the constructor, an error would be taken: The default value of an optional parameter must be constant.

So, to set a default value for a DateTime variable as the current date and time, we can use the `DateTime.now()` method 
during the initialization of the variable.

\begin{lstlisting}
class MyClass {
    DateTime _createdAt;

    MyClass({DateTime? createdAt})
        : _createdAt = createdAt ?? DateTime.now();

    DateTime get createdAt => _createdAt;
}
\end{lstlisting}

Migration our Bills to classes construction has lead to an error on .firstWhere when Account of Budget not found. 
To resolve that we have to cast our getter (otherwise it would be an error: type '() => dynamic' is not a subtype 
of type '(() => BudgetAppData)?' of 'orElse') and add orElse operator as following:

\begin{lstlisting}
  dynamic getByUuid(AppDataType property, String uuid) {
    var scope = (_data[property] as Map)['list'];
    return scope.cast<dynamic>() // cast to cover different types, 
    .firstWhere(
      (item) => item.uuid == uuid,
      orElse: () => null, // Return null if not found
    );
  }
\end{lstlisting}

Now we're going to add three pages controlled by swiping left and right for adding new bills, we'll use a combination 
of PageView and PageController. 

\begin{lstlisting}
class Page1 extends StatelessWidget {
  @override
  Widget build(BuildContext context) {
    return Container(
      color: Colors.blue,
      child: Center(
        child: Text('Page 1'),
      ),
    );
  }
}

class Page2 extends StatelessWidget {
  @override
  Widget build(BuildContext context) {
    return Container(
      color: Colors.green,
      child: Center(
        child: Text('Page 2'),
      ),
    );
  }
}

class Page3 extends StatelessWidget {
  @override
  Widget build(BuildContext context) {
    return Container(
      color: Colors.orange,
      child: Center(
        child: Text('Page 3'),
      ),
    );
  }
}

class MyApp extends StatelessWidget {
  final PageController _pageController = PageController(initialPage: 0);

  @override
  Widget build(BuildContext context) {
    return MaterialApp(
      home: Scaffold(
        body: GestureDetector(
          onHorizontalDragEnd: (DragEndDetails details) {
            if (details.primaryVelocity! > 0) {
              // Swiped right
              _pageController.previousPage(
                duration: Duration(milliseconds: 500),
                curve: Curves.ease,
              );
            } else if (details.primaryVelocity! < 0) {
              // Swiped left
              _pageController.nextPage(
                duration: Duration(milliseconds: 500),
                curve: Curves.ease,
              );
            }
          },
          child: PageView(
            controller: _pageController,
            children: [
              Page1(),
              Page2(),
              Page3(),
            ],
          ),
        ),
      ),
    );
  }
}
\end{lstlisting}

To operate with visible tabs it can be changed to:

\begin{lstlisting}
class MyApp extends StatelessWidget {
  final PageController _pageController = PageController(initialPage: 0);

  @override
  Widget build(BuildContext context) {
    return MaterialApp(
      home: DefaultTabController(
        length: 3,
        child: Scaffold(
          appBar: AppBar(
            title: Text('Swiping Pages'),
            bottom: TabBar(
              tabs: [
                Tab(text: 'Page 1'),
                Tab(text: 'Page 2'),
                Tab(text: 'Page 3'),
              ],
            ),
          ),
          body: TabBarView(
            children: [
              Page1(),
              Page2(),
              Page3(),
            ],
          ),
        ),
      ),
    );
  }
}
\end{lstlisting}

Let's combine all together:

\begin{lstlisting}
class MyApp extends StatelessWidget {
  final int tabCount = 3;
  int tabIndex = 1;
  PageController? pageController;
  TabController? tabController;

  @override
  void initState() {
    super.initState();
    pageController = PageController(initialPage: tabIndex);
    tabController = TabController(
      length: tabCount,
      vsync: const _VSync(),
      initialIndex: tabIndex,
    );
  }

  @override
  void dispose() {
    pageController?.dispose();
    tabController?.dispose();
    super.dispose();
  }

  void switchTab(int newIndex) {
    setState(() {
      tabIndex = newIndex;
      tabController?.animateTo(newIndex);
      pageController?.animateToPage(
        newIndex,
        duration: Duration(milliseconds: 300),
        curve: Curves.ease,
      );
    });
  }


  @override
  Widget build(BuildContext context) {
    return MaterialApp(
      home: GestureDetector(
        onHorizontalDragEnd: (DragEndDetails details) {
          if (details.primaryVelocity! > 0) {
            switchTab(widget.tabIndex - 1);
          } else if (details.primaryVelocity! < 0) {
            switchTab(widget.tabIndex + 1);
          }
        },
        child: Scaffold(
          appBar: TabBar(
              controller: tabController,
              onTap: switchTab,
              tabs: [
                Tab(text: 'Page 1'),
                Tab(text: 'Page 2'),
                Tab(text: 'Page 3'),
              ],
            ),
          body: PageView(
            controller: pageController,
            onPageChanged: switchTab,
            children: [
              Page1(),
              Page2(),
              Page3(),
            ],
          ),
        ),
      ),
    );
  }
}

class _VSync implements TickerProvider {
  const _VSync();

  @override
  Ticker createTicker(TickerCallback onTick) {
    return Ticker(onTick);
  }
}

class Page1 extends StatelessWidget {
  @override
  Widget build(BuildContext context) {
    return Container(
      child: Center(
        child: Text('Page 1'),
      ),
    );
  }
}

class Page2 extends StatelessWidget {
  @override
  Widget build(BuildContext context) {
    return Container(
      child: Center(
        child: Text('Page 2'),
      ),
    );
  }
}

class Page3 extends StatelessWidget {
  @override
  Widget build(BuildContext context) {
    return Container(
      child: Center(
        child: Text('Page 3'),
      ),
    );
  }
}
\end{lstlisting}

By combining both solutions, some irritating behavior is happened, - we cannot tab from `Income' to `Transfer' 
(more than one position) since `pageController` after the first movement will send an update via `onPageChanged`. 
Hopefully, we do have a solution for that by introducing delayed trigger to proceed with switching to a chosen tab:

\begin{lstlisting}
Future<void> delaySwitchTab(int delay, int newIndex) async {
  await Future.delayed(Duration(milliseconds: delay));
  switchTab(newIndex);
}

void switchTab(int newIndex) {
  if (newIndex < 0 || newIndex >= widget.tabCount) {
    return;
  }
  setState(() {
    const delay = 300;
    final currIndex = widget.tabIndex; // Saving current state for the check after 
    widget.tabIndex = newIndex;
    widget.tabController?.animateTo(newIndex);
    widget.pageController?.animateToPage(
      newIndex,
      duration: const Duration(milliseconds: delay),
      curve: Curves.ease,
    );
    // Verify that the difference is more than one
    if ((currIndex - newIndex).abs() > 1) {
      delaySwitchTab(delay, newIndex);
    }
  });
}
\end{lstlisting}

While defining functions as arguments for widgets, it can be used an expectation of that function structure

\begin{lstlisting}
typedef SetViewFunction = String Function(Currency input);

class CurrencySelector extends StatelessWidget {
  SetViewFunction? setView;

  CurrencySelector({
    Key? key,
    this.setView,
  }) {
    setView ??= setDefaultView;
  }

  String setDefaultView(Currency input) {
    // Default implementation
  }
}
\end{lstlisting}