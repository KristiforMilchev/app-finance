% Copyright 2023 The terCAD team. All rights reserved.
% Use of this content is governed by a CC BY-NC-ND 4.0 license that can be found in the LICENSE file.

\subsection{Adding Telemetry}
\markboth{Implementing Core Functionality}{Adding Telemetry}

\subsubsection{Activating Google Analytics}

To understand how users behave (user experience), where are they from (to prioritize localization), what's the scope of
troubles they're facing with (exception handling); we might use any external analytical instruments. One of them is 
Google Analytics.

The process is started from creating a Firebase account and a new project there (by clicking on the relative 
button and proceeding with obvious steps; just don't forget to enable Google Analytics while creating the project): 
\href{https://console.firebase.google.com}{https://console.firebase.google.com}.

Then we have to add packages `flutter pub add firebase\_core` and `flutter pub add firebase\_analytics`, activate 
FlutterFire CLI  via `dart pub global activate flutterfire\_cli` and `npm install -g firebase-tools` to make it 
(`flutterfire`-command) globally accessible for our environment. As a note for Windows users, be attentive to 
notifications like that (add `bin`-folder to PATH global variable, \cref{img:fs-windows-path}):

\begin{lstlisting}[language=bash]
Warning: Pub installs executables into C:\Users\...\AppData\Local\Pub\Cache\bin, which is not on your path.
\end{lstlisting}

\noindent By triggering sequence of `firebase login` and `flutterfire configure` from a command line 
we'll be able to configure the project.

\begin{lstlisting}[language=bash]
> firebase login

i  Firebase optionally collects CLI and Emulator Suite usage and error reporting information to help improve our 
products. Data is collected in accordance with Google`s privacy policy (https://policies.google.com/privacy) and 
is not used to identify you.

? Allow Firebase to collect CLI and Emulator Suite usage and error reporting information? Yes
i  To change your data collection preference at any time, run `firebase logout` and log in again.

Visit this URL on this device to log in:
https://accounts.google.com/o/oauth2/auth?...

Waiting for authentication...

+  Success! Logged in as ...


> flutterfire configure

i Found 1 Firebase projects.
V Select a Firebase project to configure your Flutter application with  fingram-9030a (Fingram)
V Which platforms should your configuration support (use arrow keys & space to select)? - macos, web, ios, android
# ... other info

Firebase configuration file lib/firebase_options.dart generated successfully with the following Firebase apps:

Platform  Firebase App Id
web       ...
android   ...
ios       ...
macos     ...

Learn more about using this file and next steps from the documentation:
  https://firebase.google.com/docs/flutter/setup
\end{lstlisting}

\noindent Once completed, we should import the generated file and adjust `main`-method:

\begin{lstlisting}
// ./lib/main.dart
import 'package:app_finance/firebase_options.dart';
import 'package:firebase_core/firebase_core.dart';

void main() async {
  WidgetsFlutterBinding.ensureInitialized();
  await Firebase.initializeApp(
    options: DefaultFirebaseOptions.currentPlatform,
  );
  runApp(/* app initialization*/);
}
\end{lstlisting}

\noindent That might be the case for web, android, ios, and macos; but we have additionally linux and windows that are 
not supported by now, so, let's patch `firebase\_options.dart`-file to return null instead of throwing an exception:

\begin{lstlisting}
// ./lib/firebase_options.dart
class DefaultFirebaseOptions {
  static FirebaseOptions? get currentPlatform {
    if (kIsWeb) {
      return web;
    }
    switch (defaultTargetPlatform) {
      case TargetPlatform.android:
        return android;
      case TargetPlatform.iOS:
        return ios;
      case TargetPlatform.macOS:
        return macos;
      case TargetPlatform.windows:
      case TargetPlatform.linux:
      default:
        return null;
    }
  }
\end{lstlisting}

Alternatively, it can be implemented own cross-functional component by following Firebase REST API (
\href{https://firebase.google.com/docs/reference/rest/database}{https://firebase.google.com/docs/reference/rest/database}
)... but it's a topic for another discussion.

After all of that modifications it's needed to evaluate `flutter clean` to avoid any errors; and after 
`flutter run`-command we would be able to view a log report on 
\href{https://analytics.google.com}{https://analytics.google.com}, \cref{img:pt-analytics}.

\img{prototyping/google-analytics-report}{Google Analytics results}{img:pt-analytics}

The negative impact here is that `firebase\_analytics` library force to use 19 version as a minimal SDK for Android, 
so, it's needed to be updated `android/app/build.gradle` file by adjusting `minSdkVersion`-property:

\begin{lstlisting}[language=yaml]
## ./android/app/build.gradle
android {
  defaultConfig {
\end{lstlisting}
{
\xpretocmd{\lstlisting}{\vspace{-12pt}}{}{}
\begin{lstlisting}[firstnumber=4, backgroundcolor=\color{backred}]
(*@\kdiff{-}@*)    minSdkVersion flutter.minSdkVersion
\end{lstlisting}
\begin{lstlisting}[firstnumber=4, backgroundcolor=\color{backgreen}]
(*@\kdiff{+}@*)    minSdkVersion 19
\end{lstlisting}
\begin{lstlisting}[firstnumber=5]
  }
}
\end{lstlisting}
}
