% Copyright 2023 The terCAD team. All rights reserved.
% Use of this content is governed by a CC BY-NC-ND 4.0 license that can be found in the LICENSE file.

\subsection{Apple Store}
\markboth{Distributing}{Apple Store}

\subsubsection{iOS Build}

Preliminaries from \href{https://docs.flutter.dev/deployment/ios}{https://docs.flutter.dev/deployment/ios} emphasize 
the necessity of Xcode but what if we do not have any device running macOS to follow this guide, but yet aspire to 
publish the app to Apple Store. In that case we can proceed with our CI/CD pipelines on GitHub:

\begin{lstlisting}[language=yaml]
build:
  name: Create ${{ matrix.target }} build
  runs-on: ${{ matrix.os }}
    strategy:
      matrix:
        target: [iOS, ...]
        include:
          - os: macos-latest
            target: iOS
            build_path: build/ios/ipa
            asset_extension: .ipa
            asset_content_type: application/zip
  steps:
  - name: Run iOS Build
    if: matrix.target == 'iOS'
    run: flutter build -v ios --build-name=${{ version }} --build-number=${{ number }} --release --no-tree-shake-icons
\end{lstlisting}

\noindent Unfortunately, this change fails, creating the expected hurdle in the form of our package not being signed:

\begin{lstlisting}[language=terminal]
No valid code signing certificates were found
For more information, please visit:
  https://help.apple.com/xcode/mac/current/#/dev3a05256b8
Error: Process completed with exit code 1.
\end{lstlisting}

Our focus doesn't lie in acquiring a personal certificate for developmental purposes (since it can be used 
\q{flutter build -v ios --no-codesign} or next steps with chosen type "\emph{[Software] Apple Development: Sign 
development versions of your iOS, iPadOS, macOS, tvOS, watchOS, and visionOS apps}"); rather, we're concerned with 
obtaining a distribution certificate. The iOS distribution certificate ensures that the app's code originates from the 
authorized organization and remains unaltered. To initiate this process, log in to your Apple Developer account and 
proceed to Certificates
(\href{https://developer.apple.com/account/resources/certificates/list}{https://developer.apple.com/account/resources/certificates/list}).

Differentiation between development and distribution is in the \q{ios/Runner.xcodeproj/project.pbxproj}-file definitions:

\begin{lstlisting}[language=terminal]
# Distribution certificate is applied
"CODE_SIGN_IDENTITY[sdk=iphoneos*]" = "iPhone Distribution";
# Development certificate would be used
"CODE_SIGN_IDENTITY[sdk=iphoneos*]" = "iPhone Developer";
\end{lstlisting}

By choosing "\emph{[Software] Apple Distribution: Sign your iOS, iPadOS, macOS, tvOS, watchOS, and visionOS apps for 
release testing using Ad Hoc distribution or for submission to the App Store}" we're asked to upload self-signed 
certificate. From macOS it can be done by using "Keychain Access", for other environments -- via \q{openssl} (download 
\q{Win32 OpenSSL} for Windows):

\begin{lstlisting}[language=terminal]
# Create a private key
$ openssl genrsa -out apple.key 2048
# Create the CSR file (C - Country Name [2 letter code])
$ openssl req -new -key apple.key \ 
    -out CertificateSigningRequest.certSigningRequest \
    -subj "/emailAddress=your@mail.to, CN=Your Name, C=XX"
\end{lstlisting}

After the certificate uploading and processing (by taking back \q{.cer}-file from site) it's needed to create 
"Identifier" with a set of permissions 
(\href{https://developer.apple.com/account/resources/identifiers/list}{https://developer.apple.com/account/resources/identifiers/list})
and initiate the app itself 
(\href{https://appstoreconnect.apple.com/apps}{https://appstoreconnect.apple.com/apps}) by defining supported 
environments and providing screenshots with description.

While we've downloaded \q{.cer}-file, it's needed to be converted into a string (base64) representation:

\begin{lstlisting}[language=terminal]
# Convert the .cer-file into .pem-format
$ openssl x509 -in apple.cer -inform DER -out apple.pem -outform PEM
# Use .pem-file and private .key to generate .p12
$ openssl pkcs12 -export -legacy -out apple.p12 -inkey apple.key -in apple.pem
Enter Export Password: #####
# Convert file to base64 string
$ base64 file_name.p12 > apple.txt
\end{lstlisting}

\noindent Usage of \q{legacy}-attribute is a key factor to avoid a security failure "SecKeychainItemImport: MAC 
verification failed" since OpenSSL 3.x has changed its default algorithm in \q{pkcs12} which is not compatible with 
embedded Security frameworks in macOS/iOS.

Then paste generated \q{.txt}-file content into secrets for GitHub together with p12 password (\cref{img:d-secrets}), 
and extend pipeline by additional Actions:

\begin{lstlisting}[language=yaml]
- name: Import Distribution Certificate
  uses: apple-actions/import-codesign-certs@v2
  with: 
    keychain: signing_distr
    p12-file-base64: ${{ secrets.APPLE_P12 }}
    p12-password: ${{ secrets.APPLE_P12_KEY }}
\end{lstlisting}

\img{distributing/github-secrets}{GitHub Repository Settings - Configuring Secrets}{img:d-secrets}

Going further with resolving "\emph{Error (Xcode): Signing for "Runner" requires a development team. Select a 
development team in the Signing \& Capabilities editor}". To fix that without Xcode some "magic" is required (since 
next properties are not defined in the configuration file, generated by Flutter)... by opening 
\q{ios/Runner.xcodeproj/project.pbxproj} and adding there a few lines regarding the Team ID and Provision Profile 
(should be taken from \href{https://appstoreconnect.apple.com/access/users}{https://appstoreconnect.apple.com/access/users}
and \href{https://developer.apple.com/account/resources/profiles/list}{https://developer.apple.com/account/resources/profiles/list}
accordingly):\\
\\

{
\xpretocmd{\lstlisting}{\vspace{-14pt}}{}{}
\begin{lstlisting}[firstnumber=172]
// ./ios/Runner.xcodeproj/project.pbxproj
  TargetAttributes = {
    331C8080294A63A400263BE5 = {
			CreatedOnToolsVersion = 14.0;
\end{lstlisting}
\begin{lstlisting}[firstnumber=176, backgroundcolor=\color{backgreen}]
(*@\kdiff{+}@*)     DevelopmentTeam = 8PVKPQ758L;
(*@\kdiff{+}@*)     provisioningProfile = "Fingrom iOS";
\end{lstlisting}
\begin{lstlisting}[firstnumber=540]
            /* Debug.xcconfig */;
  buildSettings = {
    ASSETCATALOG_COMPILER_APPICON_NAME = AppIcon;
    CURRENT_PROJECT_VERSION = "$(FLUTTER_BUILD_NUMBER)";
\end{lstlisting}
\begin{lstlisting}[firstnumber=543, backgroundcolor=\color{backgreen}]
(*@\kdiff{+}@*)   DEVELOPMENT_TEAM = 8PVKPQ758L;
(*@\kdiff{+}@*)   PROVISIONING_PROFILE_SPECIFIER = "Fingrom iOS";
\end{lstlisting}
\begin{lstlisting}[firstnumber=564]
            /* Release.xcconfig */;
  buildSettings = {
    ASSETCATALOG_COMPILER_APPICON_NAME = AppIcon;
    CURRENT_PROJECT_VERSION = "$(FLUTTER_BUILD_NUMBER)";
\end{lstlisting}
\begin{lstlisting}[firstnumber=567, backgroundcolor=\color{backgreen}]
(*@\kdiff{+}@*)   DEVELOPMENT_TEAM = 8PVKPQ758L;
(*@\kdiff{+}@*)   PROVISIONING_PROFILE_SPECIFIER = "Fingrom iOS";
\end{lstlisting}
}

Provision Profile, as a second guard statement from Apple, can be received from
\href{https://developer.apple.com/account/resources/profiles/list}{https://developer.apple.com/account/resources/profiles/list}
by choosing "\emph{App Store: Create a distribution provisioning profile to submit your app to the App Store}":

\begin{lstlisting}[language=terminal]
# Convert file to base64 string
$ base64 apple.mobileprovision > apple-provision.txt
\end{lstlisting}

\noindent Proceed with adding it into GitHub secrets (\cref{img:d-secrets}) and extending pipeline by an additional step:

\begin{lstlisting}[language=yaml]
- name: Add Provision Profile
  uses: timheuer/base64-to-file@v1.2
  with:
    fileName: 'apple.mobileprovision'
    fileDir: '~/Library/MobileDevice/Provisioning\ Profiles'
    encodedString: ${{ secrets.APPLE_IOS_PROFILE }}
\end{lstlisting}

\noindent Command \q{flutter build -v ios...} is succeeded, and we've taken \q{.app}-file that has to be 
converted into \q{.ipa}-file by compressing:

\begin{lstlisting}[language=yaml]
- name: Convert APP to IPA
  run: |
    mkdir -p Payload
    mv ./build/ios/iphoneos/Runner.app Payload
    zip -r -y Payload.zip Payload/Runner.app
    mv Payload.zip fingrom_${{ matrix.target }}.ipa
\end{lstlisting}

\img{distributing/app-store-connect}{Apple Store Connect - Build Section}{img:d-apple-store}

\noindent Turn to upload generated artifact into \q{Build}-section of the application page on 
\href{https://appstoreconnect.apple.com/}{https://appstoreconnect.apple.com/} (\cref{img:d-apple-store}) by using 
\q{altool}, but before that it's needed to generate API Key from 
\href{https://appstoreconnect.apple.com/access/api}{https://appstoreconnect.apple.com/access/api} with "App Manager"
access, then download the \q{.p8}-key (and convert to base64 for GitHub secrets), and copy \q{Issuer ID} and \q{KEY ID}.

\begin{lstlisting}[language=yaml]
- name: Upload to Apple Store
  env:
    API_CERTIFICATE_BASE64: ${{ secrets.APPLE_API_P8 }}
  run: |
    # Apply API private key
    mkdir -p ~/private_keys
    echo -n "$API_CERTIFICATE_BASE64" | base64 --decode --output AuthKey_${{ secrets.APPLE_API_KEY }}.p8
    # Validate package (is needed for "build ID" generation)
    xcrun altool --validate-app --type ios -f fingrom_${{ matrix.target }}.ipa \ 
      --apiKey ${{ secrets.APPLE_API_KEY }} \
      --apiIssuer ${{ secrets.APPLE_API_ISSUE }}
    # Upload
    xcrun altool --upload-app --type ios -f fingrom_${{ matrix.target }}.ipa \
      --apiKey ${{ secrets.APPLE_API_KEY }} \
      --apiIssuer ${{ secrets.APPLE_API_ISSUE }}
\end{lstlisting}

\noindent The culmination of this process (\cref{img:d-apple-build}) marks the successful completion of our automated 
distribution workflow. It's a moment marked by a reassuring message from the pipeline: \q{UPLOAD SUCCEEDED}. 
This message signifies not just a technical achievement but also a testament to the diligent efforts and meticulous 
planning that have propelled us toward this accomplishment. With this milestone achieved, we are now one step closer 
in sharing our application with others.

\img{distributing/app-store-upload}{Apple Store Connect - Uploaded Build}{img:d-apple-build}

\noindent P.S. With the current flow a message from Apple would be taken "ITMS-90426: Invalid Swift Support" \issue{227}{},
and the artifact is gonna to be deleted from the application' page. The problem is in our custom conversion into 
\q{.ipa}-format, that has to be replaced by \q{ipa}-typed build (\q{obfuscate} and \q{split-debug-info} is mostly to 
minimize the package size but also declared as safeguards against a reserve engineering):

\begin{lstlisting}[language=terminal]
$ flutter build ipa --build-name=${{ version }} \
  --build-number=${{ number }} \
  --release --obfuscate \
  --split-debug-info=build/ios/symbols \
  --no-tree-shake-icons \
  --export-options-plist ./ios/ExportOptions.plist
\end{lstlisting}

\noindent The key point is in a usage of \q{--export-options-plist} with the following content of \q{ExportOptions.plist}:

\begin{lstlisting}[language=xml]
<?xml version="1.0" encoding="UTF-8"?>
<!DOCTYPE plist PUBLIC "-//Apple//DTD PLIST 1.0//EN" "http://www.apple.com/DTDs/PropertyList-1.0.dtd">
<plist version="1.0">
<dict>
    <key>method</key>
    <string>app-store</string>
    <key>provisioningProfiles</key>
    <dict>
      <key>com.tercad.fingrom</key>
      <string>Fingrom iOS</string>
    </dict>
</dict>
</plist>
\end{lstlisting}

\noindent In case of any encryption usage, it's needed to be changed \q{./ios/Runner/Info.plist}:

\begin{lstlisting}[language=xml]
<key>ITSAppUsesNonExemptEncryption</key>
<true/>
<key>ITSAppEncryptionExportComplianceCode</key>
<string>NNNNNNNNNNNNNNNN</string>
\end{lstlisting}

\noindent and next form (if the application is going to be available in France marketplace) 
\href{https://www.ssi.gouv.fr/en/regulation/cryptology/how-to-submit-an-application/}{https://www.ssi.gouv.fr/en/regulation/cryptology/how-to-submit-an-application/}
to be sent to \q{controle@ssi.gouv.fr} with a title "[formalités] {organization} - {app name}", and attached the approval 
of it during the processing of "Export Compliance Information" form in "Apple Store Connect". Mentioned 
\q{ITSAppEncryptionExportComplianceCode}-code can be taken from 
\href{https://appstoreconnect.apple.com/apps}{appstoreconnect.apple.com/apps/\q{id}/features/encryption}
(\cref{img:d-apple-compliance}).

\img{distributing/app-store-compliance}{Apple Store Connect: App Services -- Encryption}{img:d-apple-compliance}

In addition, pre-release builds can be made available for external testing via 
\href{https://testflight.apple.com/a}{TestFlight}. To achieve that, we have to create \q{External Testing}-group from 
the \q{TestFlight}-tab on the page of our application, and submit the build for review (\cref{img:d-apple-review}).
After those two steps (and taken approval), we'll be able to enable external access by using \q{Enable Public Link}.

\img{distributing/app-store-review}{Apple Store Connect: App Services -- Encryption}{img:d-apple-review}


\subsubsection{macOS Build}

The process of creating \q{macOS}-package for a distribution \issue{287}{} is almost the same as for \q{iOS} with a few 
deviations.

As previously shown, the provisioning profile can be installed on the device as a whole, but the practice is to embed 
the profile within the application itself (iOS expects to find the profile at \q{MyApp.app/embedded.mobileprovision}, 
macOS expects to find the profile at \q{MyApp.app/Contents/embedded.provisionprofile}; \q{.provisionprofile}-file can 
be obtained from selecting Mac App Store in 
\href{https://developer.apple.com/account/resources/profiles/add}{https://developer.apple.com/account/resources/profiles/add}):

\begin{lstlisting}[language=yaml]
- name: Run macOS Build
  run: flutter build -v macos --release

- name: Add the Provisioning Profile
  env:
    PROVISIONING_MAC_BASE64: ${{ secrets.APPLE_MAC_PROFILE }}
  run: |
    PP_PATH=$RUNNER_TEMP/Fingrom_macOS.provisionprofile
    echo -n "$PROVISIONING_MAC_BASE64" | base64 --decode \
        --output $PP_PATH
    CONTENT=build/macos/Build/Products/Release/Fingrom.app/Contents
    cp $PP_PATH $CONTENT/embedded.provisionprofile
\end{lstlisting}


\noindent As a note, the distribution certificate "3rd Party Mac Developer Application" comes from selecting Mac App 
Distribution in
\href{https://developer.apple.com/account/resources/certificates/add}{https://developer.apple.com/account/resources/certificates/add}.
The certificate "3rd Party Mac Developer Installer" can be taken by selecting Mac Installer Distribution from the same 
page. And the application has to signed by using both of them:

\begin{lstlisting}[language=yaml]
- name: Import macOS Distribution Certificate
  if: matrix.target == 'macOS'
  uses: apple-actions/import-codesign-certs@v2
  with: 
    keychain: signing_distr
    p12-file-base64: ${{ secrets.APPLE_MAC_P12 }}
    p12-password: ${{ secrets.APPLE_P12_KEY }}

- name: Sign macOS Package
  if: matrix.target == 'macOS'
  run: |
    # Sign ID is taken from a certificate (Developer ID)
    codesign --deep --force -s "8PVKPQ758L" --entitlements \
      macos/Runner/Release.entitlements "Fingrom.app"
  # build_path: build/macos/Build/Products/Release
  working-directory: ${{ matrix.build_path }}

- name: Import macOS Install Certificate
  uses: apple-actions/import-codesign-certs@v2
  with: 
    keychain: signing_distr
    p12-file-base64: ${{ secrets.APPLE_MAC_INSTALLER }}
    p12-password: ${{ secrets.APPLE_P12_KEY }}

- name: Upload macOS Package to Apple Store
  run: |
    xcrun productbuild --component Fingrom.app \
        /Applications/ unsigned.pkg
    xcrun productsign --sign "8PVKPQ758L" unsigned.pkg Fingrom.pkg
    # Validate package
    xcrun altool --validate-app --type macos -f Fingrom.pkg \
      --apiKey ${{ secrets.APPLE_API_KEY }} \
      --apiIssuer ${{ secrets.APPLE_API_ISSUE }}
    # Upload package
    xcrun altool --upload-app --type macos -f Fingrom.pkg \
      --apiKey ${{ secrets.APPLE_API_KEY }} \
      --apiIssuer ${{ secrets.APPLE_API_ISSUE }}
  working-directory: ${{ matrix.build_path }}
\end{lstlisting}

\img{distributing/app-store-macos}{Apple Store Connect: macOS Release}{img:d-apple-macos}

\noindent The submission might fail because of the next error:

\begin{lstlisting}[language=terminal]
Error: Asset validation failed App sandbox not enabled. 
  The following executables must include the 
  "com.apple.security.app-sandbox" entitlement with a 
  Boolean value of true in the entitlements property list
\end{lstlisting}

\noindent While \q{app-sandbox} is enabled by default for both compilations (Debug and Release), the problem is in a 
missed \q{com.apple.security.inherit}. To enable sandbox inheritance, a child target must use exactly two App Sandbox 
entitlement keys:

\begin{lstlisting}[language=xml]
<!-- ./macos/Runner/Release.entitlements -->
<dict>
  <key>com.apple.security.app-sandbox</key>
  <true/>
\end{lstlisting}
{
\xpretocmd{\lstlisting}{\vspace{-14pt}}{}{}
\begin{lstlisting}[firstnumber=5, language=xml, backgroundcolor=\color{backgreen}]
(*@\kdiff{+}@*) <key>com.apple.security.inherit</key>
(*@\kdiff{+}@*) <true/>
\end{lstlisting}
}

\noindent It's important to note that the end-of-line (EOL) character for the \q{macos/Runner/Release.entitlements}-file 
should be set to "LF" (Line Feed). Using "LF" as the EOL character is a common practice in macOS development, and 
any other formatting might lead to a not recognized entitlements list.

