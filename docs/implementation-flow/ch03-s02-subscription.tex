% Copyright 2023 The terCAD team. All rights reserved.
% Use of this content is governed by a CC BY-NC-ND 4.0 license that can be found in the LICENSE file.

\subsubsection{Adding Subscriptions}

While we've chosen a Freemium business model (with a broken rule for limitations by being open-sourced, since Freemium 
[portmanteau of "free" and "premium"] is a pricing strategy by which a basic product is free, but money is charged for 
additional functionality), it's a time to land a subscription page.


\paragraph{Adding Custom References}

For Windows, Linux, and Web releases the simplest way of adding subscriptions would be to provide links to external 
services, that're specialized on crowdfunding initiatives:

\begin{itemize}
  \item  \href{https://github.com/sponsors}{https://github.com/sponsors}
  \item  \href{https://patreon.com}{https://patreon.com}
  \item  \href{https://donorbox.org}{https://donorbox.org}
  \item  \href{https://paypal.me}{https://paypal.me}
  \item  and many others...
\end{itemize}

\noindent Implementation of external links is possible by `url\_launcher`-package:

\begin{lstlisting}
import 'package:url_launcher/url_launcher.dart';

class OtherWidget extends StatelessWidget {
  const OtherWidget({super.key});

  Future<void> _launchURL(String path) async {
    final url = Uri.parse(path);
    await launchUrl(url, mode: LaunchMode.externalApplication);
  }

  Widget build(BuildContext context) {
    return ElevatedButton(
      onPressed: () => _launchURL('https://www.buymeacoffee.com/lyskouski'),
      child: Text('Buy me a coffee'),
    ),
  }
}
\end{lstlisting}


\paragraph{Enabling Google/Apply Pay}

In-app purchases are required for mobile app stores since it cannot be used any third-party systems to handle payments
by the terms and conditions of marketplaces.

To work with Google and Apple subscriptions we need to be registered there as a trusted developers / companies. Google 
Play Developer account (\href{https://play.google.com/console}{https://play.google.com/console}; set up and create a 
subscription product in the "Store Presence" section) would costs us 25 Euro at once, Apple Store Developer 
account (\href{https://developer.apple.com}{https://developer.apple.com}; via Apple Developer account create App ID and 
configure the in-app purchase settings) - roughly 100 Euro per year. Don't forget about other marketplaces as 
Galaxy and Huawei Store, by example, since they may provide some extra bonuses and coupons for a services usage.

There would be a couple of steps to verify your identity, add bank account, sign documents (tax forms) and agreements.

As an example, for Apple Store:
\begin{enumerate}
  \item Activate paid apps on page: \href{https://appstoreconnect.apple.com/agreements}{https://appstoreconnect.apple.com/agreements}
  \item Register Bundle ID: \href{https://developer.apple.com/account/resources/identifiers/bundleId/add/bundle}{https://developer.apple.com/account/resources/identifiers/bundleId/add/bundle}
  \item Enroll small-business support: \href{https://developer.apple.com/app-store/small-business-program/enroll/}{https://developer.apple.com/app-store/small-business-program/enroll/}
  \item Fulfill app definition: \href{https://appstoreconnect.apple.com/apps}{https://appstoreconnect.apple.com/apps}
  \item Via "Features" > "In-App Purchases" [Create] and "Subscriptions" [Create] declare costs
\end{enumerate}

In-app purchases are differentiated by their applicability: consumables (applicable multiple times), non-consumables 
(once), subscriptions (interval payments). And three different ways to integrate them into Flutter application:
`in\_app\_purchase`, `flutter\_inapp\_purchase`, `purchases\_flutter`.
